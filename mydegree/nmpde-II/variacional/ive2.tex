%
%
%

\chapter{An\'alisis num\'erico de un EVI II}

\section{Problema de un fluido r\'{\i}gido$-$viscopl\'astico}

El problema es el siguiente, calcular $u\in V$ tal que $\forall \ v \in V$:

\begin{displaymath}
a\ (u,v-u)+g\cdot j\ (v)-g\cdot j\ (u) \ge L\ (v)
\end{displaymath}
donde $V=H^1_0(\Omega )$ y $g>0$ y:

\begin{displaymath}
a\ (u,v) =\mu \cdot \int_{\Omega} (\nabla u \cdot \nabla v),\qquad L\ (v) =
<\ell,v> = \int_{\Omega }f\cdot v
\end{displaymath}
donde $\mu >0$.

\begin{displaymath}
j\ (v) =\int_{\Omega } |\nabla v| dx
\end{displaymath}

De ahora en adelante a este problema nos referiremos como $(P_2)$.

\newpage

\section{Teorema de existencia y unicidad}

\begin{teorema}[Existencia y unicidad de $(P_2)$]
\ \\
El problema $(P_2)$ tiene soluci\'on \'unica.
\end{teorema}

\begin{demosteorema}
\ \\
El problema $(P_2)$ cumple todas las propiedades para la existencia y unicidad
de soluci\'on de un problema EVI II\footnote{Teorema $\ref{th:exisEVIII}$ en la
p\'agina $\pageref{th:exisEVIII}$.}, todas ellas son inmediatas salvo que el
funcional $j\ (\cdot )$ sea convexo, propio y \textbf{s.c.i}.\\

Toda seminorma es convexa, en particular toda norma es convexa, por lo tanto
si $j\ (\cdot )$ es norma sobre $V$ entonces ser\'a convexa. Veamos que
$j\ (\cdot )$ es norma\footnote{Pero no ser\'a equivalente a la norma natural
de este espacio $||\cdot ||_{0,1}$.} sobre $V=H^1_0(\Omega )$.

\begin{itemize}
\item $j\ (\cdot )$ es norma sobre $V=H^1_0(\Omega )$.
\begin{enumerate}
\item $j\ (v+w) \le j\ (v)+j\ (w)$, $\forall \ v,w \in V$.
\begin{eqnarray*}
j\ (v+w)&=&\int_{\Omega } |\nabla (v+w)| \le \int_{\Omega }(|\nabla v|+
|\nabla w|) = \\ 
&=& \int_{\Omega }|\nabla v|+\int_{\Omega }|\nabla w|=j\ (v) + j\ (w)
\end{eqnarray*}

\item $j\ (\lambda \cdot v) = |\lambda |\cdot j\ (v)$, $\forall \ v\in V$ y
$\lambda \in \Re$.

\begin{displaymath}
j\ (\lambda \cdot v) =\int_{\Omega } |\nabla (\lambda \cdot v)|\le |\lambda |
\cdot \int_{\Omega } |\nabla v| = |\lambda |\cdot j\ (v)
\end{displaymath}

\item $j\ (v)= 0\Longleftrightarrow v = 0$, $\forall \ v \in V$.

\begin{displaymath}
j\ (v) = 0 \Longleftrightarrow \int_{\Omega } |\nabla v| = 0
\Longleftrightarrow |\nabla v| = 0\ c.t.p\ \Omega
\end{displaymath}
Si $|\nabla v|= 0$ en c.t.p $\Omega$\footnote{En casi todo punto de $\Omega$.}
entonces se tiene que $v = cte$ en c.t.p $\Omega$, pero como $v_{|\Gamma}=0$,
ya que $v\in H^1_0(\Omega )$ entonces se tiene que $v=0$ en c.t.p $\Omega$.

\end{enumerate}

\end{itemize}
Tenemos que $j\ (\cdot )$ es norma, en consecuencia es convexa.\\ 

Veamos que $j\ (\cdot )$ es continua, lo cual se deduce de ser norma. Para
demostrar la continuidad de $j\ (\cdot )$ utilizaremos la siguiente propiedad,
la cual se deduce de la desigualdad triangular:

\begin{displaymath}
|\ ||u||-||v||\ |\le ||u-v||
\end{displaymath}
Como $j\ (\cdot )$ es norma esta expresi\'on la podemos escribir como:

\begin{displaymath}
|j\ (u)-j\ (v)|\le j(u-v)
\end{displaymath}

\begin{itemize}
\item $j\ (\cdot )$ es continua en $V$.\\ \\
Por lo anterior se tiene que:

\begin{displaymath}
|j\ (v_2)-j\ (v_1))|\le j\ (v_2-v_1) = \int_{\Omega }|\nabla (v_2-v_1)|
\end{displaymath}
y por Cauchy$-$Schwarz se tiene:
\begin{eqnarray*}
|j\ (v_2)-j\ (v_1)|&\le &\sqrt{\int_{\Omega} 1^2}\cdot
\sqrt{\int_{\Omega}|\nabla (v_2-v_1)|^2} = \\
&=& \sqrt{\mu(\Omega )}\cdot |v_2-v_1|_{1,\Omega }
\end{eqnarray*}
Recordemos que la seminorma $|\cdot |_{1,\Omega }$ es equivalente a la norma
$||\cdot ||_{1,\Omega }$\footnote{Norma de $H^1(\Omega )$.} en $H^1_0(\Omega )$,
por lo tanto se tiene que:

\begin{displaymath}
|j\ (v_2)-j\ (v_1)|\le \sqrt{\mu (\Omega )}\cdot |v_2-v_1|_{1,\Omega }
\longrightarrow 0
\end{displaymath}
cuando $v_2\to v_1$ en $H^1_0(\Omega )$.
\end{itemize}
Luego $j\ (\cdot )$ es continua y Lipschitziana de constante
$\sqrt{\mu (\Omega )}$.\\ \\
Tenemos que $j\ (\cdot )$ es continua y tambi\'en propia, entonces dado que es
continua y convexa se tiene que es \textbf{s.c.i}.

\begin{flushright}
$\blacksquare$
\end{flushright}
\end{demosteorema}

\newpage

\section{Propiedades de regularidad}

\begin{teorema}[Brezis]
\ \\
Si $\Omega $ es de frontera suficientemente regular y $f \in L^2(\Omega )$
entonces $\mu \in H^2(\Omega )$.\\

Adem\'as si $\Omega $ es convexo se tiene que:

\begin{displaymath}
||u||_{2,\Omega }\le \frac{C(\Omega )}{\mu}\cdot ||f||_{o,\Omega }
\end{displaymath}
mayorizaci\'on independiente de $g$.
\end{teorema}

\section{Algunas propiedades de la soluci\'on}

Si $g$ es suficientemente grande, tenemos que $u\equiv 0$\footnote{El liquido
se convertir\'{\i}a en solido.}, donde $u$ es la soluci\'on de $(P_2)$.\\ \\
Para demostrar esto en el caso en que $f\in L^\infty (\Omega )$ admitiremos el
lema que vamos a demostrar a continuaci\'on de esta demostraci\'on.\\

En $(P_2)$ cogemos $v=0$ y tenemos:

\begin{equation}\label{eq:v=0}
-\mu |u|^2_{1,\Omega }-g\ j\ (u)\ge - \int_{\Omega } fu
\end{equation}
y tomando en $(P_2)$ $v=2u$ tenemos

\begin{equation} \label{eq:v=2u}
\mu |u|^2_{1,\Omega }+g\ j\ (u)\ge \int_{\Omega } fu
\end{equation}
De $(\ref{eq:v=0})$ y $(\ref{eq:v=2u})$ se tiene que:

\begin{equation} \label{eq:util}
\mu |u|^2_{1,\Omega }+g\ j\ (u) =\int_{\Omega }fu
\end{equation}
Utilizando el lema que demostraremos a continuaci\'on tendremos:

\begin{displaymath}
\mu \int_{\Omega }|\nabla u|^2 + \frac{g}{\beta}\cdot ||u||_{0,1,\Omega }\le
\int_{\Omega }fu
\end{displaymath}
Utilizando la desigualdad de Holder y teniendo en cuenta que
$f\in L^{\infty } (\Omega )$ y $u\in L^1 (\Omega )$ tenemos:

\begin{displaymath}
\mu \int_{\Omega }|\nabla u|^2+\frac{g}{\beta}\cdot ||u||_{0,1,\Omega }\le
||f||_{0,\infty ,\Omega }\cdot ||u||_{0,1,\Omega }
\end{displaymath}
es decir:

\begin{displaymath}
\mu \cdot |u|^2_{1,\Omega }+\frac{g}{\beta}\cdot ||u||_{0,1,\Omega }\le
||f||_{0,\infty ,\Omega }\cdot ||u||_{0,1,\Omega }
\end{displaymath}
de donde se deduce que:

\begin{displaymath}
\mu \cdot |u|^2_{1,\Omega }\le \{ ||f||_{0,\infty ,\Omega}-\frac{g}{\beta} \}
||u||_{1,0\Omega }
\end{displaymath}
Es decir si $g\ge\beta $ entonces $u\equiv 0$.
\begin{flushright}
$\blacksquare$
\end{flushright}

\begin{lema}
\ \\
Existe $\beta$ tal que $\forall \ v \in H^1_0 (\Omega )$ se verifica:

\begin{displaymath}
||v||_{0,1,\Omega }\le \beta \cdot j\ (v)
\end{displaymath}
Recordemos que:

\begin{displaymath}
||v||_{0,1,\Omega }=\int_{\Omega }|v|
\end{displaymath}
\end{lema}

\begin{demoslema}
\ \\
Supondremos que $\Omega $ esta acotado en una direcci\'on, por ejemplo en 
$x_2$, es decir que $\forall \ (x_1,x_2)\in \Omega$ se tiene que
$a\le x_2\le b$. Entonces basta demostrarlo $\forall \ v \in
\mathcal{D}(\Omega)$, donde:

\begin{displaymath}
\mathcal{D}(\Omega )=\{Funciones\ \mathcal{C}^{\infty}(\Omega )\ con\ soporte\
compacto\ en\ \Omega .\}
\end{displaymath}
Sea $v\in \mathcal{D}(\Omega )$ y sea $\overline{v}\in \mathcal{D}(\Re ^d)$ su
prolongaci\'on por cero a todo $\Re ^d$.

\begin{displaymath}
|\overline{v}(x_1,x_2)| = |\int_a^{x_2} \frac{\partial \overline{v}}
{\partial x_2}\ (x_1,\varepsilon)\ d\varepsilon|
\end{displaymath}
luego se tiene que:

\begin{displaymath}
|\overline{v}(x_1,x_2)|\le \int_a^{x_2}|\frac{\partial \overline{v}}
{\partial x_2}\ (x_1,\varepsilon)|\ d\varepsilon \le \int_a^b 
|\frac{\partial \overline{v}}{\partial x_2}\ (x_1,\varepsilon) |\ d\varepsilon
\end{displaymath}
Integrando en $\Re$ respecto de $x_1$ en esta \'ultima expresi\'on tendremos:

\begin{eqnarray*}
\int_{\Re} |\overline(x_1,x_2)|\ dx_1&\le& \int_{\Re} \int_a^b 
|\frac{\partial \overline{v}}{\partial x_2}\ (x_1,x_2)|\ dx_1\ dx_2\le \\
&\le &\int_{\Omega }|\nabla v|\ dx_1\ dx_2 = j\ (v)
\end{eqnarray*}
Finalmente integrando entre $a$ y $b$:

\begin{eqnarray*}
||v||_{0,1,\Omega }&=& ||\overline{v}||_{0,1,\Re^d} =\int_a^b \int_{\Re}
|\overline{v} (x_1,x_2)|\ dx_1\ dx_2\le \\
&\le & \int_a^bj\ (v)\ dx_1\ dx_2 = (b-a)\cdot j\ (v)
\end{eqnarray*}
Por densidad el resultado es v\'alido $\forall \ v \in H^1_0(\Omega )$ ya que
$j\ (\cdot )$ es continua en $H^1_0(\Omega )$.

\begin{flushright}
$\blacksquare$
\end{flushright}
\end{demoslema}

\subsection{Observaciones}

\begin{enumerate}
\item El teorema es cierto para $f\in L^2(\Omega )$. Se demuestra utilizando un
lema debido a \emph{M.Strauss$-$L.Niremberg}.\\ \\
Existe $\alpha $ tal que $\forall \ v \in H^1_0(\Omega )$ se verifica:

\begin{displaymath}
||v||_{0,\Omega }\le \alpha \cdot \int_{\Omega } |\nabla u| = \alpha \cdot
j\ (v)
\end{displaymath}
\end{enumerate}

\newpage

\section{Resultados de dualidad}

\begin{teorema}
\ \\
Para $d=2$ se tiene:

\begin{displaymath}
\Lambda = \{\mu \in L^{\infty}(\Omega)\times L^{\infty}(\Omega )\ |\mu \ (x)|
\le 1\ c.t.p.\ \Omega \}
\end{displaymath}

\begin{displaymath}
\mu = (\mu_1,\mu_2)\qquad |\mu \ (x)| = \sqrt{\mu_1\ (x)^2+\mu_2\ (x)^2 }
\end{displaymath}
La soluci\'on $u$ de $(P_2)$ esta caracterizada por la existencia de
$\lambda \in \Lambda$ tal que el par $(u,\lambda )$ verifica la propiedad
siguiente:

\begin{displaymath}
(P') \left\{ \begin{array}{lr}
\mu \int_{\Omega } \nabla u \cdot \nabla v + g 
\int_{\Omega }  \lambda \cdot \nabla v = \int_{\Omega } fv & \forall \ v \in
H^1_0 (\Omega ) \\
\\
\lambda \cdot \nabla u = |\nabla u| & c.t.p \ de\ \Omega \\
\\
(u,\lambda) \in H^1_0 (\Omega )\times \Lambda
\end{array} \right.
\end{displaymath}

\end{teorema}

\begin{demosteorema}
\ \\
Utilizaremos el teorema de Hanh$-$Banach.\\ \\ 
$\Rightarrow |$Por ser $u$ soluci\'on de $(P_2)$ se tiene:

\begin{equation} \label{eq:ateorema}
a\ (u,v-u)+g \int_{\Omega }|\nabla v|-g \int_{\Omega } |\nabla u|
\ge <\ell,v-u>=L\ (v-u)
\end{equation}
Sumando $(\ref{eq:util})$ y $(\ref{eq:ateorema})$ tenemos:

\begin{displaymath}
a\ (u,v)+g \int_{\Omega} |\nabla v|\ge <\ell,v>
\end{displaymath}
es decir:

\begin{displaymath}
<\ell,v>-a\ (u,v) \le g \int_{\Omega} |\nabla v|\qquad \forall \ v \in
H^1_0(\Omega )
\end{displaymath}
Tomando $-v$ por $v$ en la \'ultima expresi\'on:

\begin{displaymath}
<\ell,v> -a\ (u,v) \ge -g \int_{\Omega }|\nabla u|\qquad \forall \ v \in
H^1_0(\Omega )
\end{displaymath}
de donde tenemos:

\begin{equation} \label{eq:cteorema}
|<\ell,v>-a\ (u,v)|\le g \int_{\Omega }|\nabla v| = g
||\nabla v||_{0,1,\Omega }
\end{equation}
Sea $S$ el siguiente subespacio de $L^1(\Omega )\times L^1(\Omega )$:

\begin{displaymath}
S=\{q \in L^1(\Omega )\times L^1(\Omega )\ tal\ que\ q=\nabla v\ \ v\in
H^1_0(\Omega ) \}
\end{displaymath}
Definamos ahora la siguiente aplicaci\'on $T:S\longrightarrow \Re$:

\begin{displaymath}
T\ (q) = <\ell,v>-a\ (u,v)
\end{displaymath}
donde $v$ verifica $q=\nabla v$. Observar que $T\ (\cdot )$ esta bien definida
ya que no puede haber dos $v$ verificando $q=\nabla v$ ya que entonces el
gradiente de la diferencia ser\'{\i}a cero con lo cual la funci\'on ser\'{\i}a
constante y como $v \in H^1_0(\Omega )$ entonces $v=0$.

\begin{itemize}
\item $T\ (\cdot )$ es lineal. Es inmediato.
\item $T\ (\cdot )$ es continua. Tenemos que demostrar lo siguiente:

\begin{displaymath}
|T\ (q)|\le C\cdot ||q||_{0,1,\Omega }=C\cdot ||\nabla v||_{0,1,\Omega }
\end{displaymath}
Sabemos que:

\begin{displaymath}
\mu \int_{\Omega }\nabla \cdot u\nabla (v-u) +g \int_{\Omega }
|\nabla v|-g \int_{\Omega }|\nabla u|\ge \int_{\Omega }f(v-u)
\end{displaymath}
ya que $u$ es soluci\'on de $(P_2)$.\\ \\
Si sumamos esta expresi\'on con $(\ref{eq:util})$ obtenemos:

\begin{displaymath}
\int_{\Omega }fu-\mu \int_{\Omega }\nabla u \cdot \nabla v \le
-g \int_{\Omega } |\nabla v|\qquad \forall \ v \in H^1_0(\Omega )=V
\end{displaymath}
tomando valores absolutos en esta expresi\'on tenemos:

\begin{displaymath}
|\int_{\Omega }fu -\mu \int_{\Omega }\nabla u\cdot \nabla v|\le
g \int_{\Omega }|\nabla v|\qquad \forall\ v\in H^1_0(\Omega )
\end{displaymath}
es decir:

\begin{displaymath}
|T\ (q)|\le g\int_{\Omega }|\nabla v| = g\int_{\Omega }|q|=g\cdot
||q||_{0,1,\Omega }\qquad \forall \ q\in S
\end{displaymath}
Luego $T\ (\cdot )$ es continua y adem\'as se tiene que $||T||\le g$.
\end{itemize}
Aplicando el teorema de Hanh$-$Banach podemos extender a todo el espacio
\mbox{$L^1(\Omega )\times L^1(\Omega )$} la aplicaci\'on $T\ (\cdot)$, es
decir existe una aplicaci\'on lineal y continua,
$\overline{T}:L^1(\Omega )\times L^1(\Omega )\longrightarrow \Re$, tal que
$||\overline{T}||\le g$ y $\overline{T}\ (q)=T\ (q)$ $\forall \ q \in S$.\\ \\
Luego $\overline{T}\in (L^1(\Omega )\times L^1(\Omega ))^*=
L^{\infty}(\Omega )\times L^{\infty}(\Omega )$, luego existe








\begin{flushright}
$\blacksquare$
\end{flushright}
\end{demosteorema}

\subsection{Observaciones}

\begin{enumerate}
\item Si $a\ (\cdot ,\cdot )$ es sim\'etrica el problema anterior es equivalente
a resolver:

\begin{displaymath}
J\ (u) =\inf_{v\in V} J\ (v)
\end{displaymath}
donde:

\begin{displaymath} 
J\ (v) = \frac{1}{2}\ \mu \int_{\Omega }|\nabla v|^2+j\ (v)-\int_{\Omega} fv
\end{displaymath}
\end{enumerate}
