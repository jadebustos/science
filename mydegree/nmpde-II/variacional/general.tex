%
% CAPITULO 1 Generalidades sobre inecuaciones variacionales elpticas y
%		sus aproximaciones

\chapter{Generalidades sobre inecuaciones variacionales el\'{\i}pticas y sus
aproximaciones}

\section{Introducci\'on}

Las inecuaciones variacionales son una clase de problemas no lineales, 
importantes y muy utiles, que aparecen en mec\'anica y f\'{\i}sica \dots.
Principalmente consideraremos los dos siguientes tipos de inecuaciones
variacionales:

\begin{enumerate}
\item Inecuaci\'on Variacional El\'{\i}ptica (\textbf{EVI}).
\item Inecuaci\'on Variacional Parab\'olica (\textbf{PVI}).
\end{enumerate}

\section{Marco funcional}

En esta secci\'on consideraremos dos clase de \textbf{EVI}, llamados
\textbf{EVI} de primera y segunda clase.

\subsection{Notaciones}

\begin{itemize}
\item $V$ Espacio Real de Hilbert con producto escalar $(\cdot ,\cdot )$
y como norma asociada $||\cdot||$.
\item $V^*$ Espacio dual del espacio $V$.
\item $a\ (\cdot ,\cdot )\ : \ V \times V \longrightarrow \Re $ forma
bilineal, continua y $V-$el\'{\i}ptica.
\end{itemize}

La forma bilineal $a\ (\cdot ,\cdot )$ se dice que es
$V-$el\'{\i}ptica si existe una constante positiva $\alpha $ tal que:

\begin{displaymath}
a \ (v,v) \ge \alpha \ ||v||^2 \qquad \forall \ v \in V
\end{displaymath}

Normalmente $a\ (\cdot ,\cdot )$ no ser\'a sim\'etrica, aunque se
puede dar el caso en que lo sea.

\begin{itemize}
\item $L  : \ V \longrightarrow \Re $ funcional lineal y continuo. Tambi\'en
podemos encontrarlo de la siguiente forma:
\begin{displaymath}
L  (v) = \ <\ell ,v> \qquad \forall \ v \in V
\end{displaymath}
\item $K \subseteq V$ cerrado, convexo y no vac\'{\i}o.
\item $j : \ V \longrightarrow \overline{\Re}$  funcional propio,
convexo y semicontinuo inferior (\textbf{s.c.i}).

\begin{displaymath}
\overline{\Re} = \Re \cup \{+\infty \}
\end{displaymath}
\end{itemize}

Que el funcional sea propio significa:

\begin{equation} \label{eq:defpropio}
j(v)\neq + \infty \qquad para \ alg\acute{u}n \ v \in V
\end{equation}

Que el funcional sea convexo significa:

\begin{equation} \label{eq:defconvexo}
j(\lambda x + (1- \lambda) y) \le \lambda j(x) + (1-\lambda ) j(y) 
\qquad \lambda \in \Re \qquad x,y \in V
\end{equation} 

Que el funcional sea \textbf{s.c.i} significa:

\begin{equation} \label{eq:defsci}
\liminf_{v_n \to v} j(v_n) \ge j(v)
\end{equation}

\subsection{EVI de primera clase (EVI I)}

Encontrar $u \in K$ tal que $u$ sea soluci\'on del problema:

\begin{equation} \label{eq:EVI2}
(P_1) \qquad \left\{ \begin{array}{lrlr}
a \ (u,v-u) & \ge & L \ (v-u) & \forall \ v \in K \\
\\
u \in K
\end{array} \right.
\end{equation}

\subsection{EVI de segunda clase (EVI II)}

Encontrar $u \in V$ tal que $u$ sea soluci\'on del problema:

\begin{equation} \label{eq:EVI1}
(P_2) \qquad \left\{ \begin{array}{lrlr}
a \ (u,v-u) + j(v)-j(u) & \ge & L \ (v-u) & \forall \ v \in V \\
\\
u \in V
\end{array} \right.
\end{equation}

\subsection{Observaciones}

\begin{enumerate}
\item Los casos anteriormente considerados son los casos m\'as simples e
importantes. \textbf{LIONS} y \textbf{BENSOUSSAN} consideraron una
generalizaci\'on del problema $(P_1)$ llamada \textbf{Inecuaci\'on Cuasi
Variacional (QVI)} la cual aparece en las ciencias de decision. Un problema
t\'{\i}pico de (QVI) es:\newline

Encontrar $u \in V$ tal que:

\begin{equation} \label{eq:QVI}
(QVI) \qquad \left\{ \begin{array}{lrlr}
a \ (u,v-u) & \ge & L \ (v-u) & \forall \ v \in K(u) \\
\\
u \in K(u)
\end{array} \right.
\end{equation}

donde $v \longrightarrow K(v)$ es una familia de cerrados, convexos no
vac\'{\i}os de $V$.

\item Si $K = V$ y $j \equiv 0$ entonces los problemas $(P_1)$ y $(P_2)$ se
reducen a la cl\'asica ecuaci\'on variacional:

\begin{equation} \label{eq:variacional}
\left\{ \begin{array}{lrlr}
a \ (u,v) & = & L \ (v) & \forall \ v \in V \\
\\
u \in V
\end{array} \right.
\end{equation}

\item Podemos considerar $(P_1)$ como un caso particular de $(P_2)$ remplazando
$j(\ \cdot \ )$ en $(P_2)$ por la funci\'on $I_K$:

\begin{equation} \label{eq:defiIK}
I_K(v) = \left\{ \begin{array}{cccc}
0 & v & \in & K \\
\\
+ \infty & v & \notin & K
\end{array} \right.
\end{equation}

Merece la pena considerar $(P_1)$ aparte de $(P_2)$ ya que aparece de forma
natural y de esta forma tendremos una visi\'on general del problema.
\end{enumerate}
\newpage
\begin{ejercicio}
Demostrar que $I_K$ es convexo, \textbf{s.c.i} y propio.
\end{ejercicio}

\begin{demosejer}
\ %\newline
\begin{enumerate}
\item Veamos primero que $I_K$ es convexa. Tendremos que demostrar que $I_K$ verifica $(\ref{eq:defconvexo})$.\\

Podemos distinguir varios casos:

\begin{itemize}
\item $x,y \in K$ %\newline

\begin{displaymath}
I_K (\lambda x + (1-\lambda )y) = 0
\end{displaymath}

ya que como $K$ es convexo $\Longrightarrow$ $\lambda x + (1-\lambda )y \in K$,
y por la definici\'on de $I_K$. De igual forma se tiene que
$I_K(x) = I_K(y) = 0$.\newline

Luego $I_K$ verifica $(\ref{eq:defconvexo})$, en este caso.

\item $x,y \notin K$

\begin{displaymath}
I_K(x) = I_K(y) = + \infty
\end{displaymath}

Se deduce de la definici\'on de $I_K$.\newline

Luego $I_K$ verifica $(\ref{eq:defconvexo})$, en este caso.

\item $x \in K, \qquad y \notin K$

\begin{displaymath}
I_K(x) = 0 \qquad I_K(y) = +\infty
\end{displaymath}

Se deduce de la definici\'on de $I_K$.\newline

Luego $I_K$ verifica $(\ref{eq:defconvexo})$, en este caso.

\item $x \notin K, \qquad y \in K$

\begin{displaymath}
I_K(x) = +\infty \qquad I_K(y) = 0
\end{displaymath}

Se deduce de la definici\'on de $I_K$.\newline

Luego $I_K$ verifica $(\ref{eq:defconvexo})$, en este caso.
\end{itemize}

Luego $I_K$ es una funci\'on convexa.

\item Veamos que $I_K$ es propia.\newline

Por definici\'on $K$ es no vac\'{\i}o $\Longrightarrow $ por lo menos tiene un
elemento $v \in K$ $\Longrightarrow I_K(v) = 0 \neq +\infty $, luego
$j \ne +\infty $. 

\item Veamos ahora que $I_K$ es \textbf{s.c.i}.\newline

$I_K$ tiene que verificar $(\ref{eq:defsci})$.\newline

Sea $\{ v_n \}_{n \in N}$ una suceci\'on en $V$, espacio de Hilbert, la
cual converge a $v$.

\begin{itemize}
\item Si algun $v_i \in K \Longrightarrow Inf I_K(v_n) = 0 \ \forall n \in N$

Luego si demostramos que $I_K(v)=0$, es decir $v \in K$ terminamos.\newline

Como $K$ es cerrado por definici\'on entonces se tiene que:

\begin{displaymath}
\{ v_{n_i} \}_{i \in I} \to v \in K \qquad \{ v_{n_i} \}_{i \in I} \subset
\{ v_n \}_{n \in N} \qquad v_{n_i} \in K \ \forall \ i \in I
\end{displaymath}

Entonces $I_K(v) = 0$.

\item Si ning\'un $v_{n_i} \in K$ se tiene que $I_K(v_{n_i}) = +\infty$, con
lo cual $(\ref{eq:defsci})$ se verifica siempre.
\end{itemize}

Luego $I_K$ es \textbf{s.c.i}.
\end{enumerate}
\end{demosejer}

\begin{ejercicio}
Ver que $(P_1)$ es equivalente al problema de encontrar $u \in V$ tal que:
\begin{displaymath}
a \ ( a , v-u ) + I_K(v) - I_K(u) \ge L\ (v-u) \qquad v \ \in V
\end{displaymath}
\end{ejercicio}

\begin{demosejer}
\ \newline
$\Rightarrow |$ Sea $u \in K \subset V$ tal que verifica $(P_1)$, entonces por
la definici\'on de $I_K$ se tiene que $I_K(u) = 0$.\newline

Teniendo en cuenta lo anterior tenemos que $u$ verifica:

\begin{displaymath}
a\ (u,v-u)-I_K(u) \ge L\ (v-u) \qquad \forall \ v \in K
\end{displaymath}

Si adem\'as tenemos en cuenta la definici\'on de $I_K$ dada por
$(\ref{eq:defiIK})$ tendremos que:

\begin{displaymath}
a\ (u,v-u)+I_K(v)-I_K(u) \ge L\ (v-u) \qquad \forall \ v \in V
\end{displaymath}
$\Leftarrow |$ Sea $u \in V$ verificando:

\begin{displaymath}
a\ (u,v-u) + I_K(v) - I_K(u) \ge L\ (v-u) \qquad \forall \ v \in V
\end{displaymath}

Como $K \subset V$ es no vac\'{\i}o entonces se tiene que $I_K$ es propio. Si
$u,v \notin K$ entonces llegamos a contradicci\'on ya que $I_K(u)=I_K(v)=+
\infty$. Luego $u,v \in K \subset V$ con lo que $I_K(u)=I_K(v) = 0$, y 
concluimos.
\end{demosejer}

\section{Resultados de existencia y unicidad para EVI I}

\subsection{Teorema de existencia y unicidad}

\begin{teorema}[Existencia y unicidad EVI I] \label{th:exisEVII}
\ \newline \newline
El problema $(P_1)$ tiene soluci\'on \'unica.
\end{teorema}

\begin{demosteorema}
\ \newline
\begin{itemize}
\item Unicidad\newline

Sean $u_1$ y $u_2$ dos soluciones de $(P_1)$, entonces tenemos:

\begin{equation} \label{eq:unicidad1}
a\ (u_1,v-u_1) \ge L\ (v-u_1) \qquad \forall \ v \in K
\end{equation}

\begin{equation} \label{eq:unicidad2}
a\ (u_2,v-u_2) \ge L\ (v-u_2) \qquad \forall \ v \in K
\end{equation}

Sustituyendo $v$ por $u_2$ en $(\ref{eq:unicidad1})$ y por $u_1$ en 
$(\ref{eq:unicidad2})$, sumando y utilizando la $V-$elipticidad de
$a\ (\ \cdot \ , \ \cdot \ )$ tenemos:

\begin{displaymath}
\alpha \ ||u_2 -u_1 ||^2 \le a\ (u_2 - u_1, u_2 - u_1) \le 0
\end{displaymath}
y como $\alpha > 0 \Longrightarrow u_1 = u_2$.

\item Existencia\newline

Vamos a reducir el problema $(P_1)$ a un problema de punto fijo.\newline

Por el teorema de representaci\'on de \emph{Riesz-Fr\'echet}, para espacios
de Hilbert, existe $A \in \pounds(V,V)$ (lineal y continua) y $\ell \in V$
tal que:

\begin{displaymath}
(Au,v) = a(u,v) \qquad \forall \ u,v \in V \qquad y \ \ L\ (v) = (\ell ,v)
\qquad \forall \ v \in V
\end{displaymath}

El problema $(P_1)$ es equivalente al problema:

\begin{equation} \label{eq:p1ro}
(P_1^{\rho}) \left\{ \begin{array}{lrlr}
(u-\rho (Au-\ell )-u, v-u) & \le & 0 & \forall \ v \in K \\
\\
u \in K,\ \rho > 0
\end{array} \right.
\end{equation}

Veamos la equivalencia:\newline \\
$\Rightarrow |$ Sea $u \in K$ tal que es soluci\'on de $(P_1)$, veamos que
es soluci\'on de $(P_1^{\rho} )$, como $u$ verifica $(P_1)$ se tiene:

\begin{displaymath}
a\ (u,v-u) - L\ (v-u) \ge 0
\end{displaymath}
y aplicando el teorema de representaci\'on de \emph{Riesz-Fr\'echet}:

\begin{displaymath}
(Au,v-u) - (\ell ,v-u) \ge 0
\end{displaymath}
como el producto escalar es bilineal se tiene que:

\begin{displaymath}
(Au-\ell ,v-u) \ge 0
\end{displaymath}
si multiplicamos por $-\rho $ a toda la desigualdad y a continuaci\'on
sumamos y restamos $u$ en el primer factor del producto escalar tenemos que
$u \in K$ verifica:

\begin{displaymath}
(u-\rho (Au-\ell)-u,v-u) \le 0 \qquad \rho > 0
\end{displaymath}

$\Leftarrow |$ Sea $u \in K$ soluci\'on de $(P_1^{\rho})$, como es soluci\'on
de $(P_1^{\rho})$ se tiene:

\begin{eqnarray*}
0 & \ge & (u-\rho (Au-\ell )-u,v-u) = -\rho (Au-\ell,v-u) = \\
  & =   & -\rho \{(Au,v-u) -(\ell ,v-u)\} = -\rho \{ a\ (u,v-u)-L\ (v-u) \}
\end{eqnarray*}
Luego se tiene que:

\begin{displaymath}
-\rho \{ a\ (u,v-u) - L(v-u)\} \le 0 \qquad \rho > 0
\end{displaymath}
multiplicando por $-\rho$:

\begin{displaymath}
\rho \{ a\ (u,v-u)-L\ (v-u) \} \ge 0	\qquad \rho > 0
\end{displaymath}
de donde se deduce:

\begin{displaymath}
a\ (u,v-u) \ge L\ (v-u)	\qquad \rho > 0
\end{displaymath}

Luego $(P_1) \equiv (P_1^{\rho})$.\newline

Sea $P_K$ el operador de proyecci\'on de $V$ a $K$ seg\'un la norma
$||\cdot ||$.\newline

El problema $(P_1^{\rho})$ es equivalente a encontrar $u \in K$ verificando
para alg\'un $\rho > 0$:

\begin{displaymath} 
P_K(u-\rho (Au-\ell )) = u
\end{displaymath}

Definamos la siguiente aplicaci\'on $W_{\rho}\ :V \longrightarrow V$ de
forma que:

\begin{equation} \label{eq:wro}
W_{\rho} (v) = P_K(v-\rho (Av-\ell))
\end{equation}

Veamos que $W_{\rho}$ es una contracci\'on estricta.\newline

Dados $v_1,v_2 \in V$ tenemos que:

\begin{eqnarray*}
||W_{\rho}(v_1)-W_{\rho}(v_2)||^2 & = & ||P_K(v_1-\rho (Av_1-\ell ))-
P_H(v_2-\rho (Av_2-\ell )) ||^2 \le \\
& \le & || v_1 -\rho Av_1 + \rho \ell -v_2 + Av_2 - \rho \ell ||^2 = \\
& = & || v_1 - v_2 - \rho A(v_1-v_2)|| ^2 \le \\
& \le & || v_1 - v_2 ||^2 - 2\rho(A(v_1-v_2),v_1-v_2)+ \\
& + & \rho^2 ||A(v_1-v_2)||^2
\end{eqnarray*}
por definici\'on se tiene que:

\begin{displaymath}
(A(v_1-v_2),v_1-v_2) = a\ (v_1-v_2,v_1-v_2) \ge \alpha ||v_1-v_2 ||^2
\end{displaymath}
luego tenemos que:

\begin{displaymath}
||W_{\rho}(v_1)-W_{\rho}(v_2)||^2 \le (1-2\alpha \rho +||A||^2 \rho^2)
||v_1-v_2||^2
\end{displaymath}

Para que sea contracci\'on estricta se tiene que verificar que:

\begin{displaymath}
1-2\alpha \rho + ||A||^2 \rho^2 < 1 \Longleftrightarrow -2\alpha \rho +
||A||^2 \rho ^2 < 0
\end{displaymath}

Luego para que $W_{\rho}$ sea una contracci\'on estricta:

\begin{displaymath}
0 < \rho < \frac{2\alpha}{||A||^2}
\end{displaymath}

Si elegimos $\rho$ adecuadamente para que $W_{\rho}$ sea una contracci\'on
estricta, entonces $P_K$ tiene un punto fijo $u \in K \subset V$, el cual es
soluci\'on de $(P_1^{\rho})$ y como $(P_1^{\rho}) \equiv (P_1)$ tenemos que
dicho $u$ es soluci\'on de $(P_1)$.

\end{itemize}

\begin{flushright}
$\blacksquare$
\end{flushright}

\end{demosteorema}

\subsection{Observaciones}

\begin{enumerate}
\item Si $K=V$ el teorema $\ref{th:exisEVII}$ es el lema de \emph{Lax-Milgram}.
\item La demostraci\'on del teorema $\ref{th:exisEVII}$ nos da un algoritmo para
resolver el problema $(P_1)$, siempre que $P_K$ sea una contracci\'on
estricta.\newline

Recordemos la definici\'on de $P_K$ y la condici\'on que tiene que verificar
$\rho$ para que sea una contracci\'on estricta:

\begin{displaymath}
v \to P_K(v-\rho(Av - \ell) ) \qquad 0<\rho < \frac{2 \alpha}{||A||^2}
\end{displaymath}
\newpage

El algoritmo es el siguiente:

\begin{itemize}
\item Dado $u^0 \in V$
\item $u^{n+1} = P_K(u^n-\rho (Au^n -\ell))$
\end{itemize}

Entonces $u^n \longrightarrow u$ fuertemente en $V$, donde $u$ es la soluci\'on
de $(P_1)$. En la practica no es facil calcular $\ell$ y $A$ cuando
$V \neq V^{*}$. Proyectar sobre $K$ puede ser tan dificil como resolver
$(P_1)$. En general este m\'etodo no suele ser usado para calcular la soluci\'on
de $(P_1)$ si $K \neq V$, al menos no directamente.\newline

Si $a\ (\cdot, \cdot)$ es sim\'etrica entonces $J'(u) = Au - \ell$ y se tiene
que:

\begin{itemize}
\item $u^{n+1} = P_K(u^n-\rho (J'(u^n)))$
\end{itemize}

este m\'etodo es conocido como el m\'etodo de la \textbf{proyecci\'on del
gradiente}.
\end{enumerate}

\begin{ejercicio}
Probar que $(J'(u),v)=a(u,v)-L(v)$ $\forall \ u,v \in V$ y deducir que
$J'(u) = Au-\ell$ (suponiendo $a\ (\cdot ,\cdot )$ sim\'etrica).
\end{ejercicio}

\begin{demosejer}
\begin{displaymath}
J(v) = \frac{1}{2}\ a\ (v,v)- L\ (v)
\end{displaymath}

Dada una funci\'on $J\ (\cdot) \in \pounds(V,\Re )$, siendo $V$ un espacio de
Hilbert, se dice que $J\ (\cdot)$ es \textbf{diferenciable en el sentido de
Gauteaux}, si existe una funci\'on, a la que denotaremos $J'\ (\cdot)$
verificando:

\begin{equation} \label{eq:defigateaux}
\lim_{\lambda \to 0} \frac{J(u+\lambda v)-J(u)}{\lambda} = (J'(u), v)
\qquad \forall \ v \in V
\end{equation}

Aplicando la definici\'on de $(J'(u),v)$ tenemos que:

\begin{displaymath}
(J'(u),v) = \lim_{\lambda \to 0} \frac{\frac{1}{2}\ a\ (u+\lambda v,u+\lambda
v)-L\ (u+\lambda v)-\{ \frac{1}{2}\ a\ (u,u)-L\ (v)\}}{\lambda}
\end{displaymath}
teniendo en cuenta que $L\ (\cdot)$ es lineal y que $a\ (\cdot ,\cdot )$ es
bilineal y sim\'etrica se tiene que:

\begin{eqnarray*}
(J'(u),v) & = & \lim_{\lambda \to 0} \frac{\lambda a\ (u,v)+
\frac{\lambda^2}{2}\ a\ (v,v)-\lambda L\ (v)}{\lambda} = \\
& = & \lim_{\lambda \to 0} \{ a\ (u,v)-L\ (v)+\frac{\lambda}{2}\ a\ (v,v) \} =
a\ (u,v) - L\ (v)
\end{eqnarray*}

Veamos ahora que $J'(u) = Au - \ell$ $\forall \ u \in V$.\newline

Por el teorema de \emph{representaci\'on de Riesz-Fr\'echet}, para espacios
de Hilbert, tenemos que existe $A \in \pounds(V,V)$ (lineal y continua) y
$\ell \in V$ tal que:

\begin{displaymath}
(Au,v) = a\ (u,v)\qquad \forall \ u,v \in V \qquad y\ \ L\ (v)=(\ell , v)
\qquad \forall \ v \in V
\end{displaymath}
Luego se tiene que:

\begin{displaymath}
a\ (u,v) - L\ (v) = (Au,v) - (\ell ,v) = (Au-\ell , v)\qquad \forall \
u,v \in V
\end{displaymath}
es decir:

\begin{displaymath}
(J'(u),v) = a\ (u,v) - L\ (v) = (Au-\ell , v) \qquad \forall \ u,v \in V
\end{displaymath}
de esto \'ultimo se deduce que $J'(u)= Au -\ell$ $\forall \ u \in V$.

\end{demosejer}

\begin{teorema}[Existencia de soluci\'on para un problema de optimizaci\'on] \label{th:exissolopt}
\ \newline

Dada $J\ : V \longrightarrow ] -\infty ,+\infty ]$ definida de la siguiente
manera:
\begin{displaymath}
J(v) = \frac{1}{2} \ a(v,v)-L(v)
\end{displaymath}
y verificando:

\begin{itemize}
\item J es convexa, propia y \textbf{s.c.i.}.
\item J es coerciva.
\begin{equation} \label{eq:deficoerciva}
\lim_{||v|| \to +\infty} J(v) = +\infty
\end{equation}
\end{itemize}
Si $K \subset V$ es convexo, cerrado y no vac\'{\i}o entonces existe, al menos
una, soluci\'on de:\newline

Hallar $u \in K$ tal que:

\begin{displaymath}
J(u) = \inf_{v \in K} J(v)
\end{displaymath}
Si $J$ es estrictamente convexa entonces la soluci\'on es \'unica (el
m\'{\i}nimo es \'unico).

\end{teorema}

\begin{demosteorema}
\ \newline

Sea $\alpha$ tal que:

\begin{displaymath}
\alpha = \inf_{v \in K} J(v)
\end{displaymath}
Puede darse el caso $\alpha=- \infty$.\newline

Podemos construir una suceci\'on en $K$ $\{ v_n \}_{n \in N}$ tal que:

\begin{displaymath}
\lim_{n \to +\infty} J(v_n) = +\infty
\end{displaymath}
La sucesi\'on $\{ v_n\}_{n \in N}$ esta acotada, es decir $||v_n || \le C$ 
$\forall \ n \in N$, ya que si esto no fuera cierto como $J$ es coerciva se
tendr\'{\i}a lo siguiente:

\begin{displaymath}
||v_n|| \mathop{\longrightarrow}_{n \to +\infty} +\infty \Longrightarrow
\lim_{n \to +\infty} J(v_n) = +\infty \qquad !!!
\end{displaymath}
Luego las $v_n$ estan acotadas $\Longrightarrow$ $\alpha$ es finito.\newline

Si estuvieramos en dimensi\'on finita donde las $v_n$ pertenecieran a una
bola cerrada (recordemos que estan acotadas) entonces podriamos extraer una
subsucesi\'on convergente, pero como estamos en espacios de dimensi\'on
infinita no podemos hacer esto.\newline

$V$ es de \emph{Hilbert} luego es \emph{Banach reflexivo}, en la
\emph{topolog\'{\i}a d\'ebil} de $V$ podemos extraer una subsucesi\'on
convergente $v_{\sigma}\longrightarrow v^{*}$.\newline

Como $J$ es \textbf{s.c.i.} se tiene (en la topolog\'{\i}a d\'ebil) que:

\begin{displaymath}
\liminf_{n \to +\infty} J(v_n) \ge J(v^{*}) \Longrightarrow \alpha \ge J(v^{*})
\end{displaymath}
Entonces $v^{*}$ es el $u$ buscado.\newline

Adem\'as como $J$ es convexa entonces $J$ no puede tomar el valor $-\infty$,
ya que si lo tomar\'a, al ser convexa, lo tomar\'{\i}a siempre.\newline

Si $J$ es estrictamente convexa, dadas $u_1$, $u_2$ dos soluciones del problema
de optimizaci\'on, que estamos intentando resolver, se tiene:

\begin{displaymath}
J(\frac{u_1 +u_2}{2})< \frac{1}{2}\ \{ J(u_1)+ J(u_2)\} \le \alpha
\end{displaymath}

con lo cual llegamos a contradicci\'on si $u_1 \neq u_2$.
\begin{flushright}
$\blacksquare$
\end{flushright}

\end{demosteorema}

\section{Resultados de existencia y unicidad para EVI II}

\subsection{Teorema de existencia y unicidad}

\begin{teorema}[Existencia y unicidad EVI II] \label{th:exisEVIII}
\ \newline

El problema $(P_2)$ tiene soluci\'on \'unica.
\end{teorema}

\begin{demosteorema}
\ \newline
\begin{itemize}
\item Unicidad \newline

Sean $u_1$ y $u_2$ dos soluciones de $(P_2)$, entonces tenemos:

\begin{equation} \label{eq:unicidadEVI2I}
a\ (u_1,v-u_1)+j\ (v)-j\ (u_1) \ge L\ (v-u_1) \qquad \forall \ v \in V
\end{equation}

\begin{equation} \label{eq:unicidadEVI2II}
a\ (u_2,v-u_2)+j\ (v)-j\ (u_2) \ge L\ (v-u_2) \qquad \forall \ v \in V
\end{equation}

Como $j(\cdot )$ es propio, entonces existe $v_0 \in V$ tal que:

\begin{displaymath}
-\infty < j\ (v_0) < + \infty
\end{displaymath}

Entonces para $i=1,2$ se tiene que:

\begin{displaymath}
-\infty < j\ (u_i) \le j\ (v_0)- L\ (v_0-u_i)+a\ (u_i,v_0,v_i)
\end{displaymath}

lo que nos indica que $j\ (u_i)$ es finito para $i=1,2$.\newline

Si sustituimos $v$ por $u_2$ en $(\ref{eq:unicidadEVI2I})$ y sustituimos $v$
por $u_1$ en $(\ref{eq:unicidadEVI2II})$ y sumamos las expresiones obtenidas
tendremos que:

\begin{displaymath}
0 \ge a\ (u_1-u_2,u_1-u_2)
\end{displaymath}

como $a\ (\cdot,\cdot )$ es $V-$el\'{\i}ptica se tiene que:

\begin{displaymath}
0 \ge a\ (u_1-u_2,u_1-u_2) \ge \alpha\ ||u_1-u_2 ||^2 \qquad \alpha > 0
\end{displaymath}

como $\alpha > 0$ se deduce que:

\begin{displaymath}
||u_1-u_2||^2=0\Longrightarrow ||u_1-u_2|| = 0\Longrightarrow u_1-u_2=0
\Longrightarrow u_1=u_2
\end{displaymath}

\item Existencia \newline

Para cada $u \in V$ y cada $\rho > 0$ asociaremos el problema $(\pi^u_{\rho})$,
de tipo $(P_2)$, siguiente:\newline

Encontrar $w \in V$ tal que:

\begin{equation} \label{eq:piuro}
(\pi^u_{\rho}) \left\{ \begin{array}{lr}
(w,v-w)+\rho j\ (v)-\rho j\ (w) \ge \\
\ge (u,v-w)+\rho L\ (v-w)-\rho a\ (u,v-w) & \forall \ v \in V \\
\\
w \in W
\end{array} \right.
\end{equation}

La ventaja de considerar este problema sobre el problema $(P_2)$ es que la 
forma bilineal asociada al problema $(\pi^u_{\rho})$ es el producto interior de
$V$, el cual es sim\'etrico.\newline

Admitiremos, de momento, que el problema $(\pi^u_{\rho})$ tiene soluci\'on
\'unica $\forall \ u \in V$ y $\rho > 0$.\newline

Para cada $\rho > 0$ definiremos la siguiente aplicaci\'on
$f_{\rho}\ :V \longrightarrow V$, definida de la siguiente manera:

\begin{displaymath}
f_{\rho}\ (u) = w \qquad donde \ \ w \ \ verifica \ \ (\pi^u_{\rho})
\end{displaymath}

Vamos a ver que $f_{\rho}\ (\cdot)$ es una contracci\'on estricta, eligiendo
adecuadamente el par\'ametro $\rho$.\newline

Sean $u_1,u_2 \in V$ tales que $f_{\rho}\ (u_i) = w_i$ para $i=1,2$.\newline

Como $j\ (\cdot)$ es propio, se puede demostrar\footnote{Al igual que hicimos
en la unicidad.} que $j\ (u_i)$ es finito para $i=1,2$.\newline

Como $w_1$ es soluci\'on de $(\pi^{u_1}_{\rho})$ se tiene que:

\begin{equation} \label{eq:existenciaI}
\begin{array}{lr}
(w_1,w_2-w_1)+\rho j\ (w_2)-\rho j\ (w_1) &\ge \\
(u_1,w_2-w_1)+\rho L\ (w_2-w_1)-\rho a\ (u_1,w_2-w_1)
\end{array}
\end{equation}

y como $w_2$ es soluci\'on de $(\pi^{u_2}_{\rho})$ se tiene que:

\begin{equation} \label{eq:existenciaII}
\begin{array}{lr}
(w_2,w_1-w_2)+\rho j\ (w_1)-\rho j\ (w_2)&\ge \\
(u_2,w_1-w_2)+\rho L\ (w_2-w_1)-\rho a\ (u_2,w_1-w_2)
\end{array}
\end{equation}

Sumando $(\ref{eq:existenciaI})$ y $(\ref{eq:existenciaII})$ tenemos que:

\begin{equation} \label{eq:existenciacontraccion}
||w_2-w_1||^2 \le ((I-\rho A)(u_2-u_1),w_2-w_1)
\end{equation}

pero por definici\'on de $f_{\rho}\ (\cdot)$ se tiene que:

\begin{equation} \label{eq:existenciadef}
||w_2-w_1||^2=||f_{\rho}\ (u_2)-f_{\rho}\ (u_1)||^2
\end{equation}

si desarrollamos el producto escalar que aparece en el segundo miembro de la
desigualdad $(\ref{eq:existenciacontraccion})$ y si tenemos en cuenta
$(\ref{eq:existenciadef})$ tendremos:

\begin{displaymath}
||f_{\rho}\ (u_2)-f_{\rho}\ (u_1)||^2 \le ||I-\rho A||\ ||u_2-u_1||\
||w_2-w_1||
\end{displaymath}

es decir:

\begin{equation} \label{eq:existenciadesigualdad}
||f_{\rho}\ (u_2)-f_{\rho}\ (u_1)|| \le ||I -\rho A||\ ||u_2-u_1||
\end{equation}

Recordemos que:

\begin{equation} \label{eq:existencianorma}
||I-\rho A|| = \sup_{||v||=1} ||v-\rho Av||
\end{equation}

para que $f_{\rho}\ (\cdot )$ sea contracci\'on estricta se tiene que
verificar que:

\begin{displaymath}
||I-\rho A|| < 1
\end{displaymath}

Desarrollemos $||v-\rho Av||$:

\begin{displaymath}
||v-\rho Av||^2 =  (v-\rho Av, v-\rho Av)
\end{displaymath}

como el producto escalar es bilineal y sim\'etrico tenemos:

\begin{displaymath}
||v-\rho Av||^2 = (v,v)-2\rho (Av,v)+\rho^2 (Av,Av)
\end{displaymath}

utilizando ahora el teorema de \emph{Riesz-Fr\'echet}\footnote{Otra vez.},
para espacios de Hilbert, y utilizando la definici\'on de norma a partir de
un producto escalar tenemos:

\begin{displaymath}
||v-\rho Av||^2 = ||v||^2-2\rho a(v,v)+ \rho^2 ||Av||^2
\end{displaymath}

como $a\ (\cdot ,\cdot )$ es $V-$el\'{\i}ptica se tiene que existe $\alpha > 0$
tal que:

\begin{displaymath}
a\ (v,v) \ge \alpha \ ||v||^2 \qquad \forall \ v \in V
\end{displaymath}

utilizando esto \'ultimo tendremos:

\begin{displaymath}
||v-\rho Av||^2 \le (1-2\alpha \rho+\rho^2 ||A||^2)\ ||v||^2 
\end{displaymath}

juntando todos estos resultados se tiene que $f_{\rho}\ (\cdot )$ ser\'a 
una contracci\'on estricta si:

\begin{displaymath}
1-2\alpha \rho+\rho^2 ||A||^2 < 1 \Longleftrightarrow 0 < \rho <
\frac{2\ \alpha}{||A||^2}
\end{displaymath}

luego si elegimos $\rho$ adecuadamente entonces $f_{\rho}\ (\cdot )$ es una
contracci\'on estricta, es decir, admite un \'unico punto fijo $u \in V$.\\

Como $u$ es un punto fijo de $f_{\rho}\ (\cdot )$ se tiene que
$f_{\rho}\ (u) = u$ y $u$ ser\'a soluci\'on de $(\pi^u_{\rho})$, por
definici\'on de $f_{\rho}\ (\cdot )$. Como $u$ verifica $(\pi^u_{\rho})$ se
tiene que $\forall \ v \in V$:

\begin{displaymath}
(u,v-u)+\rho j\ (v)-\rho j\ (u) \ge (u,v-u)+\rho L\ (v-u)-\rho a\ (u,v-u)
\end{displaymath}

es decir:

\begin{displaymath}
a\ (u,v-u)+j\ (v)-j\ (u) \ge L\ (v-u) \qquad \forall \ v \in V
\end{displaymath}

\end{itemize}

\begin{flushright}
$\blacksquare$
\end{flushright}

\end{demosteorema}

La existencia y unicidad de la soluci\'on del problema $(\pi^u_{\rho})$ se
deduce del siguiente lema.

\begin{lema}
\ \newline
Sea $b : V\times V \longrightarrow \Re$ una forma bilineal sim\'etrica,
continua y $V-$el\'{\i}ptica con constante de el\'{\i}pticidad $\beta$. Sea
$L \in V^*$ y $j:V \longrightarrow \overline{\Re}$ funcional propio, convexo y
\textbf{s.c.i.}. Sea $J (\cdot )$ definido:

\begin{displaymath}
J(v) = \frac{1}{2}\ b\ (v,v)+j(v)-L(v)
\end{displaymath}

El problema de minimizaci\'on $(\pi)$, encontrar $u$ tal que:

\begin{equation}
(\pi) \left\{ \begin{array}{lrlr}
J(u) &\le & J(v) & \forall \ v \in V\\
\\
u \in V
\end{array} \right.
\end{equation}

tiene soluci\'on \'unica, la cual esta caracterizada por:

\begin{equation} \label{eq:caracter}
\left\{ \begin{array}{lrlr}
b\ (u,v-u)+j(v)-j(u)&\ge & L (v-u) & \forall \ v \in V\\
\\
u \in V
\end{array} \right.
\end{equation}

\end{lema}

\begin{demoslema}
\ \newline

Utilizando el teorema $\ref{th:exissolopt}$, en la p\'agina
$\pageref{th:exissolopt}$, el problema $(\pi )$ tiene soluci\'on si:

\begin{itemize}
\item $J\ (\cdot )$ es convexo.\newline

Como $b\ (\cdot ,\cdot )$ es estrictamente convexa, $j\ (\cdot )$ es convexa y
$L\ (\cdot )$ es lineal tenemos que $J\ (\cdot )$ es estrictamente convexa, lo
que nos implica, seg\'un el teorema antes mencionado, la unicidad de la
soluci\'on, suponiendo que exista.

\item $J\ (\cdot )$ es \textbf{s.c.i}.\newline

Como $b\ (\cdot ,\cdot )$ y $L\ (\cdot )$ son continuas y $j\ (\cdot )$ es
\textbf{s.c.i} tenemos que $J\ (\cdot )$ es \textbf{s.c.i}.

\item $J\ (\cdot )$ es propio.\newline

Como $b\ (\cdot ,\cdot )$ y $L\ (\cdot )$ son continuas y $j\ (\cdot )$ es
propio tenemos que $J\ (\cdot )$ es propio.

\item $J\ (\cdot )$ es coerciva.\newline

Para demostrar esto vamos a utilizar una propiedad de las funciones convexas y
\textbf{s.c.i.} que es la siguiente:

\begin{quote}
Si $j\ (\cdot )$ es una funci\'on convexa y \textbf{s.c.i.} existe una funci\'on
$\lambda \ (\cdot ) \in V^*$ y $\mu \in \Re$ tal que:

\begin{displaymath}
j\ (v) \ge <\lambda ,v> + \mu = \lambda \ (v)+\mu
\end{displaymath}

\end{quote}
Por lo tanto:

\begin{eqnarray*}
J\ (v) &\ge & \frac{\beta}{2}\ ||v||^2-||\lambda ||\ ||v||-||L||\ ||v||+\mu = \\
& = & ( \sqrt{\frac{\beta}{2}}-\frac{||\lambda ||+||L||}{2}\
\sqrt{\frac{2}{\beta}}\ )^2+\mu -\frac{(||\lambda ||+||L||)^2}{2\beta }
\end{eqnarray*}
Luego $J\ (v)\to +\infty$ cuando $||v||\to +\infty$, es decir $J\ (\cdot )$ es
coerciva.

\end{itemize}

De esto se deduce que existe soluci\'on para el problema $(\pi )$, y su unicidad
viene dada por la convexidad estricta de $J\ (\cdot )$.\newline

Una vez demostrada la existencia y unicidad del problema $(\pi )$ veamos su
caracterizaci\'on.\newline

$\Rightarrow |$Sea $u$ la soluci\'on del problema $(\pi )$ y sea $0<t\le 1$
entonces $\forall \ v \in V$ tendremos:

\begin{equation} \label{eq:minimizacion}
J\ (u) \le J(u+t(v-u))
\end{equation}
Para simplificar llamaremos $J_0\ (v)$ a $\frac{1}{2}\ b\ (v,v)-L\ (v)$, es
decir:

\begin{displaymath}
J_0\ (v) = \frac{1}{2}\ b(v,v)-L\ (v)
\end{displaymath}
Sustituyendo $J\ (\cdot )$ por su valor en $(\ref{eq:minimizacion})$, agrupando
seg\'un $J_0\ (\cdot)$ y pasando todo a un miembro de la inecuaci\'on
tendremos:

\begin{displaymath}
0 \le J_0\ (u+t(v-u))-J_0\ (u)+j\ (u+t(v-u))-j\ (u) \qquad \forall \ v \in V
\end{displaymath}
y usando la convexidad de $j\ (\cdot )$ tendremos que:

\begin{displaymath}
0 \le J_0\ (u+t(v-u))-J_0\ (u)+t\ [j\ (v)-j\ (u)]\qquad \forall \ v \in V
\end{displaymath}
si dividimos por $t$ tendremos:

\begin{displaymath} 
0 \le \frac{J_0\ (u+t(v-u))-j_0\ (u)}{t} +[j\ (v)-j\ (u)]\qquad \forall 
\ v \in V
\end{displaymath}
si sustituimos $J_0\ (\cdot )$  por su valor, y teniendo en cuenta que
$L\ (\cdot )$ es lineal y que $b\ (\cdot ,\cdot )$ es bilineal y sim\'etrica
tendremos que $\forall \ v \in V$:

\begin{displaymath}
0 \le \frac{t\ b\ (u,v-u)+\frac{t^2}{2}\ b\ (v-u,v-u)-t\ L\ (v-u)}{t} +
[j\ (v)-j\ (u)]
\end{displaymath}
simplificando y tomando el limite cuando $t \to 0$ tendremos que:

\begin{displaymath}
0 \le b\ (u,v-u)-L\ (v-u)+j\ (v)-j\ (u) \qquad \forall \ v \in V
\end{displaymath}
es decir:

\begin{displaymath}
b\ (u,v-u)+j\ (v)-j\ (u) \ge L\ (v-u)\qquad \forall \ v \in V
\end{displaymath}
$u$ es soluci\'on de $(\ref{eq:caracter})$.\newline

$\Leftarrow |$ Sea $u$ una soluci\'on de $(\ref{eq:caracter})$ entonces 
$\forall \ v \in V$ se tiene que:

\begin{equation} \label{eq:asaber}
J\ (v)-J\ (u) = \frac{1}{2}\ [b\ (v,v)-b\ (u,u)]+j\ (v)-j\ (u)-L\ (v-u)
\end{equation}
si tenemos en cuenta que:

\begin{eqnarray*}
b\ (v,v) &=& b\ (u+v-u,u+v-u)=\\
&=& b\ (u,u)+2\ b\ (u,v-u)+b\ (u-v,u-v)
\end{eqnarray*}
sustituyendo esto en $(\ref{eq:asaber})$ y simplificando tendremos:

\begin{displaymath}
J\ (v)-J\ (u) = b\ (u,v-u)+j\ (v)-j\ (u)-L\ (v-u) +\frac{1}{2}\ b\ (v-u,v-u)
\end{displaymath}
ahora bien como $u$ es soluci\'on de $(\ref{eq:caracter})$ se tiene que:

\begin{displaymath}
b\ (u,v-u)+j\ (v)-j\ (u)-L\ (v-u) \ge 0\qquad \forall \ v \in V
\end{displaymath}
y como $b\ (\cdot ,\cdot )$ es $V-$el\'{\i}ptica:

\begin{displaymath}
b\ (v-u,v-u) \ge \beta \ ||v-u||^2 \ge 0\qquad \forall \ v \in V
\end{displaymath}
de donde se deduce que:

\begin{displaymath}
J\ (v) - J\ (u) \ge 0\qquad \forall \ v \in V
\end{displaymath}
Por lo tanto $u$ es soluci\'on de $(\pi )$. Si en $(\ref{eq:caracter})$
sustituimos $b\ (\cdot ,\cdot )$ por el producto escalar en $V$, $j\ (v)$ por
$\rho \ j\ (v)$ y $L\ (v)$ por $(u,v)+\rho L\ (v)-\rho \ a\ (u,v)$ tendremos el
problema $(\pi_{\rho}^u)$, con lo que se demuestra la existencia y unicidad de
la soluci\'on de dicho problema.

\end{demoslema}

\subsection{Observaciones}

\begin{enumerate}
\item De la demostraci\'on del teorema $\ref{th:exisEVIII}$ se deduce un
algoritmo para la \mbox{resoluci\'on} del problema $(P_2)$. El algoritmo es el
siguiente:

\begin{itemize}
\item Dado $u^0 \in V$ y $0<\rho < \frac{2\alpha }{||A||^2}$
\item $(u^{n+1},v-u^{n+1})+\rho \ j\ (v)-\rho \ j\ (u^{n+1}) \ge$ \newline
$(u^n,v-u^{n+1})+\rho \ L\ (v-u^{n+1})-\rho \ a\ (u^n,v-u^{n+1})$ $\forall
\ v \in V$
\item $u^{n+1} \in V$
\end{itemize}

Se puede comprobar facilmente que $u^n \to u$ fuertemente en $V$, donde $u$
es la soluci\'on del problema $(P_2)$. Podemos encontrarnos con dificultades
si $j\ (\cdot )$ no es diferenciable. En cada iteraci\'on tenemos que resolver
un problema del mismo orden de dificultad que el problema original (el
condicionamiento del problema que estamos resolviendo depende de la buena
elecci\'on del par\'ametro $\rho$). Si $a\ (\cdot ,\cdot )$ no es sim\'etrica
el hecho de que $(\cdot ,\cdot )$ lo sea puede proporcionar alguna
simplificaci\'on.

\end{enumerate}
\newpage
\section{Aproximaci\'on EVI I}

\subsection{Discretizaci\'on de $V$ y $K$}

Sea $\{ V_h\}_h$ una familia de cerrados de $V$ con $0<h \to 0$ (en la
pr\'actica ser\'an de dimensi\'on finita).\newline

Sea $\{ K_h\}_h$ una familia de cerrados, convexos no vacios de $V$ verificando
$K_h \subset V_h \ \forall \ h$. Normalmente se tendr\'a $K_h \nsubseteq K$.\\

A\~nadiremos dos hip\'otesis sobre $K_h$:

\begin{enumerate}
\item \textbf{Hip\'otesis de consistencia}\newline
Los puntos de acumulaci\'on, para la topolog\'{\i}a d\'ebil de $V$, de las
sucesiones acotadas de $\{K_h \}$ est\'an en $K$, es decir $\{ V_h\}_h$ acotada
$V_h \in K_h$ y al estar en un espacio de Hilbert podemos extraer subsucesiones
convergentes cuyos limites est\'an en $K$.

\begin{quote}
\begin{displaymath}
\{V_h \}_h \qquad V_h \in K_h \qquad donde \ ||V_h|| \le C
\end{displaymath}

Este tipo de conjuntos no es compacto para la topolog\'{\i}a, pero si para la
topolog\'{\i}a de d\'ebil.\newline

Los puntos de acumulaci\'on para la topolog\'{\i}a d\'ebil de la sucesi\'on
$\{V_h \}$ estan en $K$ esto es necesario para demostrar que:

\begin{displaymath}
\lim_{h \to 0} ||u-u_h|| = 0
\end{displaymath}

\end{quote}

\item Existe $\chi \subset V$ tal que $\overline{\chi}=K$ y para cada $h>0$
existe $r_h : \chi \longrightarrow K_h$ tal que:

\begin{displaymath}
\lim_{h \to 0} r_h (v) = v \qquad (\equiv \lim_{h \to 0}
||r_h (v)-v||=0) \qquad \forall \ v \in \chi
\end{displaymath}

\end{enumerate}

\newpage

\subsubsection{Observaciones}

\begin{enumerate}
\item Si $K_h \subset K$ la hip\'otesis de consistencia es inmediata.
\item $\cap K_h \subset K$
\item Variante \'util de la segunda hip\'otesis:\newline

Supondremos que existe un subconjunto $\chi \subset V$ tal que
$\overline{\chi} = K$ y $r_h : \chi \longrightarrow V_h$ con la
propiedad de que para cada $v \in \chi$ existe $h_0 = h_0(v)$ con
$r_h (v) \in K_h$ $\forall \ h \le h_0(v)$ y:

\begin{displaymath}
\lim_{h \to 0} r_h (v) = v
\end{displaymath}

fuertemente en $V$.

\end{enumerate}

\subsection{Aproximaci\'on de $(P_1)$}

El problema aproximado ser\'a el siguiente:

\begin{equation}
(P^h_1) \left\{ \begin{array}{lrlr} 
a\ (u_h,v_h-u_h)&\ge & L\ (v_h-u_h) & \forall \ v_h \in K_h \\
\\
u_h \in K_h
\end{array} \right.
\end{equation}

\subsection{Teorema de existencia y unicidad}

\begin{teorema}[Existencia y unicidad EVI I aproximado]
\ \newline
$(P^h_1)$ tiene soluci\'on unica.
\end{teorema}

\begin{demosteorema}
\ \newline
En la demostraci\'on del teorema $\ref{th:exisEVII}$ coger $V_h$ en lugar de
$V$ y $K_h$ en lugar de $K$.
\begin{flushright}
$\blacksquare$
\end{flushright}

\end{demosteorema}

\newpage

\subsection{Observaciones}

\begin{enumerate}
\item En la mayor\'{\i}a de los casos sera necesario reemplazar
$a\ (\cdot, \cdot)$ y $L\ (\cdot )$ por $a_h\ (\cdot , \cdot )$ y
$L_h\ (\cdot )$ (normalmente definidos a partir de $a\ (\cdot ,\cdot)$ y
$L\ (\cdot )$ por un procedimiento de integraci\'on num\'erica).

\item En lo sucesivo utilizaremos lo siguiente:\\

Si $v_n \to v$ d\'ebil y $w_n  \to w$ fuerte entonces se tiene que:

\begin{displaymath}
a\ (v_n,w_n) \longrightarrow a\ (v,w)
\end{displaymath}

\end{enumerate}

\subsection{Convergencia}

Para demostrar la convergencia utilizaremos el siguiente lema:

\begin{lema}\label{lema:debilsci}
\ \newline
Si $a\ (\cdot ,\cdot )$ es bilineal, continua y semidefinido$-$positiva,
y $v_n \to v$ d\'ebil en $V$ entonces:

\begin{displaymath}
\liminf_{n \to \infty} a\ (v_n,v_n) \ge a\ (v,v)
\end{displaymath}
\end{lema}

\begin{demoslema}
\ \\
\begin{displaymath}
0 \le a\ (v-v_n,v-v_n)=a\ (v,v)-a\ (v,v_n)-a\ (v_n,v)+a\ (v_n,v_n)
\end{displaymath}
de esta expresi\'on se obtiene que:

\begin{displaymath}
a\ (v_n,v_n) \ge a\ (v,v_n)+a\ (v_n,v)-a\ (v,v)
\end{displaymath}
tomando limites se tiene que:

\begin{displaymath}
\liminf_{n \to \infty} a\ (v_n,v_n) \ge \liminf_{n\to \infty} \{ a\ (v,v_n)+
a\ (v_n,v) -a\ (v,v)\}
\end{displaymath}
con lo que se obtiene que:

\begin{displaymath}
\liminf_{n \to \infty} a\ (v_n,v_n) \ge a\ (v,v) + a\ (v,v)-a\ (v,v) = a\ (v,v)
\end{displaymath}

\begin{flushright}
$\blacksquare$
\end{flushright}
\end{demoslema}

\begin{teorema}[Convergencia de $(P^h_1)$]
\ \newline
Con las hip\'otesis habituales sobre $a\ (\cdot ,\cdot )$ y $L\ (\cdot )$ y
las hip\'otesis anteriores sobre $K$, $K_h$, $V$ y $V_h$ se verifica que:

\begin{displaymath}
\lim_{h \to 0} ||u_h-u|| = 0
\end{displaymath}

donde $u$ es soluci\'on de $(P_1)$ y $u_h$ es soluci\'on de $(P^h_1)$.
\end{teorema}

\begin{demosteorema}
\ \newline
\begin{itemize}
\item Estimaciones para $u_h$.\newline

Vamos a demostrar que existen constantes $C_1$ y $C_2$ independientes de $h$
tales que:

\begin{displaymath}
||u_h||^2 \le C_1 ||u_h|| + C_2 \qquad \forall \ h
\end{displaymath}

Sea $u_h$ soluci\'on de $(P^h_1)$ entonces:

\begin{displaymath}
a\ (u_h,v_h-u_h) \ge L\ (v_h-u_h) \qquad \forall \ v_h \in K_h
\end{displaymath}

es decir tenemos:

\begin{displaymath}
a\ (u_h,u_h) \le a\ (u_h,v_h)- L\ (v_h-u_h) \qquad \forall \ v_h \in K_h
\end{displaymath}

teniendo en cuenta la $V-$el\'{\i}pticidad de $a\ (\cdot ,\cdot)$ tenemos que:

\begin{equation} \label{eq:estimaciones}
\alpha ||u_h||^2 \le ||A||\ ||u_h||\ ||v_h|| + ||L||\ (||v_h||+||u_h||)\qquad
\forall \ v_h \in K_h
\end{equation}

Sea $v_0 \in \chi$ entonces se tiene que $v_h=r_h (v_0)\in K_h$ y por
las hip\'otesis que hemos a\~nadido sobre $K_h$ tenemos que
$r_h(v_0) \to v_0$ en $V$ fuerte y $||r_h(v_0)|| \le m$
acotada. Como $||v_h||=||r_h(v_0)||\le m$ podemos reescribir
$(\ref{eq:estimaciones})$ de la siguiente forma:

\begin{displaymath}
||u_h||^2 \le C_1||u_h|| + C_2
\end{displaymath}

donde

\begin{displaymath}
C_1 = \frac{1}{\alpha}\ (m ||A|| + ||L||)\qquad C_2=\frac{m}{\alpha} ||L||
\end{displaymath}
En particular se tiene que:

\begin{displaymath}
||u_h|| \le C\qquad \forall \ h
\end{displaymath}

\item Convergencia d\'ebil.\\

Como $\{ u_h \}$ esta acotada podemos extraer una subsucesi\'on convergente, la
cual tiene l\'{\i}mite en $K$\footnote{Por la hip\'otesis de consistencia.}.

\begin{displaymath}
\{ u_{h_i} \} \longrightarrow u^* \ \in K
\end{displaymath}
esta convergencia es d\'ebil en la topolog\'{\i}a de $V$.\\

Veamos que $u^*$ es soluci\'on de $P_1$.\\

Por ser $u_h$ soluci\'on de $P^1_h$ se tiene que:

\begin{displaymath}
a\ (u_{h_i}, u_{h_i}) \le a\ (u_{h_i},v_{h_i}) - L\ (v_{h_i}-u_{h_i})\qquad
\forall \ v_{h_i} \in K_{h_i}
\end{displaymath}
por la segunda hip\'otesis que hemos introducido sobre $K_h$ sabemos que existe
$v \in \chi$ tal que $r_{h_i} (v) = v_{h_i}$, entonces utilizando esta
propiedad podemos reescribir la expresi\'on anterior de la siguiente manera:

\begin{equation} \label{eq:b}
a\ (u_{h_i},u_{h_i}) \le a\ (u_{h_i},r_{h_i}\ (v))-L\ (r_{h_i} (v) - u_{h_i})
\ \forall \ v \in \chi
\end{equation}
Utilizando el lema $\ref{lema:debilsci}$ y que $r_{h_i}(v)\to v$ fuerte en $V$
y que $u_{h_i}\to u^*$ d\'ebil en $V$ en $(\ref{eq:b})$ se tiene:

\begin{equation} \label{eq:c}
\liminf_{h_i \to 0} a\ (u_{h_i},u_{h_i}) \le a\ (u^*,v)-L\ (v-u^*)\qquad \forall
\ v \in \chi
\end{equation}
Como $a\ (\cdot ,\cdot )$ es $V-$el\'{\i}ptica se tiene:

\begin{equation} \label{eq:c1}
a\ (u_{h_i}-u^*,u_{h_i}-u^*)\ge \alpha \ ||u_{h_i}-u^*||^2 \ge 0
\end{equation}
Por otro lado se tiene lo siguiente:
\begin{eqnarray} \label{eq:c2}
a\ (u_{h_i}-u^*,u_{h_i}-u^*)& =& a\ (u^*,u^*)-a\ (u_{h_i},u^*)-
\\ \nonumber
& &-a\ (u^*,u_{h_i})+a\ (u_{h_i},u_{h_i})
\end{eqnarray}
de $(\ref{eq:c1})$ y $(\ref{eq:c2})$ se obtiene:

\begin{displaymath}
a\ (u_{h_i},u^*)+a\ (u^*,u_{h_i})-a\ (u^*,u^*) \le a\ (u_{h_i},u_{u_i})
\end{displaymath}
Como $u_{h_i} \to u^*$ d\'ebil en $V$ si tomamos l\'{\i}mites en la expresi\'on
anterior:

\begin{equation} \label{eq:d}
a\ (u^*,u^*)\le \liminf_{h_i \to 0} a\ (u_{h_i},u_{h_i})
\end{equation}
de $(\ref{eq:c})$ y $(\ref{eq:d})$ obtenemos:

\begin{displaymath}
a\ (u^*,u^*) \le \liminf_{h\to 0} a\ (u_{h_i},u_{h_i})\le a\ (u^*,v)-
L\ (v-u^*)\qquad \forall \ v \in \chi
\end{displaymath}
es decir:

\begin{displaymath}
a\ (u^*,v-u^*)\ge L\ (v-u^*)\qquad \forall \ v \in \chi
\end{displaymath}
Como $\overline{\chi }=K$, $a\ (\cdot ,\cdot )$ y $L\ (\cdot )$ son continuas se
tiene que:

\begin{displaymath}
a\ (u^*,v-u^*)\ge L\ (v-u^*)\qquad \forall \ v \in K,\ u^* \in K
\end{displaymath}
Como $(P_1)$ tiene soluci\'on \'unica entonces $u^*=u$, luego $u^*$ es
soluci\'on de $(P_1)$ y $\{ u_h \}\longrightarrow u$ d\'ebil.

\item Convergencia fuerte.\\

Como $a\ (\cdot ,\cdot )$ es $V-$el\'{\i}tica se tiene:
\begin{eqnarray}\label{eq:aa}
0 &\le & \alpha \cdot ||u_h-u||^2\le a\ (u_u-u.u_h-u) =\\ \nonumber
  &= & a\ (u_h,u_h)-a\ (u_h,u)-a\ (u,u_h)+a\ (u,u)
\end{eqnarray}
donde $u_h$ es soluci\'on de $(P^h_1)$ y $u$ es soluci\'on de $(P_1)$.\\

Sabemos que existe $r_h:\chi\longrightarrow K_h$ tal que $r_h(v)=v_h$ y por
ser $u_h$ soluci\'on de $(P_1^h)$ tenemos que:

\begin{equation}\label{eq:bb}
a\ (u_h,u_h)\le a\ (u_h,r_h(v))-L\ (r_h(v)-u_h)\qquad \forall \ v \in \chi
\end{equation}
de $(\ref{eq:aa})$ y $(\ref{eq:bb})$ obtenemos:

\begin{eqnarray}\label{eq:cc}
0& \le & \alpha \cdot ||u_h-u||^2\le a\ (u_h,r_h(v))-L\ (r_h(v)-u_h) \\
\nonumber
& & -a\ (u_h,u)-a\ (u,u_h)+a\ (u,u)\qquad \forall \ v \in \chi
\end{eqnarray}
tomando l\'{\i}mites en $(\ref{eq:cc})$ tenemos:

\begin{eqnarray}\label{eq:dd}
0&\le &\alpha \cdot \liminf_{h\to 0} ||u_h-u||^2 \le \alpha \cdot
\limsup_{h\to 0} ||u_u-u||^2\le \\ \nonumber
&\le &a\ (u,v-u)-L\ (v-u)\qquad \forall \ v \in \chi
\end{eqnarray}
por densidad y continuidad $(\ref{eq:dd})$ se verifica $\forall \ v \in K$.\\

Si tomamos $v=u$ en $(\ref{eq:dd})$ tenemos:

\begin{displaymath}
\lim_{h \to 0} ||u_h-u||^2 = 0
\end{displaymath}
es decir tenemos la convergencia fuerte.

\end{itemize}

\begin{flushright}
$\blacksquare$
\end{flushright}
\end{demosteorema}
\newpage

\section{Aproximaci\'on EVI II}

\subsection{Definici\'on del problema aproximado}

Asumiremos en lo siguiente que $j:V\longrightarrow \Re$ es un funcional
continuo.

\subsection{Aproximaci\'on de V}

$h\to 0$, y sean $\{ V_h\}_h$ una familia de subespacios cerrados de $V$, en
la pr\'actica los $V_h$ ser\'an de dimensi\'on finita.\\

Supondremos que los $\{ V_h \}_h$ verifican:

\begin{itemize}
\item $\exists \ U\subset V$ tal que $\overline{U}=V$ y para cada h
$r_h:U\longrightarrow V_h$ tal que:

\begin{displaymath}
\lim_{h\to 0} r_h(v) = v
\end{displaymath}
fuerte en $V\ \forall \ v \in U$.
\end{itemize}

\subsection{Aproximaci\'on de $j\ (\cdot )$}

Aproximaremos el funcional $j\ (\cdot )$ por $\{j_h\ (\cdot ) \}_h$ donde para
cada $h$ $j_h\ (\cdot )$ satisface:

\begin{itemize}
\item $j_h:V_h \longrightarrow \overline{\Re}$
\end{itemize}
es un funcional convexo, \textbf{s.c.i} y uniformemente propio en h.\\

Que la familia $\{j_h\ (\cdot )\}_h$ sea uniformemente propia en h significa
que existe $\lambda \in V^*$ y $\mu \in \Re$ verificando lo siguiente:

\begin{displaymath}
j_h\ (v_h) \ge <\lambda,v_h>+\mu= \lambda (v_h)+\mu \qquad \forall \ v_h \in
V_h,\ \forall \ h
\end{displaymath}

Asumiremos que la familia $\{j_h\ (\cdot ) \}_h$ verifica:

\begin{itemize}
\item Si $v_h \to v$ d\'ebil en V entonces se tiene:

\begin{displaymath}
\liminf_{h\to 0} j_h\ (v_h) \ge j(v)
\end{displaymath}

\item Y se verifica:

\begin{displaymath}
\lim_{h\to 0} j_h\ (r_h(v)) = j\ (v)\qquad \forall \ v \in U
\end{displaymath}
\end{itemize}

\subsection{Observaciones}

\begin{enumerate}
\item Si $j\ (\cdot )$ es continua siempre se pueden construir $j_h\ (\cdot )$
continuas verificando las condiciones anteriores.
\item En algunos casos, en los que seremos afortunados, y se tiene que
$j_h\ (v_h) = j\ (v_h)$ $\forall \ v_h$, $\forall \ h$ entonces se verifican
trivialmente las condiciones anteriores.
\end{enumerate}

\subsection{Aproximaci\'on de $(P_2)$}

El problema aproximado ser\'a el siguiente:

\begin{equation}
(P_2^h) \left\{ \begin{array}{lrlr}
a\ (u_h,v_h-u_u)+j_h\ (v_h)-j_h (u_h)&\ge &L\ (v_h-u_h) &\forall \ v_h \in V_h \\
\\
u_h \in V_h
\end{array} \right.
\end{equation}

\subsection{Teorema de existencia y unicidad}

\begin{teorema}[Existencia y unicidad EVI II aproximado]
\ \\
$(P_2^h)$ tiene soluci\'on \'unica.
\end{teorema}

\begin{demosteorema}
\ \\
En la demostraci\'on del teorema $\ref{th:exisEVII}$ coger $V_h$ en lugar de $V$
y $j_h\ (\cdot )$ en lugar de $j\ (\cdot )$.

\begin{flushright}
$\blacksquare$
\end{flushright}
\end{demosteorema}

\newpage

\subsection{Convergencia}

\begin{teorema}
\ \\
Se verifica lo siguiente:
\begin{equation}\label{eq:convergenciaEVIIIa}
\lim_{h\to 0}||u_h-u|| = 0
\end{equation}

\begin{equation}\label{eq:convergenciaEVIIIb}
\lim_{h\to 0} j_h\ (u_h) =j\ (u)
\end{equation}
\end{teorema}

\begin{demosteorema}
\ \\
\begin{itemize}
\item Estimaciones para $u_h$.\\ \\
Vamos a demostrar que existen constantes $C_1$ y $C_2$ independientes de $h$
tales que:

\begin{displaymath}
||u_h||^2\le C_1 \cdot ||u_h|| + C_2 \qquad \forall \ h
\end{displaymath}
Sea $u_h$ soluci\'on de $(P_2^h)$ entonces $\forall \ v_n \in V_h$ se tiene:

\begin{equation}\label{eq:aaa}
a\ (u_h,u_h)+j_h\ (u_h)\le a\ (u_h,v_h)+j_h\ (v_h)-L\ (v_h-u_h)
\end{equation}
Por ser $\{j_h (\cdot )\}_h$ uniformemente propia se tiene que $\forall \ h$:

\begin{displaymath}
j_h\ (v_h) \ge \lambda (v_h)+\mu\qquad \forall \ v_h \in V_h
\end{displaymath}
donde $j_h \in V^*$, utilizando esto en $(\ref{eq:aaa})$iy teniendo en cuenta
que $a\ (\cdot ,\cdot )$ es $V-$el\'{\i}ptica tendremos:
\begin{eqnarray}\label{eq:bbb}
\alpha \cdot ||u_h||^2 &\le & ||\lambda ||\cdot ||u_h || + |\mu |+||A||\cdot
||u_h||\cdot ||v_h|| + \\ \nonumber
& & + |j_h\ (v_h)| + ||L||\cdot (||v_h||+||u_h||)
\end{eqnarray}
Existe $v_0 \in U$ tal que $r_h(v_0)=v_h$ y como se tiene que:

\begin{displaymath}
\lim_{h\to 0} j_h\ (r_h(v_0)) = j\ (v_0)
\end{displaymath}
y que $j_h\ (\cdot )$ es \textbf{s.c.i} se tiene que existe una constante
$m\ >\ 0$ tal que $||v_h|| \le m$ y $|j_h\ (v_h)|\le m$, luego podemos
reescribir $(\ref{eq:bbb})$ de la siguiente forma:

\begin{displaymath}
||u_h||^2\le C_1 \cdot ||u_h|| + C_2
\end{displaymath}
donde:

\begin{displaymath}
C_1=\frac{1}{\alpha }(||\lambda||+||A||\cdot m + ||L||)
\end{displaymath}

\begin{displaymath}
C_2 = \frac{m}{\alpha }(1+||L||)+\frac{|\mu |}{\alpha }
\end{displaymath}
Lo cual implica que $||u_h||\le C$ $\forall \ h$, es decir $\{ u_h \}_h$ esta
acotada.

\item Convergencia d\'ebil.\\ \\
Como $\{ u_h\}_h$ esta acotada entonces podemos extraer una subsucesi\'on
convergente $u_{h_i}\longrightarrow u^*$ d\'ebil en $V$.\\ \\
Sea $u_h$ soluci\'on de $(P_2^h)$, tenemos $r_{h_i}(v)=v_{h_i}\in V_h$ y
$v\in U$ entonces se tiene $\forall \ v\in U$:
\begin{eqnarray} \label{eq:ccc}
a\ (u_{h_i},u_{h_i})+j_{h_i}\ (u_{h_i}) &\le &a\ (u_{h_i},r_{h_i}(v))+
j_{h_i}\ (r_{h_i}(v))-\\ \nonumber
& &-L\ (r_{h_i}(v)-u_{h_i})
\end{eqnarray}
Como $u_{h_i}\to u^*$ d\'ebil en $V$ y $r_{h_i}(v)\to v$ fuerte en $V$ y
teniendo en cuenta las propiedades que verifica la familia
$\{ j_h\ (\cdot )\}_h$ se tiene que $\forall \ v \in U$:

\begin{equation}\label{eq:ddd}
\liminf_{h\to 0} \{a\ (u_{h_i},u_{h_i})+j_{h_i}\ (u_{h_i}) \}\le a\ (u^*,v)+
j\ (v)-L\ (v-u^*)
\end{equation}
teniendo en cuenta las propiedades de la familia $\{j_h\ (\cdot )\}_h$ se
verifica:

\begin{equation}\label{eq:eee}
a\ (u^*,u^*)+j(u^*)\le \liminf_{h\to 0} \{a\ (u_{h_i},u_{h_i})+
j_{h_i}\ (u_{h_i}) \}
\end{equation}
Utilizando $(\ref{eq:ddd})$ y $(\ref{eq:eee})$ y la densidad de $U$ se tiene
$\forall v \in V$ que:

\begin{displaymath}
a\ (u^*,v-u^*)+j\ (v)-j\ (u^*) \ge L\ (v-u^*)
\end{displaymath}
donde $u^*\in V$ es soluci\'on de $(P_2)$ entonces por la unicidad de la
soluci\'on de $(P_2)$ se tiene que $u^*=u$ entonces $u_h\to u$ d\'ebil en $V$.

\item Convergencia fuerte.\\ \\
De la $V-$el\'{\i}pticidad de $a\ (\cdot ,\cdot )$ se tiene:

\begin{displaymath}
\alpha \cdot ||u_h-u||^2+j_h\ (u_h)\le a\ (u_h-u,u_h-u)+j_h\ (u_h)
\end{displaymath}
Como $u_h$ es soluci\'on de $(P^h_2)$ y $r_h(v)=v_h\in V_h$ se tiene
$\forall \ v\in U$ que:
\begin{eqnarray*}
\alpha \cdot ||u_h-u||^2 +j_h\ (u_h)&\le &a\ (u_h,r_h(v))+j_h\ (r_h(v))-
L\ (r_h(v)-u_h)-\\
& &-a\ (u,u_h)-a\ (u_h,u)+a\ (u,u)
\end{eqnarray*}
La segunda parte de esta desigualdad cuando $h\to 0$ y $\forall \ v \in U$
tiende a:

\begin{equation} \label{eq:fff}
a\ (u,v-u)+j\ (v)-L\ (v-u)
\end{equation}
tenemos que $\forall \ v \in V$:
\begin{eqnarray*}
\liminf_{h\to 0}j_h\ (u_h)&\le & \liminf_{h\to 0}\{\alpha \cdot ||u_h-u||^2
+j_h\ (u_h)\}\le \\
&\le &\limsup_{h\to 0}\{\alpha \cdot ||u_h-u||^2+j_h\ (u_h)\}\le \\
&\le &a\ (u,v-u)+j\ (v)-L\ (v-u)
\end{eqnarray*}
Por la densidad de $U$ esta expresi\'on se verifica $\forall \ v \in V$ y
tomando $u$ en lugar de $v$ en la expresi\'on anterior y teniendo en cuenta
las propiedades de $j_h\ (\cdot )$ tenemos que:

\begin{displaymath}
j\ (u)\le \liminf_{h\to 0} j_h\ (u_h)\le \limsup_{h\to 0} \{\alpha \cdot
||u_h-u||^2+j_h\ (u_h) \}\le j\ (u)
\end{displaymath}
de donde se deduce que:

\begin{displaymath}
\lim_{h\to 0} j_h\ (u_h) = j\ (u)
\end{displaymath}
y tambi\'en se deduce que:

\begin{displaymath}
\lim_{h\to 0} ||u_h-u||= 0
\end{displaymath}

\end{itemize}

\begin{flushright}
$\blacksquare$
\end{flushright}
\end{demosteorema}
