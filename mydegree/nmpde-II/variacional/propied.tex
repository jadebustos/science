%
%
%

\chapter{Elementos de an\'alisis convexo}

Podemos encontrar un resumen de esto en el Brezis.

\begin{displaymath}
\varphi : V \longrightarrow ]-\infty ,+\infty ]
\end{displaymath}

\begin{displaymath}
dom \ \varphi =\{x\in V\ | \ \varphi (x) < \infty \}
\end{displaymath}
Se define el \textbf{ep\'{\i}grafe} de $\varphi$ como:

\begin{displaymath}
Epi\ \varphi = \{[x,\lambda ] \in V\times \Re \ |\ \varphi(x)\le \lambda \}
\end{displaymath}

\section{Propiedades de las funciones convexas y s.c.i}

\begin{enumerate}
\item $\varphi$ es \textbf{s.c.i} $\Longleftrightarrow$ Epi $\varphi$ es
cerrado en $V\times \Re$.
\item $\varphi$ es \textbf{s.c.i} $\Longrightarrow$ $\forall \ \lambda \in \Re$
el conjunto $\{ x\in V \ | \ \varphi (x)\le \lambda \}$ es cerrado.
\item $\varphi_1$ y $\varphi_2$ son \textbf{s.c.i} $\Longrightarrow$
$\varphi_1 +\varphi_2$ es \textbf{s.c.i}.
\item Si $\{\varphi_i \}_{i\in I}$ es una familia de funciones \textbf{s.c.i}
$\Longrightarrow$ la envolvente superior es \textbf{s.c.i}, es decir:

\begin{displaymath}
\varphi(x) = \sup_{i\in I} \varphi_i (x)
\end{displaymath}

es \textbf{s.c.i}.
\item $\varphi$ es convexa $\Longleftrightarrow$ Epi $\varphi$ es convexo.
\item $\varphi$ es convexa $\Longrightarrow$ $\{x\in V\ |\ \varphi (x)\le
\lambda \}$ es convexo $\forall \ \lambda \in \Re$.
\item $\varphi_1$ y $\varphi_2$ convexas $\Longrightarrow$ $\varphi_1+\varphi_2$
es convexa.
\item Si $\{\varphi_i \}_{i\in I}$ es una familia de funciones convexas
$\Longrightarrow$ la envolvente superior es convexa, es decir:

\begin{displaymath}
\varphi(x)= \sup_{i \in I} \varphi_i (x)
\end{displaymath}

es convexa.
\item Si $\varphi$ es \textbf{s.c.i} y convexa para la topolog\'{\i}a de
$||\cdot ||$ seguira siendo \textbf{s.c.i} para la topolog\'{\i}a d\'ebil.
\item $\varphi$ es \textbf{s.c.i} y convexa $\Longleftrightarrow$ existe
$\lambda \in V^*$ y $\mu \in \Re$ tales que:

\begin{displaymath}
\varphi(x) \ge <\lambda ,x> + \mu
\end{displaymath}
\end{enumerate}
Todas estas propiedades son consecuencia del teorema de Hahn-Banach.\\
