%
% LENGUJES Y SIMBOLOS
%

\section{Lenguajes y simbolos}

Para emitir un mensaje hay que hacerlo utilizando un determinado lenguaje,
com\'un al transmisor y al receptor. Todo lenguaje estar\'a formado por un
conjunto de simbolos, los cuales formar\'an las palabras del lenguaje.\\

Los simbolos tambi\'en recibiran el nombre de ``bits''.

\begin{ejemplo}
\ \\ \\
Cuando queremos comunicarnos con una persona, transmitir un mensaje, hablamos
con esa persona en un idioma conocido por ambas partes, lenguaje, y los
simbolos que utilizamos son las letras del abecedario, n\'umeros, \dots y el
medio o canal utilizado puede ser el habla o la escritura, por ejemplo.
\end{ejemplo} 

\begin{ejemplo}[I.S.B.N.]
\ \\ \\
Sistema internacional de libros. \\

Este sistema se utiliza para catalogar y numerar los libros. Est\'a formado por
palabras de $10$ cifras y los simbolos que utiliza este lenguaje son:

\begin{displaymath}
Smb_{I.S.B.N.} = \{0, 1, 2, 3, 4, 5, 6, 7, 8, 9 \}
\end{displaymath}

Una palabra de este lenguaje ser\'a:

\begin{displaymath}
a=(a_1,a_2,a_3,a_4,a_5,a_6,a_7,a_8,a_9,a_{10})\quad a_i \in Smb_{I.S.B.N.}
\quad \forall \quad i=1,\ldots,10
\end{displaymath}
\end{ejemplo}
%
\newpage
%
\section{Objetivos de la teor\'{\i}a de c\'odigos}

Los objetivos de la teor\'{\i}a de c\'odigos son los siguientes:
\begin{itemize}
\item Construir c\'odigos para la transmisi\'on de informaci\'on.
\item Dichos c\'odigos han de detectar cuando ha ocurrido un error en la
transmisi\'on de la informaci\'on.
\item Dichos c\'odigos deben corregir el mayor n\'umero posible de errores.
\end{itemize}

Con el fin de detectar y corregir los posibles errores ocurridos durante la
transmisi\'on del mensaje original introduciremos informaci\'on redundante
acerca del mensaje original, con el fin de cotejar esta informaci\'on con el 
mensaje original y poder comprobar de esta forma si se produjo alg\'un error en
la transmisi\'on del mensaje.\\

No podemos abusar de la informaci\'on redundante que introducimos. Al
introducir informaci\'on redundante en el mensaje a transmitir obtenemos un
mensaje m\'as largo de transmitir, por lo tanto ser\'a m\'as costoso y lento
el poder transmitirlo. Por lo tanto si introducimos demasiada informaci\'on
redundante tenemos la ventaja de poder detectar y corregir bastantes errores,
pero tambi\'en tenemos el inconveniente de un mayor costo al transmitir la
informaci\'on.\\

Otro objetivo que buscaremos ser\'a que cuando se origine un error en la
transmisi\'on de una palabra se produzca una palabra NO perteneciente al
c\'odigo. Por ejemplo si transmitimos una palabra ``A'', ocurre un error y,
fruto de ese error la palabra que llega al receptor es ``B'', la situaci\'on
ideal ser\'{\i}a que ``B'' NO perteneciera a nuestro c\'odigo. La explicaci\'on
de esto es que si nos llega una palabra que NO es del c\'odigo que estamos
utilizando entonces se ha producido un error, mientras que s\'{\i} la palabra
es del c\'odigo la daremos como buena y no nos percataremos de que se ha 
cometido un error en la transmisi\'on.
%
\newpage
%
\subsection{Ejemplos de la utilizaci\'on de c\'odigos}

Los c\'odigos se usan de una manera continua en muchos campos, como por ejemplo:

\begin{itemize}
\item En las comunicaciones:
\begin{itemize}
\item Radio.
\item Televisi\'on.
\item Sat\'elites.
\item Ordenadores.
\end{itemize}
\item Sistemas de grabaci\'on de datos:
\begin{itemize}
\item Voz.
\item Video.
\item CD-Rom.
\end{itemize}
\item Comunicaci\'on escrita.
\item Comunicaci\'on oral(sonora).
\end{itemize}

\section{Notaciones}

Trabajaremos siempre, salvo que no se diga lo contrario, sobre el conjunto
$\mathbb{F}_q$. Este conjunto es un conjunto finito de $q$ elementos y
supondremos siempre que $q=p^n$, donde $p$ es un n\'umero primo y $n\in
\mathbb{N}$.\\

Nuestros c\'odigos estar\'an formados por palabras de $n$ ``bits'' o simbolos,
donde cada palabra pertenece a $\mathbb{F}_q$.\\

El conjunto de todas las palabras de $n$ simbolos donde cada simbolo es un
elemento de $\mathbb{F}_q$ lo denotaremos como $\mathbb{F}^{^n}_q$.\\

Luego tendremos que:
\begin{displaymath}
\mathbb{F}^{^n}_q = \{(a_1,a_2,\dots,a_n)\ |\ a_i\in \mathbb{F}_q\quad \forall \
i=1,\dots,n\}
\end{displaymath}

Como $\mathbb{F}_q$ tiene un n\'umero finito de elementos, $q$, dado un n\'umero
entero $n<\infty$ podemos calcular la cantidad de elementos que tiene
$\mathbb{F}^{^n}_q$ al cual denotaremos como $|\mathbb{F}^{^n}_q|$.\\

El n\'umero de elementos de $\mathbb{F}^{^n}_q$, al que llamaremos orden de
$\mathbb{F}^{^n}_q$, ser\'a el n\'umero de combinaciones, distintas, que podemos
hacer de $n$ elementos de $\mathbb{F}^{^n}_q$, es decir:
\begin{displaymath}
|\mathbb{F}^{^n}_q| = q^n\ donde\ n\in \mathbb{N}
\end{displaymath}
%
Los elementos de $\mathbb{F}^{^n}_q$ los denotaremos de dos formas:
\begin{itemize}
\item $(a_1,\dots,a_n)$ fundamentalmente cuando consideremos dicho elemento como
elemento de una estructura algebraica.
\item $a_1\dots a_n$ fundamentalmente cuando consideremos dicho elemento como
elemento de un c\'odigo.
\end{itemize}
pero ambas notaciones sirven para hacer referencia al mismo elemento.

\section{Resumen}

Para transmitir informaci\'on lo que haremos ser\'a:
\begin{itemize}
\item Dividir la informaci\'on en bloques o palabras.
\item Codificar cada palabra del mensaje utilizando un c\'odigo. Este paso
supone a\~nadir a cada palabra informaci\'on redundante para poder controlar
cuando se ha producido un error en la transmisi\'on y, en algunos casos,
corregir dicho error.
\item Enviar el mensaje.
\end{itemize}
%
Una vez recibido el mensaje para poder conocer el mensaje original:
\begin{itemize}
\item Comprobar cada palabra del mensaje recibido, utilizando la informaci\'on
redundante\footnote{``Bits'' de control.}, para comprobar que no hubo ning\'un
error en la transmisi\'on del mensaje. En caso de haberlo se tienen dos
opciones:
\begin{itemize}
\item Intentar corregir el error, en el caso que el c\'odigo lo permita.
Corregir el error debe enterderse como calcular la palabra del c\'odigo que,
con mayor probabilidad, fue transmitida originalmente.
\item Solicitar, de nuevo, la informaci\'on erronea.
\end{itemize}
\item Eliminar la informaci\'on redundante introducida\footnote{Decodificar el
mensaje.}.
\end{itemize}

\subsection{Alfabeto}

\begin{definicion}[Alfabeto]
\ \\
Un \textbf{``alfabeto''} ser\'a el conjunto de si\'{\i}mbolos utilizaremos
para codificar el mensaje.
\end{definicion}
En todo alfabeto existe un s\'{\i}mbolo distinguido, que es el ``\emph{espacio
en blanco}''. Este s\'{\i}mbolo nos permite distiguir cuando termina una palabra
y comienza la siguiente.\\

Por ejemplo si utilizamos como alfabeto $\mathbb{F}_2$ los s\'{\i}mbolos 
serian $\{0,1\}$.

\subsection{Palabras}

\begin{definicion}[Palabra]
\ \\
Entenderemos por \textbf{``palabra''} a una sucesi\'on ordenada de simbolos de
un alfabeto.
\end{definicion}
En los casos que vamos a considerar las palabras ser\'an elementos de
$\mathbb{F}^{^n}_q$, puesto que vamos a utilizar palabras de longitud fija.\\ \\
%
El n\'umero de elementos que posee $\mathbb{F}^{^n}_q$ es $|\mathbb{F}_q|^{^n}$.


\subsection{C\'odigos}

\begin{definicion}[C\'odigos]
\ \\
Llamaremos \textbf{``c\'odigo''} al conjunto de palabras que utilizaremos para
codificar informaci\'on.
\end{definicion}
En los casos que vamos a considerar los c\'odigos ser\'an subconjuntos de
$\mathbb{F}^{^n}_q$, es decir $\mathcal{C}\subset \mathbb{F}^{^n}_q$.
