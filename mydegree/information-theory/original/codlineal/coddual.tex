%
% CODIGOS DUALES
%

%
\newpage
%
\section{C\'odigos duales}

\begin{definicion}[Producto escalar en $\mathbb{F}^{^n}_q$]
\ \\
En $\mathbb{F}^{^n}_q$ podemos definir el siguiente producto escalar:
\begin{displaymath}
\begin{array}{cccl}
<\ ,\ >:&\mathbb{F}^{^n}_q\times\mathbb{F}^{^n}_q&\longrightarrow&\mathbb{Z}^+\\
&(x,y)&\longrightarrow& x_1\cdot y_1+\dots+x_n\cdot y_n
\end{array}
\end{displaymath}
\end{definicion}
Se comprueba de forma inmediata que $<\ ,\ >$ verifica las condiciones para ser
producto escalar:
\begin{itemize}
\item $<x,y>\geq 0$ $\forall \ x,y\in \mathbb{F}^{^n}_q$.
\item $<x,x> > 0$ $\forall \ x\in \mathbb{F}^{^n}_q$ y $u\neq 0$.
\end{itemize}

Una vez definido un producto escalar podemos definir una relaci\'on de
ortoganilidad.
\begin{definicion}[Ortogonalidad entre vectores]
\ \\
$x$ e $y$ son ortogonales s\'{\i} y s\'olo s\'{\i} $<x,y>=0$.
\end{definicion}
Mediante esta definici\'on de ortogonalidad podemos calcular el ortogonal a
un subespacio dado.\\

Sea $\mathcal{C}[n,m]$ un c\'odigo lineal. Entonces es un subespacio vectorial
de $\mathbb{F}^{^n}_q$ tal que $\dim_{\mathbb{F}_q} \mathcal{C}[n,m]=m$.
\begin{definicion}[C\'odigo dual]
\ \\
Se define el \textbf{``c\'odigo dual''} de $\mathcal{C}[n,m]$ como el subespacio
vectorial ortogonal a $\mathcal{C}[n,m]$ y lo denotaremos como
$\mathcal{C}[n,m]^{\perp}$.
\begin{displaymath}
\mathcal{C}[n,m]^{\perp} = \{\ x\in \mathbb{F}^{^n}_q\ |\
<x,\mathcal{C}[n,m]> = 0\ \}
\end{displaymath}
\end{definicion}
\begin{observacion}
\ \\
\begin{itemize}
\item $\mathcal{C}[n,m]^{\perp}=\mathcal{C}'[n,n-m]$ ya que:
\begin{displaymath}
\dim_{\mathbb{F}_q} \mathcal{C}[n,m]^{\perp} =\dim_{\mathbb{F}_q}
\mathbb{F}^{^n}_q - \dim_{\mathbb{F}_q} \mathcal{C}[n,m] = n-m
\end{displaymath}
\item Sea $G$ la matriz generadora del c\'odigo $\mathcal{C}[n,m]$ y $H$ la
matriz generadora del c\'odigo $\mathcal{C}[n,m]^{\perp}$ entonces se tiene que
$H^t$ es una matriz de control para el c\'odigo $\mathcal{C}[n,m]$. O lo que es
lo mismo:
\begin{displaymath}
\left( \mathcal{C}[n,m]^{\perp} \right)^{\perp} = \mathcal{C}[n,m]
\end{displaymath}
\end{itemize}
\end{observacion}
De las observaciones se deduce que el c\'odigo dual determina al c\'odigo:
\begin{displaymath}
\mathcal{C}[n,m] = \{\ x\in \mathbb{F}^{^n}_q\ |\ H^t \cdot x^t=0\ \}
\end{displaymath}
donde $H^t$ es la matriz traspuesta de la matriz generadora del c\'odigo
$\mathcal{C}[n,m]^{\perp}$.
