%
% CORRECCION DE ERRORES
%

\section{Correcci\'on de errores}

Ya hemos visto como detectar cuando ha ocurrido un error en una transmisi\'on,
pero eso no es suficiente. Tambi\'en hay que saber cuando y como se puede
corregir un error. Los procesadores de error\footnote{Apartado
$(\ref{sec:ProcErr})$ en la p\'agina $\pageref{sec:ProcErr}$.} ser\'an los
encargados de detectar y corregir los errores.\\

Hemos visto que calculando la distancia m\'{\i}nima de un c\'odigo podemos
saber hasta que peso se pueden detectar errores.\\

Pero una vez detectado que se ha cometido un error, ?`como elegir la palabra
adecuada que se transmiti\'o originalmente?. Esto tambi\'en lo haremos mediante
la distancia m\'{\i}nima.
%
\newpage
%
Supongamos que recibimos una palabra $w$ y deseamos ser capaces de corregir
todos los errores de peso $1$. Una vez recibida la palabra $w$ y sabiendo que
se ha cometido un error de peso $1$ deber\'{\i}a haber una \'unica palabra
del c\'odigo $u$ tal que $d(u,w)=1$. Pero se puede dar el caso en el que existan
$u,v\in \mathcal{C}$ tales que $d(u,w)=d(v,w)=1$, entoces en esta situaci\'on,
?`que palabra elegiriamos como correcta?.\\

El mismo razonamiento se puede seguir para errores de peso $k$.
\begin{definicion}[Correcci\'on de errores de peso $k$]
\ \\
Dada una palabra $w$ en la que sabemos que existe un error de peso $k$, a lo
sumo, diremos que un c\'odigo $\mathcal{C}$ corrige errores de peso menor o
igual que $k$ $\Longleftrightarrow$ existe una y s\'olo una palabra $u\in
\mathcal{C}$ tal que $d(u,w)\leq k$ para cualquier $w$ en el que se cometa un
error de peso $k$, a lo sumo.
\end{definicion}

Para solucionar este problema elegimos c\'odigos en los que las palabras esten
bastante separadas.

\begin{teorema}[Teorema de correcci\'on de errores]\label{the:Correccion}
\ \\
Dado un c\'odigo $\mathcal{C}$ se pueden corregir, sin problemas, todos los
errores de peso menor o igual que $t$ $\Longleftrightarrow$ su distancia
m\'{\i}nima, $d_{min}$, es mayor o igual que $2\cdot t +1$ .
\end{teorema}
\underline{\textbf{Demostraci\'on}}:

$\Rightarrow |$ Supongamos que tenemos un c\'odigo que es capaz de corregir
errores de peso menor o igual que $t$. Entonces si recibimos una palabra $w$
en la que sabemos que existe un error de peso $t$, a lo sumo, entonces 
existir\'a una y s\'olo una palabra $u\in \mathcal{C}$ tal que $d(w,u)\leq t$.\\

Lo demostraremos por reducci\'on al absurdo. Supondremos que existen dos
palabras en el c\'odigo, $u_1,u_2 \in \mathcal{C}$, tales que
$d(u_1,u_2)\leq 2\cdot t$.\\ \\
%
Sea $w$ una palabra tal que:
\begin{itemize}
\item Coincide con $u_1$ en los bits en los que coinciden $u_1$ y $u_2$.
\item Coincide con $u_1$ en los $t$ primeros bits en los que $u_1$ y $u_2$ no
coinciden.
\item Coincide con $u_2$ en los bits siguientes, a los $t$ primeros, en los
que $u_1$ y $u_2$ no coinciden.
\end{itemize}
De esta construcci\'on es claro que $d(w,u_2) = t$.\\ \\
%
Supongamos que $u_1$ y $u_2$ son palabras de $n$ bits que coinciden en $m$
bits. Entonces tendremos que $d(w,u_1) = n-m-([n-m]-t) = t$.\\

Suponemos que recibimos la palabra $w$ y sabemos que se cometi\'o un error
de peso $t$, a lo sumo, en la transmisi\'on. Entonces la palabra que se
transmiti\'o originalmente es la \'unica palabra del c\'odigo, $u$, tal que
$d(w,u)\leq t$ ya que por hip\'otesis el c\'odigo corrige errores de peso
menor o igual que $t$.\\

Ahora bien, tenemos en el c\'odigo dos palabras, $u_1,u_2$, tales que verifican
que $d(w,u_i)\leq t$ para $i=1,2$. Pero esto no puede ocurrir por hip\'otesis
ya que unicamente puede haber una palabra en el c\'odigo que cumpla la
condici\'on. Hemos llegado a una contradicci\'on al suponer que
$d(u_1,u_2)\leq 2\cdot t$ para $u_1,u_2\in \mathcal{C}$. Entonces se tendr\'a
que $d(u_1,u_2) > 2\cdot t$ para $u_1,u_2\in \mathcal{C}$, es decir,
$d(u_1,u_2)\geq 2\cdot t+1$ para $u_1,u_2\in \mathcal{C}$ $\Longrightarrow$
$d_{min} \geq 2\cdot t +1$ .\\

$\Leftarrow |$ Supongamos que tenemos un c\'odigo cuya distancia m\'{\i}nima,
$d_{min}$, es mayor o igual que $2\cdot t +1$.\\

Supongamos que se ha realizado una transmisi\'on y se ha recibido la palabra
$w$, conociendo que se ha cometido un error de, a lo sumo, peso $t$. Obviamente
$w\notin \mathcal{C}$, ya que en caso contrario el error no ser\'{\i}a
detectable.\\

Supongamos que existen dos palabras en el c\'odigo $u,v\in \mathcal{C}$ tal que
$d(u,w)\leq t$ y $d(v,w)\leq t$. Entonces utilizando la desigualdad triangular
se tiene que:
\begin{displaymath}
d(u,v)\leq d(u,w)+d(w,v) = d(u,w)+d(v,w) = t+t= 2\cdot t
\end{displaymath}
Es decir $d(u,v)\leq 2\cdot t$, lo cual es falso ya que como
$u,v\in \mathcal{C}$ se tiene, por hip\'otesis, que $d(u,v)\geq 2\cdot t + 1$.\\

De todo esto se deduce que s\'olo hay una y s\'ola una palabra en el c\'odigo,
$u$, tal que $d(u,w)\leq t$, con lo cual $u$ es la palabra que se transmiti\'o
originalmente. Es decir podemos corregir errores de peso menor o igual que $t$.\\

En todo esto hemos supuesto que siempre existe una palabra en el c\'odigo, $u$,
tal que $d(u,w)\leq t$. Si esta palabra no existiera no tendr\'{\i}a sentido
este teorema, ya que es en dicha palabra en la que se produce el error, luego
siempre existe una palabra, por lo menos, en el c\'odigo verificando la
condici\'on $d(u,w)\leq t$.
\begin{flushright}
$\blacksquare$
\end{flushright}
Podemos interpretar el resultado de este teorema como:
\begin{quote}
Corregir adecuadamente cuesta el doble que detectar.
\end{quote}
\begin{definicion}[C\'odigo $t$-detector de errores]
\ \\
Diremos que un c\'odigo es un \textbf{``c\'odigo $t$-detector de errores''} si
verifica que $d_{min}=2\cdot t +1$.
\end{definicion}
Podemos mezclar los teoremas $\ref{the:Deteccion}$ y $\ref{the:Correccion}$ en
un s\'olo teorema:
\begin{teorema}[Teorema de detecci\'on y correcci\'on de errores]\label{the:DeteccionCorreccion}
\ \\
Un c\'odigo corrige errores de peso menor o igual que $t$ y detecta errores de
peso menor o igual que $s'=s+t$ $\Longleftrightarrow$
$d_{min}\geq 2\cdot t+s+1=t+s'+1$.
\end{teorema}
