%
% ESTRUCTURA DEL ANILLO Z/q[x]/(x^n-1)
%

\section{El anillo $\mathbb{F}_q[x]/(x^n-1)$}

$\mathbb{F}_q[x]/(x^n-1)$ es un anillo conmutativo con unidad y adem\'as es
un $\mathbb{F}_q$-espacio vectorial\footnote{Ejercicio $\ref{ejer:Anillo}$ en la
p\'agina $\pageref{ejer:Anillo}$.}.
\begin{displaymath}
\mathbb{F}_q[x]/(x^n-1)=<\overline{1},\overline{x},\dots,\overline{x}^{n-1}>
\end{displaymath}
como $\mathbb{F}_q$-espacio vectorial.\\

Sean $P(x),Q(x)\in \mathbb{F}_q[x]$ entonces son equivalentes m\'odulo $x^n-1$
cuando su diferencia es un m\'ultiplo de $x^n-1$.
\begin{displaymath}
P(x)\equiv Q(x)\ mod\ x^n-1 \Longleftrightarrow P(x)-Q(x)=
\stackrel{\cdot }{(x^n-1)}
\end{displaymath}

Los c\'odigos que estamos utilizando son siempre subespacios vectoriales del
$\mathbb{F}_q$-espacio vectorial $\mathbb{F}^{^n}_q$. Podemos establecer un
isomorfismo entre $\mathbb{F}^{^n}_q$ y $\mathbb{F}_q[x]/(x^n-1)$:
\begin{displaymath}
\begin{array}{rcl}
\mathbb{F}_q[x]/(P(x))&\stackrel{\sim }\longrightarrow&\mathbb{F}^{^n}_q\\
a_0+a_1\cdot \overline{x}+\dots+a_{n-1}\cdot\overline{x}^{n-1}&\longrightarrow&
(a_0,a_1,\dots,a_{n-1})
\end{array}
\end{displaymath}
\begin{proposicion}\label{pro:EquivPCicl}
\ \\
Multiplicar por $\overline{x}$ en $\mathbb{F}_q[x]/(x^n-1)$ equivale a aplicar
una permutaci\'on c\'{\i}clica de orden uno en $\mathbb{F}^{^n}_q$.
\end{proposicion}
\underline{\textbf{Demostraci\'on}}:\\
Sea $\overline{p(x)}=a_0+a_1\cdot \overline{x}+\dots+a_{n-1}\cdot
\overline{x}^{n-1}$ entonces por el isomorfismo anterior se corresponder\'a, de
forma \'unica, con $(a_0,a_1,\dots,a_{n-1})\in \mathbb{F}^{^n}_q$.
Multipliquemos $\overline{p(x)}$ por $\overline{x}$:
\begin{displaymath}
\overline{x}\cdot \overline{p(x)} = a_0\cdot \overline{x}+a_1\cdot
\overline{x}^2+\dots+a_{n-2}\cdot \overline{x}^{n-1}+a_{n-1}\cdot
\overline{x}^n
\end{displaymath}
pero como $x^n-1=0$ en $\mathbb{F}_q[x]/(x^n-1)$ tenemos que $x^n=1$ en
$\mathbb{F}_q[x]/(x^n-1)$, luego:
\begin{displaymath}
\overline{x}\cdot \overline{p(x)} = a_{n-1}+a_0\cdot \overline{x}+a_1\cdot
\overline{x}^2+\dots+a_{n-2}\cdot\overline{x}^{n-1}
\end{displaymath}
el cual se corresponde, por el isomorfismo anterior, con:
\begin{displaymath}
(a_{n-1},a_0,a_1,\dots,a_{n-2})\in\mathbb{F}^{^n}_q
\end{displaymath}
elemento que se obtiene al aplicar una permutaci\'on c\'{\i}clica a
$(a_0,a_1,\dots,a_{n-1})$, elemento que representa a
$\overline{p(x)}$ en $\mathbb{F}^{^n}_q$.
\begin{flushright}
$\blacksquare$
\end{flushright}
%
\begin{corolario}
\ \\
Multiplicar por $\overline{x}^i$ con $i=0,\dots,n-1$ en
$\mathbb{F}_q[x]/(x^n-1)$ equivale a aplicar una permutaci\'on c\'{\i}clica de
orden $i$ en $\mathbb{F}^{^n}_q$.
\end{corolario}
\underline{\textbf{Demostraci\'on}}:\\
Inmediata a partir de la proposici\'on $\ref{pro:EquivPCicl}$.
\begin{flushright}
$\blacksquare$
\end{flushright}
%
\begin{teorema}\label{the:Multiplicativo}
\ \\
Sea $\mathcal{C}[n,m]\subset \mathbb{F}_q[x]/(x^n-1)$ un c\'odigo.
$\mathcal{C}[n,m]$ es c\'{\i}clico s\'{\i} y s\'olo s\'{\i} para todo
$\overline{p(x)}\in \mathcal{C}[n,m]$ se verifica que
$\overline{x}\cdot\overline{p(x)}\in \mathcal{C}[n,m]$.
\end{teorema}
\underline{\textbf{Demostraci\'on}}:\\
Inmediata a partir de las definiciones y la proposici\'on
$\ref{pro:EquivPCicl}$.
\begin{flushright}
$\blacksquare$
\end{flushright}
%
\begin{corolario}
\ \\
S\'{\i} $\overline{p(x)}\in \mathbb{F}_q[x]/(x^n-1)$
entonces para $i\in \mathbb{Z}$ se tiene que:
\begin{displaymath}
\overline{x}^i\cdot \overline{p(x)}\in \mathbb{F}_q[x]/(x^n-1)
\end{displaymath}
\end{corolario}
\underline{\textbf{Demostraci\'on}}:\\
Es inmediato a partir del teorema $\ref{the:Multiplicativo}$ y
$\overline{x}^i=\overline{x}^{(i-1)}\cdot \overline{x}$.
\begin{flushright}
$\blacksquare$
\end{flushright}
