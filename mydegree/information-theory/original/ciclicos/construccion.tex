%
% CONSTRUCCION DE CODIGOS CICLICOS
% 

\section{Construcci\'on de c\'odigos c\'{\i}clicos}

Sea $\mathcal{C}[n,m]$ un c\'odigo lineal c\'{\i}clico. Tendremos entonces que:
\begin{displaymath}
\mathcal{C}[n,m]\subset \mathbb{F}^{^n}_q\quad y\quad
\mathcal{C}[n,m]\simeq \mathbb{F}^{^m}_q
\end{displaymath}
Adem\'as como el c\'odigo es c\'{\i}clico se corresponde con un ideal del 
anillo $\mathbb{F}_q[x]/(x^n-1)$:
\begin{displaymath}
\mathcal{C}[n,m] = (g(x))\quad con\ g(x)=a_0+a_1\cdot \overline{x}+\dots+
a_{m-1}\cdot \overline{x}^{m-1}
\end{displaymath}

\subsection{Matriz generadora de un c\'odigo c\'{\i}clico}

La matriz generadora del c\'odigo c\'{\i}clico $\mathcal{C}[n,m]$ ser\'a la
matriz de la aplicaci\'on lineal:
\begin{displaymath}
\mathbb{F}^{^m}_q\stackrel{\sim}\longrightarrow \mathcal{C}[n,m]
\end{displaymath}
Observar que $\mathbb{F}_q[x]/(x^n-1)$ es un $\mathbb{F}_q$-espacio vectorial,
adem\'as de anillo, por lo cual cualquier ideal suyo es tambi\'en subespacio
vectorial.\\

Por el teorema $\ref{the:CarCiclicos}$ sabemos que los c\'odigos c\'{\i}clicos
son los ideales del anillo $\mathbb{F}_q[x]/(x^n-1)$, o lo que es lo mismo su
polinonio generador divide a $x^n-1$, sobre $\mathbb{F}_q$.\\

Sea $x^n-1=p_1(x)\cdot p_r(x)$ la descomposici\'on en irreducibles sobre
$\mathbb{F}_q[x]$, y sea $g(x)=p_1(x)\cdot p_i(x)$ con $i<r$, entonces
tendremos que $x^n-1= g(x)\cdot h(x)$ con:
\begin{eqnarray*}
g(x)&=&g_0+g_1\cdot x+\dots+g_{n-k}\cdot x^{n-m}\\
h(x)&=&h_0+h_1\cdot x+\dots+h_k\cdot x^m
\end{eqnarray*}
Sea $\mathcal{C}=(g(x))$ y llamemos $A=\mathbb{F}_q[x]/(x^n-1)$, entonces:
\begin{displaymath}
A/\mathcal{C} = \mathbb{F}_q[x]/(g(x))
\end{displaymath}
$\mathbb{F}_q[x]/(g(x))$ es de dimensi\'on $n-m$, luego el c\'odigo ser\'a
de dimensi\'on $m$. Para encontrar una base nos bastar\'{\i}a con encontrar
$m$ vectores linealmente independientes entonces formar\'{\i}an base al estar
en un subespacio de dimensi\'on $m$.
\begin{displaymath}
\mathcal{C} = <g(x),x\cdot g(x),\dots.x^{m-1}\cdot g(x)>
\end{displaymath}
Que son base es inmediato, ya que al ser $\mathcal{C}$ un c\'odigo c\'{\i}clico
y $g(x)\in \mathcal{C}$ entonces $x^i\cdot g(x)\in \mathcal{C}$. Dichos
polinomios expresados en la base can\'onica de $A$ ser\'an:
\begin{eqnarray*}
g(x)&=& (g_0,\dots,g_{n-m},0,\stackrel{m-1)}\dots,0)\\
x\cdot g(x)&=&(0,g_0,\dots,g_{n-m},0,\stackrel{m-2)}\dots,0)\\
\dots& &\dots \dots \dots\\
x^{m-1}\cdot g(x)&=&(0,\stackrel{m-1)}\dots,0,g_0,\dots,g_{n-m})
\end{eqnarray*}
que son $m$ vectores linealmente independientes, luego generan. La matriz
generadora del c\'odigo ser\'a:
\begin{displaymath}
\left( \begin{array}{cccc}
g_0&0&\dots & 0 \\
g_1&g_0&\dots& 0 \\
g_2&g_1&\dots& \vdots \\
\vdots &\vdots &\vdots & g_0 \\
g_{n-m}&\vdots & \vdots &\vdots \\
0&g_{n-m}& \vdots &\vdots\\
\vdots &\vdots &\vdots & \vdots\\
0&0& \vdots & g_{n-m} 
\end{array}\right)
\end{displaymath}
Luego tenemos un c\'odigo de dimensi\'on $m$ y con palabras de longitud $n$,
luego nuestro c\'odigo ser\'a del tipo $\mathcal{C}[n,m]$.

\subsection{C\'odigo dual de un c\'odigo c\'{\i}clico}

Sea $x^n-1=g(x)\cdot h(x)$ y $\mathcal{C}[n,m]=(g(x))$ entonces definimos el
c\'odigo dual de $\mathcal{C}[n,m]$, y lo denotamos como
$\mathcal{C}^{\perp}[n,m]$, al c\'odigo c\'{\i}clico generado por el
polinomio $x^k\cdot h(x^{-1})$.\\

La matriz generadora de $\mathcal{C}^{\perp}[n,m]$ es la matriz de control de
$\mathcal{C}[n,m]$.
