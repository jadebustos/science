%
% ELECCION DE BUENOS CODIGOS 
%

\section{Elecci\'on de buenos c\'odigos}

Ya hemos comentado anteriormente que al transmitir una palabra se produzca un
error, y este error genere otra palabra del c\'odigo, con lo cual el error
cometido ser\'{\i}a indetectable. Por ejemplo en el ``\emph{c\'odigo de triple
repetici\'on}'', el cual est\'a formado unicamente por dos palabras
$\mathcal{C}[3,1]=\{000,111\}$ supongamos que transmitimos la palabra $000$ y se
cometen errores:
\begin{itemize}
\item \textbf{de peso $1$}, eso significar\'{\i}a que la palabra recibida
ser\'{\i}a una de las siguientes:
\begin{itemize}
\item $100$.
\item $010$.
\item $001$.
\end{itemize}
donde el error es detectable, ya que ninguna de ellas pertenece al c\'odigo.
\item \textbf{de peso $2$}, eso significar\'{\i}a que la palabra recibida
ser\'{\i}a una de las siguientes:
\begin{itemize}
\item $110$.
\item $101$.
\item $011$.
\end{itemize}
donde el error sigue siendo detectable, ya que ninguna de ellas pertenece al
c\'odigo. 
\item \textbf{de peso $3$}, eso significar\'{\i}a que la palabra recibida
ser\'a la siguiente:
\begin{itemize}
\item $111$.
\end{itemize}
donde el error es indetectable, ya que la palabra recibida pertenece al
c\'odigo.
\end{itemize}
Luego con este c\'odigo podemos detectar errores de peso $2$ ya que sus palabras
tienen una distancia entre s\'{\i} de $3$.\\

De esto se deduce que para poder detectar errores de peso $s+1$ hemos de 
utilizar c\'odigos cuyas palabras disten entre s\'{\i} $s$.

\subsection{Distancia m\'{\i}nima}

Supongamos que tenemos un c\'odigo $\mathcal{C}[n,m]\subset \mathbb{F}^{^n}_q$,
la distancia m\'{\i}nima de dicho c\'odigo es un dato importante. Esto es 
debido a que nos indica que tipos de errores se detectan en el c\'odigo.
\begin{definicion}[Distancia M\'{\i}nima]
\ \\
Llamaremos \textbf{``distancia m\'{\i}nima''} de un c\'odigo a la m\'{\i}nima
distancia entre todos los posibles pares de palabras que podamos formar con
las palabras del c\'odigo. Es decir:
\begin{displaymath}
d_{min} (\mathcal{C}[n,m]) = \mathop{\min_{a,b\in \mathcal{C}[n,m]}}_{a\neq b}
\{ \ d(a,b)\ \}
\end{displaymath}
\end{definicion}
Observar que todos los c\'odigos que utilizaremos van a ser subconjuntos de
cuerpos finitos y entonces tiene sentido hablar de ``\emph{distancia
m\'{\i}nima}''.\\ \\
%
A esta distancia para abreviar nos referiremos como $d_{min}$.\\ \\
%
Muchas veces un c\'odigo $\mathcal{C}[n,m]$ se suele denotar como
$\mathcal{C}[n,m,d_{min}]$.\\

Dado un c\'odigo $\mathcal{C}[n,m,d_{min}]$ tendremos que la distancia
m\'{\i}nima, entre dos palabras diferentes, de dicho c\'odigo es de $d_{min}$.
Luego difieren en, al menos, $d_{min}$ bits. En un c\'odigo de este tipo son
detectables errores de peso menor, estrictamente, que $d_{min}$, mientras que
los errores de peso mayor o igual que $d_{min}$ son indetectables%
\footnote{Generalmente, ya
que puede darse el caso en el que un error de peso mayor que la distancia
m\'{\i}nima sea detectable.}.
%
\newpage
%
\begin{teorema}[Teorema de detecci\'on de errores]\label{the:Deteccion}
\ \\
Un c\'odigo detecta errores de peso menor o igual que $s$ $\Longleftrightarrow$
$d_{min}$ es mayor, estrictamente, que $s$.
\end{teorema}
\underline{\textbf{Demostraci\'on}}:\\

$\Rightarrow |$ Supongamos que tenemos un c\'odigo $\mathcal{C}$ que detecta
errores de peso menor o igual que $s$.\\

Seg\'un lo visto en el apartado $(\ref{sec:DistanErr})$, en la p\'agina
$\pageref{sec:DistanErr}$, detectar un error de peso menor o igual que $s$
significa que, si dada una palabra del c\'odigo cualquiera alteramos, a
lo sumo, $s$ de sus bits, entonces la palabra resultante NO pertenece al
c\'odigo. Mientras que si alteramos m\'as de $s$ de sus bits la palabra
resultante podr\'{\i}a pertenecer al c\'odigo. Esto nos dice que dos palabras
del c\'odigo difieren en $s+1$ bits, por lo menos. O lo que es lo mismo que
la distancia m\'{\i}nima, $d_{min}$, del c\'odigo es mayor, estrictamente, que
$s$.\\

$\Leftarrow |$ Supongamos que tenemos un c\'odigo $\mathcal{C}$ con distancia
m\'{\i}nima $d_{min}$ verificando que $d_{min} > s$.\\

Entonces si, dada una palabra cualquiera del c\'odigo alteramos, a lo sumo, $s$
de sus bits entonces la palabra obtenida no pertenecer\'a al c\'odigo. Pero
seg\'un lo visto en el apartado $(\ref{sec:DistanErr})$, en la p\'agina
$\pageref{sec:DistanErr}$, esto es equivalente a detectar errores de, a lo
sumo, peso $k$.
\begin{flushright}
$\blacksquare$
\end{flushright}
De todo esto se deduce que, cuanto mayor sea la distancia m\'{\i}nima mayor
n\'umero de errores detectar\'a el c\'odigo. Los c\'odigos cuya distancia
m\'{\i}nima sea ``\emph{grande}'' ser\'an buenos c\'odigos, ya que podremos
detectar un mayor n\'umero de errores. As\'{\i} mismo cuanto mayor distancia
m\'{\i}nima tenga un c\'odigo m\'as d\'{\i}ficil ser\'a que un error en una
palabra transmitida de como resultado otra palabra del c\'odigo, con lo cual el
error ser\'{\i}a indetectable.
%
\newpage
%
\subsection{Calculo de todas las distancias m\'{\i}nimas}

Ya hemos visto que para calcular la distancia m\'{\i}mina de un c\'odigo tenemos
que calcular la distancia de cada elemento a todos y cada uno de los otros.\\

Como los c\'odigos son finitos, tienen un n\'umero finito de elementos, podemos
calcular el n\'umero de distancias que tenemos que calcular.\\

Dado un c\'odigo $\mathcal{C}$ el cual tiene $|\mathcal{C}|$ palabras. Para
calcular la distancia m\'inima del c\'odigo tendremos:
\begin{itemize}
\item Dada una palabra del c\'odigo tendremos que calcular la distancia de
dicha palabra con el resto de palabras del c\'odigo. Como el c\'odigo tiene
$|\mathcal{C}|$ palabras entonces tendremos que calcular $|\mathcal{C}|-1$
distancias.
\item Como tenemos que repetir esta operaci\'on con todas las palabras del
c\'odigo entonces tendremos que calcular $|\mathcal{C}|\cdot (|\mathcal{C} |-1)$
distancias.
\end{itemize}
Luego el n\'umero total de distancias a calcular es de $|\mathcal{C}|\cdot 
(|\mathcal{C}|-1)$ distancias.
