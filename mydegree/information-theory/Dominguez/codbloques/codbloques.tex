%
% CODIGOS DE BLOQUES
%

\chapter{C\'odigos de bloques}

Hemos visto en el cap\'{\i}tulo anterior que para enviar enviar informaci\'on
lo haremos dividiendola en bloques o palabras y mandando los bloques por
separado. Pero no hemos dicho nada sobre la longitud de las palabras que vamos
a utilizar para mandar los mensajes.\\

Cuando utilizemos un c\'odigo para mandar informaci\'on lo haremos siempre, a
no ser que se diga lo contrario, con una longitud fija de palabra.

\section{C\'odigo de bloques $(n,m)$}

\begin{definicion}[C\'odigo de bloques de tipo $(n,m)$]
\ \\
Diremos que un c\'odigo de bloques es de \textbf{``tipo $(n,m)$}'' cuando la
longitud de palabras del c\'odigo sea de $n$ bits y de esos $n$ bits utilizemos
$m$ para transmitir informaci\'on. Es decir tendremos $n-m$ bits para a\~nadir
informaci\'on redundante. A un c\'odigo de este tipo lo denotaremos como
$\mathcal{C}[n,m]$.
\end{definicion}
Obviamente se tiene que $n,m\in \mathbb{N}$.

\subsection{Longitud de un c\'odigo}

\begin{definicion}[Longitud de un c\'odigo]
\ \\
Dado un c\'odigo de tipo $(n,m)$ llamaremos \textbf{``longitud del c\'odigo''}
al n\'umero $n$, es decir, a la longitud de las palabras empleadas por el
c\'odigo.
\end{definicion}
%
\newpage
%
\subsection{Rango de un c\'odigo}

\begin{definicion}[Rango de un c\'odigo]
\ \\
Dado un c\'odigo de tipo $(n,m)$ llamaremos \textbf{``rango del c\'odigo''}
al n\'umero $m$, es decir, al n\'umero de bits utilizados para transmitir
informaci\'on.
\end{definicion}

\subsection{Raz\'on de un c\'odigo}

\begin{definicion}[Raz\'on de un c\'odigo]
\ \\
Dado un c\'odigo de tipo $(n,m)$ llamaremos \textbf{``raz\'on del c\'odigo''}
al n\'umero $\frac{m}{n}\leq 1$.
\end{definicion}
Nos interesan c\'odigos cuya raz\'on este cerca de $1$. En estos c\'odigos
introduciremos pocos bits de control, por lo tanto son c\'odigos en los que
se emplea poco tiempo en su transmisi\'on\footnote{En comparaci\'on con el
mismo c\'odigo pero con m\'as bits de control.}.

\subsection{Radio de recubrimiento de un c\'odigo}

\begin{definicion}[Radio de recubrimiento]
\ \\
Dado un c\'odigo de tipo $(n,m)$ llamaremos \textbf{``radio de recubrimiento del
c\'odigo''} a:
\begin{displaymath}
\rho (\mathcal{C}) = \max_{x\in \mathbb{F}^{^n}_q}\{\ \min_{c\in \mathcal{C}}
\{\ d(x,c)\ \}\ \} 
\end{displaymath}
\end{definicion}
La interpretaci\'on de este n\'umero es la siguiente:
\begin{quote}
$\rho(\mathcal{C})$ es el menor n\'umero positivo, $\rho$, tal que las bolas
$B_{\rho}(c)$ de radio $\rho$ y con centro en los puntos $c\in \mathcal{C}$
recubren $\mathbb{F}^{^n}_q$.
\end{quote}
\section{Codificadores}\label{sec:Codificadores}

Matem\'aticamente hablando un ``\emph{codificador}'' es una aplicaci\'on que
traduce una palabra a su equivalente respecto de un c\'odigo.
\begin{definicion}[Codificador]
\ \\
Sea $\mathcal{C}$ un c\'odigo del tipo $(n,m)$. Llamaremos
\textbf{``codificador''} a una aplicaci\'on biyectiva $C$, tal que:
\begin{eqnarray*}
C: \mathbb{F}^{^m}_2 &\stackrel{\sim}\longrightarrow& \mathcal{C}\subset \mathbb{F}^{^n}_q \\
 x &\longrightarrow & u
\end{eqnarray*}
\end{definicion}
La labor del ``\emph{codificador}'' consiste en asignar una, y solo una, palabra
del c\'odigo a cada palabra del lenguaje. Es decir, se limita a a\~nadir los
bits de control a cada palabra del lenguaje.\\

Supondremos que los ``\emph{codificadores}'' son \textbf{estandar}, es decir:
\begin{displaymath}
C(x)=u=(x_1,\cdots,x_m,c_1,\cdots,c_{n-m})\quad donde\ x_i,c_j\in \mathbb{F}_q
\end{displaymath}
donde los $c_j$ para $j=1,\cdots,n-m$ son los bits de control.

\section{Decodificadores} \label{sec:Decodificadores}
\begin{definicion}[Decodificador]
\ \\
Sea $\mathcal{C}$ un c\'odigo del tipo $(n,m)$, Llamaremos
\textbf{``decodificador''} a una aplicaci\'on biyectiva $D$, tal que:
\begin{eqnarray*}
D:\mathbb{F}^{^n}_2 \supset \mathcal{C}&\stackrel{\sim}\longrightarrow & \mathbb{F}^{^m}_2 \\
 u&\longrightarrow & x
\end{eqnarray*}
\end{definicion}
La labor del ``\emph{decodificador}'' consiste en asignar una, y solo una,
palabra del lenguaje a cada palabra del c\'odigo. Es decir, se limita a quitar
los bits de control a cada palabra del c\'odigo.

\section{Procesadores de error}\label{sec:ProcErr}

Cuando hemos recibido una transmisi\'on, ?`Como comprobar si ha habido alg\'un
error en la transmisi\'on?. Esta labor la realizan los ``\emph{procesadores de
error}''.
\begin{definicion}[Procesadores de error]
\ \\
Un \textbf{``procesador de error''}, desde el punto de vista matem\'atico, 
para un c\'odigo de tipo $(n,m)$ es una aplicaci\'on:
\begin{displaymath}
\begin{array}{cccc}
P:&\mathbb{F}^{^m}_2&\longrightarrow & \mathbb{F}_2\times \mathcal{C}[n,m] \\
 & v&\longrightarrow & (0/1,u')
\end{array}
\end{displaymath}
tal que:
\begin{itemize}
\item Si $v \in \mathcal{C}[n,m]$ entonces $P(v)=(0,v)$.
\item Si $v \notin \mathcal{C}[n,m]$ entoces $P(v)=(1,u')$. Donde $u'\in
\mathcal{C}[n,m]$ es la palabra que, con m\'as probabilidad, se transmiti\'o
originalmente. Es decir $u'$ ser\'a la palabra del c\'odigo ``m\'as parecida''
a $v$.
\end{itemize}
\end{definicion}
Cuando se recibe una palabra y esta pertenece al c\'odigo que se esta utilizando, no quiere decir que no haya habido alg\'un error en la transmisi\'on. Puede
haber ocurrido alg\'un error y la palabra resultante del error ser una palabra
del c\'odigo. Para evitar que ocurra esto se puede intentar ``\emph{separar}''
las palabras del c\'odigo lo m\'as posible, pero para ello es necesario saber
``\emph{como medir distancias}'' dentro del c\'odigo.
\begin{definicion}[Transmisi\'on correcta]
\ \\
Entenderemos por \textbf{``transmisi\'on correcta''} aquella que verifique que
$P(v)=(0/1,u)$, donde $P$ es el procesador de error, $v$ es la palabra recibida
y $u$ es la palabra transmitida originalmente.
\end{definicion}
Intentaremos utilizar c\'odigos en los que las palabras esten lo m\'as
``\emph{separadas}'' posible, ya que cuanto m\'as ``\emph{separadas}'' esten
entre s\'{\i}, m\'as dificil ser\'a que un error en una palabra del c\'odigo nos
de otra palabra del c\'odigo. Con lo cual el error ser\'{\i}a indetectable%
\footnote{A menos que se conozca, a priori, la palabra transmitida.}.\\

Una vez detectado un error en la transmisi\'on supondremos que la palabra que,
con m\'as probabildad, fue transmitida originalmente ser\'a aquella palabra del
c\'odigo que est\'e m\'as ``\emph{cercana}'' a la palabra recibida.

%
% NOCION DE DISTANCIA
%

%
% NOCION DE DISTANCIA
%

\section{Noci\'on de distancia}

Dar una distancia en un c\'odigo equivale a decir cuando dos palabras en el
c\'odigo son diferentes y en ``\emph{cuanto}'' difieren.\\

Dadas dos palabras $a=(a_1,\cdots,a_n)$ y $b=(b_1,\cdots,b_n)$, donde $a,b
\in \mathcal{C}[n,m]\subset \mathbb{F}^{^n}_q$, se tienen las siguientes
definiciones:
\begin{definicion}[Igualdad de palabras]
\ \\
Se dice que dos palabras $a$ y $b$, del mismo c\'odigo, son \textbf{``iguales''}
cuando se verifica que $\forall \ i$ se tiene que $a_i=b_i$. En caso contrario
ambas palabras son \textbf{``distintas''}.
\end{definicion}

\begin{definicion}[Error en el lugar $i$-\'esimo]
\ \\
Dadas dos palabras $a$ y $b$, del mismo c\'odigo, diremos que hay un
\textbf{``error en el lugar $i$-\'esimo''} cuando $a_i\neq b_i$.
\end{definicion}
Medir la diferencia de dos palabras dadas, de un mismo c\'odigo, equivale a
dar una noci\'on de distancia, o lo que es lo mismo una ``\emph{m\'etrica}'',
en el c\'odigo. Esta diferencia la mediremos utilizando la ``\textbf{distancia
de Hamming}''.
\begin{definicion}[Distancia de Hamming]
\ \\
Dadas dos palabras $a$ y $b$, del mismo c\'odigo, definiremos su distancia como
el numero de $i$ tales que $a_i\neq b_i$, o lo que es lo mismo, como el n\'umero
de $i$ tales que $a_i-b_i \neq 0$. La distancia entre dos palabras la
denotaremos como $d(a,b)$.
\end{definicion}
Matem\'aticamente hablando tendremos, que la ``distancia de Hamming'' vendr\'a
dada por una aplicaci\'on:
\begin{displaymath}
\begin{array}{cccc}
d:&\mathbb{F}^{^n}_q\times \mathbb{F}^{^n}_q&\longrightarrow & \mathbb{Z}^{^+}\\
&(a,b)&\longrightarrow & d(a,b)
\end{array}
\end{displaymath}
Como observaci\'on diremos que esta ``\emph{distancia}'' tambi\'en se verifica
para $\mathbb{F}^{^n}_q$, con lo cual la distancia no s\'olo est\'a definida
en el c\'odigo, sino que tambi\'en est\'a definida en el total.

\begin{definicion}[Error de peso $k$]
\ \\
Dadas dos palabras $a$ y $b$, de un mismo c\'odigo, diremos que existe un
\textbf{``error de peso $k$''} si $d(a,b) = k$.
\end{definicion}
\begin{definicion}[Peso de una palabra]
\ \\
Llamaremos \textbf{``peso de una palabra''} a su distancia con el $0$. El peso
de una palabra de $n$ bits ser\'a:
\begin{displaymath}
d((a_1,\dots ,a_n),(0,\stackrel{n)} \ldots ,0))
\end{displaymath}
y lo denotaremos como $w(u)$, con $u=(a_1,\dots,a_n)$.
\end{definicion}

\subsection{Algunos ejemplos}

Consideremos un c\'odigo $\mathcal{C}[4,1]\subset{\mathbb{F}^{^4}_2}$. Cada
palabra de $\mathbb{F}^{^4}_2$ ser\'a de la forma $a=(a_1,a_2,a_3,a_4)$.
\begin{itemize}
\item $a=(0,0,1,0)$ y $b=(1,1,1,1)$ entonces $d(a,b) = 3$ ya que $a_1\neq b_1$,
$a_2\neq b_2$ y $a_4\neq b_4$. Es decir hay $3$ bits que no coinciden. Tenemos
un error de peso $3$.
\item $a=(0,1,0,1)$ y $b=(0,1,0,1)$ entonces $d(a,b) = 0$ ya que $a_i=b_i$
$\forall \ i$. Tenemos un error de peso $0$, ambas palabras son iguales.
\item $a=(1,0,0,1)$ y $b=(0,1,1,0)$ entonces $d(a,b) = 4$  ya que $a_i\neq b_i$
$\forall \ i$. Es decir ambas palabras no coinciden en ning\'un bit. Tenemos
un error de peso $4$.
\end{itemize}

\subsection{Propiedades de las distancias}

Hemos definido antes una aplicaci\'on a la que hemos llamado ``\emph{distancia
de Hamming}''. Una ``\emph{distancia}'' es una aplicaci\'on que cumple una
serie de condiciones.\\

Dado un conjunto cualquiera $A$, diremos que una aplicaci\'on, $d$, es una
distancia si es de la forma:
\begin{eqnarray*}
d: A\times A&\longrightarrow& \mathbb{R}^{^+}\\
(a,b)&\longrightarrow& d(a,b)
\end{eqnarray*}
y verifica las siguientes condiciones:
\begin{itemize}
\item $d(a,b)\geq 0$ $\forall \ a,b \in A$ y $d(a,b)=0$ $\Longleftrightarrow$
$a=b$.
\item $d(a,b)=d(b,a)$ $\forall \ a,b\in A$.
\item ``\emph{Desigualdad triangular}''
\begin{displaymath}
d(a,c)\leq d(a,b) + d(b,c)\quad \forall \ a,b,c\in A
\end{displaymath}
\end{itemize}
En nuestros casos $A=\mathbb{F}^{^n}_q$ y en lugar de considerar
$\mathbb{R}^{^+}$ consideraremos $\mathbb{Z}^{^+}$.

\subsection{Detecci\'on de errores mediante las distancias}\label{sec:DistanErr}

Para detectar errores en una transmisi\'on utilizaremos una propiedad de las
distancias:
\begin{quote}
La distancia entre dos palabras es nula $\Longleftrightarrow$ s\'{\i} ambas
palabras son la misma. $d(a,b)=0$ $\Longleftrightarrow$ $a=b$.
\end{quote}
Cuando hayamos recibido una palabra en una transmisi\'on lo que haremos para
detectar si ocurrio alg\'un error en la transmisi\'on ser\'a:
\begin{itemize}
\item Calcular la distancia de la palabra recibida, $u$, a todas las del
c\'odigo.
\item Si alguna de las distancias es nula entonces se tiene que la palabra
recibida $u$ pertenece al c\'odigo. En caso contrario la palabra recibida $u$
no pertenece al c\'odigo, con lo cual habr\'a ocurrido alg\'un error en la
transmisi\'on.
\end{itemize}


%
% ELECCION DE BUENOS CODIGOS
%

%
% ELECCION DE BUENOS CODIGOS 
%

\section{Elecci\'on de buenos c\'odigos}

Ya hemos comentado anteriormente que al transmitir una palabra se produzca un
error, y este error genere otra palabra del c\'odigo, con lo cual el error
cometido ser\'{\i}a indetectable. Por ejemplo en el ``\emph{c\'odigo de triple
repetici\'on}'', el cual est\'a formado unicamente por dos palabras
$\mathcal{C}[3,1]=\{000,111\}$ supongamos que transmitimos la palabra $000$ y se
cometen errores:
\begin{itemize}
\item \textbf{de peso $1$}, eso significar\'{\i}a que la palabra recibida
ser\'{\i}a una de las siguientes:
\begin{itemize}
\item $100$.
\item $010$.
\item $001$.
\end{itemize}
donde el error es detectable, ya que ninguna de ellas pertenece al c\'odigo.
\item \textbf{de peso $2$}, eso significar\'{\i}a que la palabra recibida
ser\'{\i}a una de las siguientes:
\begin{itemize}
\item $110$.
\item $101$.
\item $011$.
\end{itemize}
donde el error sigue siendo detectable, ya que ninguna de ellas pertenece al
c\'odigo. 
\item \textbf{de peso $3$}, eso significar\'{\i}a que la palabra recibida
ser\'a la siguiente:
\begin{itemize}
\item $111$.
\end{itemize}
donde el error es indetectable, ya que la palabra recibida pertenece al
c\'odigo.
\end{itemize}
Luego con este c\'odigo podemos detectar errores de peso $2$ ya que sus palabras
tienen una distancia entre s\'{\i} de $3$.\\

De esto se deduce que para poder detectar errores de peso $s+1$ hemos de 
utilizar c\'odigos cuyas palabras disten entre s\'{\i} $s$.

\subsection{Distancia m\'{\i}nima}

Supongamos que tenemos un c\'odigo $\mathcal{C}[n,m]\subset \mathbb{F}^{^n}_q$,
la distancia m\'{\i}nima de dicho c\'odigo es un dato importante. Esto es 
debido a que nos indica que tipos de errores se detectan en el c\'odigo.
\begin{definicion}[Distancia M\'{\i}nima]
\ \\
Llamaremos \textbf{``distancia m\'{\i}nima''} de un c\'odigo a la m\'{\i}nima
distancia entre todos los posibles pares de palabras que podamos formar con
las palabras del c\'odigo. Es decir:
\begin{displaymath}
d_{min} (\mathcal{C}[n,m]) = \mathop{\min_{a,b\in \mathcal{C}[n,m]}}_{a\neq b}
\{ \ d(a,b)\ \}
\end{displaymath}
\end{definicion}
Observar que todos los c\'odigos que utilizaremos van a ser subconjuntos de
cuerpos finitos y entonces tiene sentido hablar de ``\emph{distancia
m\'{\i}nima}''.\\ \\
%
A esta distancia para abreviar nos referiremos como $d_{min}$.\\ \\
%
Muchas veces un c\'odigo $\mathcal{C}[n,m]$ se suele denotar como
$\mathcal{C}[n,m,d_{min}]$.\\

Dado un c\'odigo $\mathcal{C}[n,m,d_{min}]$ tendremos que la distancia
m\'{\i}nima, entre dos palabras diferentes, de dicho c\'odigo es de $d_{min}$.
Luego difieren en, al menos, $d_{min}$ bits. En un c\'odigo de este tipo son
detectables errores de peso menor, estrictamente, que $d_{min}$, mientras que
los errores de peso mayor o igual que $d_{min}$ son indetectables%
\footnote{Generalmente, ya
que puede darse el caso en el que un error de peso mayor que la distancia
m\'{\i}nima sea detectable.}.
%
\newpage
%
\begin{teorema}[Teorema de detecci\'on de errores]\label{the:Deteccion}
\ \\
Un c\'odigo detecta errores de peso menor o igual que $s$ $\Longleftrightarrow$
$d_{min}$ es mayor, estrictamente, que $s$.
\end{teorema}
\underline{\textbf{Demostraci\'on}}:\\

$\Rightarrow |$ Supongamos que tenemos un c\'odigo $\mathcal{C}$ que detecta
errores de peso menor o igual que $s$.\\

Seg\'un lo visto en el apartado $(\ref{sec:DistanErr})$, en la p\'agina
$\pageref{sec:DistanErr}$, detectar un error de peso menor o igual que $s$
significa que, si dada una palabra del c\'odigo cualquiera alteramos, a
lo sumo, $s$ de sus bits, entonces la palabra resultante NO pertenece al
c\'odigo. Mientras que si alteramos m\'as de $s$ de sus bits la palabra
resultante podr\'{\i}a pertenecer al c\'odigo. Esto nos dice que dos palabras
del c\'odigo difieren en $s+1$ bits, por lo menos. O lo que es lo mismo que
la distancia m\'{\i}nima, $d_{min}$, del c\'odigo es mayor, estrictamente, que
$s$.\\

$\Leftarrow |$ Supongamos que tenemos un c\'odigo $\mathcal{C}$ con distancia
m\'{\i}nima $d_{min}$ verificando que $d_{min} > s$.\\

Entonces si, dada una palabra cualquiera del c\'odigo alteramos, a lo sumo, $s$
de sus bits entonces la palabra obtenida no pertenecer\'a al c\'odigo. Pero
seg\'un lo visto en el apartado $(\ref{sec:DistanErr})$, en la p\'agina
$\pageref{sec:DistanErr}$, esto es equivalente a detectar errores de, a lo
sumo, peso $k$.
\begin{flushright}
$\blacksquare$
\end{flushright}
De todo esto se deduce que, cuanto mayor sea la distancia m\'{\i}nima mayor
n\'umero de errores detectar\'a el c\'odigo. Los c\'odigos cuya distancia
m\'{\i}nima sea ``\emph{grande}'' ser\'an buenos c\'odigos, ya que podremos
detectar un mayor n\'umero de errores. As\'{\i} mismo cuanto mayor distancia
m\'{\i}nima tenga un c\'odigo m\'as d\'{\i}ficil ser\'a que un error en una
palabra transmitida de como resultado otra palabra del c\'odigo, con lo cual el
error ser\'{\i}a indetectable.
%
\newpage
%
\subsection{Calculo de todas las distancias m\'{\i}nimas}

Ya hemos visto que para calcular la distancia m\'{\i}mina de un c\'odigo tenemos
que calcular la distancia de cada elemento a todos y cada uno de los otros.\\

Como los c\'odigos son finitos, tienen un n\'umero finito de elementos, podemos
calcular el n\'umero de distancias que tenemos que calcular.\\

Dado un c\'odigo $\mathcal{C}$ el cual tiene $|\mathcal{C}|$ palabras. Para
calcular la distancia m\'inima del c\'odigo tendremos:
\begin{itemize}
\item Dada una palabra del c\'odigo tendremos que calcular la distancia de
dicha palabra con el resto de palabras del c\'odigo. Como el c\'odigo tiene
$|\mathcal{C}|$ palabras entonces tendremos que calcular $|\mathcal{C}|-1$
distancias.
\item Como tenemos que repetir esta operaci\'on con todas las palabras del
c\'odigo entonces tendremos que calcular $|\mathcal{C}|\cdot (|\mathcal{C} |-1)$
distancias.
\end{itemize}
Luego el n\'umero total de distancias a calcular es de $|\mathcal{C}|\cdot 
(|\mathcal{C}|-1)$ distancias.


%
% CORRECCION DE ERRORES
%

%
% CORRECCION DE ERRORES
%

\section{Correcci\'on de errores}

Ya hemos visto como detectar cuando ha ocurrido un error en una transmisi\'on,
pero eso no es suficiente. Tambi\'en hay que saber cuando y como se puede
corregir un error. Los procesadores de error\footnote{Apartado
$(\ref{sec:ProcErr})$ en la p\'agina $\pageref{sec:ProcErr}$.} ser\'an los
encargados de detectar y corregir los errores.\\

Hemos visto que calculando la distancia m\'{\i}nima de un c\'odigo podemos
saber hasta que peso se pueden detectar errores.\\

Pero una vez detectado que se ha cometido un error, ?`como elegir la palabra
adecuada que se transmiti\'o originalmente?. Esto tambi\'en lo haremos mediante
la distancia m\'{\i}nima.
%
\newpage
%
Supongamos que recibimos una palabra $w$ y deseamos ser capaces de corregir
todos los errores de peso $1$. Una vez recibida la palabra $w$ y sabiendo que
se ha cometido un error de peso $1$ deber\'{\i}a haber una \'unica palabra
del c\'odigo $u$ tal que $d(u,w)=1$. Pero se puede dar el caso en el que existan
$u,v\in \mathcal{C}$ tales que $d(u,w)=d(v,w)=1$, entoces en esta situaci\'on,
?`que palabra elegiriamos como correcta?.\\

El mismo razonamiento se puede seguir para errores de peso $k$.
\begin{definicion}[Correcci\'on de errores de peso $k$]
\ \\
Dada una palabra $w$ en la que sabemos que existe un error de peso $k$, a lo
sumo, diremos que un c\'odigo $\mathcal{C}$ corrige errores de peso menor o
igual que $k$ $\Longleftrightarrow$ existe una y s\'olo una palabra $u\in
\mathcal{C}$ tal que $d(u,w)\leq k$ para cualquier $w$ en el que se cometa un
error de peso $k$, a lo sumo.
\end{definicion}

Para solucionar este problema elegimos c\'odigos en los que las palabras esten
bastante separadas.

\begin{teorema}[Teorema de correcci\'on de errores]\label{the:Correccion}
\ \\
Dado un c\'odigo $\mathcal{C}$ se pueden corregir, sin problemas, todos los
errores de peso menor o igual que $t$ $\Longleftrightarrow$ su distancia
m\'{\i}nima, $d_{min}$, es mayor o igual que $2\cdot t +1$ .
\end{teorema}
\underline{\textbf{Demostraci\'on}}:

$\Rightarrow |$ Supongamos que tenemos un c\'odigo que es capaz de corregir
errores de peso menor o igual que $t$. Entonces si recibimos una palabra $w$
en la que sabemos que existe un error de peso $t$, a lo sumo, entonces 
existir\'a una y s\'olo una palabra $u\in \mathcal{C}$ tal que $d(w,u)\leq t$.\\

Lo demostraremos por reducci\'on al absurdo. Supondremos que existen dos
palabras en el c\'odigo, $u_1,u_2 \in \mathcal{C}$, tales que
$d(u_1,u_2)\leq 2\cdot t$.\\ \\
%
Sea $w$ una palabra tal que:
\begin{itemize}
\item Coincide con $u_1$ en los bits en los que coinciden $u_1$ y $u_2$.
\item Coincide con $u_1$ en los $t$ primeros bits en los que $u_1$ y $u_2$ no
coinciden.
\item Coincide con $u_2$ en los bits siguientes, a los $t$ primeros, en los
que $u_1$ y $u_2$ no coinciden.
\end{itemize}
De esta construcci\'on es claro que $d(w,u_2) = t$.\\ \\
%
Supongamos que $u_1$ y $u_2$ son palabras de $n$ bits que coinciden en $m$
bits. Entonces tendremos que $d(w,u_1) = n-m-([n-m]-t) = t$.\\

Suponemos que recibimos la palabra $w$ y sabemos que se cometi\'o un error
de peso $t$, a lo sumo, en la transmisi\'on. Entonces la palabra que se
transmiti\'o originalmente es la \'unica palabra del c\'odigo, $u$, tal que
$d(w,u)\leq t$ ya que por hip\'otesis el c\'odigo corrige errores de peso
menor o igual que $t$.\\

Ahora bien, tenemos en el c\'odigo dos palabras, $u_1,u_2$, tales que verifican
que $d(w,u_i)\leq t$ para $i=1,2$. Pero esto no puede ocurrir por hip\'otesis
ya que unicamente puede haber una palabra en el c\'odigo que cumpla la
condici\'on. Hemos llegado a una contradicci\'on al suponer que
$d(u_1,u_2)\leq 2\cdot t$ para $u_1,u_2\in \mathcal{C}$. Entonces se tendr\'a
que $d(u_1,u_2) > 2\cdot t$ para $u_1,u_2\in \mathcal{C}$, es decir,
$d(u_1,u_2)\geq 2\cdot t+1$ para $u_1,u_2\in \mathcal{C}$ $\Longrightarrow$
$d_{min} \geq 2\cdot t +1$ .\\

$\Leftarrow |$ Supongamos que tenemos un c\'odigo cuya distancia m\'{\i}nima,
$d_{min}$, es mayor o igual que $2\cdot t +1$.\\

Supongamos que se ha realizado una transmisi\'on y se ha recibido la palabra
$w$, conociendo que se ha cometido un error de, a lo sumo, peso $t$. Obviamente
$w\notin \mathcal{C}$, ya que en caso contrario el error no ser\'{\i}a
detectable.\\

Supongamos que existen dos palabras en el c\'odigo $u,v\in \mathcal{C}$ tal que
$d(u,w)\leq t$ y $d(v,w)\leq t$. Entonces utilizando la desigualdad triangular
se tiene que:
\begin{displaymath}
d(u,v)\leq d(u,w)+d(w,v) = d(u,w)+d(v,w) = t+t= 2\cdot t
\end{displaymath}
Es decir $d(u,v)\leq 2\cdot t$, lo cual es falso ya que como
$u,v\in \mathcal{C}$ se tiene, por hip\'otesis, que $d(u,v)\geq 2\cdot t + 1$.\\

De todo esto se deduce que s\'olo hay una y s\'ola una palabra en el c\'odigo,
$u$, tal que $d(u,w)\leq t$, con lo cual $u$ es la palabra que se transmiti\'o
originalmente. Es decir podemos corregir errores de peso menor o igual que $t$.\\

En todo esto hemos supuesto que siempre existe una palabra en el c\'odigo, $u$,
tal que $d(u,w)\leq t$. Si esta palabra no existiera no tendr\'{\i}a sentido
este teorema, ya que es en dicha palabra en la que se produce el error, luego
siempre existe una palabra, por lo menos, en el c\'odigo verificando la
condici\'on $d(u,w)\leq t$.
\begin{flushright}
$\blacksquare$
\end{flushright}
Podemos interpretar el resultado de este teorema como:
\begin{quote}
Corregir adecuadamente cuesta el doble que detectar.
\end{quote}
\begin{definicion}[C\'odigo $t$-detector de errores]
\ \\
Diremos que un c\'odigo es un \textbf{``c\'odigo $t$-detector de errores''} si
verifica que $d_{min}=2\cdot t +1$.
\end{definicion}
Podemos mezclar los teoremas $\ref{the:Deteccion}$ y $\ref{the:Correccion}$ en
un s\'olo teorema:
\begin{teorema}[Teorema de detecci\'on y correcci\'on de errores]\label{the:DeteccionCorreccion}
\ \\
Un c\'odigo corrige errores de peso menor o igual que $t$ y detecta errores de
peso menor o igual que $s'=s+t$ $\Longleftrightarrow$
$d_{min}\geq 2\cdot t+s+1=t+s'+1$.
\end{teorema}


%
% DETECTAR O CORREGIR
%

%
% DETECTAR O CORREGIR
%

\section{?`Detectar o corregir?}

Si observamos los ejercicios de este cap\'{\i}tulo, en la p\'agina
$\pageref{sec:EjerciciosCodBloques}$, podemos ver que en los c\'odigos
de ``\emph{triple repetici\'on}'' y de ``\emph{triple control}'' tenemos una
de las dos siguientes opciones:
\begin{itemize}
\item Detectar y corregir un error.
\item Detectar dos errores y no corregir.
\end{itemize}
%
\newpage
%
\subsection{Probabilidad de error en el canal}

Para saber cual de las dos opciones utilizar necesitaremos conocer
``\emph{algo}'' m\'as sobre el canal de transmisi\'on. Ese ``\emph{algo}''
ser\'a la ``\emph{probabilidad de errores en el canal}''.
\begin{teorema}[Resultados sobre la probabilidad de error]
\ \\
Supongamos que tenemos un canal sim\'etrico, aleatorio, binario y con una
probabilidad de error $P$. En dicho canal transmitiremos palabras de longitud
$n$ mediante un c\'odigo de bloques.
\begin{enumerate}
\item La probabilidad de que ocurra un error concreto de peso $k$ ser\'a:
\begin{displaymath}
P^k\cdot (1-P)^{n-k}
\end{displaymath}
\item La probabilidad de que ocurra un error cualquiera de peso $k$ ser\'a:
\begin{displaymath}
{n \choose k}\cdot P^k\cdot (1-P)^{n-k}
\end{displaymath}
\end{enumerate}
\end{teorema}
\underline{\textbf{Demostraci\'on}}:
\begin{enumerate}
\item Si se ha cometido un error especifico de peso $k$ entonces tendremos que
$k$ bits son incorrectos y los $n-k$ restantes son correctos.\\

La probabildad de tener un bit en particular incorrecto es $P$, por lo tanto
la probabilidad de que un bit en particular sea correcto es $1-P$. Como el
canal es aleatorio todas estas probabilidades son independientes.\\

De todo esto y de que tenemos $k$ bits erroneos y $n-k$ correctos se tiene que
la probabilidad buscada es:
\begin{displaymath}
P^k\cdot (1-P)^{n-k}
\end{displaymath}
\item El n\'umero total de palabras con un error de peso $k$ es el mismo que
la cantidad de formas de elegir $k$ lugares de error, que es ${n\choose k}$. 
Como los errores son excluyentes, que s\'{\i} ocurre uno no puede
ocurrir otro, tenemos entonces que la probabilidad de que ocurra uno de ellos
es la suma de las probabilidades de cada error, es decir:
\begin{displaymath}
{n\choose k}\cdot P^k\cdot (1-P)^{n-k}
\end{displaymath}
\end{enumerate}
\begin{flushright}
$\blacksquare$
\end{flushright}
\begin{teorema}[Probabilidad de transmisiones correctas]
\ \\
Supondremos que utilizamos un c\'odigo de bloques para transmitir un mensaje
sobre un canal binario y sim\'etrico. Tenemos los siguientes resultados:
\begin{enumerate}
\item La probabilidad de que un procesador de error produzca la palabra correcta
es la suma de las probabilidades de error de los errores que el procesador de
error puede corregir.
\item Si los errores que el procesador de error puede corregir son
independientes de la palabra transmitida entonces la probabilidad de que el
mensaje se haya recibido correctamente es la suma de las probabilidades de que
se den los errores que el procesador corrige.
\end{enumerate}
\end{teorema}
\underline{\textbf{Demostraci\'on}}:
\begin{enumerate}
\item Los errores que el procesador corrige son excluyentes, ya que si se
produce un error no se puede producir otro. Entonces la probabilidad de que uno
ocurra es la suma de sus probabilidades individuales.
\item Como los errores que el procesador corrige son independientes de la
palabra entonces la probabilidad de transmisi\'on correcta es la suma de las
probabilidades individuales de los errores que el procesador de error corrige.
\end{enumerate}
\begin{flushright}
$\blacksquare$
\end{flushright}

\subsection{Ejemplos sobre probabilidad de error}

Supongamos que vamos a transmitir un mensaje de $10000$ bits por un canal con
una probabilidad de error de $\frac{1}{1000}$. Esto significa que hay una
probabildad de error en cada bit de $\frac{1}{1000}$, o lo que es lo mismo, que
hay una probabilidad de transmisi\'on correcta en un bit de
$1-\frac{1}{1000}=0.999$.\\

Sin utilizar ning\'un tipo de codificaci\'on la probabildad de transmisi\'on
correcta es $(0.999)^{10000}\simeq0.000045$.
%
\newpage
%
\subsubsection{C\'odigo del bit de control de paridad}

Este c\'odigo detecta hasta un error en cada palabra transmitida de longitud
ocho, pero no lo corrige.
\begin{itemize}
\item Probabilidad de que no haya errores en una palabra.\\

Como una palabra tiene $8$ bits la probabilidad de que no haya error en una
palabra ser\'a:
$$(0.999)^8\simeq 0.992028$$
\item Probabilidad de que haya un error en una palabra.\\

La probabilidad de que ocurra un error en una palabra es la probabilidad de
que en una palabra haya siete bits correctos y uno incorrecto:
$$(0.999)^7\cdot \frac{8}{1000}\simeq 0.007944$$
\item Probabilidad de transmisi\'on correcta.\\

Como queremos transmitir $1000$ bits con este c\'odigo los agruparemos en
bloques de siete m\'as el de control. Estas ser\'an las palabras, luego la
probabilidad de transmisi\'on correcta ser\'a la probabilidad de que no haya
ning\'un error en la transmisi\'on, ya que el c\'odigo no corrige errores:
$$(0.992028)^{\frac{10000}{7}}\simeq 0.000011$$
\item Probabilidad de que no ocurra un error indetectable.\\

La probabilidad de que no ocurra un error indetectable es la misma de que ocurra
un error detectable, es decir, de que no ocurra ning\'un error o de que ocurra
un error:
\begin{displaymath}
((0.992028)+(0.007944))^{\frac{10000}{7}}\simeq 0.960789
\end{displaymath}
\end{itemize}
%
\newpage
%
\subsubsection{C\'odigo de triple repetici\'on}

Podemos ver en el ejercicio $\ref{ejer:TripleRepeticion}$ que en este c\'odigo
se presentan dos situaciones:
\begin{enumerate}
\item Supondremos que es m\'as probable que se cometan dos errores que uno
en cada palabra transmitida. En este caso no se podr\'a corregir el error pero
se detectar\'an errores simples y dobles.
\begin{itemize}
\item Probabilidad de que no haya errores en una palabra.\\

Como una palabra tiene $3$ bits la probabilidad de que no haya error en una
palabra:
$$(0.999)^3\simeq 0.997003$$
\item Probabilidad de transmisi\'on correcta.\\

Como hemos de transmitir $10000$ bits y con este c\'odigo por cada bit de
informaci\'on hemos de transmitir $3$ bits, tendremos que vamos a transmitir un
total de $30000$ bits. Luego la probabilidad de transmisi\'on correcta ser\'a:
$$(0.999)^{30000}\simeq 9.218214^{-14}$$
\item Probabilidad de que no ocurra un error indetectable.\\

En el caso que estamos considerando el c\'odigo detecta que ha ocurrido un error
y supondr\'a que es un error doble, luego el \'unico caso en el que no
detectar\'a un error ser\'a cuando ocurra un error triple, es decir, que se
inviertan todos los bits de una palabra transmitida. Como la probabilidad de
error del canal es $\frac{1}{1000}$ entonces tenemos que la probabilidad de
que se haya un error triple ser\'a:
$$\Big(\frac{1}{1000}\Big)^3= 10^{-9}$$
Como vamos a transmitir $10000$ bloques de tres bits entonces la probabilidad
de que no ocurra un error triple, indetectable, ser\'a:
$$(1-10^{-9})^{10000}\simeq 0.99999$$
\end{itemize}
%
\newpage
%
Podemos ver que este c\'odigo tiene un probabilidad practicamente nula de que
ocurran errores indetectables, pero la probabilidad de que ocurran errores, 
detectables, en cada palabra es alt\'{\i}sima. Y teniendo en cuenta que el
c\'odigo no corrige errores necesitaremos una gran cantidad de retransmisiones,
que, junto con el hecho de que la raz\'on del c\'odigo es baja, $\frac{1}{3}$,
hace que transmitir de esta manera sea lento y poco recomendable.
%
%
\item Supondremos que es m\'as probable que se cometa un error que dos en 
cada palabra transmitida. En este caso podremos corregir el error.\\
\begin{itemize}
\item Probabilidad de que no haya errores en una palabra.\\

Como una palabra tiene $3$ bits la probabilidad de que no haya error en una
palabra:
$$(0.999)^3\simeq 0.997003$$
\item Probabilidad de que haya un error en una palabra.\\

La probabilidad de que ocurra un error en una palabra es la probabilidad de
que en una palabra haya dos bits correctos y uno incorrecto:
$$(0.999)^2\cdot \frac{3}{1000}\simeq 0.002994$$
\item Probabilidad de transmisi\'on correcta.\\

La probabilidad de transmisi\'on correcta es la probabilidad de que no ocurra
ning\'un error o de que ocurra uno\footnote{El c\'odigo corrige un error.} en
todas y cada una de las palabras transmitidas, que son $10000$, entonces:
$$((0.997003)+(0.002994))^{10000}\simeq 0.970445$$
\item Probabilidad de que no ocurra un error indetectable.\\

La probabilidad de que no ocurra un error indetectable es la misma de que ocurra
un error detectable, es decir, de que no ocurra ning\'un error o de que ocurra
un error:
\begin{displaymath}
((0.997003)+(0.002994))^{10000}\simeq 0.970445
\end{displaymath}
\end{itemize}
\end{enumerate}

\subsubsection{C\'odigo del triple control}

Supondremos que es m\'as probable que se cometa un error que dos en
cada palabra transmitida. En este caso podemos corregir el error.
\begin{itemize}
\item Probabilidad de que no haya errores en una palabra.\\

Como una palabra tiene $6$ bits la probabilidad de que no haya errores en una
palabra:
$$(0.999)^6\simeq 0.994015$$
\item Probabilidad de que haya un error en una palabra.\\

La probabilidad de que ocurra un error en una palabra es la probabilidad de que
haya cinco bits correctos y uno incorrecto:
$$(0.999)^5\cdot \frac{6}{1000}\simeq 0.005970$$
\item Probabilidad de transmisi\'on correcta.\\

La probabilidad de transmisi\'on correcta es la probabilidad de que no ocurra
ning\'un error o de que ocurra uno\footnote{El c\'odigo corrige un error.} en
todas y cada una de las palabras transmitidas, que son $\frac{10000}{3}$:
$$((0.994015)+(0.005970))^{\frac{10000}{3}}\simeq 0.951229$$
\item Probabilidad de que no ocurra un error indetectable.\\

La probabilidad de que no ocurra un error indetectable es la misma de que ocurra
un error detectable, es decir, de que no ocurra ning\'un error o de que ocurra
un error:
\begin{displaymath}
((0.994015)+(0.005970))^{\frac{10000}{3}}\simeq 0.951229
\end{displaymath}
\end{itemize}


%
% EL TEOREMA DE SHANNON
%

%
% EL TEOREMA DE SHANNON
%

%
\newpage
%
\section{El teorema de Shannon}

\emph{Claude Shannon} demostr\'o en $1.948$ que existe una constante llamada
``\emph{capacidad del canal}'', $C(P)$, para cualquier canal sim\'etrico,
binario y con probabilidad $P$ tal que siempre existen c\'odigos de bloques
donde la probabilidad de transmisi\'on correcta est\'a arbitrariamente
pr\'oxima a $C(P)$.
\begin{teorema}[de Shannon, $1.948$]
\ \\
Dado un canal sim\'etrico, binario y con probabilidad $P$ siempre existen
c\'odigos de bloques donde la probabilidad de transmisi\'on correcta est\'a
arbitrariamente pr\'oxima a:
\begin{displaymath}
C(P) = 1 + P \cdot \log_2 P+(1-P)\cdot \log_2 (1-P)
\end{displaymath}
La constante $C(P)$ recibe el nombre de ``\emph{capacidad del canal}''.
\end{teorema}

Supongamos que la probabilidad de error del canal es $P=\frac{1}{2}$ entonces
se tiene que $C(\frac{1}{2})=0$, es decir, la probabilidad de transmisi\'on 
correcta en un canal de este tipo es practicamente nula. Con lo cual no es 
aconsejable trabajar con canales de este tipo.


%
% EJEMPLOS
%

%
% EJEMPLOS
%

\section{Ejemplos}

\subsection{C\'odigo del bit de control de paridad}

Este c\'odigo es un c\'odigo de $2^7=128$ palabra palabras. Es un subconjunto de
$\mathbb{F}^{^8}_2$ el cual tiene $2^8=256$ palabras, con lo cual el n\'umero
de patrones de error que tendremos ser\'a $2^8-2^7=128$.

\subsubsection{Tabla est\'andar del c\'odigo}

Como el c\'odigo tiene $2^7=128$ palabras la tabla tendr\'a $2^7=128$ columnas
y el n\'umero de filas ser\'a $2^{8-7}=2$.\\

Debido a la gran cantidad de palabras de este c\'odigo no construiremos su
tabla est\'andar.

\subsubsection{Sindromes}

Sea $H$ la matriz de control, en forma est\'andar, del c\'odigo:
\begin{displaymath}
\left( \begin{array}{cccccccc}
1&1&1&1&1&1&1&1
\end{array} \right)
\end{displaymath}

\begin{table}[!h]
\begin{displaymath}
\begin{array}{|c|c|}
\hline
Errores & Sindromes \\
\hline
00000000 & 0 \\
\hline
10000000 & 1 \\
\hline
\end{array}
\end{displaymath}
\caption{Tabla de sindromes del c\'odigo del bit de control de paridad.}
\end{table}

\subsection{C\'odigo de triple repetici\'on}

Este c\'odigo es un c\'odigo de $2^1=2$ palabras. Es un subconjunto de
$\mathbb{F}^{^3}_2$ el cual tiene $2^3=8$ palabras, con lo cual el n\'umero
de patrones de error que tendremos ser\'a $2^3-2^1=6$.

\subsubsection{Tabla est\'andar del c\'odigo}

Como el c\'odigo tiene $2^1=2$ palabras la tabla tendr\'a $2^1=2$ columnas y
el n\'umero de filas ser\'a $2^{3-1}=4$.\\

La primera fila estar\'a formada por las palabras del c\'odigo con la
condici\'on de que el $0$ sea el primer elemento. Luego la primera fila ser\'a:
\begin{displaymath}
\begin{array}{cc}
000&111
\end{array}
\end{displaymath}
Para la segunda fila cogeremos un elemento de $\mathbb{F}^{^3}_2$ que no este en
la fila cero, y el primer elemento de cada fila ha de ser el de m\'{\i}nimo
peso de dicha fial. El elemento $100$ no esta en la fila cero, luego elegimos
ese elemento como primer elemento de la fila uno ya que es de peso uno y no es
posible encontrar otro de peso menor. El resto de elementos de la fila ser\'an:
\begin{displaymath}
Tb(\mathcal{C}[3,1])_{1,i}=Tb(\mathcal{C}[3,1])_{1,0}+Tb(\mathcal{C}[3,1])_{0,i}
\quad i=0,1
\end{displaymath}
Luego la segunda fila ser\'a:
\begin{displaymath}
\begin{array}{cc}
100&011
\end{array}
\end{displaymath}
Siguiendo el mismo razonamiento elegiremos como primer elemento de la fila tres
un elemento que no haya aparecido en las filas anteriores y de peso m\'{\i}nimo,
por ejemplo $010$ y siguiendo el razonamiento anterior completaremos la tabla.
\begin{eqnarray*}
Tb(\mathcal{C}[3,1])_{2,0}&=&010\\
Tb(\mathcal{C}[3,1])_{3,0}&=&001
\end{eqnarray*}
\begin{table}[!h]
\begin{displaymath}
\begin{array}{|c|c|}
\hline
000&111\\
\hline
100&011\\
\hline
010&101\\
\hline
001&110\\
\hline
\end{array}
\end{displaymath}
\caption{Tabla est\'andar del c\'odigo de triple repetici\'on.}\label{tab:TablaII}
\end{table}

\subsubsection{Correcci\'on de errores}

Supongamos que recibimos la palabra $011$, que no pertenece al c\'odigo. Para
corregir el c\'odigo haremos:
\begin{itemize}
\item Localizamos el lugar de dicha palabra en la tabla.
\begin{displaymath}
Tb(\mathcal{C}[3,1])_{1,1}=011
\end{displaymath}
\item Elegimos como palabra correcta la palabra que est\'e en la misma columna y
en la fila cero.
\begin{displaymath}
Tb(\mathcal{C}[3,1])_{0,1}=111
\end{displaymath}
\end{itemize}
El error cometido vendr\'a dado por el primer elemento de la fila en la que se
encuentre la palabra recibida:
\begin{displaymath}
Tb(\mathcal{C}[3,1])_{1,0}=100
\end{displaymath}
Luego en la transmisi\'on se ha cometido un error de peso $w(100)=1$ en el
primer bit.

\subsubsection{Sindromes}

Supondremos que queremos corregir los mismos errores que se corrigen con la
tabla $\ref{tab:TablaII}$.\\ \\
%
Sea $H$ la matriz de control, en forma est\'andar, del c\'odigo:
\begin{displaymath}
H=\left( \begin{array}{ccc}
1&1&0\\
1&0&1
\end{array} \right)
\end{displaymath}

\begin{table}[!h]
\begin{displaymath}
\begin{array}{|c|c|}
\hline
Errores & Sindromes \\
\hline
000 & 00 \\
\hline
100 & 11 \\
\hline
010 & 10 \\
\hline
001 & 01 \\
\hline
\end{array}
\end{displaymath}
\caption{Tabla de sindromes del c\'odigo de triple repetici\'on.}\label{tab:TabSindromes}
\end{table}

Para corregir errores supongamos que recibimos la palabra $101$, calculamos
su sindrome que ser\'a $10$. Entonces utilizando la tabla
$\ref{tab:TabSindromes}$ buscamos el error que tiene sindrome $10$, dicho
error es $010$. Luego la palabra transmitida ser\'a $101-010=111$. Recordar
que $-1=1$ en $\mathbb{F}_2$.

\subsection{C\'odigo de triple control}

Este c\'odigo es un c\'odigo de $2^3=8$ palabras. Es un subconjunto de
$\mathbb{F}^{^6}_2$ el cual tiene $2^6=64$ palabras, con lo cual el n\'umero
de patrones de error que tendremos ser\'a $2^6-2^3=56$.

\subsubsection{Tabla est\'andar del c\'odigo}

Como el c\'odigo tiene $2^3=8$ palabras la tabla tendr\'a $2^3=8$ columnas y
el n\'umero de filas ser\'a $2^{6-3}=8$.\\ 

La primera fila estar\'a formada por las palabras del c\'odigo con la
condici\'on de que el $0$ sea el primer elemento. Luego la primera fila
ser\'a:
\begin{displaymath}
\begin{array}{cccccccc}
000000&100110&010101&001011&111000&011110&101101&110011
\end{array}
\end{displaymath}
Para la segunda fila cogeremos un elemento de $\mathbb{F}^{^6}_2$ que no este
en la fila cero, y el primer elemento de cada fila ha de ser el de
m\'{\i}nimo peso de dicha fila. El elemento $100000$ no est\'a en la fila 
cero,
luego elegimos ese elemento como primer elemento de la fila uno ya que es de
peso uno y no es posible encontrar otro elemento de peso menor. El resto de
elementos de la fila ser\'an:
\begin{displaymath}
Tb(\mathcal{C}[6,3])_{1,i}=Tb(\mathcal{C}[6,3])_{1,0}+Tb(\mathcal{C}[6,3])_{0,
i}
\quad i=1,\dots,7
\end{displaymath}
Luego la segunda fila ser\'a:
\begin{displaymath}
\begin{array}{cccccccc}
100000&000110&110101&101011&011000&111110&001101&010011
\end{array}
\end{displaymath}
%
\newpage
%
Siguiendo el mismo razonamiento escogeremos como primer elemento de la fila 
dos un elemento de $\mathbb{F}^{^6}_2$ que no aparezca en las filas cero y uno. 
Como hay elementos de peso uno que no aparecen en dichas filas elegiremos como
primer elemento de la fila dos $010000$. Y siguiendo el mismo razonamiento
construiremos la fila y eligiremos los siguientes elementos:
\begin{eqnarray*}
Tb(\mathcal{C}[6,3])_{3,0}&=& 001000\\
Tb(\mathcal{C}[6,3])_{4,0}&=& 000100\\
Tb(\mathcal{C}[6,3])_{5,0}&=& 000010\\
Tb(\mathcal{C}[6,3])_{6,0}&=& 000001
\end{eqnarray*}
Para elegir el primer elemento de la fila siete observaremos que todas las
palabras de $\mathbb{F}^{^6}_2$ de peso uno estan en alguna de las filas
anteriores, con lo cual el elemento de la fila siete de menor peso tendr\'a un
peso mayor o igual que dos. El elemento $100001$ no esta en ninguna de las  
filas anteriores con lo cual lo elegiremos como primer elemento de la fila
siete.\\ \\ 
%
%
\begin{table}[!h]
\begin{displaymath}
\begin{array}{|c|c|c|c|c|c|c|c|}
\hline
000000&100110&010101&001011&111000&011110&101101&110011\\
\hline
100000&000110&110101&101011&011000&111110&001101&010011\\
\hline
010000&110110&000101&011011&101000&001110&111101&100011\\
\hline
001000&101110&011101&000011&110000&010110&100101&111011\\
\hline
000100&100010&010001&001111&111100&011010&101001&110111\\
\hline
000010&100100&010111&001001&111010&011100&101111&110001\\
\hline
000001&100111&010100&001010&111001&011111&101100&110010\\
\hline
100001&000111&110100&101010&011001&111111&001100&010010\\
\hline
\end{array}
\end{displaymath}
\caption{Tabla est\'andar del c\'odigo de triple control.}\label{tab:Tabla}
\end{table}
%
%
En la tabla $\ref{tab:Tabla}$ podemos ver una tabla est\'andar para el c\'odigo
de triple repetici\'on. No es la \'unica tabla est\'andar. Para obtener otras
tablas est\'andar bastar\'a con elegir distintos elementos como primer elemento
de cada fila obteniendo las mismas filas, pero en ordenadas de distinta forma.
%
\newpage
%
\subsubsection{Correcci\'on de errores}

Supongamos que recibimos la palabra $111100$, la cual no pertenece al c\'odigo.
Para corregir el error haremos:
\begin{itemize}
\item Localizamos el lugar de dicha palabra en la tabla.
\begin{displaymath}
Tb(\mathcal{C}[6,3])_{4,4}=111100
\end{displaymath}
\item Elegimos como palabra correcta la palabra que est\'e en la misma columna y
en la fila cero.
\begin{displaymath}
Tb(\mathcal{C}[6,3])_{0,4}=111000
\end{displaymath}
\end{itemize}
El error cometido vendr\'a dado por el primer elemento de la fila en la que se
encuentre la palabra recibida:
\begin{displaymath}
Tb(\mathcal{C}[6,3])_{4,0}=000100
\end{displaymath}
Luego en la transmisi\'on se ha cometido un error de peso $w(000100)=1$ en el
cuarto bit.

\subsubsection{Sindromes}

Supondremos que queremos corregir los mismos errores que se corrigen con la
tabla $\ref{tab:Tabla}$.\\ \\
%
Sea $H$ la matriz de control, en forma est\'andar, del c\'odigo:
\begin{displaymath}
H=\left( \begin{array}{cccccc}
1&1&0&1&0&0\\
1&0&1&0&1&0\\
0&1&1&0&0&1
\end{array} \right)
\end{displaymath}

\begin{table}[!h]
\begin{displaymath}
\begin{array}{|c|c|}
\hline
Error  & Sindrome \\
\hline
000000 & 000 \\
\hline
100000 & 110 \\
\hline
010000 & 101 \\
\hline
001000 & 011 \\
\hline
000100 & 100 \\
\hline
000010 & 010 \\
\hline
000001 & 001 \\
\hline
100001 & 111 \\
\hline
\end{array}
\end{displaymath}
\caption{Tabla de sindromes del c\'odigo de triple control.}\label{tab:TablaSindromes}
\end{table}
Obviamente requiere menor costo almacenar la tabla $\ref{tab:TablaSindromes}$
que almacenar la tabla $\ref{tab:Tabla}$.\\

Para corregir errores supongamos que recibimos la palabra $100011$, calculamos
su sindrome que ser\'a $101$. Entonces utilizando la tabla
$\ref{tab:TablaSindromes}$ buscamos que error tiene sindrome $101$, dicho error
es $010000$. Luego la palabra transmitida ser\'a $100011-010000=110011$.
Recordar que $-1=1$ en $\mathbb{F}_2$.


%
% EJERCICIOS
%

%
% EJERCICIOS
%

\section{Ejercicios}

\begin{ejercicio}
\ \\
Dada una palabra cualquiera del c\'odigo de triple control, comprobar que es lo
que falla s\'{\i} se produce un fallo en cualquiera de los seis bits.
\end{ejercicio}
\textbf{\underline{Soluci\'on:}}\\
Sea $110011$ una palabra perteneciente al c\'odigo de triple control.\\ \\
%
Todas las palabras del c\'odigo son de la forma:
\begin{displaymath}
\mathcal{C}[6,3]=\{\ (u_1,u_2,u_3,u_1+u_2,u_1+u_3,u_2+u_3)\ u_i \in \mathbb{F}_2
\ \}
\end{displaymath}
Llamaremos a los bit de control:
\begin{displaymath}
c_1 = u_1+u_2\quad c_2=u_1+u_3\quad c_3=u_2+u_3
\end{displaymath}
Introduciremos un error en cada bit y veremos que bits de control fallan,
suponiendo siempre que los bits de informaci\'on son correctos.\\ 
\begin{center}
\begin{tabular}{|c|c|c|c|}
\hline
Bit de error & Palabra erronea & Bits de control & Bits correctos\\
\hline
Primer bit & $010011$ & $011$ & $101$ \\
\hline
Segundo bit & $100011$ & $011$ & $110$ \\
\hline
Tercer bit & $111011$ & $011$ & $000$ \\
\hline
Cuarto bit & $110111$ & $111$ & $011$ \\
\hline
Quinto bit & $110001$ & $001$ & $011$ \\
\hline
Sexto bit & $110010$ & $010$ & $011$ \\
\hline
\end{tabular}
\end{center}
\begin{flushright}
$\blacksquare$
\end{flushright}

