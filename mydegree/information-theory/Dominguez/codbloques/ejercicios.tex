%
% EJERCICIOS DEL TEMA DE CODIGO DE BLOQUES
%

\section{Ejercicios}\label{sec:EjerciciosCodBloques}

%
% COMPROBAR QUE LA DISTANCIA DE HAMMING ES UNA DISTANCIA
%
\begin{ejercicio}
Comprobar que la distancia de Hamming es una distancia.
\end{ejercicio}
\underline{\textbf{Soluci\'on}}:\\
Sea $\mathcal{C}$ un c\'odigo, entonces la distancia de Hamming ser\'a una
aplicaci\'on de la forma:
\begin{displaymath}
d:\mathcal{C}\times \mathcal{C} \longrightarrow \mathbb{Z}^{^+}
\end{displaymath}
Recordemos que la distancia de Hamming, entre dos palabras, es el n\'umero de
bits en el que no coinciden.
\begin{itemize}
\item $d(x,y)\geq 0$ y $d(x,y)=0$ $\Longleftrightarrow$ $x=y$.\\

Como la distancia entre dos palabras es el n\'umero de bits en el que no
coinciden esta distancia siempre ser\'a mayor o igual que cero.\\

La distancia entre dos palabras ser\'a cero s\'{\i} y s\'olo s\'{\i} coinciden
en todos sus bits, luego ambas palabras ser\'an las mismas.
\item $d(x,y)=d(y,x)$.\\

El n\'umero de bits en el que no coincide $x$ con $y$ es el mismo que el de $y$
con $x$.
%
\newpage
%
\item Desigualdad triangular: $d(x,z)\leq d(x,y)+d(y,z)$.\\

Sean $x,y,z\in \mathcal{C}$ tres palabras cualesquiera del c\'odigo.\\ \\
%
Definamos los siguientes conjuntos:
\begin{displaymath}
U=\{i\ |\ x_i\neq z_i \}
\end{displaymath}
$U$ es el conjunto de indices en los que $x$ y $z$ no coinciden.
\begin{displaymath}
S=\{i\ |\ x_i\neq z_i\ y\ x_i=y_i \}
\end{displaymath}
$S$ es el conjunto de indices en los que $x$ y $z$ no coinciden y en los que
$x$ e $y$ coinciden.
\begin{displaymath}
T=\{i\ |\ x_i\neq z_y\ y\ x_i\neq y_i\}
\end{displaymath}
$T$ es el conjunto de indices en los que $x$ y $z$ no coinciden y en los que
$x$ e $y$ tampoco coinciden.\\ \\
%
Entenderemos por $\#U$, $\#S$ y $\#T$ al n\'umero de elementos de los conjuntos
$U$, $S$ y $T$ respectivamente.\\ \\
%
De las definiciones anteriores se deduce que:
\begin{equation}\label{eq:Primera}
\#U=d(x,z)=\#S+\#T
\end{equation}
Como $d(x,y)$ es el n\'umero de indices en los que no coinciden $x$ e $y$
tenemos entonces:
\begin{equation}\label{eq:Segunda}
d(x,y)\geq \#T
\end{equation}
De la definici\'on de $d(y,z)$ y de $S$ se tiene que:
\begin{equation}\label{eq:Tercera}
d(y,z)\geq \#S
\end{equation}
Si sumamos miembro a miembro $(\ref{eq:Segunda})$ y $(\ref{eq:Tercera})$ y
tenemos en cuenta $(\ref{eq:Primera})$ tenemos: $$d(x,z)\leq d(x,y)+d(y,z)$$
que es lo que queriamos demostrar.
\end{itemize}
\begin{flushright}
$\blacksquare$
\end{flushright}
%
\newpage
%

%
% CALCULO DE DISTANCIA MINIMA, ERRORES QUE DETECTA Y CORRIGE PARA EL
% CODIGO DEL BIT DE CONTROL DE PARIDAD
%
\begin{ejercicio}
Comprobar que en el c\'odigo del ``\emph{bit de control de paridad}'':
\begin{itemize}
\item $d_{min}=2$.
\item Detecta errores simples, $s'=1$.
\item No corrige ning\'un error, $t=0$.
\end{itemize}
\end{ejercicio}
\underline{\textbf{Soluci\'on}}:
\begin{itemize}
\item \textbf{Distancia m\'{\i}nima.}
\begin{itemize}
\item $d_{min}=0$ no puede darse nunca, por definici\'on de $d_{min}$.
\item $d_{min}=1$ este caso tampoco puede darse. Supongamos que tenemos una
palabra cualquiera del c\'odigo, $a=(a_1,\cdots,a_7,c_1)$, y que existiera 
otra palabra del c\'odigo, $b$, tal que $d(a,b) = 1$. Esto implicar\'{\i}a que
todos los bits de ambas palabras son iguales excepto uno. Obviamente el
\'ultimo bit no podr\'{\i}a ser ya que si a una palabra del c\'odigo le
cambiamos el \'ultimo bit la palabra resultante no pertenece al c\'odigo%
\footnote{No se cumple la condici\'on de tener una cantidad par de unos en
la palabra.}. Luego para que $d(a,b)=1$ tendr\'{\i}amos que alterar uno de
los siete primeros bits, y en ese caso la palabra resultante tampoco 
pertenecer\'{\i}a al c\'odigo\footnote{Ya que tendriamos que cambiar el
\'ultimo bit para cumplir la condici\'on de paridad}. Luego deberiamos cambiar
dos bits para que la palabra resultante perteneciera al c\'odigo entonces
$d(a,b)=2$. Luego en el c\'odigo no existen $a,b\in \mathcal{C}$ tales que
$d(a,b)=1$.
\item $d_{min}=2$ De lo dicho en el apartado anterior se deduce que en el 
codigo existen palabras $a,b\in \mathcal{C}$ tal que $d(a,b)=2$. Para ello
basta con coger una palabra cualquiera del c\'odigo, por ejemplo
$a=(0,0,1,1,0,0,1,1)$, alterar uno cualquiera de sus siete primeros bits y
poner como bit de control el correspondiente para cumplir la condici\'on de
paridad. Por ejemplo si cambiamos $a_2=0$ por $1$ el bit de control
pasar\'{\i}a de ser $1$ a ser $0$, con lo cual $b=(0,1,1,1,0,0,1,0)$
$\Longrightarrow$ $d(a,b)=2$, y como $a,b\in \mathcal{C}$ se tiene que 
$d_{min}=2$.
\end{itemize}
\item \textbf{Errores que detecta.}\\
Para ello utilizamos el teorema $\ref{the:Deteccion}$ de detecci\'on de errores.
Por este teorema para detectar errores de peso $s'$ se tiene que cumplir que
$d_{min}\geq s'+1$. Pero 
como tenemos que $d_{min} = 2$ entonces $s'=1$. Luego este c\'odigo detecta
errores simples.
\item \textbf{Errores que corrige.}\\
Para ello utilizamos el teorema $\ref{the:DeteccionCorreccion}$ de
detecci\'on y correcci\'on de
errores. Por este teorema para corregir errores de peso $t$ se tiene que cumplir
que $d_{min}\geq t +s'+1$. Pero como tenemos que $d_{min} = 2$ y $s'=1$ entonces
$t=0$. Luego este c\'odigo no corrige errores.
\end{itemize}
\begin{flushright}
$\blacksquare$
\end{flushright}
%
% CALCULO DE LA DISTANCIA MINIMA, ERRORES QUE DETECTA Y CORRIGE PARA EL
% CODIGO DE TRIPLE CONTROL 
%
\begin{ejercicio}\label{ejer:TripleControl}
Comprobar que en el c\'odigo de ``\emph{triple control}'':
\begin{itemize}
\item $d_{min}=3$.
\item Detecta errores dobles o simples, $s'=2$ o $s'=1$.
\item No corrige ning\'un error o corrige uno, $t=0$ o $t=1$.
\end{itemize}
\end{ejercicio}
\underline{\textbf{Soluci\'on}}:\\
Recordemos que:
\begin{displaymath}
\mathcal{C}=\{(x,y,z,a,b,c)\in \mathbb{F}^{^6}_2\quad donde\ a=x+y,\ b=x+z,\
c=y+z\}
\end{displaymath}
o lo que es lo mismo:
\begin{displaymath}
\mathcal{C}=\{\ (x,y,z,x+y,x+z,y+z)\ donde\ x,y,z\in \mathbb{F}_2\ \}
\end{displaymath}
\begin{itemize}
\item \textbf{Distancia m\'{\i}nima.}
\begin{itemize}
\item $d_{min}=0$ no puede darse nunca, por definici\'on de $d_{min}$.
\item $d_{min}=1$ tampoco puede darse debido a que si tenemos dos palabras
$a,b\in \mathcal{C}$ tales que $d(a,b) = 1$ entonces ambas palabras coincidirian
en todos sus bits menos en uno. El bit donde no coinciden no puede ser ninguno
de los tres \'ultimos\footnote{Bits de control.}, ya que si coinciden en los
tres primeros han de coincidir en los tres \'ultimos para que ambas palabras
pertenezcan al c\'odigo. Luego si se diferencian en un, bit forzosamente ha de
ser uno de los tres primeros. Si se diferecian en uno de los tres primeros bits,tambi\'en se han de diferenciar en dos de los tres \'ultimos\footnote{Para
satisfacer las condiciones de paridad de unos.} luego su distancia no ser\'{\i}auno, sino tres. Por lo tanto en el c\'odigo no existen palabras cuya distancia
entre s\'{\i} sea uno.
\item $d_{min}=2$ por el mismo motivo que en el caso anterior si dos palabras
del c\'odigo se diferencian en dos bits, estos bits han de ser dos de los tres
primeros. Si dos palabras del c\'odigo se diferencian en dos de sus tres
primeros bits, forzosamente se tienen que diferenciar en dos de los tres
\'ultimos y, problablemente, tambi\'en en el otro bit del final. Con lo
cual la distancia entre ambas palabras ser\'{\i}a, como poco, de cuatro. Luego
en el c\'odigo no puede haber palabras cuya distancia entre s\'{\i} sea dos.
\item $d_{min}=3$ cojamos una palabra cualquiera del c\'odigo, por ejemplo
$a=(0,0,1,0,1,1)$. Cambiemos uno de sus tres primeros bits y pongamos los
tres \'ultimos de tal forma que la palabra obtenida pertenezca al c\'odigo, por
ejemplo $b=(1,0,1,1,0,1)$. Entonces tenemos que $d(a,b)=3$. Luego la
$d_{min}=3$.
\end{itemize}
\item \textbf{Errores que detecta.} \\
Para ello utilizamos el teorema $\ref{the:Deteccion}$ de detecci\'on de errores.
Por este teorema para detectar errores de peso $s'$ se tiene que cumplir que
$d_{min}\geq s'+1$. Pero como tenemos que $d_{min}=3$ entonces $s'=2$ o $s'=1$.
Luego este c\'odigo detecta errores dobles o simples.
\item \textbf{Errores que corrige.} \\
Para ello utilizamos el teorema $\ref{the:DeteccionCorreccion}$ de
detecci\'on y correcci\'on de errores. Por este teorema para detectar errores
de peso $s'$ y corregir errores de peso $t$ se tiene que cumplir que
$d_{min}\geq t+s'+1$. Pero tenemos dos casos:
\begin{itemize}
\item $s'=1$, el c\'odigo detecta errores simples. En este caso tendremos lo
siguiente:
\begin{displaymath}
3=d_{min}\geq t+s'+1=t+1+1=t+2
\end{displaymath}
con lo que la soluci\'on es $t=1$. Luego si el c\'odigo
detecta un error lo corrige.
\item $s'=2$, el c\'odigo detecta errores dobles. En este caso tendremos lo
siguiente:
\begin{displaymath}
3=d_{min} \geq t+s'+1=t+2+1=t+3
\end{displaymath}
con lo que la \'unica soluci\'on es $t=0$. Luego si el c\'odigo detecta
dos errores no los corrige.
\end{itemize}
\end{itemize}
\begin{flushright}
$\blacksquare$
\end{flushright}

Seg\'un lo que hemos visto cuando utilizemos el c\'odigo de triple repetici\'on 
tenemos dos posibles opciones:
\begin{itemize}
\item Detectar cuando ocurren errores simples o dobles.\\ \\
En este caso detectaremos cuando ha ocurrido un error, sin saber si el error es
de tipo simple o doble, pero no podremos corregirlo.
\item Detectar y corregir errores simples.\\ \\
En este caso supondremos que cualquier error que se cometa es simple,
corrigiendolo. En el caso de suponer que los errores que se comenten son
simples,
y corregirlos, cuando se cometa un error doble estaremos suponiendo que el error
es simple y al corregirlo estaremos corrigiendo mal.
\end{itemize}

La forma de utilizar el c\'odigo depende de las probabilidades de error del
canal, si la probabilidad de errores dobles o simples son similares es
conveniente utilizar el c\'odigo para detectar errores. Sin embargo,
si predomina
una probabilidad sobre la otra utilizaremos el c\'odigo para corregir, 
suponiendo que todos los errores que se cometan ser\'an del tipo de error que
m\'as probabilidad tenga. Es claro que, cuando se cometa el error que menos
probabilidades tiene estaremos corrigiendo mal.
%
% CALCULO DE LA DISTANCIA MINIMA, ERRORES QUE DETECTA Y CORRIGE PARA EL
% CODIGO DE TRIPLE REPETICION
%
\begin{ejercicio}\label{ejer:TripleRepeticion}
Comprobar que en el c\'odigo de ``\emph{triple repetici\'on}'':
\begin{itemize}
\item $d_{min}=3$.
\item Detecta errores dobles o simples, $s'=2$ o $s'=1$.
\item No corrige ning\'un error o corrige uno, $t=0$ $t=1$.
\end{itemize}
\end{ejercicio}
\underline{\textbf{Soluci\'on}}:
\begin{itemize}
\item \textbf{Distancia m\'{\i}nima.}\\
Como el c\'odigo s\'olo tiene dos palabras la distancia m\'{\i}nima se calcula
de forma muy sencilla $d_{min} = d(000,111) = 3$.
\item \textbf{Errores que detecta.}\\
Para ello utilizamos el teorema $\ref{the:Deteccion}$ de detecci\'on de errores.
Por este teorema para detectar errores de peso $s'$ se tiene que cumplir que
$d_{min}\geq s'+1$. Pero como tenemos que $d_{min}=3$ entonces $s'=2$ o $s'=1$.
Luego este c\'odigo detecta errores dobles o simples.
\item \textbf{Errores que corrige.}\\
Para ello utilizamos el teorema $\ref{the:DeteccionCorreccion}$ de detecci\'on
y correcci\'on de errores. Por este teorema para detectar errores de peso $s'$
y corregir errores de peso $t$ se tiene que cumplir que $d_{min}\geq t+s'+1$.
Pero tenemos dos casos:
\begin{itemize}
\item $s'=1$, el c\'odigo detecta errores simples. En este caso tendremos lo
siguiente:
\begin{displaymath}
3=d_{min}\geq t+s'+1=t+1+1=t+2
\end{displaymath}
con lo que la soluci\'on es $t=1$. Luego si el c\'odigo detecta un error lo
corrige.
\item $s'=2$, el c\'odigo detecta errores dobles. En este caso tendremos lo
siguiente:
\begin{displaymath}
3=d_{min}\geq t+s'+1=t+2+1=t+3
\end{displaymath}
con lo que la \'unica soluci\'on es $t=0$. Luego si el c\'odigo detecta dos
errores no los corrige.
\end{itemize}
\end{itemize}
\begin{flushright}
$\blacksquare$
\end{flushright}

Seg\'un lo que hemos visto cuando utilizemos el c\'odigo de triple control
tenemos dos posibles opciones:
\begin{itemize}
\item Detectar cuando ocurren errores simples o dobles.\\ \\
Detectamos cuando ocurren errores simples o dobles, pero no podemos corregir.
\item Detectar y corregir errores simples.\\ \\
Detectamos errores simples y los corregimos.
\end{itemize}

La forma de utilizar el c\'odigo depende de las probabilidades de error que
se den en el canal que estemos utilizando.
%
\newpage
%

%
% PROPIEDAD QUE SE DEDUCE DE LA DESIGUALDAD TRIANGULAR
%
\begin{ejercicio}
Dado un c\'odigo $\mathcal{C}\subset \mathbb{F}^{^n}_q$ tal que
$d_{min} = 2\cdot t + 1$ comprobar que si dado
$z\in \mathbb{F}^{^n}_q$ tal que existe $c\in \mathcal{C}$ verificando que
$d(z,c)\leq t$ entonces la distancia de $z$ a cualquier otra palabra del
c\'odigo es estrictamente mayor que $t$.
\end{ejercicio}
\underline{\textbf{Soluci\'on}}:\\
Dado $z\in \mathbb{F}^{^n}_q$ supongamos que existe $c\in \mathcal{C}$ tal que
$d(z,c)\leq t$. Ahora bien como $d_{min} = 2\cdot t+1$ tendremos que
$d(c,c')\geq 2\cdot t+1$ $\forall \ c'\in \mathcal{C}$ con $c'\neq c$.
Por la desigualdad triangular tendremos:
\begin{displaymath}
2\cdot t+1 \leq d(c,c')\leq d(c,z)+d(z,c')
\end{displaymath}
para todo $z\in \mathbb{F}^{^n}_q$, en particular para nuestro $z$. Como por
hip\'otesis se tiene que $d(z,c)\leq t$:
\begin{displaymath}
2\cdot t +1 \leq t + d(z,c')
\end{displaymath}
o lo que es lo mismo:
\begin{displaymath}
t + 1 \leq d(z,c') \Longrightarrow d(z,c') > t\quad \forall \ c'\neq c \in \mathcal{C}
\end{displaymath}
\begin{flushright}
$\blacksquare$
\end{flushright}

Este ejercicio nos indica que si conocemos la distancia m\'{\i}nima, $d_{min}$,
de un c\'odigo $\mathcal{C}$ y calculamos $t$ de tal manera que verifique
$d_{min} = 2\cdot t +1$ entonces en el caso de recibir una transmisi\'on 
incorrecta, $z$, tal que $d(z,c)\leq t$ se tiene que cualquier otra palabra del
c\'odigo, $c'$, verifica que $d(z,c')> t$. Esto quiere decir que podemos
corregir el error cometido ya que s\'olo existe una palabra del c\'odigo
a distancia $t$ de $w$, entonces esa palabra del c\'odigo es la que se
transmiti\'o originalmente, luego $c$ es la palabra transmitida originalmente.
