%
% DETECTAR O CORREGIR
%

\section{?`Detectar o corregir?}

Si observamos los ejercicios de este cap\'{\i}tulo, en la p\'agina
$\pageref{sec:EjerciciosCodBloques}$, podemos ver que en los c\'odigos
de ``\emph{triple repetici\'on}'' y de ``\emph{triple control}'' tenemos una
de las dos siguientes opciones:
\begin{itemize}
\item Detectar y corregir un error.
\item Detectar dos errores y no corregir.
\end{itemize}
%
\newpage
%
\subsection{Probabilidad de error en el canal}

Para saber cual de las dos opciones utilizar necesitaremos conocer
``\emph{algo}'' m\'as sobre el canal de transmisi\'on. Ese ``\emph{algo}''
ser\'a la ``\emph{probabilidad de errores en el canal}''.
\begin{teorema}[Resultados sobre la probabilidad de error]
\ \\
Supongamos que tenemos un canal sim\'etrico, aleatorio, binario y con una
probabilidad de error $P$. En dicho canal transmitiremos palabras de longitud
$n$ mediante un c\'odigo de bloques.
\begin{enumerate}
\item La probabilidad de que ocurra un error concreto de peso $k$ ser\'a:
\begin{displaymath}
P^k\cdot (1-P)^{n-k}
\end{displaymath}
\item La probabilidad de que ocurra un error cualquiera de peso $k$ ser\'a:
\begin{displaymath}
{n \choose k}\cdot P^k\cdot (1-P)^{n-k}
\end{displaymath}
\end{enumerate}
\end{teorema}
\underline{\textbf{Demostraci\'on}}:
\begin{enumerate}
\item Si se ha cometido un error especifico de peso $k$ entonces tendremos que
$k$ bits son incorrectos y los $n-k$ restantes son correctos.\\

La probabildad de tener un bit en particular incorrecto es $P$, por lo tanto
la probabilidad de que un bit en particular sea correcto es $1-P$. Como el
canal es aleatorio todas estas probabilidades son independientes.\\

De todo esto y de que tenemos $k$ bits erroneos y $n-k$ correctos se tiene que
la probabilidad buscada es:
\begin{displaymath}
P^k\cdot (1-P)^{n-k}
\end{displaymath}
\item El n\'umero total de palabras con un error de peso $k$ es el mismo que
la cantidad de formas de elegir $k$ lugares de error, que es ${n\choose k}$. 
Como los errores son excluyentes, que s\'{\i} ocurre uno no puede
ocurrir otro, tenemos entonces que la probabilidad de que ocurra uno de ellos
es la suma de las probabilidades de cada error, es decir:
\begin{displaymath}
{n\choose k}\cdot P^k\cdot (1-P)^{n-k}
\end{displaymath}
\end{enumerate}
\begin{flushright}
$\blacksquare$
\end{flushright}
\begin{teorema}[Probabilidad de transmisiones correctas]
\ \\
Supondremos que utilizamos un c\'odigo de bloques para transmitir un mensaje
sobre un canal binario y sim\'etrico. Tenemos los siguientes resultados:
\begin{enumerate}
\item La probabilidad de que un procesador de error produzca la palabra correcta
es la suma de las probabilidades de error de los errores que el procesador de
error puede corregir.
\item Si los errores que el procesador de error puede corregir son
independientes de la palabra transmitida entonces la probabilidad de que el
mensaje se haya recibido correctamente es la suma de las probabilidades de que
se den los errores que el procesador corrige.
\end{enumerate}
\end{teorema}
\underline{\textbf{Demostraci\'on}}:
\begin{enumerate}
\item Los errores que el procesador corrige son excluyentes, ya que si se
produce un error no se puede producir otro. Entonces la probabilidad de que uno
ocurra es la suma de sus probabilidades individuales.
\item Como los errores que el procesador corrige son independientes de la
palabra entonces la probabilidad de transmisi\'on correcta es la suma de las
probabilidades individuales de los errores que el procesador de error corrige.
\end{enumerate}
\begin{flushright}
$\blacksquare$
\end{flushright}

\subsection{Ejemplos sobre probabilidad de error}

Supongamos que vamos a transmitir un mensaje de $10000$ bits por un canal con
una probabilidad de error de $\frac{1}{1000}$. Esto significa que hay una
probabildad de error en cada bit de $\frac{1}{1000}$, o lo que es lo mismo, que
hay una probabilidad de transmisi\'on correcta en un bit de
$1-\frac{1}{1000}=0.999$.\\

Sin utilizar ning\'un tipo de codificaci\'on la probabildad de transmisi\'on
correcta es $(0.999)^{10000}\simeq0.000045$.
%
\newpage
%
\subsubsection{C\'odigo del bit de control de paridad}

Este c\'odigo detecta hasta un error en cada palabra transmitida de longitud
ocho, pero no lo corrige.
\begin{itemize}
\item Probabilidad de que no haya errores en una palabra.\\

Como una palabra tiene $8$ bits la probabilidad de que no haya error en una
palabra ser\'a:
$$(0.999)^8\simeq 0.992028$$
\item Probabilidad de que haya un error en una palabra.\\

La probabilidad de que ocurra un error en una palabra es la probabilidad de
que en una palabra haya siete bits correctos y uno incorrecto:
$$(0.999)^7\cdot \frac{8}{1000}\simeq 0.007944$$
\item Probabilidad de transmisi\'on correcta.\\

Como queremos transmitir $1000$ bits con este c\'odigo los agruparemos en
bloques de siete m\'as el de control. Estas ser\'an las palabras, luego la
probabilidad de transmisi\'on correcta ser\'a la probabilidad de que no haya
ning\'un error en la transmisi\'on, ya que el c\'odigo no corrige errores:
$$(0.992028)^{\frac{10000}{7}}\simeq 0.000011$$
\item Probabilidad de que no ocurra un error indetectable.\\

La probabilidad de que no ocurra un error indetectable es la misma de que ocurra
un error detectable, es decir, de que no ocurra ning\'un error o de que ocurra
un error:
\begin{displaymath}
((0.992028)+(0.007944))^{\frac{10000}{7}}\simeq 0.960789
\end{displaymath}
\end{itemize}
%
\newpage
%
\subsubsection{C\'odigo de triple repetici\'on}

Podemos ver en el ejercicio $\ref{ejer:TripleRepeticion}$ que en este c\'odigo
se presentan dos situaciones:
\begin{enumerate}
\item Supondremos que es m\'as probable que se cometan dos errores que uno
en cada palabra transmitida. En este caso no se podr\'a corregir el error pero
se detectar\'an errores simples y dobles.
\begin{itemize}
\item Probabilidad de que no haya errores en una palabra.\\

Como una palabra tiene $3$ bits la probabilidad de que no haya error en una
palabra:
$$(0.999)^3\simeq 0.997003$$
\item Probabilidad de transmisi\'on correcta.\\

Como hemos de transmitir $10000$ bits y con este c\'odigo por cada bit de
informaci\'on hemos de transmitir $3$ bits, tendremos que vamos a transmitir un
total de $30000$ bits. Luego la probabilidad de transmisi\'on correcta ser\'a:
$$(0.999)^{30000}\simeq 9.218214^{-14}$$
\item Probabilidad de que no ocurra un error indetectable.\\

En el caso que estamos considerando el c\'odigo detecta que ha ocurrido un error
y supondr\'a que es un error doble, luego el \'unico caso en el que no
detectar\'a un error ser\'a cuando ocurra un error triple, es decir, que se
inviertan todos los bits de una palabra transmitida. Como la probabilidad de
error del canal es $\frac{1}{1000}$ entonces tenemos que la probabilidad de
que se haya un error triple ser\'a:
$$\Big(\frac{1}{1000}\Big)^3= 10^{-9}$$
Como vamos a transmitir $10000$ bloques de tres bits entonces la probabilidad
de que no ocurra un error triple, indetectable, ser\'a:
$$(1-10^{-9})^{10000}\simeq 0.99999$$
\end{itemize}
%
\newpage
%
Podemos ver que este c\'odigo tiene un probabilidad practicamente nula de que
ocurran errores indetectables, pero la probabilidad de que ocurran errores, 
detectables, en cada palabra es alt\'{\i}sima. Y teniendo en cuenta que el
c\'odigo no corrige errores necesitaremos una gran cantidad de retransmisiones,
que, junto con el hecho de que la raz\'on del c\'odigo es baja, $\frac{1}{3}$,
hace que transmitir de esta manera sea lento y poco recomendable.
%
%
\item Supondremos que es m\'as probable que se cometa un error que dos en 
cada palabra transmitida. En este caso podremos corregir el error.\\
\begin{itemize}
\item Probabilidad de que no haya errores en una palabra.\\

Como una palabra tiene $3$ bits la probabilidad de que no haya error en una
palabra:
$$(0.999)^3\simeq 0.997003$$
\item Probabilidad de que haya un error en una palabra.\\

La probabilidad de que ocurra un error en una palabra es la probabilidad de
que en una palabra haya dos bits correctos y uno incorrecto:
$$(0.999)^2\cdot \frac{3}{1000}\simeq 0.002994$$
\item Probabilidad de transmisi\'on correcta.\\

La probabilidad de transmisi\'on correcta es la probabilidad de que no ocurra
ning\'un error o de que ocurra uno\footnote{El c\'odigo corrige un error.} en
todas y cada una de las palabras transmitidas, que son $10000$, entonces:
$$((0.997003)+(0.002994))^{10000}\simeq 0.970445$$
\item Probabilidad de que no ocurra un error indetectable.\\

La probabilidad de que no ocurra un error indetectable es la misma de que ocurra
un error detectable, es decir, de que no ocurra ning\'un error o de que ocurra
un error:
\begin{displaymath}
((0.997003)+(0.002994))^{10000}\simeq 0.970445
\end{displaymath}
\end{itemize}
\end{enumerate}

\subsubsection{C\'odigo del triple control}

Supondremos que es m\'as probable que se cometa un error que dos en
cada palabra transmitida. En este caso podemos corregir el error.
\begin{itemize}
\item Probabilidad de que no haya errores en una palabra.\\

Como una palabra tiene $6$ bits la probabilidad de que no haya errores en una
palabra:
$$(0.999)^6\simeq 0.994015$$
\item Probabilidad de que haya un error en una palabra.\\

La probabilidad de que ocurra un error en una palabra es la probabilidad de que
haya cinco bits correctos y uno incorrecto:
$$(0.999)^5\cdot \frac{6}{1000}\simeq 0.005970$$
\item Probabilidad de transmisi\'on correcta.\\

La probabilidad de transmisi\'on correcta es la probabilidad de que no ocurra
ning\'un error o de que ocurra uno\footnote{El c\'odigo corrige un error.} en
todas y cada una de las palabras transmitidas, que son $\frac{10000}{3}$:
$$((0.994015)+(0.005970))^{\frac{10000}{3}}\simeq 0.951229$$
\item Probabilidad de que no ocurra un error indetectable.\\

La probabilidad de que no ocurra un error indetectable es la misma de que ocurra
un error detectable, es decir, de que no ocurra ning\'un error o de que ocurra
un error:
\begin{displaymath}
((0.994015)+(0.005970))^{\frac{10000}{3}}\simeq 0.951229
\end{displaymath}
\end{itemize}
