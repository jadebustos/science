%
% EJEMPLOS DE CODIGOS DE BLOQUES
%

\section{Ejemplos}

\subsection{C\'odigo del bit de control de paridad}

Este c\'odigo ser\'a un c\'odigo del tipo $(8,7)$, es decir:
\begin{itemize}
\item El c\'odigo utiliza palabras de $8$ bits de longitud.
\item El c\'odigo utiliza $7$ bits para transmitir informaci\'on.
\end{itemize}
\begin{displaymath}
\mathcal{C}[8,7]=\{(a_1,\cdots,a_7,a_8)\quad donde\ a_i\in \mathbb{F}_2 \}
\subset \mathbb{F}^{^8}_2
\end{displaymath}
donde $a_8$ es tal que el n\'umero de $1$ en la palabra es par, es decir:
\begin{displaymath}
a_8 = \sum_{i=1}^7 a_i
\end{displaymath}
donde la suma es la suma en $\mathbb{F}_2$, s\'{\i} la suma es par
entonces cero y en caso contrario uno.

\subsubsection{Codificador}

El codificador para este c\'odigo ser\'a el siguiente:
\begin{eqnarray*}
C:\mathbb{F}^{^7}_2 &\stackrel{\sim} \longrightarrow & \mathcal{C} \\
(a_1,\cdots,a_7)&\longrightarrow & (a_1,\cdots,a_7,c_1=\sum_{i=1}^7 a_i)
\end{eqnarray*}

\subsubsection{Decodificador}

El decodificador para este c\'odigo ser\'a el siguiente:
\begin{eqnarray*}
D:\mathcal{C}&\stackrel{\sim} \longrightarrow& \mathbb{F}^{^7}_2 \\
(a_1,\cdots,a_7,c_1)&\longrightarrow & (a_1,\cdots,a_7)
\end{eqnarray*}

\subsubsection{Distancia m\'{\i}nima del c\'odigo}

La distancia m\'{\i}nima para este c\'odigo es $2$.\\

Para calcular la distancia m\'{\i}nima de este c\'odigo tendremos que calcular
$2^7 \cdot (2^7 -1)=16.256$ distancias y quedarnos con la menor. 

\subsubsection{Raz\'on del c\'odigo}

Este c\'odigo utiliza $8$ bits para transmitir, de los cuales $7$ son de
informaci\'on su raz\'on es $\frac{7}{8}$.

\subsection{C\'odigo de triple repetici\'on} 

Este c\'odigo ser\'a un c\'odigo del tipo $(3,1)$, es decir:
\begin{itemize}
\item El c\'odigo utiliza palabras de $3$ bits de longitud.
\item El c\'odigo utiliza $1$ bit para transmitir informaci\'on.
\end{itemize}
\begin{displaymath}
\mathcal{C}[3,1] = \{000,111\}\subset \mathbb{F}^{^3}_2
\end{displaymath}

\subsubsection{Codificador}

El codificador para este c\'odigo ser\'a el siguiente:\\
\begin{eqnarray*}
C:\mathbb{F}_2&\stackrel{\sim}\longrightarrow & \mathcal{C} \\
0&\longrightarrow & 000 \\
1&\longrightarrow & 111
\end{eqnarray*}

\subsubsection{Decodificador}

El decodificador para este c\'odigo ser\'a el siguiente:
\begin{eqnarray*}
D:\mathcal{C}&\stackrel{\sim}\longrightarrow & \mathbb{F}_2\\
000&\longrightarrow & 0 \\
111&\longrightarrow & 1
\end{eqnarray*}

\subsubsection{Distancia m\'{\i}nima}

La distancia m\'{\i}nima para este c\'odigo es $3$.\\

Para calcular la distancia m\'{\i}nima de este c\'odigo tendremos que calcular
$2^1\cdot (2^1-1)= 1$ distancia y quedarnos con la menor.

\subsubsection{Raz\'on del c\'odigo}

Este c\'odigo utiliza $3$ bits para informaci\'on, de los cuales
$1$ es de informaci\'on su raz\'on es $\frac{1}{3}$.

\subsection{C\'odigo del triple control}

Este c\'odigo ser\'a un c\'odigo del tipo $(6,3)$, es decir:
\begin{itemize}
\item El c\'odigo utiliza palabras de $6$ bits de longitud.
\item El c\'odigo utiliza $3$ bits para transmitir informaci\'on.
\end{itemize}
\begin{displaymath}
\mathcal{C}[6,3]=\{(a_1,a_2,a_3,c_1,c_2,c_3)\quad donde\ a_i,c_i\in
\mathbb{F}_2\} \subset \mathbb{F}^{^6}_2
\end{displaymath}
donde $c_1$, $c_2$ y $c_3$ estan determinados por $a_1$, $a_2$ y $a_3$.

\subsubsection{Codificador}

El codificador para este c\'odigo ser\'a el siguiente:
\begin{eqnarray*}
C:\mathbb{F}^{^3}_2&\stackrel{\sim}\longrightarrow & \mathcal{C} \\
(a_1,a_2,a_3)&\longrightarrow & (a_1,a_2,a_3,c_1,c_2,c_3)
\end{eqnarray*}

\subsubsection{Decodificador}

El decodificador para este c\'odigo ser\'a el siguiente:
\begin{eqnarray*}
D:\mathcal{C}&\stackrel{\sim}\longrightarrow & \mathbb{F}^{^3}_2 \\
(a_1,a_2,a_3,c_1,c_2,c_3) &\longrightarrow & (a_1,a_2,a_3)
\end{eqnarray*}

\subsubsection{Distancia m\'{\i}nima}

La distancia m\'{\i}nima para este c\'odigo es de $3$.\\

Para calcular la distancia m\'{\i}nima de este c\'odigo tendremos que calcular
$2^3\cdot (2^3-1) = 56$ y quedarnos con la menor.

\subsubsection{Raz\'on del c\'odigo}

Este c\'odigo utiliza $6$ bits para transmitir, de los cuales $3$ son de 
informaci\'on su raz\'on es $\frac{3}{6}=\frac{1}{2}$.
