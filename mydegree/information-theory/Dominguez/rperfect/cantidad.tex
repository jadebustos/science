%
% CANTIDAD DE CODIGOS r-PERFETOS
%

\subsection{?`Cuantos c\'odigos $r$-perfectos existen?}

\begin{lema}\label{lem:Disjuntos}
\ \\
Un c\'odigo $r$-perfecto, $\mathcal{C}[n,m]$ es uni\'on disjunta de discos
cerrados centrados en palabras del c\'odigo y radio $r$. Para c\'odigos
bin\'arios.
\end{lema}
\underline{\textbf{Demostraci\'on}}:\\
Como $\mathcal{C}[n,m]$ es un c\'odigo $r$-perfecto se tiene que:
\begin{equation}\label{eq:Recubrimiento}
\mathbb{F}^{n}_2 = \bigcup_{u\in \mathcal{C}[n,m]} \overline{D}(u,r)
\end{equation}
donde las $\overline{D}(u,r)$ para $u\in \mathcal{C}[n,m]$ son disjuntas entre
s\'{\i}, es decir:
\begin{displaymath}
\overline{D}(u_i,r)\bigcap \overline{D}(u_j,r) = \emptyset\quad para\ i\neq j
\end{displaymath}
La ecuaci\'on $(\ref{eq:Recubrimiento})$ se deduce de la definici\'on de 
c\'odigo $r$-perfecto:
\begin{itemize}
\item Que $\mathbb{F}^{^n}_2$ es la uni\'on de discos cerrados se deduce del
hecho de que al ser $\mathcal{C}[n,m]$ un c\'odigo $r$-perfecto dada
$v\in\mathbb{F}^{^n}_2$, una palabra cualquiera, siempre existe una palabra
$u\in\mathcal{C}[n,m]$ tal que $d(u,v)\leq r$.
\item Y el hecho de que los discos sean disjuntos se deduce del hecho de que
las palabras $u\in \mathcal{C}[n,m]$ anteriores son unicas\footnote{Por
definici\'on de c\'odigo $r$-perfecto.}.
\end{itemize}
\begin{flushright}
$\blacksquare$
\end{flushright}

En $\mathbb{F}^{^n}_2$ tenemos $2^n$ palabras y en $\mathcal{C}[n,m]$ tenemos
$2^m$ palabras. De esto y de los lemas $\ref{lem:Cantidad}$ y
$\ref{lem:Disjuntos}$ se deduce que un c\'odigo $r$-perfecto verifica:
\begin{displaymath}
2^n=2^m\cdot\left({n\choose 0}+{n\choose 1}+\cdots+{n\choose r} \right)
\end{displaymath}
Luego tendremos tantos c\'odigos $r$-perfectos como $n$ y $m$ verifiquen la
condici\'on anterior.
%
\newpage
%
\begin{proposicion}
\ \\
Si existe un c\'odigo $\mathcal{C}[n,m]$ $r$-perfecto entonces no existe
ning\'un c\'odigo $\mathcal{C}[n,m']$ con $m'>m$ y que tenga distancia 
m\'{\i}nima mayor que $2\cdot r$.
\end{proposicion}
\underline{\textbf{Demostraci\'on}}:\\
Sea $\mathcal{C}[n,m]$ un c\'odigo $r$-perfecto. Entonces por los lemas
$\ref{lem:Cantidad}$ y $\ref{lem:Disjuntos}$ tendremos que:
\begin{equation}\label{eq:Primer}
2^n=2^m\cdot |\overline{D}_r\ |
\end{equation}
Sea $\mathcal{C}'[n,m']$ con $m'>m$ tal que $d_{min}>2\cdot r$ es decir,
$d_{min}\geq2\cdot r+1$ entonces el c\'odigo corrige errores de peso menor o
igual que $r$, luego $\mathcal{C}'[n,m]$ es $r$-perfecto. Entonces:
\begin{displaymath}
2^n=2^{m'}\cdot |\overline{D}_r\ |
\end{displaymath}
Teniendo en cuenta esta \'ultima ecuaci\'on y $(\ref{eq:Primer})$ llegamos a
la conclusi\'on de que $m'=m$.
\begin{flushright}
$\blacksquare$
\end{flushright}
Luego para una distancia m\'{\i}nima fijada, $2\cdot r+1$, los c\'odigos
$r$-perfectos tienen los bits de informaci\'on, $m$, m\'aximos. Para c\'odigos
con igual longitud de palabra.
