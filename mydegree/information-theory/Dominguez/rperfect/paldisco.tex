% 
% NUMERO DE PALABRAS EN UN DISCO CERRADO
%

\subsection{N\'umero de palabras en un disco cerrado}

\begin{lema}[N\'umero de palabras en un disco cerrado]\label{lem:Cantidad}
\ \\
Dado un disco cerrado $\overline{D}(u,r)$ donde $u\in\mathbb{F}^{^n}_2$ y
$r\in \mathbb{Z}$ se tiene que el n\'umero de palabras que hay en dicho
disco, $|\overline{D}(u,r)\ |$, es:
\begin{displaymath}
|\overline{D}(u,r)\ | = {n \choose 0}+{n \choose 1}+\cdots+{n\choose r}
\end{displaymath}
\end{lema}
\underline{\textbf{Desmostraci\'on}}:\\
El n\'umero de palabras en $\overline{D}(u,r)$ es el n\'umero de palabras que
podemos construir variando hasta $r$ bits.\\

Dada $u$ hay ${n \choose i}$ palabras $v$ tales que $d(u,v) = i$, luego por
la definici\'on de disco cerrado y distancia tendremos que:
\begin{displaymath}
|\overline{D}(u,r)\ | = {n\choose 0}+{n\choose 1}+\cdots+{n\choose r}
\end{displaymath}
\begin{flushright}
$\blacksquare$
\end{flushright}
\begin{observacion}\ \\
\begin{itemize}
\item De este lema se deduce que el n\'umero de palabras que hay en un disco
cerrado\footnote{Con la distancia de Hamming.} \'unicamente depende de su
radio $r$ y nunca de su centro. Luego utilaremos la siguiente notaci\'on:
\begin{displaymath}
|\overline{D}(u,r)\ | = |\overline{D}_r\ |
\end{displaymath}
\end{itemize}
\end{observacion}
