%
% PRELIMINARES
%

\section{Preliminares}

Consideremos el c\'odigo $Ham(3)$, si rotamos c\'{\i}clicamente sus palabras:
\begin{displaymath}
a_1a_2a_3a_4a_5a_6a_7\rightarrow a_7a_1a_2a_3a_4a_5a_6
\end{displaymath}
obtenemos otro c\'odigo lineal, $K$ con las mismas propiedades que $Ham(3)$:
\begin{itemize}
\item $n=7$
\item $m=4$
\item $d_{min}=3$ 
\end{itemize}
y verificando que las \'unicas palabras que tienen en com\'un ambos c\'odigos
son el $0$ y el $1$.
%
\newpage
%
Adem\'as podemos a\~nadir en los dos c\'odigos un octavo bit de ``control de
paridad'' y hacer que mantengan las propiedades. Luego tendremos unos nuevos
c\'odigos $Ham(3)'$ y $K'$ de modo que:
\begin{itemize}
\item $n=8$
\item $m=4$
\item $d_{min}=3$
\end{itemize}

Estos nuevos c\'odigos lineales que hemos construidos nos van a servir para
construir otros c\'odigos lineales entendiendolos como subespacios. Esto lo
podemos hacer aprovechando la estructura de espacios vectoriales que tienen
los c\'odigos $Ham(3)'$ y $K'$, la cual nos permite realizar operaciones con
dichos espacios:
\begin{itemize}
\item Sumas.
\item Productos directos.
\item Intersecciones.
\item Uniones.
\end{itemize}

Las propiedades que tienen estos c\'odigos se obtienen de las propiedades de las
operaciones con subespacios:
\begin{eqnarray*}
\dim (Ham(3)'\times K')&=&\dim Ham(3)'+\dim K'\\
\dim (Ham(3)'+K')&=&\dim Ham(3)'+\dim K' -\dim (Ham(3)'\bigcap K')
\end{eqnarray*}
