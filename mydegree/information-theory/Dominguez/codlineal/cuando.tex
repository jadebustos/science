%
% CUANDO UN CODIGO ES LINEAL
%

%
\newpage
%
\section{?`Cuando un c\'odigo es lineal?}

Podemos distinguir unos c\'odigos de bloques especiales que son los c\'odigos
lineales.
\begin{definicion}[C\'odigos Lineales]
\ \\
Dado un cuerpo cualquiera, $\mathbb{K}$, y un $\mathbb{K}$-espacio vectorial de
la forma $\mathbb{K}^{^n}$ diremos que un subconjunto $\mathcal{C}\subset
\mathbb{K}^{^n}$ es un \textbf{``c\'odigo lineal''} si $\mathcal{C}$ es un
subespacio vectorial de $\mathbb{K}^{^n}$.
\end{definicion}
Nosotros consideraremos $\mathbb{K}=\mathbb{F}_q$ y
$\mathbb{K}^{^n}=\mathbb{F}^{^n}_q$.

\subsection{Propiedades de los c\'odigos lineales}

Dado que los c\'odigos lineales son subespacios vectoriales poseen unas
propiedades especiales.\\ \\ 
%
Sea $\mathcal{C}\subset \mathbb{F}^{^n}_2$ un c\'odigo lineal, entonces se
verifican las siguientes propiedades:
\begin{itemize}
\item La suma de dos palabras del c\'odigo es otra palabra del c\'odigo.
\begin{displaymath}
(x_1,\cdots,x_n)+(x_1',\cdots,x_n') = (x_1+x_1',\cdots,x_n+x_n')\in \mathcal{C}
\end{displaymath}
\item Multiplicar una palabra del c\'odigo por un elemento del cuerpo es otra
palabra del c\'odigo.
\begin{displaymath}
x\odot (x_1,\cdots,x_n) = (x\cdot x_1,\cdots, x\cdot x_n)\in \mathcal{C}
\end{displaymath}
\item El $0$ siempre es una palabra del c\'odigo.
\begin{displaymath}
(0,\stackrel{n)}\cdots,0)\in \mathcal{C}
\end{displaymath}
\end{itemize}
