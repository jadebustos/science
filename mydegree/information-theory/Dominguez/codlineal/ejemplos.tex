%
% EJEMPLOS
%

\section{Ejemplos}
%
% BIT DE CONTROL DE PARIDAD
%
\subsection{C\'odigo del bit de control de paridad}

Es gracias a la estructura algebraica que tiene $(\mathbb{F}_2,+,\cdot )$, 
cuerpo, podemos expresar este c\'odigo como:
\begin{displaymath}
\mathcal{C}=\{\ (x_1,\cdots,x_7,\sum_{i=1}^7x_i)\quad \{x_i\}_{i=1}^7\in
\mathbb{F}_2\ \}
\end{displaymath}

\subsubsection{El c\'odigo del bit de control de paridad es lineal}

Para ser un c\'odigo lineal $\mathcal{C}$ ha de ser un subespacio vectorial
de $\mathbb{F}^{^8}_2$.\\ \\
%
Para ver que es un subespacio vectorial veamos que:
\begin{itemize}
\item La suma de dos palabras del c\'odigo est\'a en el c\'odigo.\\

Consideremos dos palabras cualesquiera del c\'odigo:
$$(x_1,\cdots,x_7,\sum_{i=1}^7 x_i )$$ con $\{x_i\}_{i=1}^7\in \mathbb{F}_2$, 
y $$(x_1',\cdots,x_7',\sum_{i=1}^7 x_i' )$$ con $\{x_i'\}_{i=1}^7\in
\mathbb{F}_2$:
\begin{eqnarray*}
(x_1,\cdots,x_7,\sum_{i=1}^7 x_i)+(x_1',\cdots,x_7',\sum_{i=1}^7 x_i')=\\
= (x_1+x_1',\cdots,x_7+x_7',\sum_{i=1}^7 (x_i+x_i')\ )\in \mathcal{C}
\end{eqnarray*}
%
\newpage
%
\item El producto de un escalar por una palabra del c\'odigo es otra palabra 
del c\'odigo.\\ 

Consideremos una palabra cualquiera del c\'odigo:
$$(x_1,\cdots,x_7,\sum_{i=1}^7 x_i)$$
con $\{x_i\}_{i=1}^7\in \mathbb{F}_2$, y un elemento cualquiera del cuerpo,
$x\in \mathbb{F}_2$:
\begin{displaymath}
x\odot (x_1,\cdots,x_7,\sum_{i=1}^7 x_i) = (x\cdot x_1,\cdots, x\cdot x_7,
\sum_{i=1}^7 (x\cdot x_i)\ )\in \mathcal{C}
\end{displaymath}
\end{itemize}
Luego $\mathcal{C}$ es un subespacio vectorial de $\mathbb{F}^{^8}_2$. Es
un c\'odigo lineal.\\ \\
%
Una base de este subespacio vectorial es:
\begin{eqnarray*}
e_1&=&(1,0,0,0,0,0,0,1)\\
e_2&=&(0,1,0,0,0,0,0,1)\\
e_3&=&(0,0,1,0,0,0,0,1)\\
e_4&=&(0,0,0,1,0,0,0,1)\\
e_5&=&(0,0,0,0,1,0,0,1)\\
e_6&=&(0,0,0,0,0,1,0,1)\\
e_7&=&(0,0,0,0,0,0,1,1)
\end{eqnarray*}
Luego $\mathcal{C}$ es un subespacio vectorial tal que
$\dim_{\mathbb{F}_2} \mathcal{C}=7$.

\subsubsection{Codificador}

El codificador de este c\'odigo ser\'a una aplicaci\'on lineal:
\begin{eqnarray*}
C:\mathbb{F}^{^7}_2&\stackrel{\sim}\longrightarrow&\mathcal{C}\subset
\mathbb{F}^{^8}_2\\
x&\longrightarrow&(x_1,\dots,x_7,\sum_{i=1}^7x_i)
\end{eqnarray*}
Las columnas de la matriz que define esta aplicaci\'on lineal, codificador,
ser\'an $C(e_i)$, donde $i=1,\dots,7$ y los $e_i$ son vectores de una 
base del subespacio $\mathcal{C}$.\\

\begin{figure}[!h]
\begin{displaymath}
C\equiv \left( \begin{array}{ccccccc}
1&0&0&0&0&0&0\\
0&1&0&0&0&0&0\\
0&0&1&0&0&0&0\\
0&0&0&1&0&0&0\\
0&0&0&0&1&0&0\\
0&0&0&0&0&1&0\\
0&0&0&0&0&0&1\\
1&1&1&1&1&1&1
\end{array} \right)
\end{displaymath}
\caption{Matriz generadora, en forma est\'andar, del c\'odigo del bit de
control de paridad.}
\end{figure}
%
Para codificar una palabra:
\begin{displaymath}
\left( \begin{array}{ccccccc}
1&0&0&0&0&0&0\\
0&1&0&0&0&0&0\\
0&0&1&0&0&0&0\\
0&0&0&1&0&0&0\\
0&0&0&0&1&0&0\\
0&0&0&0&0&1&0\\
1&1&1&1&1&1&1
\end{array} \right) \cdot
\left( \begin{array}{c}
x_1\\
x_2\\
x_3\\
x_4\\
x_5\\
x_6\\
x_7
\end{array} \right) =
\left( \begin{array}{c}
x_1\\
x_2\\
x_3\\
x_4\\
x_5\\
x_6\\
x_7\\
x_1+\dots+x_7
\end{array} \right)
\end{displaymath}

La matriz generadora de este c\'odigo se puede expresar en otras bases como:
\begin{displaymath}
\left( \begin{array}{ccccccc}
1&0&0&0&0&0&0\\
1&1&0&0&0&0&0\\
0&1&1&0&0&0&0\\
0&0&1&1&0&0&0\\
0&0&0&1&1&0&0\\
0&0&0&0&1&1&0\\
0&0&0&0&0&1&1\\
0&0&0&0&0&0&1
\end{array} \right)
\end{displaymath}
la cual no est\'a en forma est\'andar.

\subsubsection{Ecuaciones param\'etricas}

Para calcular las ecuaciones param\'etricas utilizaremos la matriz generadora
en forma est\'andar.
\begin{displaymath}
\left( \begin{array}{ccccccc}
1&0&0&0&0&0&0\\
0&1&0&0&0&0&0\\
0&0&1&0&0&0&0\\
0&0&0&1&0&0&0\\
0&0&0&0&1&0&0\\
0&0&0&0&0&1&0\\
1&1&1&1&1&1&1
\end{array} \right) \cdot
\left( \begin{array}{c}
\lambda_1\\
\lambda_2\\
\lambda_3\\
\lambda_4\\
\lambda_5\\
\lambda_6\\
\lambda_7
\end{array} \right) =
\left( \begin{array}{c}
\lambda_1\\
\lambda_2\\
\lambda_3\\
\lambda_4\\
\lambda_5\\
\lambda_6\\
\lambda_7\\
\lambda_1+\dots+\lambda_7
\end{array} \right)
\end{displaymath}
Las ecuaciones param\'etricas son:
\begin{figure}[!h]
\begin{displaymath}
\left\{ \begin{array}{ccl}
x_1&=&\lambda_1\\
x_2&=&\lambda_2\\
x_3&=&\lambda_3\\
x_4&=&\lambda_4\\
x_5&=&\lambda_5\\
x_6&=&\lambda_6\\
x_7&=&\lambda_7\\
x_8&=&\lambda_1+\lambda_2+\lambda_3+\lambda_4+\lambda_5+\lambda_6+\lambda_7
\end{array} \right.
\end{displaymath}
\caption{Ecuaciones param\'etricas del c\'odigo del bit de control de 
paridad.}
\end{figure}

\subsubsection{Ecuaciones impl\'{\i}citas}

Como $\dim_{\mathbb{F}_2} \mathbb{F}^{^8}_2 = 8$ y
$\dim_{\mathbb{F}_2} \mathcal{C} = 7$ tendremos $8-7=1$ ecuaci\'on
impl\'{\i}cita.\\ \\
%
Planteemos el sistema de ecuaciones:
\begin{displaymath}
\left( \begin{array}{cccccccc}
1&0&0&0&0&0&0&1 \\
0&1&0&0&0&0&0&1 \\
0&0&1&0&0&0&0&1 \\
0&0&0&1&0&0&0&1 \\
0&0&0&0&1&0&0&1 \\
0&0&0&0&0&1&0&1 \\
0&0&0&0&0&0&1&1 \\
\end{array} \right) \cdot
\left( \begin{array}{c}
x_1 \\
x_2 \\
x_3 \\
x_4 \\
x_5 \\
x_6 \\
x_7 \\
x_8
\end{array} \right) =
\left( \begin{array}{c}
0 \\
0 \\
0 \\
0 \\
0 \\
0 \\
0 
\end{array} \right)
\end{displaymath}
De donde nos sale que
$\mathcal{C}^{^\circ} = <(1,1,1,1,1,1,1,1)>$.
Luego la ecuaci\'on impl\'{\i}cita del c\'odigo ser\'a:
\begin{figure}[!h]
$$x_1+x_2+x_3+x_4+x_5+x_6+x_7+x_8 = 0$$
\caption{Ecuaci\'on impl\'{\i}cita del c\'odigo del bit de control de 
paridad.}
\end{figure}

\subsubsection{Matriz de control}

La matriz de control utilizando las ecuaciones impl\'{\i}citas de este c\'odigo
es:
\begin{figure}[!h]
\begin{displaymath}
\left( \begin{array}{cccccccc}
1&1&1&1&1&1&1&1
\end{array} \right)
\end{displaymath}
\caption{Matriz de control, en forma est\'andar, para el c\'odigo del
bit de control de paridad.}
\end{figure}

%
% CODIGO DE TRIPLE REPETICION   
%
\subsection{C\'odigo de triple repetici\'on}

Es gracias a la estructura algebraica que tiene $(\mathbb{F}_2,+,\cdot )$,
cuerpo, podemos expresar este c\'odigo como:
\begin{displaymath}
\mathcal{C}=\{\ (x,x,x)\quad x\in \mathbb{F}_2\ \}
\end{displaymath}

\subsubsection{El c\'odigo de triple repetici\'on es lineal}

Para ser un c\'odigo lineal $\mathcal{C}$ ha de ser un subespacio vectorial de
$\mathbb{F}^{^3}_2$.\\ \\
%
Para ver que es un subespacio vectorial veamos que:
\begin{itemize}
\item La suma de dos palabras del c\'odigo est\'a en el c\'odigo.\\

Como el c\'odigo s\'olo tiene dos palabras veamos que su suma es otra palabra
del c\'odigo.
\begin{displaymath}
(0,0,0)+(1,1,1)=(1,1,1)\in \mathcal{C}
\end{displaymath}
%
\newpage
%
\item El producto de un escalar por una palabra del c\'odigo es otra palabra
del c\'odigo.\\

Como el c\'odigo unicamente tiene dos palabras veamoslo para cada una de
ellas:\\ \\
%
Para $(0,0,0)$:
\begin{displaymath}
x\odot (0,0,0) =
\left\{ \begin{array}{ll}
(0,0,0)\in \mathcal{C} & para\ x = 0 \\
 & \\
(0,0,0)\in \mathcal{C} & para\ x = 1 
\end{array} \right.
\end{displaymath}
Para $(1,1,1)$:
\begin{displaymath}
x\odot (1,1,1) =
\left\{ \begin{array}{ll}
(0,0,0)\in \mathcal{C} & para\ x = 0\\
 & \\
(1,1,1)\in \mathcal{C} & para\ x = 1
\end{array} \right.
\end{displaymath}
\end{itemize}
Luego $\mathcal{C}$ es un subespacio vectorial de $\mathbb{F}^{^3}_2$. Es un
c\'odigo lineal.\\ \\
%
Una base de este subespacio vectorial es:
\begin{eqnarray*}
e_1&=&(1,1,1)
\end{eqnarray*}
Luego $\mathcal{C}$ es un subespacio vectorial tal que
$\dim_{\mathbb{F}_2}\mathcal{C} = 1$.

\subsubsection{Codificador}

El codificador de este c\'odigo ser\'a una aplicaci\'on lineal:
\begin{eqnarray*}
C:\mathbb{F}_2&\stackrel{\sim}\longrightarrow&\mathcal{C}\subset \mathbb{
F}^{^3}_2\\
x&\longrightarrow&(x,x,x)
\end{eqnarray*}
La columna de la matriz que define esta aplicaci\'on lineal, codificador,
ser\'a
$C(e_1)$, donde $e_1$ es un vector de la base del subespacio $\mathcal{C}$.\\

\begin{figure}[!h]
\begin{displaymath}
C\equiv \left( \begin{array}{c}
1\\
1\\
1
\end{array} \right)
\end{displaymath}
\caption{Matriz generadora, en forma est\'andar, del c\'odigo de triple
repetici\'on.}
\end{figure}
%
Para codificar cualquiera de las dos palabras:
\begin{displaymath}
\left( \begin{array}{c}
1\\
1\\
1
\end{array} \right) \cdot
\left( \begin{array}{c}
0
\end{array} \right) =
\left( \begin{array}{c}
0\\
0\\
0\\
\end{array} \right) \ o \
\left( \begin{array}{c}
1\\
1\\
1
\end{array} \right) \cdot
\left( \begin{array}{c}
1
\end{array} \right) =
\left( \begin{array}{c}
1\\
1\\
1
\end{array} \right)
\end{displaymath}

\subsubsection{Ecuaciones param\'etricas}

Para calcular las ecuaciones param\'etricas utilizaremos la matriz generadora
en forma est\'andar.
\begin{displaymath}
\left( \begin{array}{c}
1\\
1\\
1
\end{array} \right) \cdot
\left( \begin{array}{c}
\lambda_1
\end{array} \right) =
\left( \begin{array}{c}
\lambda_1\\
\lambda_1\\
\lambda_1
\end{array} \right)
\end{displaymath}
Las ecuaciones param\'etricas son:
\begin{figure}[!h]
\begin{displaymath}
\left\{ \begin{array}{ccl}
x_1&=&\lambda_1\\
x_2&=&\lambda_1\\
x_3&=&\lambda_1
\end{array} \right.
\end{displaymath}
\caption{Ecuaciones param\'etricas del c\'odigo de triple repetici\'on.}
\end{figure}
\subsubsection{Ecuaciones impl\'{\i}citas}

Como $\dim_{\mathbb{F}_2} \mathbb{F}^{^3}_2 = 3$ y
$\dim_{\mathbb{F}_2} \mathcal{C} = 1$ tendremos $3-1=2$ ecuaciones
impl\'{\i}citas.\\ \\
%
Planteemos el sistema de ecuaciones:
\begin{displaymath}
\left( \begin{array}{ccc}
1&1&1
\end{array} \right) \cdot
\left( \begin{array}{c}
x_1 \\
x_2 \\
x_3
\end{array} \right) =
\left( \begin{array}{c}
0
\end{array} \right)
\end{displaymath}
De donde nos sale que $\mathcal{C}^{^\circ} = <(1,1,0),(1,0,1)>$. Luego las
ecuaciones impl\'{\i}citas del c\'odigo ser\'an:
\begin{figure}[!h]
\begin{eqnarray*}
x+y&=&0\\
x+z&=&0
\end{eqnarray*}
\caption{Ecuaciones impl\'{\i}citas del c\'odigo de triple repetici\'on.}
\end{figure}
%
\newpage
%
\subsubsection{Matriz de control}

La matriz de control utilizando las ecuaciones impl\'{\i}citas de este c\'odigo
es:
\begin{figure}[!h]
\begin{displaymath}
\left( \begin{array}{ccc}
1&1&0\\
1&0&1
\end{array} \right)
\end{displaymath}
\caption{Matriz de control, en forma est\'andar, del c\'odigo de triple
repetici\'on.}
\end{figure}

%
% CODIGO DE TRIPLE CONTROL 
%
\subsection{C\'odigo de triple control}

Es gracias a la estructura algebraica que tiene $(\mathbb{F}_2,+,\cdot )$,
cuerpo, podemos expresar este c\'odigo como:
\begin{displaymath}
\mathcal{C}=\{\ (x,y,z,x+y,x+z,y+z)\quad x,y,z\in \mathbb{F}_2\ \}
\end{displaymath}

\subsubsection{El c\'odigo de triple control es lineal}

Para ser un c\'odigo lineal $\mathcal{C}$ ha de ser un subespacio vectorial de
$\mathbb{F}^{^6}_2$.\\ \\
%
Para ver que es un subespacio vectorial veamos que:
\begin{itemize}
\item La suma de dos palabras del c\'odigo est\'a en el c\'odigo.\\

Consideremos dos palabras cualesquiera del c\'odigo:
\begin{displaymath}
(x,y,z,x+y,x+z,y+z)
\end{displaymath}
con $x,y,z\in \mathbb{F}_2$, y
\begin{displaymath}
(x',y',z',x'+y',x'+z',y'+z')
\end{displaymath}
con $x',y',z'\in \mathbb{F}_2$:
\begin{eqnarray*}
(x,y,z,x+y,x+z,y+z)+(x',y',z',x'+y',x'+z',y'+z') =& &\\
=(x+x',y+y',z+z',x+x'+y+y',x+x'+z+z',y+y'+z+z')\in \mathcal{C}
\end{eqnarray*}
\item El producto de un escalar por una palabra del c\'odigo es otra palabra
del c\'odigo.\\

Consideremos una palabra cualquiera del c\'odigo:
\begin{displaymath}
(x,y,z,x+y,x+z,y+z)
\end{displaymath}
con $x,y,z\in \mathbb{F}_2$, y un elemento cualquiera del cuerpo,
$w\in \mathbb{F}_2$:
\begin{displaymath}
w\odot (x,y,z,x+y,x+z,y+z) = (w\cdot x,w\cdot y,w\cdot z, w\cdot x+w\cdot y,
w\cdot x+w\cdot z,w\cdot y+w\cdot z) \in \mathcal{C}
\end{displaymath}
\end{itemize}
Luego $\mathcal{C}$ es un subespacio vectorial de $\mathbb{F}^{^6}_2$. Es un
c\'odigo lineal.\\ \\
%
Una base de este espacio vectorial es:
\begin{eqnarray*}
e_1&=&(1,0,0,1,1,0)\\
e_2&=&(0,1,0,1,0,1)\\
e_3&=&(0,0,1,0,1,1)\\
\end{eqnarray*}
Luego $\mathcal{C}$ es un subespacio vectorial tal que
$\dim_{\mathbb{F}_2}\mathcal{C}=3$.

\subsubsection{Codificador}

El codificador de este c\'odigo ser\'a una aplicaci\'on lineal:
\begin{eqnarray*}
C:\mathbb{F}^{^3}_2&\stackrel{\sim}\longrightarrow& \mathcal{C}\subset
\mathbb{F}^{^6}_2 \\
(x,y,z)&\longrightarrow&(x,y,z,x+y,x+z,y+z)
\end{eqnarray*}
Las columnas de la matriz que define esta aplicaci\'on lineal, codificador,
ser\'an $C(e_i)$, donde $i=1,2,3$ y los $e_i$ son vectores de una
base del subespacio $\mathcal{C}$.\\

\begin{figure}[!h]
\begin{displaymath}
C\equiv \left( \begin{array}{ccc}
1&0&0\\
0&1&0\\
0&0&1\\ 
1&1&0\\
1&0&1\\
0&1&1
\end{array} \right)
\end{displaymath}
\caption{Matriz generadora, en forma est\'andar, del c\'odigo de triple 
control.}
\end{figure}
%
Para codificar una palabra:
\begin{displaymath}
\left( \begin{array}{ccc}
1&0&0\\
0&1&0\\
0&0&1\\
1&1&0\\
1&0&1\\
0&1&1
\end{array} \right)  \cdot
\left( \begin{array}{c}
x_1\\
x_2\\
x_3
\end{array} \right) =
\left( \begin{array}{c}
x_1\\
x_2\\
x_3\\
x_1+x_2\\
x_1+x_3\\
x_2+x_3
\end{array} \right)
\end{displaymath}

La matriz generadora de este c\'odigo se puede expresar en otras bases como:
\begin{displaymath}
\left( \begin{array}{ccc}
1&1&0\\
1&1&1\\
0&1&1\\
0&0&1\\
1&0&1\\
1&0&0
\end{array} \right)
\end{displaymath}
la cual no est\'a en forma est\'andar.

\subsubsection{Ecuaciones param\'etricas}

Para calcular las ecuaciones param\'etricas utilizaremos la matriz generadora
en forma est\'andar.
\begin{displaymath}
\left( \begin{array}{ccc}
1&0&0\\
0&1&0\\
0&0&1\\
1&1&0\\
1&0&1\\
0&1&1
\end{array} \right)  \cdot
\left( \begin{array}{c}
\lambda_1\\
\lambda_2\\
\lambda_3
\end{array} \right) =
\left( \begin{array}{c}
\lambda_1\\
\lambda_2\\
\lambda_3\\
\lambda_1+\lambda_2\\
\lambda_1+\lambda_3\\
\lambda_2+\lambda_3
\end{array} \right)
\end{displaymath}
\begin{figure}[!h]
\begin{displaymath}
\left\{ \begin{array}{ccl}
x_1&=&\lambda_1\\
x_2&=&\lambda_2\\
x_3&=&\lambda_3\\
x_4&=&\lambda_1+\lambda_2\\
x_5&=&\lambda_1+\lambda_3\\
x_6&=&\lambda_2+\lambda_3
\end{array} \right.
\end{displaymath}
\caption{Ecuaciones param\'etricas del c\'odigo de triple control.}
\end{figure}
%
\newpage
%
\subsubsection{Ecuaciones impl\'{\i}citas}

Como $\dim_{\mathbb{F}_2} \mathbb{F}^{^6}_2 = 6$ y
$\dim_{\mathbb{F}_2} \mathcal{C} = 3$ tendremos $6-3=3$ ecuaciones
impl\'{\i}citas.\\ \\
%
Planteemos el sistema de ecuaciones:
\begin{displaymath}
\left( \begin{array}{cccccc}
1&0&0&1&1&0 \\
0&1&0&1&0&1 \\
0&0&1&0&1&1
\end{array} \right) \cdot
\left( \begin{array}{c}
x_1 \\
x_2 \\
x_3 \\
x_4 \\
x_5 \\
x_6
\end{array} \right) =
\left( \begin{array}{c}
0 \\
0 \\
0
\end{array} \right)
\end{displaymath}
De donde nos sale que $\mathcal{C}^{^\circ}=<(1,1,0,1,0,0),(1,0,1,0,1,0),
(0,1,1,0,0,1)>$.
\begin{figure}[!h]
\begin{eqnarray*}
x_1+x_2+x_4&=&0\\
x_1+x_3+x_5&=&0\\
x_2+x_3+x_6&=&0
\end{eqnarray*}
\caption{Ecuaciones impl\'{\i}citas del c\'odigo de triple control.}
\end{figure}
%
\newpage
%
\subsubsection{Matriz de control}

La matriz de control utilizando las ecuaciones impl\'{\i}citas del c\'odigo
es:
\begin{figure}[!h]
\begin{displaymath}
\left( \begin{array}{cccccc}
1&1&0&1&0&0\\
1&0&1&0&1&0\\
0&1&1&0&0&1
\end{array} \right)
\end{displaymath}
\caption{Matriz de control, en forma est\'andar, del c\'odigo de triple
control.}
\end{figure}
