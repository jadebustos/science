%
% CODIGOS LINEALES
%

\chapter{C\'odigos lineales}

Hemos visto que los c\'odigos de bloques, con sus codificadores,
decodificadores, procesadores de error se definen sobre un alfabeto
$\mathbb{F}_q$, el cual es un conjunto cualquiera. Pero si sobre este
conjunto podemos establecer alguna estructura algebraica, la cual nos permita
realizar operaciones sobre los elementos del conjunto, podremos entonces
simplificar el c\'odigo. Esta simplificaci\'on nos permitir\'a trabajar de
una forma m\'as comoda.

\section{Requisitos para c\'odigos lineales}

Cuando trabajemos con c\'odigos lineales tendremos que $(\mathbb{F}_q,+,\cdot)$
es cuerpo, por ejemplo $\mathbb{F}_2=\mathbb{Z}/2=\mathbb{Z}_2=\{0,1\}$.\\

Dado un cuerpo cualquiera $A$ se tiene que $A^{^n}=A\times \stackrel{n)}\cdots
\times A$ es un \mbox{$A$-espacio vectorial} con las siguientes operaciones:
\begin{itemize}
\item $(x_1,\cdots,x_n)+(x_1',\cdots,x_n')=(x_1+x_1',\cdots,x_n+x_n')$, donde
tenemos que $(x_1,\cdots,x_n)$ y $(x_1',\cdots,x_n')$ son dos palabras
cualesquiera de $A^{^n}$.
\item $x\odot (x_1,\cdots,x_n)=(x\cdot x_1,\cdots, x\cdot x_n)$, donde
tenemos que $x\in A$, $(x_1,\cdots,x_n)\in A^{^n}$ y $\odot$ es el producto
por escalares definido en $A^{^n}$.
\end{itemize}
Nosotros trabajaremos con cuerpos de la forma $\mathbb{F}_q$ y con
$\mathbb{F}^{^n}_q$ que, por lo visto antes, ser\'an
\mbox{$\mathbb{F}_q$-espacios vectoriales}.

%
% CUANDO UN CODIGO ES LINEAL
%

%
% CUANDO UN CODIGO ES LINEAL
%

%
\newpage
%
\section{?`Cuando un c\'odigo es lineal?}

Podemos distinguir unos c\'odigos de bloques especiales que son los c\'odigos
lineales.
\begin{definicion}[C\'odigos Lineales]
\ \\
Dado un cuerpo cualquiera, $\mathbb{K}$, y un $\mathbb{K}$-espacio vectorial de
la forma $\mathbb{K}^{^n}$ diremos que un subconjunto $\mathcal{C}\subset
\mathbb{K}^{^n}$ es un \textbf{``c\'odigo lineal''} si $\mathcal{C}$ es un
subespacio vectorial de $\mathbb{K}^{^n}$.
\end{definicion}
Nosotros consideraremos $\mathbb{K}=\mathbb{F}_q$ y
$\mathbb{K}^{^n}=\mathbb{F}^{^n}_q$.

\subsection{Propiedades de los c\'odigos lineales}

Dado que los c\'odigos lineales son subespacios vectoriales poseen unas
propiedades especiales.\\ \\ 
%
Sea $\mathcal{C}\subset \mathbb{F}^{^n}_2$ un c\'odigo lineal, entonces se
verifican las siguientes propiedades:
\begin{itemize}
\item La suma de dos palabras del c\'odigo es otra palabra del c\'odigo.
\begin{displaymath}
(x_1,\cdots,x_n)+(x_1',\cdots,x_n') = (x_1+x_1',\cdots,x_n+x_n')\in \mathcal{C}
\end{displaymath}
\item Multiplicar una palabra del c\'odigo por un elemento del cuerpo es otra
palabra del c\'odigo.
\begin{displaymath}
x\odot (x_1,\cdots,x_n) = (x\cdot x_1,\cdots, x\cdot x_n)\in \mathcal{C}
\end{displaymath}
\item El $0$ siempre es una palabra del c\'odigo.
\begin{displaymath}
(0,\stackrel{n)}\cdots,0)\in \mathcal{C}
\end{displaymath}
\end{itemize}


%
% DISTANCIA MINIMA
%

%
% DISTANCIA MINIMA 
%

\section{Calculo de la distancia m\'{\i}nima}

Debido a la estructura de espacios vectoriales que poseen estos c\'odigos se
puede simplificar el calculo de la distancia m\'{\i}nima, $d_{min}$. Seg\'un
la definici\'on de distancia m\'{\i}nima el n\'umero de distancias que tendremos
que calcular para encontrala ser\'a de $\frac{|\mathcal{C}|^2}{2}$, es decir,
tendr\'{\i}amos que calcular tantas distancias como parejas de dos elementos
distintos del c\'odigo podamos tener dividido por dos, ya que la distancia es
sim\'etrica. \\

\begin{proposicion}[Distancia m\'{\i}nima para c\'odigos lineales]
\ \\
Sea $\mathcal{C}\subset \mathbb{F}^{^n}_q$ un c\'odigo lineal, entonces:
$$d_{min} = \min_{u \in \mathcal{C}\atop u\neq 0} \{\ d(u,0)\ \}$$
\end{proposicion}
\underline{\textbf{Demostraci\'on}}:\ \\
Supongamos que la distancia m\'{\i}nima es:
$$d_{min} = d(u,v)\quad u,v\in \mathcal{C}$$
entonces por la definici\'on de la distancia de Hamming se tiene que:
$$d(u,v)=d(u-v,0)$$ y como el c\'odigo $\mathcal{C}$ es un c\'odigo lineal se
tiene entonces que $u-v\in \mathcal{C}$. Es decir:
$$d_{min}= \min_{u\in \mathcal{C} \atop u\neq 0} \{\ d(u,0)\ \}$$
\begin{flushright}
$\blacksquare$
\end{flushright}
\begin{observacion}
\ \\
De esta proposici\'on se deduce que para calcular la distancia m\'{\i}nima de 
un c\'odigo lineal basta con calcular la distancia de todas las palabras, salvo
la palabra cero, al cero y tomar el m\'{\i}nimo de esas distancias. Con lo cual
unicamente tendremos que calcular $|\mathcal{C}|-1$ distancias en lugar de las
$\frac{|\mathcal{C}|^2}{2}$ distancias que deberiamos calcular para un c\'odigo
no lineal.
\end{observacion}


%
% CODIFICADORES
%

%
% CODIFICADORES
%

\section{Codificadores}

Seg\'un el apartado $\ref{sec:Codificadores}$, en la p\'agina
$\pageref{sec:Codificadores}$, un codificador es una aplicaci\'on biyectiva.

\begin{definicion}[Codificador para c\'odigos lineales]
\ \\
Sea $\mathcal{C}\subset \mathbb{F}^{^n}_q$ un c\'odigo lineal. Un
\textbf{``codificador''} para un c\'odigo lineal es un isomorfismo%
\footnote{Aplicaci\'on lineal biyectiva.} del siguiente tipo:
\begin{eqnarray*}
C:\mathbb{F}^{^m}_q&\stackrel{\sim}\longrightarrow&\mathcal{C}\\
x&\longrightarrow& u
\end{eqnarray*}
tal que codifica palabras de longitud $m$ con palabras de longitud $n$.
\end{definicion}
\begin{observacion}
\ \\
\begin{itemize}
\item Al ser el codificador una aplicaci\'on biyectiva cada palabra tiene una, y
s\'olo una, forma de codificarse. 
\item Todas las posibles palabras que puedan formar un mensaje se pueden
codificar con una palabra del c\'odigo. Esto es gracias a que el codificador es
epiyectivo, y, por lo dicho en el punto anterior dicha palabra es \'unica.
\item Como el codificador es una aplicaci\'on lineal podemos representar el
codificador por la matriz que representa a la aplicaci\'on lineal una vez
fijadas las correspondientes bases. A est\'a matriz se la conoce como
``\textbf{matriz generadora}''.
\item Como la matriz depende de las bases escogidas, entonces la matriz del
codificador no es \'unica.
\item La matriz de un codificador de un c\'odigo lineal estar\'a formada por
tantas columnas como dimensi\'on tenga el c\'odigo. Y la columna $i$-\'esima
del codificador ser\'a el vector $i$-esimo de la base, del c\'odigo, 
codificado seg\'un el c\'odigo.
\item Como el codificador es un isomorfismo, aplicaci\'on lineal biyectiva, la
imagen de una base es otra base, entonces las columnas de la matriz generadora
son base del c\'odigo $\mathcal{C}$.
\end{itemize}
\end{observacion}

\subsection{Matriz generadora}

\begin{definicion}[Matriz generadora]
\ \\
Dado un c\'odigo lineal y su codificador llamaremos \textbf{``matriz
generadora''} de c\'odigo a la matriz de la aplicaci\'on lineal que representa
al codificador respecto de unas bases fijadas.
\end{definicion}
Para codificar una palabra basta con multiplicarla por la matriz generadora del
c\'odigo y obtendremos la palabra codificada. Por ejemplo, sea $C$ la matriz
generadora de un c\'odigo y queremos codificar la palabra $(1,1,0)$:
\begin{displaymath}
\left( \begin{array}{ccc}
1&0&0\\
0&1&0\\
0&0&1\\
1&1&0\\
1&0&1\\
0&1&1
\end{array} \right) \cdot
\left( \begin{array}{c}
1\\
1\\
0
\end{array} \right) =
\left( \begin{array}{c}
1\\
1\\
0\\
0\\
1\\
1
\end{array} \right)
\end{displaymath}
Luego la palabra $(1,1,0)$ codificada es $(1,1,0,0,1,1)$.\\ 

La siguiente proposici\'on nos dice cuando una matriz es una matriz generadora
de un c\'odigo, y en caso afirmativo, de que tipo de c\'odigo se trata.

\begin{proposicion}
\ \\
Sea $\mathcal{C}[n,m]$ un c\'odigo lineal y sea $C$ una matriz de orden
$n\times m$. $C$ es una matriz generadora para el c\'odigo $\mathcal{C}[n,m]$
s\'{\i} y s\'olo s\'{\i} tiene rango $m$ y sus columnas son palabras del
c\'odigo.
\end{proposicion}
\underline{\textbf{Demostraci\'on}}:\\
$\Rightarrow |$ Sea $C$ una matriz generadora para el c\'odigo
$\mathcal{C}[n,m]$. \\

Por definici\'on de matriz generadora sus columnas ser\'an palabras del
c\'odigo, y como el c\'odigo lineal es de dimensi\'on $m$, dimensi\'on
del subespacio imagen de la aplicaci\'on lineal que determina $C$, entonces
el rango de $C$ es $m$.\\ \\
%
$\Leftarrow |$ Sean $C_1$ , \dots, $C_m$ las columnas de la matriz $C$, las
cuales son, por hip\'otesis, palabras del c\'odigo.\\

Sea $a=(a_1,\dots,a_m )$ una palabra entonces $C\cdot a^t = a_1\cdot
C_1^t+\dots +a_m\cdot C_m^t$ que es una combinaci\'on lineal de
$\{C_i \}_{i=1}^m$ y como el c\'odigo $\mathcal{C}[n,m]$ es lineal se tiene
que las combinaciones lineales de palabras del c\'odigo son palabras del 
c\'odigo. Luego la matriz $C$ transforma cualquier palabra, de longitud $m$, 
en una palabra del c\'odigo.\\

La dimensi\'on del subespacio imagen es el rango de la matriz $C$, luego la
dimensi\'on del subespacio imagen es $m$.\\

Como la matriz $C$ codifica palabras de longitud $m$ en palabras de longitud
$n$ en un subespacio de dimensi\'on $m$ entonces $C$ es la matriz generadora
de un c\'odigo $\mathcal{C}[n,m]$.
\begin{flushright}
$\blacksquare$
\end{flushright}
%
\newpage
%
\subsection{Forma est\'andar de la matriz generadora}

La expresi\'on de la ``\textbf{matriz generadora}'' no es unica, sino que
depende de las bases escogidas. Luego eligiremos un convenio para elegir
las bases en las que expresaremos dicha matriz.
\begin{definicion}[Forma est\'andar de la matriz generadora]
\ \\
Sea $\mathcal{C}\subset \mathbb{F}^{^n}_q$ un c\'odigo lineal que utilizaremos
para codificar palabras de longitud $m$, $\mathbb{F}^{^m}_q$. Sea $C$ la
matriz del codificador, una vez fijadas las bases correspondientes. Diremos
que $C$ est\'a en \textbf{``forma est\'andar''} cuando los $m$ primeros elementos
de la palabra codificada $C(x)\in \mathcal{C}\subset \mathbb{F}^{^n}_q$
coincidan con $x\in \mathbb{F}^{^m}_q$.
\end{definicion}
\begin{proposicion}
\ \\
Sea $\mathcal{C}[n,m]$ un c\'odigo lineal. La matriz generadora est\'a en 
forma est\'andar s\'{\i} y s\'olo s\'{\i} en las $m$ primeras columnas
aparece la matriz identidad de orden $m$.
\end{proposicion}
\underline{\textbf{Demostraci\'on}}:\\
Tanto el directo como el rec\'{\i}proco son una mera comprobaci\'on.
\begin{flushright}
$\blacksquare$
\end{flushright}


%
% ECUACIONES DE LOS CODIGOS LINEALES
%

%
% ECUACIONES DE LOS CODIGOS LINEALES
%

\section{Ecuaciones de los c\'odigos lineales}

Como hemos visto un c\'odigo lineal $\mathcal{C}$ es un subespacio vectorial.
Utilizando la teor\'{\i}a de espacios vectoriales podemos calcular las
ecuaciones de un subespacio vectorial cualquiera. La utilidad de estas
ecuaciones radica en que nos indican las condiciones que ha de cumplir un
vector para pertenecer al c\'odigo.\\

Sea $\mathcal{C}\subset \mathbb{F}^{^n}_q$ un c\'odigo lineal, donde
$\mathbb{F}^{^n}_q$ es un $\mathbb{F}_q$-espacio vectorial tal que
$\dim_{\mathbb{F}_q} \mathbb{F}^{^n}_q=n$. Supongamos que
$\dim_{\mathbb{F}_q} \mathcal{C} = m$, con $m<n$ y que el espacio de palabras
a codificar es $\mathbb{F}^{^m}_q$.
%
\newpage
%
\subsection{Ecuaciones param\'etricas}\label{sec:EcuParametricas}

Una vez conocida la matriz generadora del c\'odigo se pueden calcular las
ecuaciones param\'etricas del c\'odigo. Dichas ecuaciones nos permitir\'an
calcular todas las palabras del c\'odigo.\\

Sea $C$ la matriz generadora del c\'odigo, para calcular las ecuaciones
pa\-ra\-m\'etricas del c\'odigo bastar\'a con aplicar la matriz generadora a un
vector gen\'erico del espacio de palabras a codificar.
\begin{displaymath}
C\cdot \left( \begin{array}{c}
\lambda_1\\
\lambda_2\\
\vdots\\
\lambda_m
\end{array} \right) =
\left( \begin{array}{c}
\theta_1\\
\theta_2\\
\vdots\\
\theta_n
\end{array} \right)
\end{displaymath}
luego las ecuaciones param\'etricas ser\'an:
\begin{displaymath}
\left\{ \begin{array}{ccc}
x_1 &=&\theta_1\\
x_2 &=&\theta_2\\
\cdots&\cdots&\cdots\\
x_n &=&\theta_n
\end{array} \right.
\end{displaymath}
Tendremos que $\theta_i = f_i (\lambda_1,\dots,\lambda_m)$ para $i=1,\dots,n$ y
dando valores a $\{\lambda_j \}_{j=1}^m$ en el cuerpo $\mathbb{F}_q$ tendremos
todos los elementos del c\'odigo.

\subsection{Ecuaciones impl\'{\i}citas}\label{sec:EcuImplicitas}

Para calcular las ecuaciones impl\'{\i}citas de $\mathcal{C}$ calcularemos el
subespacio incidente a $\mathcal{C}$, $\mathcal{C}^{^{\circ}}\subset
(\mathbb{F}^{^n}_q)^{^*}$. El subespacio incidente est\'a formado por las
formas lineales que se anulan sobre $\mathcal{C}$.\\

El n\'umero de ecuaciones impl\'{\i}citas que tendr\'a nuestro c\'odigo
$\mathcal{C}$ ser\'a:
$$dim_{\mathbb{F}_q} \mathbb{F}^{^n}_q - dim_{\mathbb{F}_q} \mathcal{C}=n-m$$
Como $\dim_{\mathbb{F}_q} \mathcal{C} = m$ tendremos que
$\mathcal{C}=<c_1,\ldots,c_m>$. Para calcular las ecuaciones tendremos que
resolver un sistema de ecuaciones que tendr\'a m\'as de una soluci\'on.
%
\newpage
%
Sean $\{x_i^j\}_{j=1}^m$ con $i=1,\dots,n$ la $i$-\'esima coordenada del
vector $j$-\'esimo de la base de $\mathcal{C}$. El sistema que nos da las
ecuaciones ser\'a:
\begin{displaymath}
\left( \begin{array}{cccc}
x_1^1 & \ldots & \ldots & x_n^1 \\
x_1^2 & \ddots & & x_n^2 \\
\vdots & &\ddots & \vdots \\
x_1^m &\ldots &\ldots & x_n^m
\end{array} \right) \cdot
\left( \begin{array}{c}
x_1 \\
\vdots \\
\vdots \\
x_n
\end{array} \right) =
\left( \begin{array}{c}
0 \\
\vdots \\
\vdots \\
0
\end{array} \right)
\end{displaymath}

Cualquier $x=(x_1,\ldots,x_n)$ que verifique este sistema de ecuaciones
ser\'a un elemento de $\mathcal{C}$, o lo que es lo mismo cualquier punto que no
verifique el sistema de ecuaciones anterior NO pertenecer\'a a $\mathcal{C}$.


%
% DETECCION DE ERRORES
%

%
% DETECCION DE ERRORES
%

\section{Detecci\'on de errores}

Supondremos que tenemos un c\'odigo lineal $\mathcal{C}\subset
\mathbb{F}^{^n}_q$ tal que $\dim_{\mathbb{F}_q} \mathcal{C}=m$, con $m<n$. De
esto se deduce que utilizaremos el c\'odigo para codificar palabras de longitud
$m$ luego es un c\'odigo $\mathcal{C}[n,m]$.\\

Se detecta un error cuando una de las palabras recibidas no pertenece al
c\'odigo. En el caso de los c\'odigos lineales es facil comprobar cuando una
palabra pertenece o no al c\'odigo.\\

Dado que los c\'odigos lineales son subespacios vectoriales tienen asociadas
unas ecuaciones:
\begin{itemize}
\item Ecuaciones param\'etricas\footnote{Apartado $(\ref{sec:EcuParametricas})$,
en la p\'agina $\pageref{sec:EcuParametricas}$.}.
\item Ecuaciones impl\'{\i}citas\footnote{Apartado $(\ref{sec:EcuImplicitas})$,
en la p\'agina $\pageref{sec:EcuImplicitas}$.}.
\end{itemize}
Para comprobar s\'{\i} una palabra pertenece o no al c\'odigo basta con ver
s\'{\i} verifica las ecuaciones que definen al c\'odigo, param\'etricas o
impl\'{\i}citas.
%
\newpage
%
\subsection{Matriz de control}

\begin{definicion}[Matriz de control]
\ \\
Para un c\'odigo lineal $\mathcal{C}$ diremos que una matriz $M$ de orden
$k\times n$ es una \textbf{``matriz de control''} del c\'odigo si verifica que
$M\cdot u^t = 0$ para todo $u\in \mathbb{F}^{^n}_q$ s\'{\i} y
s\'olo s\'{\i} $u\in \mathcal{C}$.
\end{definicion}
$k$ puede ser cualquiera, pero normalmente $k=n-m$ ya que esa es la dimensi\'on
del subespacio incidente a $\mathcal{C}$.\\ \\
%
La matriz de control la podemos obtener de:
\begin{itemize}
\item Las ecuaciones impl\'{\i}citas de $\mathcal{C}$.
\item La base del subespacio incidente a $\mathcal{C}$.
\end{itemize}
La matriz de control es una matriz cuyas filas son una base de ``vectores''
incidentes a $\mathcal{C}$.

\subsection{Forma est\'andar de la matriz de control}

Al igual que en el caso de la matriz generadora la expresi\'on de la 
\textbf{``matriz de control''} depende de la base escogida en el subespacio
incidente a $\mathcal{C}$.

\begin{definicion}[Forma est\'andar de la matriz de control]
\ \\
Sea $M$ una matriz de control de orden $(n-m)\times n$, diremos que est\'a en
\textbf{``forma est\'andar''} si $M=(M_1,M_2)$, donde $M_2$ es la matriz
identidad de orden $(n-m)\times (n-m)$.
\end{definicion}


%
% EJEMPLOS
%

%
% EJEMPLOS
%

\section{Ejemplos}

\subsection{C\'odigo del bit de control de paridad}

Este c\'odigo es un c\'odigo de $2^7=128$ palabra palabras. Es un subconjunto de
$\mathbb{F}^{^8}_2$ el cual tiene $2^8=256$ palabras, con lo cual el n\'umero
de patrones de error que tendremos ser\'a $2^8-2^7=128$.

\subsubsection{Tabla est\'andar del c\'odigo}

Como el c\'odigo tiene $2^7=128$ palabras la tabla tendr\'a $2^7=128$ columnas
y el n\'umero de filas ser\'a $2^{8-7}=2$.\\

Debido a la gran cantidad de palabras de este c\'odigo no construiremos su
tabla est\'andar.

\subsubsection{Sindromes}

Sea $H$ la matriz de control, en forma est\'andar, del c\'odigo:
\begin{displaymath}
\left( \begin{array}{cccccccc}
1&1&1&1&1&1&1&1
\end{array} \right)
\end{displaymath}

\begin{table}[!h]
\begin{displaymath}
\begin{array}{|c|c|}
\hline
Errores & Sindromes \\
\hline
00000000 & 0 \\
\hline
10000000 & 1 \\
\hline
\end{array}
\end{displaymath}
\caption{Tabla de sindromes del c\'odigo del bit de control de paridad.}
\end{table}

\subsection{C\'odigo de triple repetici\'on}

Este c\'odigo es un c\'odigo de $2^1=2$ palabras. Es un subconjunto de
$\mathbb{F}^{^3}_2$ el cual tiene $2^3=8$ palabras, con lo cual el n\'umero
de patrones de error que tendremos ser\'a $2^3-2^1=6$.

\subsubsection{Tabla est\'andar del c\'odigo}

Como el c\'odigo tiene $2^1=2$ palabras la tabla tendr\'a $2^1=2$ columnas y
el n\'umero de filas ser\'a $2^{3-1}=4$.\\

La primera fila estar\'a formada por las palabras del c\'odigo con la
condici\'on de que el $0$ sea el primer elemento. Luego la primera fila ser\'a:
\begin{displaymath}
\begin{array}{cc}
000&111
\end{array}
\end{displaymath}
Para la segunda fila cogeremos un elemento de $\mathbb{F}^{^3}_2$ que no este en
la fila cero, y el primer elemento de cada fila ha de ser el de m\'{\i}nimo
peso de dicha fial. El elemento $100$ no esta en la fila cero, luego elegimos
ese elemento como primer elemento de la fila uno ya que es de peso uno y no es
posible encontrar otro de peso menor. El resto de elementos de la fila ser\'an:
\begin{displaymath}
Tb(\mathcal{C}[3,1])_{1,i}=Tb(\mathcal{C}[3,1])_{1,0}+Tb(\mathcal{C}[3,1])_{0,i}
\quad i=0,1
\end{displaymath}
Luego la segunda fila ser\'a:
\begin{displaymath}
\begin{array}{cc}
100&011
\end{array}
\end{displaymath}
Siguiendo el mismo razonamiento elegiremos como primer elemento de la fila tres
un elemento que no haya aparecido en las filas anteriores y de peso m\'{\i}nimo,
por ejemplo $010$ y siguiendo el razonamiento anterior completaremos la tabla.
\begin{eqnarray*}
Tb(\mathcal{C}[3,1])_{2,0}&=&010\\
Tb(\mathcal{C}[3,1])_{3,0}&=&001
\end{eqnarray*}
\begin{table}[!h]
\begin{displaymath}
\begin{array}{|c|c|}
\hline
000&111\\
\hline
100&011\\
\hline
010&101\\
\hline
001&110\\
\hline
\end{array}
\end{displaymath}
\caption{Tabla est\'andar del c\'odigo de triple repetici\'on.}\label{tab:TablaII}
\end{table}

\subsubsection{Correcci\'on de errores}

Supongamos que recibimos la palabra $011$, que no pertenece al c\'odigo. Para
corregir el c\'odigo haremos:
\begin{itemize}
\item Localizamos el lugar de dicha palabra en la tabla.
\begin{displaymath}
Tb(\mathcal{C}[3,1])_{1,1}=011
\end{displaymath}
\item Elegimos como palabra correcta la palabra que est\'e en la misma columna y
en la fila cero.
\begin{displaymath}
Tb(\mathcal{C}[3,1])_{0,1}=111
\end{displaymath}
\end{itemize}
El error cometido vendr\'a dado por el primer elemento de la fila en la que se
encuentre la palabra recibida:
\begin{displaymath}
Tb(\mathcal{C}[3,1])_{1,0}=100
\end{displaymath}
Luego en la transmisi\'on se ha cometido un error de peso $w(100)=1$ en el
primer bit.

\subsubsection{Sindromes}

Supondremos que queremos corregir los mismos errores que se corrigen con la
tabla $\ref{tab:TablaII}$.\\ \\
%
Sea $H$ la matriz de control, en forma est\'andar, del c\'odigo:
\begin{displaymath}
H=\left( \begin{array}{ccc}
1&1&0\\
1&0&1
\end{array} \right)
\end{displaymath}

\begin{table}[!h]
\begin{displaymath}
\begin{array}{|c|c|}
\hline
Errores & Sindromes \\
\hline
000 & 00 \\
\hline
100 & 11 \\
\hline
010 & 10 \\
\hline
001 & 01 \\
\hline
\end{array}
\end{displaymath}
\caption{Tabla de sindromes del c\'odigo de triple repetici\'on.}\label{tab:TabSindromes}
\end{table}

Para corregir errores supongamos que recibimos la palabra $101$, calculamos
su sindrome que ser\'a $10$. Entonces utilizando la tabla
$\ref{tab:TabSindromes}$ buscamos el error que tiene sindrome $10$, dicho
error es $010$. Luego la palabra transmitida ser\'a $101-010=111$. Recordar
que $-1=1$ en $\mathbb{F}_2$.

\subsection{C\'odigo de triple control}

Este c\'odigo es un c\'odigo de $2^3=8$ palabras. Es un subconjunto de
$\mathbb{F}^{^6}_2$ el cual tiene $2^6=64$ palabras, con lo cual el n\'umero
de patrones de error que tendremos ser\'a $2^6-2^3=56$.

\subsubsection{Tabla est\'andar del c\'odigo}

Como el c\'odigo tiene $2^3=8$ palabras la tabla tendr\'a $2^3=8$ columnas y
el n\'umero de filas ser\'a $2^{6-3}=8$.\\ 

La primera fila estar\'a formada por las palabras del c\'odigo con la
condici\'on de que el $0$ sea el primer elemento. Luego la primera fila
ser\'a:
\begin{displaymath}
\begin{array}{cccccccc}
000000&100110&010101&001011&111000&011110&101101&110011
\end{array}
\end{displaymath}
Para la segunda fila cogeremos un elemento de $\mathbb{F}^{^6}_2$ que no este
en la fila cero, y el primer elemento de cada fila ha de ser el de
m\'{\i}nimo peso de dicha fila. El elemento $100000$ no est\'a en la fila 
cero,
luego elegimos ese elemento como primer elemento de la fila uno ya que es de
peso uno y no es posible encontrar otro elemento de peso menor. El resto de
elementos de la fila ser\'an:
\begin{displaymath}
Tb(\mathcal{C}[6,3])_{1,i}=Tb(\mathcal{C}[6,3])_{1,0}+Tb(\mathcal{C}[6,3])_{0,
i}
\quad i=1,\dots,7
\end{displaymath}
Luego la segunda fila ser\'a:
\begin{displaymath}
\begin{array}{cccccccc}
100000&000110&110101&101011&011000&111110&001101&010011
\end{array}
\end{displaymath}
%
\newpage
%
Siguiendo el mismo razonamiento escogeremos como primer elemento de la fila 
dos un elemento de $\mathbb{F}^{^6}_2$ que no aparezca en las filas cero y uno. 
Como hay elementos de peso uno que no aparecen en dichas filas elegiremos como
primer elemento de la fila dos $010000$. Y siguiendo el mismo razonamiento
construiremos la fila y eligiremos los siguientes elementos:
\begin{eqnarray*}
Tb(\mathcal{C}[6,3])_{3,0}&=& 001000\\
Tb(\mathcal{C}[6,3])_{4,0}&=& 000100\\
Tb(\mathcal{C}[6,3])_{5,0}&=& 000010\\
Tb(\mathcal{C}[6,3])_{6,0}&=& 000001
\end{eqnarray*}
Para elegir el primer elemento de la fila siete observaremos que todas las
palabras de $\mathbb{F}^{^6}_2$ de peso uno estan en alguna de las filas
anteriores, con lo cual el elemento de la fila siete de menor peso tendr\'a un
peso mayor o igual que dos. El elemento $100001$ no esta en ninguna de las  
filas anteriores con lo cual lo elegiremos como primer elemento de la fila
siete.\\ \\ 
%
%
\begin{table}[!h]
\begin{displaymath}
\begin{array}{|c|c|c|c|c|c|c|c|}
\hline
000000&100110&010101&001011&111000&011110&101101&110011\\
\hline
100000&000110&110101&101011&011000&111110&001101&010011\\
\hline
010000&110110&000101&011011&101000&001110&111101&100011\\
\hline
001000&101110&011101&000011&110000&010110&100101&111011\\
\hline
000100&100010&010001&001111&111100&011010&101001&110111\\
\hline
000010&100100&010111&001001&111010&011100&101111&110001\\
\hline
000001&100111&010100&001010&111001&011111&101100&110010\\
\hline
100001&000111&110100&101010&011001&111111&001100&010010\\
\hline
\end{array}
\end{displaymath}
\caption{Tabla est\'andar del c\'odigo de triple control.}\label{tab:Tabla}
\end{table}
%
%
En la tabla $\ref{tab:Tabla}$ podemos ver una tabla est\'andar para el c\'odigo
de triple repetici\'on. No es la \'unica tabla est\'andar. Para obtener otras
tablas est\'andar bastar\'a con elegir distintos elementos como primer elemento
de cada fila obteniendo las mismas filas, pero en ordenadas de distinta forma.
%
\newpage
%
\subsubsection{Correcci\'on de errores}

Supongamos que recibimos la palabra $111100$, la cual no pertenece al c\'odigo.
Para corregir el error haremos:
\begin{itemize}
\item Localizamos el lugar de dicha palabra en la tabla.
\begin{displaymath}
Tb(\mathcal{C}[6,3])_{4,4}=111100
\end{displaymath}
\item Elegimos como palabra correcta la palabra que est\'e en la misma columna y
en la fila cero.
\begin{displaymath}
Tb(\mathcal{C}[6,3])_{0,4}=111000
\end{displaymath}
\end{itemize}
El error cometido vendr\'a dado por el primer elemento de la fila en la que se
encuentre la palabra recibida:
\begin{displaymath}
Tb(\mathcal{C}[6,3])_{4,0}=000100
\end{displaymath}
Luego en la transmisi\'on se ha cometido un error de peso $w(000100)=1$ en el
cuarto bit.

\subsubsection{Sindromes}

Supondremos que queremos corregir los mismos errores que se corrigen con la
tabla $\ref{tab:Tabla}$.\\ \\
%
Sea $H$ la matriz de control, en forma est\'andar, del c\'odigo:
\begin{displaymath}
H=\left( \begin{array}{cccccc}
1&1&0&1&0&0\\
1&0&1&0&1&0\\
0&1&1&0&0&1
\end{array} \right)
\end{displaymath}

\begin{table}[!h]
\begin{displaymath}
\begin{array}{|c|c|}
\hline
Error  & Sindrome \\
\hline
000000 & 000 \\
\hline
100000 & 110 \\
\hline
010000 & 101 \\
\hline
001000 & 011 \\
\hline
000100 & 100 \\
\hline
000010 & 010 \\
\hline
000001 & 001 \\
\hline
100001 & 111 \\
\hline
\end{array}
\end{displaymath}
\caption{Tabla de sindromes del c\'odigo de triple control.}\label{tab:TablaSindromes}
\end{table}
Obviamente requiere menor costo almacenar la tabla $\ref{tab:TablaSindromes}$
que almacenar la tabla $\ref{tab:Tabla}$.\\

Para corregir errores supongamos que recibimos la palabra $100011$, calculamos
su sindrome que ser\'a $101$. Entonces utilizando la tabla
$\ref{tab:TablaSindromes}$ buscamos que error tiene sindrome $101$, dicho error
es $010000$. Luego la palabra transmitida ser\'a $100011-010000=110011$.
Recordar que $-1=1$ en $\mathbb{F}_2$.


%
% EJERCICIOS
%
%
% EJERCICIOS
%

\section{Ejercicios}

\begin{ejercicio}
\ \\
Dada una palabra cualquiera del c\'odigo de triple control, comprobar que es lo
que falla s\'{\i} se produce un fallo en cualquiera de los seis bits.
\end{ejercicio}
\textbf{\underline{Soluci\'on:}}\\
Sea $110011$ una palabra perteneciente al c\'odigo de triple control.\\ \\
%
Todas las palabras del c\'odigo son de la forma:
\begin{displaymath}
\mathcal{C}[6,3]=\{\ (u_1,u_2,u_3,u_1+u_2,u_1+u_3,u_2+u_3)\ u_i \in \mathbb{F}_2
\ \}
\end{displaymath}
Llamaremos a los bit de control:
\begin{displaymath}
c_1 = u_1+u_2\quad c_2=u_1+u_3\quad c_3=u_2+u_3
\end{displaymath}
Introduciremos un error en cada bit y veremos que bits de control fallan,
suponiendo siempre que los bits de informaci\'on son correctos.\\ 
\begin{center}
\begin{tabular}{|c|c|c|c|}
\hline
Bit de error & Palabra erronea & Bits de control & Bits correctos\\
\hline
Primer bit & $010011$ & $011$ & $101$ \\
\hline
Segundo bit & $100011$ & $011$ & $110$ \\
\hline
Tercer bit & $111011$ & $011$ & $000$ \\
\hline
Cuarto bit & $110111$ & $111$ & $011$ \\
\hline
Quinto bit & $110001$ & $001$ & $011$ \\
\hline
Sexto bit & $110010$ & $010$ & $011$ \\
\hline
\end{tabular}
\end{center}
\begin{flushright}
$\blacksquare$
\end{flushright}

