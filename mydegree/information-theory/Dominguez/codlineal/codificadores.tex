%
% CODIFICADORES
%

\section{Codificadores}

Seg\'un el apartado $\ref{sec:Codificadores}$, en la p\'agina
$\pageref{sec:Codificadores}$, un codificador es una aplicaci\'on biyectiva.

\begin{definicion}[Codificador para c\'odigos lineales]
\ \\
Sea $\mathcal{C}\subset \mathbb{F}^{^n}_q$ un c\'odigo lineal. Un
\textbf{``codificador''} para un c\'odigo lineal es un isomorfismo%
\footnote{Aplicaci\'on lineal biyectiva.} del siguiente tipo:
\begin{eqnarray*}
C:\mathbb{F}^{^m}_q&\stackrel{\sim}\longrightarrow&\mathcal{C}\\
x&\longrightarrow& u
\end{eqnarray*}
tal que codifica palabras de longitud $m$ con palabras de longitud $n$.
\end{definicion}
\begin{observacion}
\ \\
\begin{itemize}
\item Al ser el codificador una aplicaci\'on biyectiva cada palabra tiene una, y
s\'olo una, forma de codificarse. 
\item Todas las posibles palabras que puedan formar un mensaje se pueden
codificar con una palabra del c\'odigo. Esto es gracias a que el codificador es
epiyectivo, y, por lo dicho en el punto anterior dicha palabra es \'unica.
\item Como el codificador es una aplicaci\'on lineal podemos representar el
codificador por la matriz que representa a la aplicaci\'on lineal una vez
fijadas las correspondientes bases. A est\'a matriz se la conoce como
``\textbf{matriz generadora}''.
\item Como la matriz depende de las bases escogidas, entonces la matriz del
codificador no es \'unica.
\item La matriz de un codificador de un c\'odigo lineal estar\'a formada por
tantas columnas como dimensi\'on tenga el c\'odigo. Y la columna $i$-\'esima
del codificador ser\'a el vector $i$-esimo de la base, del c\'odigo, 
codificado seg\'un el c\'odigo.
\item Como el codificador es un isomorfismo, aplicaci\'on lineal biyectiva, la
imagen de una base es otra base, entonces las columnas de la matriz generadora
son base del c\'odigo $\mathcal{C}$.
\end{itemize}
\end{observacion}

\subsection{Matriz generadora}

\begin{definicion}[Matriz generadora]
\ \\
Dado un c\'odigo lineal y su codificador llamaremos \textbf{``matriz
generadora''} de c\'odigo a la matriz de la aplicaci\'on lineal que representa
al codificador respecto de unas bases fijadas.
\end{definicion}
Para codificar una palabra basta con multiplicarla por la matriz generadora del
c\'odigo y obtendremos la palabra codificada. Por ejemplo, sea $C$ la matriz
generadora de un c\'odigo y queremos codificar la palabra $(1,1,0)$:
\begin{displaymath}
\left( \begin{array}{ccc}
1&0&0\\
0&1&0\\
0&0&1\\
1&1&0\\
1&0&1\\
0&1&1
\end{array} \right) \cdot
\left( \begin{array}{c}
1\\
1\\
0
\end{array} \right) =
\left( \begin{array}{c}
1\\
1\\
0\\
0\\
1\\
1
\end{array} \right)
\end{displaymath}
Luego la palabra $(1,1,0)$ codificada es $(1,1,0,0,1,1)$.\\ 

La siguiente proposici\'on nos dice cuando una matriz es una matriz generadora
de un c\'odigo, y en caso afirmativo, de que tipo de c\'odigo se trata.

\begin{proposicion}
\ \\
Sea $\mathcal{C}[n,m]$ un c\'odigo lineal y sea $C$ una matriz de orden
$n\times m$. $C$ es una matriz generadora para el c\'odigo $\mathcal{C}[n,m]$
s\'{\i} y s\'olo s\'{\i} tiene rango $m$ y sus columnas son palabras del
c\'odigo.
\end{proposicion}
\underline{\textbf{Demostraci\'on}}:\\
$\Rightarrow |$ Sea $C$ una matriz generadora para el c\'odigo
$\mathcal{C}[n,m]$. \\

Por definici\'on de matriz generadora sus columnas ser\'an palabras del
c\'odigo, y como el c\'odigo lineal es de dimensi\'on $m$, dimensi\'on
del subespacio imagen de la aplicaci\'on lineal que determina $C$, entonces
el rango de $C$ es $m$.\\ \\
%
$\Leftarrow |$ Sean $C_1$ , \dots, $C_m$ las columnas de la matriz $C$, las
cuales son, por hip\'otesis, palabras del c\'odigo.\\

Sea $a=(a_1,\dots,a_m )$ una palabra entonces $C\cdot a^t = a_1\cdot
C_1^t+\dots +a_m\cdot C_m^t$ que es una combinaci\'on lineal de
$\{C_i \}_{i=1}^m$ y como el c\'odigo $\mathcal{C}[n,m]$ es lineal se tiene
que las combinaciones lineales de palabras del c\'odigo son palabras del 
c\'odigo. Luego la matriz $C$ transforma cualquier palabra, de longitud $m$, 
en una palabra del c\'odigo.\\

La dimensi\'on del subespacio imagen es el rango de la matriz $C$, luego la
dimensi\'on del subespacio imagen es $m$.\\

Como la matriz $C$ codifica palabras de longitud $m$ en palabras de longitud
$n$ en un subespacio de dimensi\'on $m$ entonces $C$ es la matriz generadora
de un c\'odigo $\mathcal{C}[n,m]$.
\begin{flushright}
$\blacksquare$
\end{flushright}
%
\newpage
%
\subsection{Forma est\'andar de la matriz generadora}

La expresi\'on de la ``\textbf{matriz generadora}'' no es unica, sino que
depende de las bases escogidas. Luego eligiremos un convenio para elegir
las bases en las que expresaremos dicha matriz.
\begin{definicion}[Forma est\'andar de la matriz generadora]
\ \\
Sea $\mathcal{C}\subset \mathbb{F}^{^n}_q$ un c\'odigo lineal que utilizaremos
para codificar palabras de longitud $m$, $\mathbb{F}^{^m}_q$. Sea $C$ la
matriz del codificador, una vez fijadas las bases correspondientes. Diremos
que $C$ est\'a en \textbf{``forma est\'andar''} cuando los $m$ primeros elementos
de la palabra codificada $C(x)\in \mathcal{C}\subset \mathbb{F}^{^n}_q$
coincidan con $x\in \mathbb{F}^{^m}_q$.
\end{definicion}
\begin{proposicion}
\ \\
Sea $\mathcal{C}[n,m]$ un c\'odigo lineal. La matriz generadora est\'a en 
forma est\'andar s\'{\i} y s\'olo s\'{\i} en las $m$ primeras columnas
aparece la matriz identidad de orden $m$.
\end{proposicion}
\underline{\textbf{Demostraci\'on}}:\\
Tanto el directo como el rec\'{\i}proco son una mera comprobaci\'on.
\begin{flushright}
$\blacksquare$
\end{flushright}
