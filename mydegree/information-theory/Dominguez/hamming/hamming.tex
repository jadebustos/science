%
% CODIGOS BINARIOS DE HAMMING
%

\section{C\'odigos binarios de Hamming}

Estos c\'odigos son binarios, sobre $\mathbb{F}_2$.
\begin{definicion}[C\'odigo binario de Hamming]
\ \\
El \textbf{``c\'odigo binario de Hamming''} de orden $k$ es aquel cuya matriz
de control $H_k$ es aquella matriz que tiene por columnas todas las palabras
binarias no nulas de longitud $k$. A este c\'odigo lo denotaremos como $Ham(k)$.
\end{definicion}
\begin{figure}[!h]
\begin{displaymath}
H_3=\left( \begin{array}{ccccccc}
1&0&1&1&1&0&0\\
1&1&1&0&0&1&0\\
0&1&1&1&0&0&1
\end{array} \right)
\end{displaymath}
\caption{Matriz de control, en forma est\'andar, de $Ham(3)$.}\label{fig:Hamm3}
\end{figure}
%
\begin{figure}[!h]
\begin{displaymath}
H_4=\left( \begin{array}{ccccccccccccccc}
1&0&0&1&1&0&1&0&1&1&1&1&0&0&0\\
1&1&0&1&0&1&1&1&1&0&0&0&1&0&0\\
0&1&1&0&1&0&1&1&1&1&0&0&0&1&0\\
0&0&1&1&0&1&0&1&1&1&1&0&0&0&1
\end{array} \right)
\end{displaymath}
\caption{Matriz de control, en forma est\'andar, de $Ham(4)$.}\label{fig:Hamm4}
\end{figure}
Por contrucci\'on estos c\'odigos son binarios y con todas las columnas de su
matriz de control distintas y no nulas, entonces por lo visto en el
cap\'{\i}tulo anterior tendremos que estos c\'odigos corrijen todos los errores
de peso uno.\\

Observando las matrices de control de los c\'odigos de Hamming $Ham(3)$ y
$Ham(4)$, en las figuras $\ref{fig:Hamm3}$ y $\ref{fig:Hamm4}$ respectivamente,
podemos observar lo siguiente:\\
\begin{center}
\begin{tabular}{|c|c|c|c|}
\hline
C\'odigo & Bits totales & Bits de informaci\'on & Tipo de c\'odigo \\
\hline
$Ham(3)$ & $7$  & $4$  & $\mathcal{C}[7,4]$ \\
\hline
$Ham(4)$ & $15$ & $11$ & $\mathcal{C}[15,11]$ \\
\hline
\end{tabular}
\end{center}

\subsection{Los c\'odigos de Hamming son lineales}

Por definici\'on un c\'odigo de Hamming es aquel que tiene como matriz de
control a $H_k$. Luego suponiendo que $H_k$ es la matriz incidente de un
subespacio vectorial nos basta con calcular que vectores se anulan al
multiplicarlos por $H_k$ y estos formar\'an un subespacio vectorial.

\subsection{Par\'ametros de los c\'odigos de Hamming}

\begin{proposicion}[Par\'ametros de los c\'odigos de Hamming]
\ \\
El c\'odigo de Hamming $Ham(k)$ tiene los siguientes par\'ametros:
\begin{enumerate}
\item La longitud de las palabras de $Ham(k)$ es $2^k-1$.
\item El rango del c\'odigo $Ham(k)$ es $2^k-k-1$.
\item La distancia m\'{\i}nima, $d_{min}$, de $Ham(k)$ es $3$.
\end{enumerate}
\end{proposicion}
\underline{\textbf{Demostraci\'on}}:\\
\begin{enumerate}
\item La longitud de las palabras de $Ham(k)$ es $2^k-1$.\\

Como estos c\'odigos son binarios las palabras del c\'odigo estar\'an formadas
por dos elementos, en particular las columnas de su matriz de control.\\

La columnas de su matriz de control son todas las palabras de longitud $k$ no
nulas. Los elementos de cada columna pueden ser dos, los elementos de
$\mathbb{F}_2$, y cada columna tiene $k$ elementos entonces el n\'umero de
posibles columnas de $k$ elementos de $\mathbb{F}_2$ es $2^k$, como no estamos
considerando el elemento nulo entonces tendremos que el n\'umero de columnas no
nulas de longitud $k$ es de $2^k-1$.\\

Ahora bien, como el c\'odigo es lineal tendremos que:
\begin{displaymath}
H_k\cdot u^t = 0
\end{displaymath}
donde $H_k$ es la matriz de control de $Ham(k)$ y $u^t\in Ham(k)$. Entonces para
poder multiplicar $H_k$ por el vector columna $u^t$ se tiene que $u^t$ ha de
tener el mismo n\'umero de columnas que la matriz $H_k$ y como esta tiene
$2^k-1$ columnas entonces el vector $u^t$ tiene $2^k-1$ filas, luego tiene una
longitud de $2^k-1$.

\item El rango del c\'odigo $Ham(k)$ es $2^k-k-1$.\\

El rango de un c\'odigo lineal coincide con el rango de su matriz de control.
Para el c\'odigo $Ham(k)$ se tiene que es del tipo
$\mathcal{C}[n=2^k-1,m]$ y la matriz de control un c\'odigo lineal del tipo
$\mathcal{C}[n,m]$ es de orden $(n-m)\times n$. En el caso de los c\'odigos
lineales $Ham(k)$ se tiene que $n-m=k$ y $n=2^k-1$ entonces tendremos que
$m=2^k-k-1$. De donde se deduce que $Ham(k)$ es un c\'odigo
$\mathcal{C}[2^k-1,2^k-k-1]$, luego su rango es $2^k-k-1$.

\item La distancia m\'{\i}nima, $d_{min}$, de $Ham(k)$ es $3$.\\

Por el teorema $\ref{the:DistMin}$, en la p\'agina $\pageref{the:DistMin}$, se
tiene que $Ham(k)$ corrige errores de peso $1$ luego $d_{min} = 3$.
\end{enumerate}
\begin{flushright}
$\blacksquare$
\end{flushright}
%
\newpage
%
\subsection{Palabras del c\'odigo Ham(k)}

Ya hemos visto como se construyen los c\'odigos de Hamming, vienen determinados
por su matriz de control. Como estos c\'odigos son lineales podemos utilizar su
matriz de control para calcular las palabras que forman el c\'odigo.\\

El c\'odigo de Hamming $Ham(3)$ utiliza siete bits para transmitir
informaci\'on, de los cuales tres son de control. Utilizando la matriz de
control en forma est\'andar tendremos que el c\'odigo ser\'a:
\begin{displaymath}
Ham(3)\subset \mathbb{F}^{^7}_2 =
\{\ (x_1,x_2,x_3,x_4,c_1,c_2,c_3)\ |\ x_i,c_j \in \mathbb{F}_2 \ \}
\end{displaymath}
Para determinar las palabras del c\'odigo utilizaremos las ecuaciones
impl\'{\i}citas del mismo:
\begin{displaymath}
\left( \begin{array}{ccccccc}
1&0&1&1&1&0&0\\
1&1&1&0&0&1&0\\
0&1&1&1&0&0&1
\end{array} \right) \cdot
\left( \begin{array}{c}
x_1\\
x_2\\
x_3\\
x_4\\
c_1\\
c_2\\
c_3
\end{array} \right) =
\left( \begin{array}{c}
0\\
0\\
0
\end{array} \right)
\end{displaymath}
%
%
\begin{figure}[!h]
\begin{eqnarray*}
c_1&=&x_1+x_3+x_4\\
c_2&=&x_1+x_2+x_3\\
c_3&=&x_2+x_3+x_4
\end{eqnarray*}
\caption{Ecuaciones impl\'{\i}citas del c\'odigo $Ham(3)$.}\label{fig:ImpHamm3}
\end{figure}
Como sabemos que el c\'odigo es binario y utiliza cuatro bits para transmitir
informaci\'on tendremos que el c\'odigo $Ham(3)$ tiene $2^4=16$ palabras.\\

Todas las palabras del c\'odigo se pueden calcular a partir de las ecuaciones
impl\'{\i}citas del c\'odigo, figura $\ref{fig:ImpHamm3}$.\\

Para calcular las palabras de cualquier c\'odigo $Ham(k)$ basta con proceder de
la misma manera que hemos hecho con el c\'odigo $Ham(3)$.

\subsection{La matriz generadora de Ham(k)}

El c\'odigo lineal $Ham(k)$ es un c\'odigo del tipo
$\mathcal{C}[2^k-1,2^k-k-1]$, como ya habiamos visto. Luego su matriz generadora
ser\'a una matriz de orden $(2^k-1)\times(2^k-k-1)$.\\

La matriz generadora la podemos dividir en dos partes. La primera parte ser\'a
la matriz identidad de orden $(2^k-k-1)\times(2^k-k-1)$. Mientras que la
segunda parte ser\'a una matriz de orden $k\times(2^k-k-1)$.\\

Cada columna de la matriz generadora, siempre en forma est\'andar, es una base
de $Ham(k)$ de tal forma que la primera matriz en la que hemos subdividido la
matriz generadora sea una matriz identidad. Una vez calculadas la ecuaciones
impl\'{\i}citas de $Ham(k)$ podemos calcular una base para tener la matriz
generadora en forma est\'andar, con lo cual obtenemos que la matriz generadora,
en forma est\'andar, para $Ham(3)$ ser\'a:
\begin{displaymath}
\left( \begin{array}{cccc}
1&0&0&0\\
0&1&0&0\\
0&0&1&0\\
0&0&0&1\\
1&0&1&1\\
1&1&1&0\\
0&1&1&1\\
\end{array} \right)
\end{displaymath}
