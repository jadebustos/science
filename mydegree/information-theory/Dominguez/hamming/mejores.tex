%
% MEJORES CODIGOS
%

\section{Ham(k) es el mejor c\'odigo con par\'ametros n, m y d}

El c\'odigo $Ham(k)$ es $1$-perfecto, luego no existe ning\'un c\'odigo con
$m$ mayor que este c\'odigo. Luego su raz\'on, $\frac{m}{n}$, es la mayor que
se puede dar para c\'odigos con longitud $n$ y distancia m\'{\i}nima $3$.

\subsection{Ham(3) vs triple control}

Nos referiremos al c\'odigo de triple control como $CTC$. Los sindromes
de ambos c\'odigos son de tres bits.\\

\begin{table}[!h]
\begin{center}
\begin{tabular}{|c|c|c|}
\hline
Sindromes& Errores $CTC$& Errores $Ham(3)$ \\
\hline
$000$&$000000$&$0000000$\\
\hline
$100$&$000100$&$0000100$\\
\hline
$010$&$000010$&$0000010$\\
\hline
$001$&$000001$&$0000001$\\
\hline
$110$&$100000$&$1000000$\\
\hline
$101$&$010000$&$0001000$\\
\hline
$011$&$001000$&$0100000$\\
\hline
$111$&$100001$&$0010000$\\
\hline
\end{tabular}
\end{center}
\caption{Sindromes de $Ham(3)$ y $CTC$.}\label{tab:Sindromes}
\end{table}
%
\newpage
%
Con el c\'odigo $CTC$ podemos corregir todos los errores de peso uno y adem\'as
un s\'olo error de peso dos. ?`Como elegir el error de peso dos que vamos a
corregir?. La \'unica forma en la que la elecci\'on de un error de peso dos nos
dar\'{\i}a una correcci\'on fiable ser\'{\i}a cuando en el canal de
transmisi\'on unicamente se comet\'{\i}era un error de peso dos de todos los
posibles errores de peso dos, algo poco probable.\\

Por el contrario en el $Ham(3)$ se corrigen todos los errores de peso menor o
igual que uno y ning\'un error m\'as, es decir, cualquier palabra en la que
se haya cometido un error se puede transformar en una palabra mediante el
cambio de un \'unico bit. El cambio de este bit estar\'{\i}a determinado por
el sindrome de la palabra recibida.
%
\newpage
%
\begin{proposicion}
\ \\
Para cada palabra $v\in \mathbb{F}^{^n}_2$ con $n=2^k-1$ existe una \'unica
palabra $u\in Ham(k)$ tal que $d(u,v)\leq 1$.
\end{proposicion}
\underline{\textbf{Demostraci\'on}}:\\
\begin{itemize}
\item Veamos la existencia.\\

Sea $v\in \mathbb{F}^{^n}_2$. $H_k\cdot v^t=0$ s\'{\i} y s\'olo s\'{\i}
$v\in Ham(k)$. En caso contrario el sindrome ser\'{\i}a una palabra no nula
de longitud $k$, y como todas las palabras de longitud $k$ aparecen en la
tabla de sindromes\footnote{Se puede ver en la tabla $\ref{tab:Sindromes}$
donde $k=3$.} su sindrome se corresponder\'a con un $e_i$, donde $e_i$ denota
a aquella palabra en la que todos sus bits son nulos excepto el $i$-\'esimo
que es uno. Luego se tendr\'a que $u=v-e_i$ y $u\in Ham(k)$.
\item Veamos la unicidad.\\

La unicidad se deduce de que la distancia m\'{\i}nima de los c\'odigos $Ham(k)$
es $3$, es decir, detecta y corrige todos los errores de peso uno. Que es el
error que estamos corrigiendo.
\end{itemize}
\begin{flushright}
$\blacksquare$
\end{flushright}

Para la transmisi\'on de un mensaje por un canal binario y sim\'etrico con 
probabilidad de error $p=\frac{1}{1000}$ tenemos que si queremos transmitir un
mensaje de $10000$ bits la probabilidad de transmisi\'on correcta ser\'a:
\begin{displaymath}
((0.999)^7+7\cdot(0.001)\cdot(0.999)^6)^{\frac{10000}{4}}\simeq 0.949
\end{displaymath}

Para el c\'odigo de triple control est\'a probabilidad era $0.951$. Luego la
probabilidad de transmisi\'on correcta en ambos c\'odigos es muy similar.\\

La raz\'on de $Ham(3)$ es $\frac{4}{7}\simeq 0.5714$ mientras que la raz\'on
del c\'odigo de triple control es $\frac{3}{6} = 0.5$. Esto nos indica que es
m\'as eficiente $Ham(3)$ ya que para mandar el mismo mensaje utilizar\'a menos
bits que el c\'odigo de triple control.\\

Para una transmisi\'on de $10000$ bits el c\'odigo de triple control mandar\'a
$20000$ bits, mientras que el $Ham(3)$ mandar\'a $17500$.
