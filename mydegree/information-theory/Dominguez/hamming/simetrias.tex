%
% SIMETRIAS EN LOS CODIGOS DE HAMMING
%

\section{Simetrias en Ham(3)}

\begin{definicion}[Palabras complementarias]
\ \\
Para una palabra binaria $u$ se define su \textbf{``palabra complementaria''}
$v$ como aquella que se obtiene al cambiar ceros por unos y unos por ceros en la
palabra $u$.
\end{definicion}

\begin{figure}[!h]
\begin{center}
\begin{tabular}{cc}
$0000000$&$1111111$\\
$1000110$&$0111001$\\
$0100011$&$1011100$\\
$0010111$&$1101000$\\
$0001101$&$1110010$\\
$1100101$&$0011010$\\
$1010001$&$0101110$\\
$1001011$&$0110100$\\
\end{tabular}
\end{center}
\caption{Palabras del c\'odigo $Ham(3)$.}\label{fig:PalHam3}
\end{figure}
En la tabla $\ref{fig:PalHam3}$ se puede ver que en el c\'odigo $Ham(3)$ esta
una palabra y su complementaria.

\begin{definicion}[Rotaciones c\'{\i}clicas]
\ \\
Dada una palabra $u$ \textbf{``rotar c\'{\i}clicamente''} $u$ significa
desplazar hacia la derecha todos sus bits y poner como primer bit su \'ultimo
bit. S\'{\i} tenemos que $u=a_1a_2a_3a_4a_5a_6a_7a_8$ entonces al rotar
c\'{\i}clicamente $u$ obtendremos la siguiente palabra:
\begin{displaymath}
a_8a_1a_2a_3a_4a_5a_6a_7
\end{displaymath}
\end{definicion}

Utilizando la tabla $\ref{fig:PalHam3}$ podemos ver que dada una palabra 
cualquiera de $Ham(3)$ su rotaci\'on c\'{\i}clica tambi\'en pertenece a
$Ham(3)$. Entonces es obvio que rotar c\'{\i}clicamente $k$ veces una palabra
en $Ham(3)$ produce otra palabra de $Ham(3)$.

\begin{definicion}[C\'odigos c\'{\i}clicos]
\ \\
Reciben el nombre de c\'odigos c\'{\i}clicos aquellos c\'odigos en los que la
rotaci\'on c\'{\i}clica de cualquier palabra del c\'odigo produce otra palabra
del c\'odigo.
\end{definicion}

De todo esto se deduce la siguiente proposici\'on:
\begin{proposicion}
\ \\
El c\'odigo $Ham(3)$ es un c\'odigo c\'{\i}clico en el cual las palabras
complementarias de una del c\'odigo pertenecen al c\'odigo.
\end{proposicion}
