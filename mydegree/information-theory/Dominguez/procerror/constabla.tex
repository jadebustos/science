%
% CONSTRUCCION DE LA TABLA DE UN CODIGO
%

%
\newpage
%
\section{Construcci\'on de la tabla de un c\'odigo lineal}

En la tabla estar\'an los $q^n$ elementos de $\mathbb{F}^{^n}_q$. La tabla
la organizaremos por filas y columnas.\\ \\
%
Como $\mathcal{C}$ tiene $q^m$ elementos, es de dimensi\'on $m$, organizaremos
la tabla con $q^{n-m}$ filas y $q^m$ columnas. De esta forma en la tabla
tendremos: $$q^{n-m}\cdot q^m=q^{n-m+m}=q^n$$ elementos.\\ \\
%
La justificaci\'on de elegir esta estructura para la tabla es que de esta manera
pondremos como primera fila a todos los elementos del c\'odigo.\\

Para referirnos individualmente a cada elemento de la tabla lo haremos
utilizando la siguiente notaci\'on $Tb(\mathcal{C}[n,m])_{i,j}$ donde 
$i=0,\dots,q^{n-m}-1$ nos indica la fila y $j=0,\dots,q^m-1$ nos
indica la columna.
\begin{definicion}[Fila $0$ de la tabla]
\ \\
La fila $0$ de la tabla estar\'a formada por todos los elementos del c\'odigo
$\mathcal{C}[n,m]$ en cualquier orden, pero con la condici\'on de que el primer
elemento sea el $0$.
\end{definicion}
%
%
\begin{definicion}[Fila $i$ de la tabla]
\ \\
Supondremos que ya hemos construido todas las columnas hasta la $i-1$ inclusive.
Entonces la construcci\'on de la fila $i$ la haremos en dos pasos:
\begin{enumerate}
\item Eligiremos una palabra de $\mathbb{F}^{^n}_q$ que no este en ninguna de
las filas anteriores y la situaremos en $Tb(\mathcal{C}[n,m])_{i,0}$.
\item Para calcular el resto de los elementos de la fila lo haremos de la
siguiente forma:
\begin{displaymath}
Tb(\mathcal{C}[n,m])_{i,j} = Tb(\mathcal{C}[n,m])_{i,0}+
Tb(\mathcal{C}[n,m])_{0,j}\qquad j=1,\dots,q^m-1
\end{displaymath}
\end{enumerate}
\end{definicion}
%
%
\begin{definicion}[Tabla est\'andar de un c\'odigo]
\ \\
Llamaremos \textbf{``tabla est\'andar''} de un c\'odigo a una tabla construida
de la manera anterior.
\end{definicion}
