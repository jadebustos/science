%
% CONSTRUCCION DE LA TABLA DE UN CODIGO
%

\section{Estructura de una tabla est\'andar}

Ya hemos visto como se construye la tabla de un c\'odigo, pero la 
cons\-tru\-cci\'on que hemos hecho no nos asegura que en la tabla esten todos 
los elementos de $\mathbb{F}^{^n}_q$.
%
%
\begin{teorema}
\ \\
En una tabla est\'andar para un c\'odigo $\mathcal{C}[n,m]$ aparecen todas las
palabras de $\mathbb{F}^{^n}_q$ una sola vez.
\end{teorema}
\underline{\textbf{Demostraci\'on}}:\\
Dados dos elementos cualesquiera de la tabla:
\begin{eqnarray*}
Tb(\mathcal{C}[n,m])_{i,k}&=&Tb(\mathcal{C}[n,m])_{i,0}+
Tb(\mathcal{C}[n,m])_{0,k}\\
Tb(\mathcal{C}[n,m])_{j,l}&=&Tb(\mathcal{C}[n,m])_{j,0}+
Tb(\mathcal{C}[n.m])_{0,l}
\end{eqnarray*}
S\'{\i} ambos elementos son iguales entonces su diferencia es $0$, y como
$\mathcal{C}[n,m]$ es lineal se tiene que $0\in \mathcal{C}[n,m]$ entonces,
y debido tambi\'en a la linealidad del c\'odigo, ambos elementos deben
pertenecer al c\'odigo, con lo cual $i=j$. Debido a la construcci\'on de la
tabla tendremos que $i=j=0$ ya que los elementos del c\'odigo est\'an en la
primera fila. Teniendo en cuenta esto tendremos que:
\begin{displaymath}
Tb(\mathcal{C}[n,m])_{i,k}-Tb(\mathcal{C}[n,m])_{j,l} =
Tb(\mathcal{C}[n,m])_{0,k}-Tb(\mathcal{C}[n,m])_{0,l}
\end{displaymath}
Tenemos dos posibilidades:
\begin{itemize}
\item $k=l$ entonces como $i=j$ tendremos que ambos elementos son el mismo. Es
decir ambos elementos ocupan el mismo lugar en la tabla, son el mismo. Luego no
aparece repetido en la tabla.
\item $k\neq l$ no se puede dar ya que estamos en un c\'odigo lineal y si la
diferencia de dos palabras del c\'odigo es cero entonces ambas palabras han de
ser la misma, con lo cual $k=l$.
\end{itemize}
El n\'umero de elementos de $\mathbb{F}^{^n}_q$ es $q^n$.\\ \\
% 
En la tabla tenemos $q^{n-m}$ filas y $q^m$ columnas entonces el n\'umero de
elementos de la tabla es $q^{n-m}\cdot q^m=q^{n-m+m}=q^n$. Es decir en la tabla
hay tantos elementos como en $\mathbb{F}^{^n}_q$ y como en la tabla no se
repiten elementos entonces tenemos que est\'an todos los elementos.
\begin{flushright}
$\blacksquare$
\end{flushright}
%
%
\begin{lema}[Diferencias horizontales]\label{lem:DifHorizontales}
\ \\
Dado un c\'odigo lineal $\mathcal{C}[n,m]$, con tabla $Tb(\mathcal{C}[n,m])$,
dos elementos de la misma fila difieren en un elemento del c\'odigo.
\end{lema}
\underline{\textbf{Demostraci\'on}}:\\
Dos elementos cualesquiera de la fila $i$-\'esima ser\'an de la forma:
\begin{eqnarray*}
Tb(\mathcal{C}[n,m])_{i,k}&=&Tb(\mathcal{C}[n,m])_{i,0}+
Tb(\mathcal{C}[n,m])_{0,k}\\
Tb(\mathcal{C}[n,m])_{i,l}&=&Tb(\mathcal{C}[n,m])_{i,0}+
Tb(\mathcal{C}[n,m])_{0,l}
\end{eqnarray*}
La diferencia entre ambos elementos es:
\begin{displaymath}
Tb(\mathcal{C}[n,m])_{i,k}-Tb(\mathcal{C}[n,m])_{i,l}=
Tb(\mathcal{C}[n,m])_{0,k}-Tb(\mathcal{C}[n,m])_{0,l}
\end{displaymath}
Por construcci\'on los elementos de la fila cero son elementos del c\'odigo, y,
como este es lineal la diferencia de dos elementos del c\'odigo es otro 
elemento del c\'odigo.
\begin{flushright}
$\blacksquare$
\end{flushright}
%
%
\begin{lema}[Diferencias verticales]\label{lem:DifVerticales}
\ \\
Dado un c\'odigo lineal $\mathcal{C}[n,m]$, con tabla $Tb(\mathcal{C}[n,m])$,
dos elementos distintos de la misma columna difieren en una palabra que no
pertenece al c\'odigo.
\end{lema}
\underline{\textbf{Demostraci\'on}}:\\
Dos elementos cualesquiera, $i\neq j$, de la columna $k$-\'esima ser\'an de
la forma:
\begin{eqnarray*}
Tb(\mathcal{C}[n,m])_{i,k}&=&Tb(\mathcal{C}[n,m])_{i,0}+
Tb(\mathcal{C}[n,m])_{0,k}\\
Tb(\mathcal{C}[n,m])_{j,k}&=&Tb(\mathcal{C}[n,m])_{j,0}+
Tb(\mathcal{C}[n,m])_{0,k}
\end{eqnarray*}
La diferencia entre ambos elementos es:
\begin{displaymath}
Tb(\mathcal{C}[n,m])_{i,k}-Tb(\mathcal{C}[n,m])_{j,k}=
Tb(\mathcal{C}[n,m])_{i,0}-Tb(\mathcal{C}[n,m])_{j,0}
\end{displaymath}
Los elementos del c\'odigo est\'an en la fila cero y
$Tb(\mathcal{C}[n,m])_{i,0}$, $Tb(\mathcal{C}[n,m])_{j,0}$ no pueden, ambos,
estar en la fila cero. Luego al menos uno de ellos no pertenecer\'a al c\'odigo,
y al ser un c\'odigo lineal cualquier operaci\'on que realizemos entre dos
palabras, con una que no pertenezca al c\'odigo, ser\'a otra palabra que no
estar\'a en el c\'odigo.
\begin{flushright}
$\blacksquare$
\end{flushright}
%
%
\begin{proposicion}[Propiedades de la tabla est\'andar]\label{pro:Propiedades}
\ \\
Sea $\mathcal{C}[n,m]$ un c\'odigo lineal, entonces se tiene que:
\begin{eqnarray*}
Tb(\mathcal{C}[n,m])_{i,k}-Tb(\mathcal{C}[n,m])_{i,l}&=&
Tb(\mathcal{C}[n,m])_{j,k}-Tb(\mathcal{C}[n,m])_{j,l}\quad \forall \ i,j,k,l\\
Tb(\mathcal{C}[n,m])_{i,k}-Tb(\mathcal{C}[n,m])_{j,k}&=&
Tb(\mathcal{C}[n,m])_{i,l}-Tb(\mathcal{C}[n,m])_{j,l}\quad \forall \ i,j,k,l
\end{eqnarray*}
donde $Tb(\mathcal{C}[n,m])$ es la tabla est\'andar del c\'odigo.
\end{proposicion}
\underline{\textbf{Demostraci\'on}}:\\
Por construcci\'on de la tabla tenemos:
\begin{eqnarray*}
Tb(\mathcal{C}[n,m])_{i,k}&=& Tb(\mathcal{C}[n,m])_{i,0}+
Tb(\mathcal{C}[n,m])_{0,k}\\
Tb(\mathcal{C}[n,m])_{i,l}&=& Tb(\mathcal{C}[n,m])_{i,0}+
Tb(\mathcal{C}[n,m])_{0,l}\\
Tb(\mathcal{C}[n,m])_{j,k}&=& Tb(\mathcal{C}[n,m])_{j,0}+
Tb(\mathcal{C}[n,m])_{0,k}\\
Tb(\mathcal{C}[n,m])_{j,l}&=& Tb(\mathcal{C}[n,m])_{j,0}+
Tb(\mathcal{C}[n,m])_{0,l}
\end{eqnarray*}
Basta con calcular las diferencias:
\begin{eqnarray*}
Tb(\mathcal{C}[n,m])_{i,k}-Tb(\mathcal{C}[n,m])_{i,l}&=&
Tb(\mathcal{C}[n,m])_{0,k}-Tb(\mathcal{C}[n,m])_{0,l}\\
Tb(\mathcal{C}[n,m])_{j,k}-Tb(\mathcal{C}[n,m])_{j,l}&=&
Tb(\mathcal{C}[n,m])_{0,k}-Tb(\mathcal{C}[n,m])_{0,l}
\end{eqnarray*}
Ambas diferencias son iguales. Y:
\begin{eqnarray*}
Tb(\mathcal{C}[n,m])_{i,k}-Tb(\mathcal{C}[n,m])_{j,k}&=&
Tb(\mathcal{C}[n,m])_{i,0}-Tb(\mathcal{C}[n,m])_{j,0}\\
Tb(\mathcal{C}[n,m])_{i,l}-Tb(\mathcal{C}[n,m])_{j,l}&=&
Tb(\mathcal{C}[n,m])_{i,0}-Tb(\mathcal{C}[n,m])_{j,0}
\end{eqnarray*}
Ambas diferencias son iguales.
\begin{flushright}
$\blacksquare$
\end{flushright}
%
%
Ya hemos visto como se construye la tabla est\'andar de un c\'odigo lineal 
$\mathcal{C}[n,m]$, y que en dicha tabla estan todos los elementos de
$\mathbb{F}^{^n}_q$.\\

Adem\'as de esto podemos ver la relaci\'on que guardan entre s\'{\i} los 
elementos de cada fila.
%
\newpage
%
\begin{teorema}\label{the:ElementosFila}
\ \\
Sea $\mathcal{C}[n,m]$ un c\'odigo lineal y $Tb(\mathcal{C}[n,m])$ su tabla en
forma est\'andar. Sean $Tb(\mathcal{C}[n,m])_{i,k}$ y
$Tb(\mathcal{C}[n,m])_{j,l}$ dos elementos de la tabla, entonces est\'an en la
misma fila s\'{\i} y s\'olo s\'{\i} su diferencia es una palabra del c\'odigo.
\end{teorema}
\underline{\textbf{Demostraci\'on}}:\\
$\Rightarrow |$ Es el lema $\ref{lem:DifHorizontales}$ en la p\'agina
$\pageref{lem:DifHorizontales}$.\\ \\
%
$\Leftarrow |$ Sean dos elementos cualesquiera de $Tb(\mathcal{C}[n,m])$:
\begin{eqnarray*}
Tb(\mathcal{C}[n,m])_{i,k}&=&Tb(\mathcal{C}[n,m])_{i,0}+
Tb(\mathcal{C}[n,m])_{0,k}\\
Tb(\mathcal{C}[n,m])_{j,l}&=&Tb(\mathcal{C}[n,m])_{j,0}+
Tb(\mathcal{C}[n,m])_{0,l}
\end{eqnarray*}
tal que su diferencia sea una palabra del c\'odigo.\\ \\
%
Su diferencia ser\'a:
\begin{equation} \label{eq:Diferencia}
Tb(\mathcal{C}[n,m])_{i,k}-Tb(\mathcal{C}[n,m])_{j,l} = x + y
\end{equation}
donde:
\begin{displaymath}
x=Tb(\mathcal{C}[n,m])_{0,k}-Tb(\mathcal{C}[n,m])_{0,l}\quad e\quad
y=Tb(\mathcal{C}[n,m])_{i,0}-Tb(\mathcal{C}[n,m])_{j,0}
\end{displaymath}
$x$ es una palabra del c\'odigo ya que es la diferencia de dos palabras
del c\'odigo. Es la diferencia de dos palabras de la fila cero.\\ \\
%
Despejando $y$ en $(\ref{eq:Diferencia})$ tenemos:
\begin{displaymath}
y=(Tb(\mathcal{C}[n,m])_{i,k}-Tb(\mathcal{C}[n,m])_{j,l})-x
\end{displaymath}
luego como $\mathcal{C}[n,m]$ es un c\'odigo lineal e $y$ es diferencia de dos
palabras del c\'odigo se tiene que $y\in \mathcal{C}[n,m]$.\\

Supongamos ahora que $i\neq j$ entonces como $Tb(\mathcal{C}[n,m])_{i,0}$ y
$Tb(\mathcal{C}[n,m])_{j,0}$ pertenecen a la misma columna y distinta fila
su diferencia no pertenecer\'a al c\'odigo seg\'un el lema
$\ref{lem:DifVerticales}$. Luego $y\notin \mathcal{C}[n,m]$ por ser la
diferencia de una palabra del c\'odigo y otra que no lo es, llegamos a 
contradicci\'on ya que $y$ s\'{\i} que pertenece al c\'odigo. Entoces $i=j$
luego ambas palabras est\'an en la misma fila.
\begin{flushright}
$\blacksquare$
\end{flushright}
%
\newpage
%
La interpretaci\'on matem\'atica de este teorema es:
\begin{corolario}
\ \\
Sea un c\'odigo lineal $\mathcal{C}[n,m]\subset \mathbb{F}^{^n}_q$. El
 $\mathbb{F}_q$-espacio vectorial $\mathbb{F}^{^n}_q/\mathcal{C}[n,m]$ tiene
tantas clases de equivalencia como filas tiene su tabla est\'andar,
$Tb(\mathcal{C}[n,m])$. Cada clase de equivalencia estar\'a formada por los
elementos de una fila. 
\end{corolario}
\underline{\textbf{Demostraci\'on}}:\\
Es inmediata a partir del teorema $\ref{the:ElementosFila}$ y del hecho de que
dos elementos de la misma clase de equivalencia m\'odulo $\mathcal{C}[n,m]$ 
difieren en un elemento de $\mathcal{C}[n,m]$.
\begin{flushright}
$\blacksquare$
\end{flushright}
\begin{observacion}\ \\
\begin{itemize}
\item Dos filas tienen un elemento en com\'un s\'{\i} y s\'olo s\'{\i} son la
misma fila. Se deduce del hecho de que las clases de equivalencia son disjuntas
entre s\'{\i} y que cada fila forma una clase de equivalencia.
\end{itemize}
\end{observacion}
A partir de ahora en la tabla de un c\'odigo est\'andar eligiremos como primer
elemento de cada fila el elemento que menor peso tenga de la fila.
