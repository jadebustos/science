%
% PROCESAMIENTO DE ERROR
%

\section{Procesamiento de errores}

Un error es la diferencia entre la palabra recibida y la transmitida. En los
c\'odigos lineales, gracias a su estructrura algebraica, ese error lo podemos
expresar como la diferencia entre dos palabras. Si se transmiti\'o la palabra
$u$ y se recibio la palabra $v$ el error cometido en la transmisi\'on del
mensaje lo podemos expresar como $e=v-u$.
\begin{definicion}[Error cometido para c\'odigos lineales]
\ \\
Sea $\mathcal{C}[n,m]$ un c\'odigo lineal. S\'{\i} se transmiti\'o la palabra
$u$ y se recibi\'o la palabra $v$ diremos que el error que ocurri\'o en la
transmisi\'on es $e=v-u$.
\end{definicion}
Seg\'un esta definici\'on tendremos:
\begin{itemize}
\item La palabra recibida ser\'a: $v=u+e$.
\item La palabra transmitida ser\'a: $u=v-e$.
\end{itemize}
%
%
\begin{proposicion} \label{pro:ProcesarError}
\ \\
Sea $\mathcal{C}[n,m]$ un c\'odigo lineal y sea $P$ un procesador de error para
dicho c\'odigo. S\'{\i} se recibe una palabra $w$ entonces $P$ corregir\'a $w$
como $u=w-e$, donde $e$ es el primer elemento de la fila en la que se encuentra
$w$.
\end{proposicion}
\underline{\textbf{Desmostraci\'on}}:\\
$w=Tb(\mathcal{C}[n,m])_{i,k}$ entonces $e=Tb(\mathcal{C}[n,m])_{i,0}$.\\ \\
%
Por la proposici\'on $\ref{pro:Propiedades}$, en la p\'agina
$\pageref{pro:Propiedades}$, tendremos que:
\begin{displaymath}
Tb(\mathcal{C}[n,m])_{i,k}-Tb(\mathcal{C}[n,m])_{0,k}=
Tb(\mathcal{C}[n,m])_{i,0}-Tb(\mathcal{C}[n,m])_{0,0}
\end{displaymath}
donde $Tb(\mathcal{C}[n,m])_{0,k} = u$ y por contrucci\'on de la tabla tenemos
que: $$Tb(\mathcal{C}[n,m])_{0,0} = (0,\stackrel{n)}\dots,0)$$
De donde se deduce que $w-u=e-0$ entoces $u=w-e$.
\begin{flushright}
$\blacksquare$
\end{flushright}
Seg\'un esta proposici\'on un procesador de error corrige errores en funci\'on
del primer elemento de cada fila, luego los errores que corrige un procesador
de errores son aquellos que est\'an en primer lugar en cada fila.
%
\newpage
%
\begin{ejemplo}
\ \\
Sea $\mathcal{C}[m,m]$ un c\'odigo lineal binario con tabla
$Tb(\mathcal{C}[n,m])$:
\begin{itemize}
\item $Tb(\mathcal{C}[n,m])_{1,0}=100000$ nos indica que con esta tabla se
pueden corregir errores de peso uno en los que el error est\'e en el primer bit.
\item $Tb(\mathcal{C}[n,m])_{2,0}=010000$ nos indica que con esta tabla se
pueden corregir errores de peso uno en los que el error est\'e en el segundo
bit.
\item $Tb(\mathcal{C}[n,m])_{3,0}=001000$ nos indica que con esta tabla se
pueden corregir errores de peso uno en los que el error est\'e en el tercer bit.
\item $Tb(\mathcal{C}[n,m])_{4,0}=000100$ nos indica que con esta tabla se
pueden corregir errores de peso uno en los que el error est\'e en el cuarto bit.
\item $Tb(\mathcal{C}[n,m])_{5,0}=000010$ nos indica que con esta tabla se
pueden corregir errores de peso uno en los que el error est\'e en el quinto
bit.
\item $Tb(\mathcal{C}[n,m])_{6,0}=000001$ nos indica que con esta tabla se
pueden corregir errores de peso uno en los que el error est\'e en el \'ultimo
bit.
\item $Tb(\mathcal{C}[n,m])_{7,0}=100001$ nos indica que con esta tabla se
pueden corregir errores de peso dos en los que el error est\'e en el primer y
\'ultimo bit.
\end{itemize}
\begin{flushright}
$\blacksquare$
\end{flushright}
\end{ejemplo}
%
%
\begin{proposicion}\label{pro:Cercania}
\ \\
Sea $\mathcal{C}[n,m]\subset \mathbb{F}^{^n}_q$ un c\'odigo lineal
con una tabla est\'andar
$Tb(\mathcal{C}[n,m])$ en la que el primer elemento de cada fila es el de
menor peso de dicha fila. Sea $Tb(\mathcal{C}[n,m])_{i,k}$ una palabra
cualquiera de la tabla entonces se tiene que:
\begin{displaymath}
d(Tb(\mathcal{C}[n,m])_{i,k},Tb(\mathcal{C}[n,m])_{0,k})\leq
d(Tb(\mathcal{C}[n,m])_{i,k},Tb(\mathcal{C}[n,m])_{0,j}) 
\end{displaymath}
donde $j=0,\dots,q^{m}-1$.
\end{proposicion}
\underline{\textbf{Demostraci\'on}}:\\
Tenemos por definici\'on de distancia que:
\begin{displaymath}
d(Tb(\mathcal{C}[n,m])_{i,k},Tb(\mathcal{C}[n,m])_{0,j}) =
w(Tb(\mathcal{C}[n,m])_{i,k}-Tb(\mathcal{C}[n,m]_{0,j}))
\end{displaymath}
Seg\'un varia $j$ tenemos que $T(\mathcal{C}[n,m])_{0,j}$ varia en
$\mathcal{C}[n,m]$, luego por las propiedades de la tabla tendremos que
$Tb(\mathcal{C}[n,m])_{i,k}-Tb(\mathcal{C}[n,m])_{0,j}$ varia en el
conjunto formado por los elementos de la fila $i$. Luego el problema de
encontrar el m\'{\i}nimo de:
\begin{displaymath}
w(Tb(\mathcal{C}[n,m])_{i,k}-Tb(\mathcal{C}[n,m])_{0,j})
\end{displaymath}
equivale a encontrar la palabra de m\'{\i}nimo peso de la fila $i$, pero por
cons\-trucci\'on dicha palabra es $Tb(\mathcal{C}[n,m])_{i,0}$.
\begin{displaymath}
\min_{j=0,\dots,q^{m}-1}\{\ w(Tb(\mathcal{C}[n,m])_{i,k}-
Tb(\mathcal{C}[n,m])_{0,j})\ \} = Tb(\mathcal{C}[n,m]_{i,0})
\end{displaymath}
Luego:
\begin{displaymath}
d(Tb(\mathcal{C}[n,m])_{i,k},Tb(\mathcal{C}[n,m])_{0,j})\geq
w(Tb(\mathcal{C}[n,m])_{i,0})\quad j=0,\dots,q^{m}-1
\end{displaymath}
Pero como $Tb(\mathcal{C}[n,m])_{i,k}=Tb(\mathcal{C}[n,m])_{i,0}+
Tb(\mathcal{C}[n,m])_{0,k}$, o lo que es lo mismo $Tb(\mathcal{C}[n,m])_{i,0}=
Tb(\mathcal{C}[n,m])_{i,k}-Tb(\mathcal{C}[n,m])_{0,k}$ tenemos entonces:
\begin{displaymath}
d(Tb(\mathcal{C}[n,m])_{i,k},Tb(\mathcal{C}[n,m])_{0,j})\geq
d(Tb(\mathcal{C}[n,m])_{i,k},Tb(\mathcal{C}[n,m]_{0,k}))
\end{displaymath}
para $j=0,\dots,q^{m}-1$.
\begin{flushright}
$\blacksquare$
\end{flushright}

\subsection{Correcci\'on de errores}

Seg\'un la proposici\'on $\ref{pro:ProcesarError}$ para corregir un error
tendremos que realizar los siguientes pasos:
\begin{itemize}
\item Localizar la palabra recibida en una tabla est\'andar.
\item Suponiendo que la palabra recibida es $Tb(\mathcal{C}[6,3])_{i,k}$ 
entonces tomaremos como palabra transmitida la palabra que est\'e en la misma
columna pero en la fila cero, $Tb(\mathcal{C}[6,3])_{0,k}$. Siendo el error
cometido el primer elemento de la fila en la que se encuentre la palabra
recibida, $Tb(\mathcal{C}[6,3])_{i,0}$.
\item El n\'umero de errores que corrige viene dado por el n\'umero de filas
que posee la tabla del c\'odigo, y los patrones de error que corrige son los
primeros elementos de cada fila.
\item La proposici\'on $\ref{pro:Cercania}$ nos indica que corregiremos siempre
por la palabra m\'as cercana del c\'odigo, seg\'un la distancia de Hamming.
\end{itemize}
