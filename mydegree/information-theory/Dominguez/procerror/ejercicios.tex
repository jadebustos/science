%
% EJERCICIOS
%

\section{Ejercicios}
%
% SINDROMES
%
\begin{ejercicio}
\ \\
Programar la correcci\'on de errores de un c\'odigo lineal utilizando
sindromes.
\end{ejercicio}
\underline{\textbf{Soluci\'on}}:\\
Se ha programado el funcionamiento de un procesador de error para el 
c\'odigo de triple control utilizando la tabla de sindromes.\\

El programa est\'a escrito en lenguaje $C$ y tanto el c\'odigo como los
binarios para \emph{MS-DOS} estan el el disco adjunto.\\ \\
%
El funcionamiento del programa es el siguiente:
\begin{itemize}
\item El programa pide, por teclado, una palabra de seis digitos bin\'arios. 
Para prevenir el funcionamiento an\'omalo del programa se ha recurrido a 
admitir una cadena, de hasta $80$ caracteres, como entrada. El programa
seleccionar\'a los seis primeros caracteres num\'ericos y los convertir\'a a
$\mathbb{F}_2$, considerando a estos como la palabra recibida.
\item El programa conoce, a priori, las palabras del c\'odigo, su matriz de
control y los patrones de error.
\item Seguidamente calcular\'a el sindrome de la palabra recibida, as\'{\i}
como los sindromes de todas palabras de peso uno, errores, y el de una palabra
de peso dos, para detectar cuando ocurre un error de peso dos.
\item S\'{\i} el error es peso menor o igual que uno el programa da la palabra
transmitida originalmente.
\item S\'{\i} el error es de peso dos el programa indica que ocurrio un error
de peso dos, y no nos da la palabra que se transmiti\'o originalmente. Para
poder calcular esta palabra ser\'{\i}a necesario tener alguna hip\'otesis
adicional sobre que error de peso dos es el que m\'as se da en el canal de
transmisi\'on.
\end{itemize}

\begin{flushright}
$\blacksquare$
\end{flushright}
%
% TABLA ESTANDAR
%
\begin{ejercicio}
\ \\
Programar la la correcci\'on de errores de un c\'odigo lineal utilizando una
tabla est\'andar.
\end{ejercicio}
\underline{\textbf{Soluci\'on}}:\\
Se ha programado el funcionamiento de un procesador de error para el 
c\'odigo de triple control utilizando una tabla est\'andar.\\

El programa est\'a escrito en lenguaje $C$ y tanto el c\'odigo como los
binarios para \emph{MS-DOS} estan el el disco adjunto.\\ \\
%
El funcionamiento del programa es el siguiente:
\begin{itemize}
\item El programa pide, por teclado, una palabra de seis digitos bin\'arios. 
Para prevenir el funcionamiento an\'omalo del programa se ha recurrido a 
admitir una cadena, de hasta $80$ caracteres, como entrada. El programa
seleccionar\'a los seis primeros caracteres num\'ericos y los convertir\'a a
$\mathbb{F}_2$, considerando a estos como la palabra recibida.
\item El programa conoce, a priori, las palabras del c\'odigo y los patrones
de error.
\item Seguidamente calcular\'a la tabla est\'andar a partir de las palabras
del c\'odigo y los patrones de error.
\item S\'{\i} el error es peso menor o igual que uno el programa da la palabra
transmitida originalmente.
\item S\'{\i} el error es de peso dos el programa indica que ocurrio un error
de peso dos, y no nos da la palabra que se transmiti\'o originalmente. Para
poder calcular esta palabra ser\'{\i}a necesario tener alguna hip\'otesis
adicional sobre que error de peso dos es el que m\'as se da en el canal de
transmisi\'on.
\end{itemize}

\begin{flushright}
$\blacksquare$
\end{flushright}
