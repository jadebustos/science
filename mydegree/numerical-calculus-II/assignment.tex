\documentclass[a4paper,12pt]{article}

\addtolength{\voffset}{-4cm}
\usepackage[spanish]{babel}

\pagestyle{headings}


\begin{document}

\title{\textbf{Numerical Calculus II homework}}
\author{Jos\'e \'Angel de Bustos P\'erez}
\maketitle

\begin{abstract}
Use the Broyden and Newton-Raphson methods to solve the following.
\end{abstract}

In calculating the shape of a gravity-flow discarge chute that will minimize transit time of discharged granular particles C. Chiarella, W. Charlton and A.W. Roberts solve the following equations by Newton's method:
%
\begin{eqnarray}
  f_n(\theta_1,\dots,\theta_N) & = & \frac{\sin{\theta_{n+1}}}{\upsilon_{n+1}}\cdot (1 - \mu \cdot \omega_{n+1}) - \frac{\theta_n}{\upsilon_n} \cdot (1 - \mu \cdot \omega_n) = 0 \label{eqn:general} \\
  f_N(\theta_1,\dots,\theta_N) & = & \Delta y \cdot \sum_{i=1}^N \tan{\theta_i} - X = 0 \label{eqn:last}
\end{eqnarray}
%
Where:
%
\begin{eqnarray*}
  \upsilon_n^2 & = & v_0^2 + 2\cdot g\cdot n \cdot \Delta y - 2\cdot \mu \cdot \Delta y \cdot \sum_{j=1}^n \frac{1}{\cos{\theta_j}} \\
  \omega_n & = & -\Delta y \cdot \upsilon_n \cdot \sum_{i=1}^N \frac{1}{\upsilon_i^3 \cdot \cos{\theta_i}}
\end{eqnarray*}
%
for each $n=1,\dots,N$.\\

The constant $\upsilon_0$ is the initial velocity of the granular material, $X$ is de $x$-coordinate of the end the chute, $\mu$ is the friction force, $N$ is the number of segments and $g$ = $32.2$ $ft/s^2$ is the gravitational constant. The variable $\theta_i$ is the angle of the $i$-th chute segment.\\

\newpage

Solve (\ref{eqn:general}) and (\ref{eqn:last}) for:
%
\begin{eqnarray}
  \theta & = & (\theta_1,\dots,\theta_N)^t\\
  \mu & = & 0 \\
  X & = & 2 \\
  \Delta y & = & 0.2 \\
  N & = & 20 \\
  \upsilon_0 & = & 0
\end{eqnarray}

Solution will be considered a valid one when:
%
\begin{displaymath}
|| \theta^{k)} - \theta^{k-1)}||_{\infty} < 10^{-2}  
\end{displaymath}
%

\end{document}
