%
% SERIES TEMPORALES
%

\section{Series Temporales}

Las ``\emph{Series Temporales}'' tambi\'en reciben el nombre de ``\emph{Series
Cronol\'ogicas}'' o ``\emph{Series Hist\'oricas}''. Se basan en el estudio del
comportamiento de una determinada variable, observaciones, con el fin de poder
predecir el comportamiento de dicha variable a lo largo del tiempo.

\subsection{?`Para que sirven?}

Intentaremos crear un modelo matem\'atico para predecir la evoluci\'on de una
variable a lo largo del tiempo. Esta variable puede ser de cualquier tipo:

\begin{itemize}
\item \textbf{Econ\'omica}: \emph{\'{\i}ndice de precios al consumo},
\emph{demanda de electricidad}, \emph{producto nacional bruto}, \emph{ventas
de un determinado tipo}, \ldots
\item \textbf{F\'{\i}sica}: \emph{velocidad del viento}, \emph{temperatura en
un proceso}, \emph{concentraci\'on en la atm\'osfera de un contaminante}, \ldots
\item \textbf{Social}: \emph{n\'umero de nacimientos}, \emph{n\'umero de
defunciones}, \emph{votos a un partido pol\'{\i}tico}, \ldots 
\end{itemize}

\newpage

\subsection{?`Que se necesita?}

Necesitaremos disponer de datos sobre la evoluci\'on de la variable que nos 
interesa a intervalos regulares de tiempo (horas, d\'{\i}as, meses, a\~nos,
\ldots). Utilizando estos datos intentaremos ``adivinar'' el comportamiento de
dicha variable en el futuro, suponiendo que el comportamiento de la variable es
uniforme a lo largo del tiempo, es decir que su comportamiento en el
futuro\footnote{O al menos en un futuro no muy lejano.} es similar al 
comportamiento que tuvo dicha variable en el pasado. Es decir que su
comportamiento no experimenta cambios bruscos entre dos espacios de tiempo
proximos.\\

Este tipo de analisis recibe el nombre de \emph{an\'alisis univariante}, el cual
es util para predicciones a corto plazo. Si necesitaramos hacer predicciones a
medio o largo plazo necesitariamos tener en cuenta una serie de variables
relacionadas con la variable que queremos estudiar. Este tipo de an\'alisis
recibe el nombre de \emph{an\'alisis multivariante}.
 
\subsection{Modelo matem\'atico}

El modelo matem\'atico para una \emph{serie temporal} es el concepto de
``\emph{proceso estoc\'astico}''. Supondremos que el valor observado de la serie
en un instante $t$ es una extracci\'on al azar de una variable aleatoria
definida en dicho instante $t$.\\

Una serie de $n$ datos ser\'a una muestra de un vector de $n$ variables
aleatorias ordenadas en el tiempo $(z_1,\ldots,z_n)$. Se denomina
\emph{proceso estoc\'astico} al conjunto de estas variables $\{z_t\}_{t=1}^n$,
y la serie observada se considera una realizaci\'on o trayectoria del proceso.\\

Resumiendo, una ``\emph{serie temporal}'' es una sucesi\'on de valores
observados de una variable en diferentes instantes de tiempo, en la que las
observaciones aparecen ordenadas cronol\'ogicamente. Los instantes de tiempo en
los que se toman las mediciones de la variable suelen ser intervalos regulares
de tiempo.\\

Hay ``\emph{series temporales}'' que tambi\'en reciben el nombre de
``\emph{series aleatorias}'' ya que son pr\'actiamente impredecibles, como por
ejemplo pueden ser:
\begin{itemize}
\item Los n\'umeros premiados en la Loter\'{\i}a Nacional.
\item El n\'umero resultante del lanzamiento de un dado equilibrado.
\end{itemize}

Sin embargo tambi\'en existen ``\emph{series temporales}'' cuyo comportamiento
es tan regular que se pueden hacer predicciones muy precisas:
\begin{itemize}
\item La posici\'on de los astros.
\item El horario de las mareas.
\end{itemize}

En un lugar intermedio, entre estos dos tipos de ``\emph{series temporales}'',
tenemos otro tipo de series que en su comportamiento tienen dos componentes,
una componente ``\emph{regular}'' y otra ``\emph{irregular}''. Un ejemplo de 
estas series pueden ser las ``\emph{series temporales econ\'omicas}''%
\footnote{Si no todas, al menos la mayor\'{\i}a.}.

\subsection{Procesos estoc\'asticos}

Un ``\emph{proceso estoc\'astico}'', $\{X_t\}$, es una colecci\'on de variables
aleatorias, $X_t$, ordenadas seg\'un un par\'ametro discreto, $t$, que para 
nosotros ser\'a el tiempo.\\

Como hemos observado antes los modelos estoc\'asticos de series temporales
conciben una serie temporal dada como una colecci\'on de observaciones
muestrales, cada una de ellas correspondiente a una variable del proceso.

\subsection{Definiciones sobre procesos estoc\'asticos}

\begin{definicion}[Funci\'on de medias]\label{def:funciondemedias}
Llamaremos ``funci\'on de medias'' del proceso a una funci\'on, dependiente del
tiempo, que proporciona las medias de las distribuciones marginales $z_t$ para
cada instante $t$:
\begin{displaymath}
\mu_t = E[z_t]
\end{displaymath}
\end{definicion}
Si todas la variables tienen la misma media, entonces la ``\emph{funci\'on de
medias}'' es constante y diremos que el proceso es \emph{estable} en la media.

\begin{definicion}[Funci\'on de varianzas]\label{def:funciondevarianzas}
Llamaremos ``funci\'on de varianzas'' del proceso a una funci\'on, dependiente
del tiempo, que proporciona las varianzas en cada instante $t$:
\begin{displaymath}
\sigma^2_t = Var\ (z_t)
\end{displaymath}
\end{definicion}
Si esta funci\'on es constante en el tiempo diremos que el proceso es
\emph{estable} en la varianza.

\begin{definicion}[Funci\'on de autocovarianzas]\label{def:fdeautocovarianzas}
Llamaremos ``funci\'on de autocorrelaci\'on'' del proceso a la funci\'on que
describe las covarianzas en dos instantes cualesquiera:
\begin{displaymath}
Cov\ (t,t+j)=Cov\ (z_t,z_{t+j}) = E\ [(z_t-\mu_t)(z_{t+j}-\mu_{t+j})]
\end{displaymath}
\end{definicion}

\begin{definicion}[Funci\'on de autocorrelaci\'on]\label{def:fdeautocorrelacion}
Llamaremos ``funci\'on de autocorrelaci\'on'' del proceso a la estandarizaci\'on
de la funci\'on de covarianzas:
\begin{displaymath}
\varrho_{(t,t+j)}=\frac{Cov\ (t,t+j)}{\sigma_t,\sigma_{t+j}}
\end{displaymath}
\end{definicion}
%
Estas dos \'ultimas funciones dependen de dos par\'ametros, $t$ y $j$, donde:
\begin{itemize}
\item $t$ es el instante inicial.
\item $j$ es el intervalo entre observaciones.
\end{itemize}
%
En muchos fen\'omenos din\'amicos se observa una condici\'on de estabilidad en
la cual la dependencia entre dos observaciones s\'olo depende del intervalo
entre ellas y no del origen considerado, es decir:
\begin{displaymath}
Cov\ (t_1,t_{1+k})=Cov\ (t_2,t_{2+k}) = \gamma_k\qquad \forall k
\end{displaymath}
es decir la relaci\'on entre $z_t$ y $z_{t+j}$ es siempre igual a la relaci\'on
entre $z_t$ y $z_{t-j}$.

\subsection{Procesos estacionarios}

Diremos que un proceso estoc\'astico (serie temporal) es estacionario en
``sentido d\'ebil'' si existen y son estables la media, la varianza y las
covarianzas, es decir, si para todo $t$:
\begin{enumerate}
\item $\mu_t$ = $\mu$ = cte
\item $\sigma^2_t$ = $\sigma^2$ = cte
\item $Cov\ (t,t+k)=Cov\ (t,t-k)=\gamma_k$ $\forall k$
\end{enumerate}
%
Teniendo en cuenta que $\gamma_0=\sigma^2$ la ``funci\'on de autocorrelaci\'on''
para un proceso estacionario es:
\begin{displaymath}
\varrho_{(t,t+k)} = \varrho_k = \frac{\gamma_k}{\gamma_0}
\end{displaymath}
adem\'as se verifica que $\varrho_{-k}=\varrho_{k}$.\\

Llamaremos ``\emph{funci\'on de autocorrelaci\'on simple}''(fas) o
``\emph{correlograma}'' a la representaci\'on de los coeficientes de
autocorrelaci\'on en funci\'on del retardo.\\

Si la dependencia entre observaciones tiende a cero al aumentar el retardo
diremos que el proceso es ``\emph{erg\'odico}''. En lo sucesivo supondremos
que los procesos estacionarios siempre son ``\emph{erg\'odicos}''.

\subsection{Procesos estacionales}

Que un proceso sea estacional implica que la serie temporal que lo representa
sigue una pauta regular de comportamiento peri\'odico.\\

Si la serie es mensual, por ejemplo, diremos que existir\'a estacionalidad si
los eneros tienden a ser similares en distintos a\~nos, y lo mismo con el resto
de meses. La estacionalidad de una serie la hace no estacionaria, ya que el
valor medio $\mu$ variar\'a de unos meses a otros.\\

\subsection{Procesos de ruido blanco}

Un proceso estacionario muy importante es el definido por:
\begin{enumerate}
\item $E\ [z_t] = 0$
\item $Var\ (z_t) = \sigma^2$
\item $Cov\ (z_t,z_{t-k})=0$ $\forall \ k$
\end{enumerate}
este proceso recibe el nombre de ``\emph{proceso de ruido}''. Si todas las
variables de un proceso de este tipo tienen una distribuci\'on normal entonces
diremos que es un proceso de ``\emph{ruido blanco}''.

\newpage
