%
% REDES NEURONALES 
%

\section{Redes Neuronales}

\subsection{?`Que son?}

Las ``\emph{Redes Neuronales}'' surgieron como consecuencia del intento de
modelizar el funcionamiento del cerebro humano, y en particular su capacidad de
aprendizaje, para poder dotar de ``inteligencia'' a los ordenadores.\\

Estas redes son modelos simplificados de las redes de neuronas que forman el
cerebro y tambi\'en reciben el nombre de ``\textbf{RNA}''\footnote{Redes
Neuronales Artificiales.} y al igual que el modelo biol\'ogico intentan 
``\emph{aprender}'' de los datos de los que le son suministrados.

\subsection{?`Para que sirven?}

Las \textbf{RNA} se utilizan para resolver aquellos problemas en los que su
complejidad hace casi imposible su resoluci\'on mediante las t\'ecnicas
habituales de programaci\'on, ya que el algoritmo utilizado para su
resoluci\'on es altamente complejo y dificilmente programable mediante los
sistemas de computaci\'on secuenciales.\\

Un ejemplo de estos problemas son los ``\emph{Sistemas de Reconocimiento Optico
de Caracteres}''\footnote{O.C.R.}. Estos sistemas tratan, basicamente, de que
un ordenador reconozca las letras y n\'umeros del abecedario a partir de una
imagen digital(reconocimiento de patrones).

\subsection{Historia}

Las primeras investigaciones en este campo empezaron en los a\~nos $40$ con los
trabajos de \emph{McCulloh} y \emph{Pitts}. Posteriormente \emph{Donald O. Hebb}
estableci\'o una conexi\'on entre la psicolog\'{\i}a y  f\'{\i}siologia en la
organizaci\'on del comportamiento, proponiendo un m\'etodo de aprendizaje que se
sigue aplicando en muchas redes neuronales actualmente como una de las reglas
m\'as potentes.
%
\newpage
%
Posteriormente \emph{Bernard Widrow} y \emph{Marcian Hoff} desarrollaron un
algoritmo de red adaptativa basado en un modelo simple de neurona al que
llamaron \textbf{Adaline} y \emph{Frank Rosenblantt} desarrollo el
\textbf{Perceptron} que era capaz de clasificar patrones contaminados con
ruido a la entrada\footnote{La informaci\'on original se deteriora antes de ser
procesada.} y fue inicialmente aplicado a ``O.C.R.''.\\

En $1.969$ hay un declive en la investigaci\'on sobre \textbf{RNA} a partir de
la publicaci\'on de un trabajo de \emph{Minsky} y \emph{Papert} que demostraba
que el \textbf{Perceptron} no era capaz de resolver problemas no lineales
simples, como la funci\'on l\'ogica \emph{XOR}\footnote{Una red de perceptrones
s\'{\i} era capaz de resolver el problema \emph{XOR} pero no existian
algoritmos que pudieran entrenar todos los casos.}.

En los a\~nos $80$ el campo de la computaci\'on neuronal resurge como
consecuencia de los trabajos de:

\begin{itemize}
\item \textbf{Kohonen}, que aport\'o trabajos sobre memorias adaptativas y
aprendizaje competitivo, quedando como recompensa la llamada \emph{red de
Kohonen}, en honor al investigador.
\item \textbf{Grossberg}, que trabaj\'o sobre dise\~no y construcci\'on de
modelos neuronales, introduciendo la funci\'on de transferencia de tipo
\emph{sigmoide} con la que era posible establecer valores de activaci\'on
reales y acotados. A partir de estos postulados se desarroll\'o su
\emph{Teor\'{\i}a de la Resonancia Adaptativa (ART)}.
\item \textbf{Hopfield}, que desarroll\'o un modelo que consist\'{\i}a en un
sistema de elementos totalmente interconectados y que buscaba la situaci\'on de
m\'{\i}nima energ\'{\i}a.
\item \textbf{Rumelhart}, que desarroll\'o el algoritmo de aprendizaje
``\emph{Backpropagation}'' o propagaci\'on del error hacia atr\'as.
\end{itemize}
%
\newpage
%
\subsection{Caracter\'{\i}sticas de las RNA}

Las \textbf{RNA} presentan una serie de caracter\'{\i}sticas que son:

\begin{description}
\item[Paralelismo:] debido a su dise\~no son entidades altamente paralelas, ya
que cada elemento de la red puede funcionar independientemente del resto (dentro
de una misma capa).
\item[Adaptabilidad:] las \textbf{RNA} son capaces de autoorganizarse y
aprender. Las redes no son programadas como los ordenadores tradicionales, sino
que son entrenadas a trav\'es de la repetida presentaci\'on de ejemplos.
\item[Aprendizaje adaptativo:] consigue el conocimiento en base a una etapa
caracter\'{\i}stica que es el aprendizaje, que se basa en el entrenamiento.
\item[Autoorganizaci\'on:] En base a ese entrenamiento la red va a crearse su
propia representaci\'on de todo el dominio de la informaci\'on que se le puede
dar. Es decir, aunque no se la entrene para todas las posibles entradas que le
pueden llegar es capaz de responder a todas ellas de una manera coherente.
\item[Reconocimiento de patrones:] las \textbf{RNA} tienen la ``habilidad'' de
clasificar patrones. Dado un patr\'on de entrada encuentra el patr\'on de
salida asociado.
\item[Reconstrucci\'on de patrones:] la red toma un patr\'on de entrada
incompleto\footnote{Del que se ha perdido informaci\'on.} y es capaz de
insertar la informaci\'on perdida en el patr\'on recuperando el mejor patr\'on
asociado con la entrada.
\item[Tolerancia a fallos:] la destrucci\'on o eliminaci\'on de elementos de
proceso de una red no causar\'a que la red falle e forma global. Dado que la
informaci\'on en una red es distribuida, peque\~nas porciones de informaci\'on
pueden perderse sin afectar seriamente al funcionamiento de la red.
\end{description}

\begin{ejemplo}[RNA aplicada al O.C.R.] \label{ej:OCRSimple}\ \\
Podemos emplear una \textbf{RNA} para el reconocimiento de
patrones. Supongamos que utilizamos una matriz rectangular para la
representaci\'on de las diferentes letras del abecedario. La matriz tendr\'a
dimensi\'on $7\times 5$.
%
\newpage
%
\begin{figure}[!ht]
\begin{displaymath}
\begin{tabular}{ccccc}
0&1&1&1&0\\
1&0&0&0&1\\
1&0&0&0&1\\
1&1&1&1&1\\
1&0&0&0&1\\
1&0&0&0&1\\
1&0&0&0&1
\end{tabular}
\end{displaymath}
\caption{Ideal de la letra A} \label{fig:IdealA}
\end{figure}
\begin{figure}[!ht]
\begin{displaymath}
\begin{tabular}{ccccc}
1&1&1&1&0\\
1&0&0&0&1\\
1&0&0&0&1\\
1&1&1&1&1\\
1&0&0&0&1\\
1&0&0&0&1\\
1&0&0&0&1
\end{tabular}
\end{displaymath}
\caption{Letra A borrosa} \label{fig:ABorrosa}
\end{figure}

Podemos entrenar una \textbf{RNA} para que reconozca la entrada mostrada en la
figura $\ref{fig:IdealA}$ como la letra \emph{A}. Pero por alg\'un motivo puede
ocurrir que se suministre la figura $\ref{fig:ABorrosa}$ como letra \emph{A},
puede ocurrir un error en la recepci\'on de los datos, o simplemente una
persona diferente a nosotros puede tener ``otro ideal de letra
\emph{A}''\footnote{Cada persona tiene un tipo de escritura propio.}. Mediante
el entrenamiento podemos lograr que una \textbf{RNA} reconozca el ``patr\'on''
de la letra \emph{A} en la entrada mostrada en la figura $\ref{fig:ABorrosa}$ y
lo asocie a la letra \emph{A}.
\end{ejemplo}

\subsection{Elementos de una RNA}

Una \textbf{RNA} consta de una serie de elementos interconectados entre s\'{\i}
a los que denominaremos ``EP'', los cuales son elementos simples de procesado,
neuronas. Las conexiones entre ``EP's'' reciben el nombre de
``\emph{sinapsis}'', las cuales ir\'an ponderadas con unos determinados valores
llamados ``\emph{pesos sin\'apticos}'' que var\'{\i}aran a medida que la red
aprende, por lo cual necesitaremos de un algoritmo\footnote{Algoritmo de
aprendizaje de la red.} que permita saber c\'omo y
cu\'anto deben variar estos.
\subsection{Elementos de proceso o EP's}

Los ``EP's'' son los elementos individuales de una \textbf{RNA} y los podemos
dividir en tres tipos:

\begin{itemize}
\item \textbf{Entrada:} reciben datos del ``mundo exterior'' para procesarlos.
\item \textbf{Salida:} emiten los datos procesados.
\item \textbf{Ocultos:} procesan los datos recibidos del ``mundo exterior'' o
de otros ``EP's'' ocultos y emiten la informaci\'on generada a otros ``EP's''
ocultos o a los de salida.
\end{itemize}

Todo ``EP'' tiene, como m\'{\i}nimo, una entrada y una salida la cual puede ser
aplicada a las entradas de otros ``EP's''(incluido el mismo EP).\\

Como hemos dicho antes todos los ``EP's'' que forman una \textbf{RNA} est\'an
interconectados entre s\'{\i}. Esta interconexi\'on permite transmitir los
datos de salida de un ``EP'', en forma de un n\'umero real, a todos los ``EP's''
con los que se encuentre conectado.

\begin{figure}[!ht]
\setlength{\unitlength}{3947sp}%
%
\begingroup\makeatletter\ifx\SetFigFont\undefined%
\gdef\SetFigFont#1#2#3#4#5{%
  \reset@font\fontsize{#1}{#2pt}%
  \fontfamily{#3}\fontseries{#4}\fontshape{#5}%
  \selectfont}%
\fi\endgroup%
\begin{picture}(6624,3928)(589,-3854)
\thinlines
\put(1576,-1261){\circle{670}}
\put(1576,-2161){\circle{670}}
\put(1576,-3061){\circle{670}}
\put(3976,-811){\circle{670}}
\put(3976,-1711){\circle{670}}
\put(3976,-2611){\circle{670}}
\put(3976,-3511){\circle{670}}
\put(6376,-1711){\circle{670}}
\put(6376,-2611){\circle{670}}
\put(1951,-1261){\vector( 4, 1){1641.177}}
\put(1951,-1261){\vector( 4,-1){1658.823}}
\put(1951,-1261){\vector( 4,-3){1704}}
\put(3601,-961){\makebox(1.6667,11.6667){\SetFigFont{5}{6}{\rmdefault}{\mddefault}{\updefault}.}}
\put(1951,-2161){\vector( 4, 1){1641.177}}
\put(1951,-2161){\vector( 4,-1){1658.823}}
\put(1951,-2161){\vector( 4, 3){1632}}
\put(1951,-3136){\vector( 3, 1){1642.500}}
\put(1951,-3136){\vector( 4, 3){1668}}
\put(1951,-3136){\vector( 3, 4){1602}}
\put(1951,-1261){\vector( 3,-4){1629}}
\put(1951,-2161){\vector( 4,-3){1656}}
\put(1951,-3136){\vector( 4,-1){1658.823}}
\put(4351,-811){\vector( 2,-1){1650}}
\put(4351,-811){\vector( 1,-1){1687.500}}
\put(4351,-1786){\vector( 1, 0){1575}}
\put(4351,-1786){\makebox(1.6667,11.6667){\SetFigFont{5}{6}{\rmdefault}{\mddefault}{\updefault}.}}
\put(4351,-1786){\vector( 2,-1){1590}}
\put(4351,-2611){\vector( 2, 1){1590}}
\put(4351,-2611){\vector( 1, 0){1575}}
\put(4351,-3511){\vector( 1, 1){1575}}
\put(4351,-3511){\vector( 2, 1){1560}}
\put(601,-1261){\vector( 1, 0){600}}
\put(601,-2161){\vector( 1, 0){600}}
\put(601,-3061){\vector( 1, 0){600}}
\put(6751,-1711){\vector( 1, 0){450}}
\put(6751,-2611){\vector( 1, 0){450}}
\put(1501,-1336){\makebox(0,0)[lb]{\smash{\SetFigFont{12}{14.4}{\rmdefault}{\mddefault}{\updefault}$1$}}}
\put(1501,-2236){\makebox(0,0)[lb]{\smash{\SetFigFont{12}{14.4}{\rmdefault}{\mddefault}{\updefault}$2$}}}
\put(1501,-3136){\makebox(0,0)[lb]{\smash{\SetFigFont{12}{14.4}{\rmdefault}{\mddefault}{\updefault}$3$}}}
\put(3901,-886){\makebox(0,0)[lb]{\smash{\SetFigFont{12}{14.4}{\rmdefault}{\mddefault}{\updefault}$4$}}}
\put(3901,-1786){\makebox(0,0)[lb]{\smash{\SetFigFont{12}{14.4}{\rmdefault}{\mddefault}{\updefault}$5$}}}
\put(3901,-2686){\makebox(0,0)[lb]{\smash{\SetFigFont{12}{14.4}{\rmdefault}{\mddefault}{\updefault}$6$}}}
\put(3901,-3586){\makebox(0,0)[lb]{\smash{\SetFigFont{12}{14.4}{\rmdefault}{\mddefault}{\updefault}$7$}}}
\put(6301,-1786){\makebox(0,0)[lb]{\smash{\SetFigFont{12}{14.4}{\rmdefault}{\mddefault}{\updefault}$8$}}}
\put(6301,-2686){\makebox(0,0)[lb]{\smash{\SetFigFont{12}{14.4}{\rmdefault}{\mddefault}{\updefault}$9$}}}
\put(901,-61){\makebox(0,0)[lb]{\smash{\SetFigFont{12}{14.4}{\rmdefault}{\mddefault}{\updefault}EP's de Entrada}}}
\put(3301,-61){\makebox(0,0)[lb]{\smash{\SetFigFont{12}{14.4}{\rmdefault}{\mddefault}{\updefault}EP's Ocultos}}}
\put(5851,-61){\makebox(0,0)[lb]{\smash{\SetFigFont{12}{14.4}{\rmdefault}{\mddefault}{\updefault}EP's de Salida}}}
\end{picture}

\caption{Ejemplo de RNA} \label{fig:RNA}
\end{figure}

\begin{ejemplo}[RNA aplicada al O.C.R.]\ \\
Siguiendo con el ejemplo $\ref{ej:OCRSimple}$, en la p\'agina
$\pageref{ej:OCRSimple}$, como cada letra esta representada por una matriz de
dimensi\'on $7\times 5$ entonces tendremos $7\times 5 = 35$ ``EP's'' de entrada
y tendremos tantos ``EP's'' como letras queramos reconocer con nuestra
\textbf{RNA}, por ejemplo $26$ para reconocer el siguiente conjunto de
caracteres:
\begin{center}
A B C D E F G H I J K L M N \~N O P Q R S T U V X Y Z
\end{center}
Como cada caracter est\'a representado por una matriz de $7\times 5$, si
ponemos todas las filas de la matriz seguidas tendremos $35$ digitos, cada uno
asociado con un ``EP'' de entrada. As\'{\i} mismo como el conjunto de
caracteres a reconocer es de $26$ tendremos $26$ ``EP's'' de salida, cada
``EP'' estar\'a asociado a un caracter del conjunto.
\end{ejemplo}

En algunas \textbf{RNA} aparece un ``EP'' especial, denominado \emph{bias} por
los inform\'aticos, el cual no es m\'as que un ``EP'' con salida constante que
est\'a conectado a todas la neuronas de una capa. En lo sucesivo supondremos que
dicho ``EP'' no existe o que tiene salida nula.
%
\newpage
%
\subsection{Partes de un EP}

Como hemos dicho antes una \textbf{RNA} esta formada por un conjunto de
``EP's'' interconectados entre s\'{\i}. Al estar, todos ellos, interconectados
entre s\'{\i} tiene que haber una forma en la que la informaci\'on procesada
por un ``EP'' se transmita a los dem\'as para seguir siendo procesada. De esto
se encargan las diferentes partes de un ``EP''.\\ \\
En un ``EP'' podemos encontrar:

\begin{itemize}
\item \textbf{Pesos Sin\'apticos}, valores asociados a la conexi\'on de un
``EP'' con otros ``EP's''. Estos valores se van modificando a medida que la
\textbf{RNA} va aprendiendo, mediante entrenamiento, para que de esta forma la
\textbf{RNA} se comporte adecuadamente.
\item \textbf{Entrada de Red o Entrada Total}, es una suma ponderada de los
valores de entrada al ``EP'', procedentes de los ``EP's'' conectados con \'el,
y de los pesos sin\'apticos asociados con dicho ``EP''.
\item \textbf{Funci\'on de Activaci\'on}, esta funci\'on nos va a indicar si el
``PE'' est\'a activo o no, y depende de la \emph{entrada de red} y/o \emph{valor
de la funci\'on de activaci\'on en el instante anterior}.
\item \textbf{Funci\'on de Salida}, en esta funci\'on se procesa el valor
obtenido en la \emph{funci\'on de activaci\'on} y el valor obtenido es el valor
de salida de la neurona, es decir, la informaci\'on que suministra a otras
neuronas. Normalmente se utiliza como \emph{funci\'on de salida} la funci\'on
identidad, con lo cual el valor devuelto por la \emph{funci\'on de activaci\'on}
es la informaci\'on mandada a otras neuronas. En lo sucesivo supondremos que
esta funci\'on es la funci\'on identidad, salvo que expresamente se diga lo
contrario.
\end{itemize}
Un ejemplo de esto aplicado a la capa oculta de la \textbf{RNA} que aparece en
la figura $\ref{fig:RNA}$ de la p\'agina $\pageref{fig:RNA}$:

\subsubsection{Pesos sin\'apticos para la RNA de la figura $\ref{fig:RNA}$}
\begin{center}
\begin{tabular}{|c|c|c|}
\hline
EP&EP's que suministran informaci\'on al EP &Pesos sin\'apticos\\
\hline
$4$&$1$, $2$ y $3$& $w_{1,4}$, $w_{2,4}$ y $w_{3,4}$\\
\hline
$5$&$1$, $2$ y $3$& $w_{1,5}$, $w_{2,5}$ y $w_{3,5}$\\
\hline
$6$&$1$, $2$ y $3$& $w_{1,6}$, $w_{2,6}$ y $w_{3,6}$\\
\hline
$7$&$1$, $2$ y $3$& $w_{1,7}$, $w_{2,7}$ y $w_{3,7}$\\
\hline
\end{tabular}
\end{center}
Los pesos asociados a cada ``EP'' se suelen determinar experimentalmente
mediante el ``entrenamiento'' de la red. El ``entrenamiento'' se suele realizar
introduciendo en la \textbf{RNA} las posibles entradas que se pueden presentar 
en la pr\'actica, o al menos un subconjunto significativo de las mismas, y
comparar la salida obtenida con la salida deseada y modificar los pesos mediante
alg\'un algoritmo de tal forma que el error cometido por la \textbf{RNA} sea
nulo o practicamente nulo.

\subsubsection{Entrada de red para la \textbf{RNA} de la figura $\ref{fig:RNA}$}

La \emph{entrada de red}, en el instante $t$, de los diferentes ``EP's'' es:

\begin{displaymath}
s_j(t)=x_1(t)\cdot w_{1,j}+x_2(t)\cdot w_{2,j}+x_3(t)\cdot w_{3,j} =\sum_{i=1}^3
x_i(t)\cdot w_{i,j}
\end{displaymath}
para $j=4,5,6,7$. Donde $x_i(t)$, $i=1,2,3$, es la entrada desde el $i$-esimo
``EP'', o lo que es lo mismo la salida del $i$-esimo ``EP''.

\subsubsection{Funci\'on de activaci\'on para la \textbf{RNA} de la figura
$\ref{fig:RNA}$}

La \emph{funci\'on de activaci\'on} no tiene que ser la misma para todos los
``EP's'' de la \textbf{RNA} y depender\'a del tipo de problema que estemos
resolviendo. Denotemos como $F_j$ a la funci\'on de activaci\'on para el
$j$-esimo ``EP''. Normalmente la activaci\'on, $A_j$, para el $j$-esimo ``EP''
en el momento depende de la \emph{entrada de red} en el momento $t$, es decir:

\begin{displaymath}
A_j(t+1) = F_j(s_j(t))
\end{displaymath}
Tambi\'en se puede dar el caso que la activaci\'on en el momento $t+1$ dependa
de la activaci\'on en el momento anterior $t$:

\begin{displaymath}
A_j(t+1)=F_j(A_j(t),s_j(t))
\end{displaymath}
A las \emph{funciones de activaci\'on} tambi\'en se las conoce con el nombre de 
\emph{funciones de transferencia}.\\

La \emph{funci\'on log\'{\i}stica} es comunmente utilizada como \emph{funci\'on de activaci\'on}.
\begin{displaymath}
F_j(t)=\frac{1}{1+e^{-t}}
\end{displaymath}
Esta funci\'on toma todos los valores comprendidos entre $0$ y $1$.\\

En algunos casos se tiene definido un valor ``\emph{umbral}'', $\theta$, y
dependiendo de si la ``entrada de red'' es mayor o no que este valor el ``EP''
correspondiente se activar\'a o no. En caso de activarse mandar\'{\i}a la 
se\~nal $1$ y en caso contrario la se\~nal $0$(permanecer\'{\i}a inactiva).

\subsection{Notaci\'on para redes neuronales}

En lo sucesivo utilizaremos la siguiente notaci\'on para una red neuronal:
\begin{itemize}
\item $n_s$ n\'umero de neuronas en la capa $s$ de la red neuronal.
\item $f(x)$ funci\'on de activaci\'on para la red neuronal.
\item $w_{j,i}[s]$ peso sin\'aptico entre la neurona $i$-\'esima de la capa
$(s-1)$ con la neurona $j$-\'esima en la capa $s$.
\item $x_j[s](p)$ salida de la $j$-\'esima neurona de la capa $s$ y patr\'on
$p$.
\begin{displaymath}
x_j[s](p) = f(I_j[s](p))
\end{displaymath}
\item $I_j[s](p)$ entrada total de red de la neurona $j$-\'esima en la capa $s$
y patr\'on $p$.
\begin{displaymath}
I_j[s](p) = \sum_{i=1}^{n_{(s-1)}} w_{j,i}[s]\cdot x_i[s-1](p)
\end{displaymath}
\end{itemize}

\subsection{Redes Feedforward}

En este tipo de redes las salidas de los ``EP's'' de una capa unicamente son
entrada de los ``EP's'' de la capa siguiente. Normalmente este tipo de redes
son las m\'as r\'apidas de todas y representan sistemas lineales. \\

Podemos ver un ejemplo de una red de este tipo en la figura $\ref{fig:RNA}$ 
situada en la p\'agina $\pageref{fig:RNA}$.

\subsection{Redes Feedback} 

En este tipo de redes las salidas de los ``EP's'' de una capa pueden ser la
entrada de los ``EP's'' de las capas anteriores. Este tipo de redes representa
sistemas no lineales.

\subsection{Redes Feedlateral}

En este tipo de redes las salidas de los ``EP's'' de una capa pueden ser la 
entrada de los ``EP's'' situados en la misma capa. Este tipo de redes representa
sistemas no lineales.

\subsection{Redes Recurrentes}

En este tipo de redes pueden existir lazos cerrados, es decir, un ``EP'' le
manda informaci\'on a otro ``EP'' y este \'ultimo despu\'es de procesar la
informaci\'on le manda informaci\'on al primer ``EP''.

\subsection{El aprendizaje en las RNA}

Basicamente el aprendizaje en las Redes Neuronales consiste en encontrar los
``pesos sin\'apticos'' adecuados para que la red pueda realizar de una forma
eficiente su trabajo.

\subsubsection{Aprendizaje por correcci\'on de error}

Consiste en tomar un conjunto de pares de datos, entradas con la correspondiente
salida en la red, y minimizar el error cometido por la red variando los ``pesos
sin\'apticos''. El conjunto de pares de datos recibe el nombre de
``\emph{conjunto de entrenamiento}''.

\subsubsection{Aprendizaje por refuerzo}

No se dispone de un conjunto completo del comportamiento deseado de la red, ya
que no se conoce la salida deseada exacta para cada entrada, sino que se conoce
el comportamiento de una manera global para diferentes entradas. La relaci\'on
entrada-salida se realiza a trav\'es de un proceso de \'exito o fracaso,
produciendo este una se\~nal de refuerzo que mide el buen funcionamiento del
sistema. La funci\'on del supervisor es m\'as la de un cr\'{\i}tico que la de
un maestro.

\subsubsection{Aprendizaje estoc\'astico}

Consiste, b\'asicamente, en realizar cambios aleatorios de los valores de los 
pesos y evaluar su efecto a partir del objetivo deseado.
