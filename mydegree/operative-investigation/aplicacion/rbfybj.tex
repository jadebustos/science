%
% MODELO HIBRIDO
%

\section{El modelo hibrido}

Hemos ido mejorando los modelos de redes.\\

En primer lugar teniamos una red $RBF$, la cual pod\'{\i}a producir predicciones
poco fiables y no teniamos ninguna forma de saber que predicciones eran fiables
y cuales no. Para mejorar esto modificamos la red e introdujimos un factor de
seguridad, el cual nos indicaba cuando una predicci\'on era fiable o no.\\

Aunque la red $RBF$ con factor de seguridad ofrece m\'as garantias que la red
$RBF$ sin factor de seguridad no tenemos resueltos todos los problemas. En el 
caso de obtener una predicci\'on no fiable la red $RBF$ no nos da una
predicci\'on fiable para ese caso, \'unicamente nos indica que la predicci\'on
no es fiable.\\

Para solucionar esto utilizamos un modelo hibrido, es decir un modelo que
utiliza una red neuronal y un m\'etodo de predicci\'on, en nuestro caso el de
$Box$ y $Jenkins$, y con ambos resultados nos ofrece una predicci\'on.
%
\begin{figure}[!h]
\setlength{\unitlength}{3947sp}%
%
\begingroup\makeatletter\ifx\SetFigFont\undefined%
\gdef\SetFigFont#1#2#3#4#5{%
  \reset@font\fontsize{#1}{#2pt}%
  \fontfamily{#3}\fontseries{#4}\fontshape{#5}%
  \selectfont}%
\fi\endgroup%
\begin{picture}(7437,5697)(514,-5623)
\thinlines
\put(3976,-811){\circle{670}}
\put(6376,-1711){\circle{670}}
\put(3976,-2611){\circle{670}}
\put(1501,-1711){\circle{670}}
\put(3976,-1711){\circle{670}}
\put(6376,-811){\circle{670}}
\put(4351,-811){\vector( 2,-1){1650}}
\put(4351,-2611){\vector( 2, 1){1590}}
\put(526,-1711){\vector( 1, 0){600}}
\put(1876,-1711){\vector( 2, 1){1740}}
\put(1876,-1711){\vector( 1, 0){1725}}
\put(1876,-1711){\vector( 2,-1){1740}}
\put(4351,-1711){\vector( 1, 0){1650}}
\put(2401,-4261){\framebox(975,450){}}
\put(4351,-2611){\vector( 1, 1){1650}}
\put(4351,-1711){\vector( 2, 1){1680}}
\put(4351,-811){\vector( 1, 0){1650}}
\put(3376,-4036){\line( 1, 0){1500}}
\put(4876,-4036){\vector( 1, 3){997.500}}
\put(826,-5161){\vector( 1, 0){900}}
\put(826,-5161){\line( 0, 1){3450}}
\put(1726,-5461){\framebox(1350,600){}}
\put(5176,-5611){\framebox(1950,900){}}
\put(6376,-2086){\vector( 0,-1){2625}}
\put(3076,-5161){\vector( 1, 0){2100}}
\put(6751,-811){\line( 1, 0){225}}
\put(6976,-811){\vector( 0,-1){3900}}
\put(7126,-5161){\vector( 1, 0){675}}
\put(3301,-61){\makebox(0,0)[lb]{\smash{\SetFigFont{12}{14.4}{\rmdefault}{\mddefault}{\updefault}EP's Ocultos}}}
\put(901,-61){\makebox(0,0)[lb]{\smash{\SetFigFont{12}{14.4}{\rmdefault}{\mddefault}{\updefault}EP de Entrada}}}
\put(1426,-1786){\makebox(0,0)[lb]{\smash{\SetFigFont{12}{14.4}{\rmdefault}{\mddefault}{\updefault}X}}}
\put(3826,-886){\makebox(0,0)[lb]{\smash{\SetFigFont{12}{14.4}{\rmdefault}{\mddefault}{\updefault}$C_1$}}}
\put(3826,-1786){\makebox(0,0)[lb]{\smash{\SetFigFont{12}{14.4}{\rmdefault}{\mddefault}{\updefault}$C_2$}}}
\put(3826,-2686){\makebox(0,0)[lb]{\smash{\SetFigFont{12}{14.4}{\rmdefault}{\mddefault}{\updefault}$C_3$}}}
\put(2800,-4080){\makebox(0,0)[lb]{\smash{\SetFigFont{12}{14.4}{\rmdefault}{\mddefault}{\updefault}$w_0$}}}
\put(3826,-3961){\makebox(0,0)[lb]{\smash{\SetFigFont{12}{14.4}{\rmdefault}{\mddefault}{\updefault}Constante}}}
\put(6176,-886){\makebox(0,0)[lb]{\smash{\SetFigFont{12}{14.4}{\rmdefault}{\mddefault}{\updefault}$I_1[3]$}}}
\put(6250,-1786){\makebox(0,0)[lb]{\smash{\SetFigFont{12}{14.4}{\rmdefault}{\mddefault}{\updefault}CF}}}
\put(5851,-61){\makebox(0,0)[lb]{\smash{\SetFigFont{12}{14.4}{\rmdefault}{\mddefault}{\updefault}EP's de Salida}}}
\put(6826,-3286){\makebox(0,0)[lb]{\smash{\SetFigFont{12}{14.4}{\rmdefault}{\mddefault}{\updefault}y}}}
\put(6076,-3286){\makebox(0,0)[lb]{\smash{\SetFigFont{12}{14.4}{\rmdefault}{\mddefault}{\updefault}CF}}}
\put(1900,-5236){\makebox(0,0)[lb]{\smash{\SetFigFont{12}{14.4}{\rmdefault}{\mddefault}{\updefault}B-J Modelo}}}
\put(5350,-5200){\makebox(0,0)[lb]{\smash{\SetFigFont{12}{14.4}{\rmdefault}{\mddefault}{\updefault}Predicci\'on Hibrida}}}
\put(7951,-5236){\makebox(0,0)[lb]{\smash{\SetFigFont{12}{14.4}{\rmdefault}{\mddefault}{\updefault}Y}}}
\end{picture}

\caption{Modelo de red hibrida utilizando $RBF$ y el modelo de $Box$ y
$Jenkins$.}
\end{figure}
%
\newpage
%
En este modelo tenemos tres datos para realizar la predicci\'on $Y$:
\begin{itemize}
\item La predicci\'on dada por la red $RBF$, $y_{RBF}$.
\item El factor de seguridad, $CF$.
\item La predicci\'on dada por el modelo de $Box$ y $Jenkins$, $y_{B-J}$.
\end{itemize}
Podemos combinar estos datos de varias formas para obtener la predicci\'on
deseada sobre el comportamiento de la serie temporal.

\subsection{M\'etodo de tolerancia}

En este m\'etodo el usuario dar\'a una tolerancia, $Tol$, para el factor de
seguridad. Esta tolerancia se determinar\'a experimentalmente para obtener un
grado alto de precisi\'on sobre las predicciones.\\ \\
%
El algoritmo para este m\'etodo es el siguiente:
\begin{quote}
if ( $CF\geq Tol$ ) then
\begin{quote}
$Y=y_{RBF}$
\end{quote}
else
\begin{quote}
$Y=y_{B-J}$
\end{quote}
\end{quote}

\subsection{M\'etodo de las aproximaciones de igual importancia}

Este m\'etodo es muy similar al anterior y en \'el tambi\'en se ha de
especificar una tolerancia, $Tol$, la cual se determina de igual forma que en el
m\'etodo anterior.\\

En el m\'etodo anterior despreciamos la predicci\'on dada por el modelo de
$Box$ y $Jenkins$ cuando $CF\geq Tol$, suponiendo esta predicci\'on como peor
que la obtenida por la red $RBF$, pero esto puede no ser cierto. Es por esta
raz\'on que en este m\'etodo se utiliza la media de ambas predicciones cuando
$CF\geq Tol$. El algoritmo para este m\'etodo es el siguiente:
\begin{quote}
if ( $CF\geq Tol$ ) then
\begin{quote}
$Y=\frac{y_{RBF}+y_{B-J}}{2}$
\end{quote}
else
\begin{quote}
$Y=y_{B-J}$
\end{quote}
\end{quote}
\newpage
\subsection{M\'etodo de la aproximaci\'on seg\'un el factor de seguridad}

Este m\'etodo combina las predicciones dadas tanto por la red $RBF$ como por el
modelo de $Box$ y $Jenkins$, pero da tanta importancia a la predicci\'on de la
red $RBF$ como seguridad nos indique el factor de seguridad. Es decir si el 
factor de seguridad es $0.80$ en la predicci\'on final el $80\ \%$ ser\'a
aportado por $y_{RBF}$ y el $20\ \%$ restante ser\'a aportado por $y_{B-J}$.\\

La predicci\'on por este m\'etodo vendr\'a dada por la siguiente f\'ormula:
\begin{displaymath}
Y=(CF\cdot y_{RBF})\cdot (1-CF)\cdot y_{B-J}
\end{displaymath}
