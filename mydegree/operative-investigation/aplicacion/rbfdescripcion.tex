%
% DESCRIPCION DE ESTA RED
% 

\subsection{Descripci\'on de la red RBF}

Este tipo de red es una red de tres capas:
\begin{itemize}
\item La primera capa es la capa de entrada de datos, la cual suministra los
datos de entrada a los EP's de la segunda capa. 
\item La segunda capa, capa oculta, se comporta de una forma muy diferente a
otro tipo de redes. Cada EP de esta capa representa a un grupo de datos
centrados en un punto fijo y con radio conocido.\\

Por comodidad supondremos que en esta capa tenemos tres EP's.
\item La tercera capa, capa de salida, es la capa que procesa los datos
obtenidos en la segunda capa y nos devuelve la predicci\'on para la serie
temporal.\\
\end{itemize}
\begin{figure}[!h]
\setlength{\unitlength}{3947sp}%
%
\begingroup\makeatletter\ifx\SetFigFont\undefined%
\gdef\SetFigFont#1#2#3#4#5{%
  \reset@font\fontsize{#1}{#2pt}%
  \fontfamily{#3}\fontseries{#4}\fontshape{#5}%
  \selectfont}%
\fi\endgroup%
\begin{picture}(6837,4347)(514,-4273)
\thinlines
\put(3976,-811){\circle{670}}
\put(6376,-1711){\circle{670}}
\put(3976,-2611){\circle{670}}
\put(1501,-1711){\circle{670}}
\put(3976,-1711){\circle{670}}
\put(4351,-811){\vector( 2,-1){1650}}
\put(4351,-2611){\vector( 2, 1){1590}}
\put(6751,-1711){\vector( 1, 0){450}}
\put(526,-1711){\vector( 1, 0){600}}
\put(1876,-1711){\vector( 2, 1){1740}}
\put(1876,-1711){\vector( 1, 0){1725}}
\put(1876,-1711){\vector( 2,-1){1740}}
\put(4351,-1711){\vector( 1, 0){1650}}
\put(3376,-4036){\line( 1, 0){3000}}
\put(6376,-4036){\vector( 0, 1){1950}}
\put(2401,-4261){\framebox(975,450){}}
\put(3301,-61){\makebox(0,0)[lb]{\smash{\SetFigFont{12}{14.4}{\rmdefault}{\mddefault}{\updefault}EP's Ocultos}}}
\put(901,-61){\makebox(0,0)[lb]{\smash{\SetFigFont{12}{14.4}{\rmdefault}{\mddefault}{\updefault}EP de Entrada}}}
\put(5851,-61){\makebox(0,0)[lb]{\smash{\SetFigFont{12}{14.4}{\rmdefault}{\mddefault}{\updefault}EP de Salida}}}
\put(1426,-1786){\makebox(0,0)[lb]{\smash{\SetFigFont{12}{14.4}{\rmdefault}{\mddefault}{\updefault}X}}}
\put(2476,-886){\makebox(0,0)[lb]{\smash{\SetFigFont{12}{14.4}{\rmdefault}{\mddefault}{\updefault}$w_{1,1}[2]$}}}
\put(2476,-1636){\makebox(0,0)[lb]{\smash{\SetFigFont{12}{14.4}{\rmdefault}{\mddefault}{\updefault}$w_{2,1}[2]$}}}
\put(2476,-2486){\makebox(0,0)[lb]{\smash{\SetFigFont{12}{14.4}{\rmdefault}{\mddefault}{\updefault}$w_{3,1}[2]$}}}
\put(3826,-886){\makebox(0,0)[lb]{\smash{\SetFigFont{12}{14.4}{\rmdefault}{\mddefault}{\updefault}$C_1$}}}
\put(3826,-1786){\makebox(0,0)[lb]{\smash{\SetFigFont{12}{14.4}{\rmdefault}{\mddefault}{\updefault}$C_2$}}}
\put(3826,-2686){\makebox(0,0)[lb]{\smash{\SetFigFont{12}{14.4}{\rmdefault}{\mddefault}{\updefault}$C_3$}}}
\put(4851,-886){\makebox(0,0)[lb]{\smash{\SetFigFont{12}{14.4}{\rmdefault}{\mddefault}{\updefault}$w_{1,1}[3]$}}}
\put(4851,-1636){\makebox(0,0)[lb]{\smash{\SetFigFont{12}{14.4}{\rmdefault}{\mddefault}{\updefault}$w_{1,2}[3]$}}}
\put(4851,-2486){\makebox(0,0)[lb]{\smash{\SetFigFont{12}{14.4}{\rmdefault}{\mddefault}{\updefault}$w_{1,3}[3]$}}}
\put(2800,-4080){\makebox(0,0)[lb]{\smash{\SetFigFont{12}{14.4}{\rmdefault}{\mddefault}{\updefault}$w_0$}}}
\put(3826,-3961){\makebox(0,0)[lb]{\smash{\SetFigFont{12}{14.4}{\rmdefault}{\mddefault}{\updefault}Constante}}}
\put(7351,-1786){\makebox(0,0)[lb]{\smash{\SetFigFont{12}{14.4}{\rmdefault}{\mddefault}{\updefault}y}}}
\put(6176,-1786){\makebox(0,0)[lb]{\smash{\SetFigFont{12}{14.4}{\rmdefault}{\mddefault}{\updefault}$I_1[3]$}}}
\end{picture}

\caption{Red neuronal del tipo $RBF$.}
\end{figure}

Una forma l\'ogica de escoger los $C_i$ ser\'{\i}a agrupando todos los patrones
de entrenamiento en tantos discos como EP's tengamos en la capa oculta y tomar
como $C_i$ el centro de estos discos y como radio el radio de los mismos.
