%
% INCONVENIENTES DE LAS REDES RBF
%

\subsection{Inconvenientes de las redes RBF}

Las redes $RBF$ dan predicciones poco fiables cuando el vector de entrada est\'a
lejos del conjunto de entrenamiento. Est\'a lejan\'{\i}a nos la da la distancia
que estemos utilizando, normalmente ser\'a la distancia euclidea.\\

En el caso de las funciones gausianas: 
\begin{displaymath}
\phi_i(x)=e^{-(\frac{x}{r_i})^2} = \frac{1}{e^{(\frac{x}{r_i})^2}}
\end{displaymath}
cuando el vector de entrada est\'a lejos del centro de los grupos de datos $C_i$
se tiene que:
\begin{displaymath}
\frac{1}{e^{(\frac{x}{r_i})^2}}\to 0\quad donde\ x=d(X,C_i)
\end{displaymath}
Luego las redes $RBF$ utilizando funciones gausianas son buenas solamente cuando
el vector de entrada est\'a proximo a los centros $C_i$. En caso contrario las
predicciones obtenidas son poco fiables.
