%
% FUNCIONAMIENTO DE ESTA RED
%

\subsection{Funcionamiento de una red RBF}

Supongamos que tenemos un vector $X=(x_1,\dots,x_n)$, donde $\{x_i\}_{i=1}^n$
son valores de una serie temporal cualquiera.\\

El EP de la primera capa, capa de entrada de datos, suministra el vector $X$ a
los EP's de la segunda capa, por lo tanto los pesos sin\'apticos de estas
conexiones ser\'an $1$:
\begin{itemize}
\item $w_{1,1}[2]=1$.
\item $w_{2,1}[2]=1$.
\item $w_{3,1}[2]=1$.
\end{itemize}
%
\newpage
%
Los EP's de la segunda capa funcionan de la siguiente manera:
\begin{itemize}
\item Calculan la distancia del vector $X$ al centro del grupo de datos que
representan $C_i=(c_{1,i},\dots,c_{n,i})$ para $i=1,2,3$.\\

Para el calculo de la distancia normalmente se emplea la distancia euclidea:
\begin{displaymath}
d(X,C_i) =\sqrt{\sum_{j=1}^n(x_j-c_{j,i})^2}\quad para\ i=1,2,3
\end{displaymath}
aunque tambi\'en puede usarse cualquier otra distancia.
\item La distancia $d(X,C_i)$ es transformada por alguna funci\'on y el
resultado obtenido es la salida del nodo.\\

Una funci\'on tipicamente usada para transformar las distancias $d(X,C_i)$ es
la siguiente:
\begin{displaymath}
\phi_i(x) = e^{-(\frac{x}{r_i})^2}
\end{displaymath}
donde $r_i$ es el radio del $i$-\'esimo EP de la segunda capa, el cual es
conocido. Estas funciones reciben el nombre de ``\emph{funciones gausianas}''.
\end{itemize}
Los pesos sin\'apticos asociados a las conexiones de esta capa con la capa de
salida se determinan de forma experimental mediante algoritmos como el de 
``backpropagation'' que ya vimos.\\ \\
%
El EP de la tercera capa funciona de la siguiente manera:
\begin{itemize}
\item Calcula la entrada total de red:
\begin{displaymath}
I_1[3] = I_1[3](X) = \sum_{i=1}^3 w_{1,i}[3]\cdot \phi_i(d(X,C_i))
\end{displaymath}
\item Se le suma un valor constante a la entrada total de red y el valor
obtenido es la predicci\'on sobre el comportamiento de la serie temporal:
\begin{displaymath}
y = w_0+I_1[3][X]
\end{displaymath}
\end{itemize}
