%
% REDES RBF CON FACTOR DE SEGURIDAD
%

\section{Redes RBF con factor de seguridad}

En los apartados anteriores hemos visto como en algunos casos las predicciones
de una red $RBF$ no son fiables. Luego no todas las predicciones de una red de
este tipo son buenas predicciones, por lo tanto es muy importante identificar
que predicciones son buenas y cuales no lo son.\\

La forma m\'as directa de solucionar este problema es tener una red $RBF$ que
distinga las buenas predicciones de las malas.\\

Las funciones gausianas $\phi_i$ toman los valores entre $1.0$ y $0.0$. Para
distinguir las buenas de las malas predicciones utilizaremos un dato al que
denominaremos ``\emph{factor de seguridad}'', el cual vendr\'a dado por los
valores que toman las funciones $\phi_i$.\\

La salida de las neuronas de la segunda capa ser\'a de la forma:
\begin{displaymath}
\phi_i(d(X,C_i)) = e^{-(\frac{d(X,C_i)}{r_i})^2}
\end{displaymath}
sabemos que $\phi_i(x)\in [0,1]$ y el exponente de la funci\'on exponencial
de $\phi_i$ nos indica lo siguiente:
\begin{itemize}
\item Si $\phi_i(d(X,C_i))\approx 1$ entonces $X$ est\'a pr\'oximo al centro
$C_i$ y su aportaci\'on a la predicci\'on final es buena.
\item Si $\phi_i(d(X,C_i))\approx 0$ entonces $X$ no est\'a pr\'oximo al centro
$C_i$ y su aportaci\'on a la predicci\'on final puede estropear esta.
\end{itemize}
En el caso de que todos los EP's tengan valores de $\phi_i$ pr\'oximos a cero la
predicci\'on obtenida no ser\'a buena.\\

El problema es como obtener una medida de la precisi\'on de la predicci\'on
utilizando estos valores. A primera vista se le puede ocurrir a uno utilizar
el m\'aximo de los $\phi_i$, pero esto representa un problema y es que puede
haber varios EP's con $\phi_i \approx 1$ que hagan que la predicci\'on sea
buena y, por el hecho de existir un $\phi_i$ grande desechar esta
predicci\'on.
%
\newpage
%
Para dar una medida de la precisi\'on de la predicci\'on se utiliza la siguiente
f\'ormula recursiva:
\begin{equation}\label{eq:FormulaRecursiva}
CF_i=CF_{i-1}+(1-CF_{i-1})\cdot max_{i}(\{\phi_j(d(X,C_j))\}_{j=1}^N)
\end{equation}
donde:
\begin{itemize}
\item $N$ es el n\'umero de EP's en la capa oculta.
\item $max_i$ ser\'a el $i$-\'esimo mayor valor de los $\phi_j$.
\item $CF_0 = 0$ por definici\'on.
\item $CF_i$ es el $i$-\'esimo factor de seguridad.
\item $CF$ es el factor de seguridad.
\end{itemize}
%
\begin{figure}[!h]
\setlength{\unitlength}{3947sp}%
%
\begingroup\makeatletter\ifx\SetFigFont\undefined%
\gdef\SetFigFont#1#2#3#4#5{%
  \reset@font\fontsize{#1}{#2pt}%
  \fontfamily{#3}\fontseries{#4}\fontshape{#5}%
  \selectfont}%
\fi\endgroup%
\begin{picture}(6762,4347)(514,-4273)
\thinlines
\put(3976,-811){\circle{670}}
\put(6376,-1711){\circle{670}}
\put(3976,-2611){\circle{670}}
\put(1501,-1711){\circle{670}}
\put(3976,-1711){\circle{670}}
\put(6376,-811){\circle{670}}
\put(4351,-811){\vector( 2,-1){1650}}
\put(4351,-2611){\vector( 2, 1){1590}}
\put(6751,-1711){\vector( 1, 0){450}}
\put(526,-1711){\vector( 1, 0){600}}
\put(1876,-1711){\vector( 2, 1){1740}}
\put(1876,-1711){\vector( 1, 0){1725}}
\put(1876,-1711){\vector( 2,-1){1740}}
\put(4351,-1711){\vector( 1, 0){1650}}
\put(2401,-4261){\framebox(975,450){}}
\put(4351,-2611){\vector( 1, 1){1650}}
\put(4351,-1711){\vector( 2, 1){1680}}
\put(4351,-811){\vector( 1, 0){1650}}
\put(3376,-4036){\line( 1, 0){1500}}
\put(4876,-4036){\vector( 1, 3){997.500}}
\put(6751,-811){\vector( 1, 0){450}}
\put(3301,-61){\makebox(0,0)[lb]{\smash{\SetFigFont{12}{14.4}{\rmdefault}{\mddefault}{\updefault}EP's Ocultos}}}
\put(901,-61){\makebox(0,0)[lb]{\smash{\SetFigFont{12}{14.4}{\rmdefault}{\mddefault}{\updefault}EP de Entrada}}}
\put(1426,-1786){\makebox(0,0)[lb]{\smash{\SetFigFont{12}{14.4}{\rmdefault}{\mddefault}{\updefault}X}}}
\put(3826,-886){\makebox(0,0)[lb]{\smash{\SetFigFont{12}{14.4}{\rmdefault}{\mddefault}{\updefault}$C_1$}}}
\put(3826,-1786){\makebox(0,0)[lb]{\smash{\SetFigFont{12}{14.4}{\rmdefault}{\mddefault}{\updefault}$C_2$}}}
\put(3826,-2686){\makebox(0,0)[lb]{\smash{\SetFigFont{12}{14.4}{\rmdefault}{\mddefault}{\updefault}$C_3$}}}
\put(2800,-4080){\makebox(0,0)[lb]{\smash{\SetFigFont{12}{14.4}{\rmdefault}{\mddefault}{\updefault}$w_0$}}}
\put(3826,-3961){\makebox(0,0)[lb]{\smash{\SetFigFont{12}{14.4}{\rmdefault}{\mddefault}{\updefault}Constante}}}
\put(6176,-886){\makebox(0,0)[lb]{\smash{\SetFigFont{12}{14.4}{\rmdefault}{\mddefault}{\updefault}$I_1[3]$}}}
\put(6250,-1786){\makebox(0,0)[lb]{\smash{\SetFigFont{12}{14.4}{\rmdefault}{\mddefault}{\updefault}CF}}}
\put(7351,-836){\makebox(0,0)[lb]{\smash{\SetFigFont{12}{14.4}{\rmdefault}{\mddefault}{\updefault}y}}}
\put(7276,-1786){\makebox(0,0)[lb]{\smash{\SetFigFont{12}{14.4}{\rmdefault}{\mddefault}{\updefault}CF}}}
\put(5851,-61){\makebox(0,0)[lb]{\smash{\SetFigFont{12}{14.4}{\rmdefault}{\mddefault}{\updefault}EP's de Salida}}}
\end{picture}

\caption{Red neuronal del tipo $RBF$ con factor de seguridad.}
\end{figure}
%

Normalmente se hace que $i$ varie entre $1$ y $N$, pero tambi\'en se puede hacer
que varie entre $1$ y $M$ donde $M<N$. El factor de seguridad que mejor nos
indica la precisi\'on de las predicciones ser\'a $CF=CF_N$.\\

En este tipo de red el calculo de los pesos se realiza de igual forma que en la
red $RBF$ sin factor de seguridad:
\begin{itemize}
\item Los pesos de interconexi\'on entre la primera y segunda capa son todos
uno:
\begin{displaymath}
w_{1,1}[2]=w_{2,1}[2]=w_{3,1}[2]=1
\end{displaymath}
\item Los pesos de interconexi\'on entre la segunda y tercera capa se calculan
de la siguiente forma:
\begin{itemize}
\item Aquellos que van al EP de predicci\'on se calculan como en el caso
anterior. Estos pesos son:
\begin{displaymath}
w_{1,1}[3]\quad w_{1,2}[3]\quad w_{1,3}[3]
\end{displaymath}
\item Aquellos que van al EP que nos da el factor de seguridad son todos iguales
a uno:
\begin{displaymath}
w_{2,1}[3]=w_{2,2}[3]=w_{2,3}[3]=1
\end{displaymath}
\end{itemize}
\end{itemize}
