%
% APLICACION A SERIES TEMPORALES 
%

\chapter{Aplicaci\'on a series temporales}

La predicci\'on de series temporales es importante para gobiernos y empresas, ya
que para este tipo de organismos es importante conocer, a priori, estimaciones
sobre demandas, costos o ingresos que se producir\'an en el futuro y, de esta
manera poder obrar en consecuencia.\\

El m\'etodo m\'as comunmente utilizado para la predicci\'on de series temporales
es el m\'etodo de $Box$ y $Jenkins$, el cual hemos visto en el cap\'{\i}tulo 
anterior. Las razones por las que este m\'etodo es tan usado son dos:
\begin{itemize}
\item Es un modelo sencillo.
\item Sus predicciones son ``precisas''.
\end{itemize}
Pero este m\'etodo tambi\'en tiene desvenjas y son que, debido a que utiliza
un conjunto de valores previos para predecir el futuro comportamiento de la
serie temporal, le es imposible encontrar patrones sutiles en el comportamiento
de la serie temporal.\\

Con el paso del tiempo aumenta la velocidad y potencia de los ordenadores, lo
que favorece la busqueda de otros m\'etodos para poder predecir el
comportamiento de las series temporales en el futuro, uno de estos m\'etodos es
el uso de \textbf{RNA}. Las \textbf{RNA} son utiles para el reconocimiento de
patrones, como vimos en el primer cap\'{\i}tulo.
%
\newpage
%

%
% REDES NEURONALES RBF
%

%
% REDES NEURONALES RBF
%

\section{Redes neuronales RBF}

De todos los tipos de \textbf{RNA} uno de los m\'as empleados es el
$RBF$\footnote{Radial Basis Function.}. Este tipo de red ha sido empleado en
la predicci\'on de series temporales con \'exito ya que puede ser entrenado para
encontrar complejas relaciones entre datos.\\

Pero no todo son ventajas, este tipo
de red tambi\'en tiene inconvenientes y es que si utilizamos datos ``lejanos'' a
los patrones de entrenamiento de la red los resultados obtenidos no son muy
fiables.\\

Las redes de tipo $RBF$ y el m\'odelo de $Box$ y $Jenkins$ se complementan muy
bien ya que las redes $RBF$ pueden encontrar patrones en los datos, pero no
pueden dar resultados significativos si empleamos, como datos de entrada, datos
``lejanos'' de los patrones de entrenamiento. Y por el contrario el modelo de
$Box$ y $Jenkins$ da buenos resultados, en general, pero al no poder determinar
patrones de comportamiento en los datos de la serie puede dar predicciones
imprecisas.\\

Ambos m\'etodos se pueden combinar para obtener un modelo m\'as fiable.


%
% DESCRIPCION DE ESTA RED
%

%
% DESCRIPCION DE ESTA RED
% 

\subsection{Descripci\'on de la red RBF}

Este tipo de red es una red de tres capas:
\begin{itemize}
\item La primera capa es la capa de entrada de datos, la cual suministra los
datos de entrada a los EP's de la segunda capa. 
\item La segunda capa, capa oculta, se comporta de una forma muy diferente a
otro tipo de redes. Cada EP de esta capa representa a un grupo de datos
centrados en un punto fijo y con radio conocido.\\

Por comodidad supondremos que en esta capa tenemos tres EP's.
\item La tercera capa, capa de salida, es la capa que procesa los datos
obtenidos en la segunda capa y nos devuelve la predicci\'on para la serie
temporal.\\
\end{itemize}
\begin{figure}[!h]
\setlength{\unitlength}{3947sp}%
%
\begingroup\makeatletter\ifx\SetFigFont\undefined%
\gdef\SetFigFont#1#2#3#4#5{%
  \reset@font\fontsize{#1}{#2pt}%
  \fontfamily{#3}\fontseries{#4}\fontshape{#5}%
  \selectfont}%
\fi\endgroup%
\begin{picture}(6837,4347)(514,-4273)
\thinlines
\put(3976,-811){\circle{670}}
\put(6376,-1711){\circle{670}}
\put(3976,-2611){\circle{670}}
\put(1501,-1711){\circle{670}}
\put(3976,-1711){\circle{670}}
\put(4351,-811){\vector( 2,-1){1650}}
\put(4351,-2611){\vector( 2, 1){1590}}
\put(6751,-1711){\vector( 1, 0){450}}
\put(526,-1711){\vector( 1, 0){600}}
\put(1876,-1711){\vector( 2, 1){1740}}
\put(1876,-1711){\vector( 1, 0){1725}}
\put(1876,-1711){\vector( 2,-1){1740}}
\put(4351,-1711){\vector( 1, 0){1650}}
\put(3376,-4036){\line( 1, 0){3000}}
\put(6376,-4036){\vector( 0, 1){1950}}
\put(2401,-4261){\framebox(975,450){}}
\put(3301,-61){\makebox(0,0)[lb]{\smash{\SetFigFont{12}{14.4}{\rmdefault}{\mddefault}{\updefault}EP's Ocultos}}}
\put(901,-61){\makebox(0,0)[lb]{\smash{\SetFigFont{12}{14.4}{\rmdefault}{\mddefault}{\updefault}EP de Entrada}}}
\put(5851,-61){\makebox(0,0)[lb]{\smash{\SetFigFont{12}{14.4}{\rmdefault}{\mddefault}{\updefault}EP de Salida}}}
\put(1426,-1786){\makebox(0,0)[lb]{\smash{\SetFigFont{12}{14.4}{\rmdefault}{\mddefault}{\updefault}X}}}
\put(2476,-886){\makebox(0,0)[lb]{\smash{\SetFigFont{12}{14.4}{\rmdefault}{\mddefault}{\updefault}$w_{1,1}[2]$}}}
\put(2476,-1636){\makebox(0,0)[lb]{\smash{\SetFigFont{12}{14.4}{\rmdefault}{\mddefault}{\updefault}$w_{2,1}[2]$}}}
\put(2476,-2486){\makebox(0,0)[lb]{\smash{\SetFigFont{12}{14.4}{\rmdefault}{\mddefault}{\updefault}$w_{3,1}[2]$}}}
\put(3826,-886){\makebox(0,0)[lb]{\smash{\SetFigFont{12}{14.4}{\rmdefault}{\mddefault}{\updefault}$C_1$}}}
\put(3826,-1786){\makebox(0,0)[lb]{\smash{\SetFigFont{12}{14.4}{\rmdefault}{\mddefault}{\updefault}$C_2$}}}
\put(3826,-2686){\makebox(0,0)[lb]{\smash{\SetFigFont{12}{14.4}{\rmdefault}{\mddefault}{\updefault}$C_3$}}}
\put(4851,-886){\makebox(0,0)[lb]{\smash{\SetFigFont{12}{14.4}{\rmdefault}{\mddefault}{\updefault}$w_{1,1}[3]$}}}
\put(4851,-1636){\makebox(0,0)[lb]{\smash{\SetFigFont{12}{14.4}{\rmdefault}{\mddefault}{\updefault}$w_{1,2}[3]$}}}
\put(4851,-2486){\makebox(0,0)[lb]{\smash{\SetFigFont{12}{14.4}{\rmdefault}{\mddefault}{\updefault}$w_{1,3}[3]$}}}
\put(2800,-4080){\makebox(0,0)[lb]{\smash{\SetFigFont{12}{14.4}{\rmdefault}{\mddefault}{\updefault}$w_0$}}}
\put(3826,-3961){\makebox(0,0)[lb]{\smash{\SetFigFont{12}{14.4}{\rmdefault}{\mddefault}{\updefault}Constante}}}
\put(7351,-1786){\makebox(0,0)[lb]{\smash{\SetFigFont{12}{14.4}{\rmdefault}{\mddefault}{\updefault}y}}}
\put(6176,-1786){\makebox(0,0)[lb]{\smash{\SetFigFont{12}{14.4}{\rmdefault}{\mddefault}{\updefault}$I_1[3]$}}}
\end{picture}

\caption{Red neuronal del tipo $RBF$.}
\end{figure}

Una forma l\'ogica de escoger los $C_i$ ser\'{\i}a agrupando todos los patrones
de entrenamiento en tantos discos como EP's tengamos en la capa oculta y tomar
como $C_i$ el centro de estos discos y como radio el radio de los mismos.


%
% FUNCIONAMIENTO DE ESTA RED
%

%
% FUNCIONAMIENTO DE ESTA RED
%

\subsection{Funcionamiento de una red RBF}

Supongamos que tenemos un vector $X=(x_1,\dots,x_n)$, donde $\{x_i\}_{i=1}^n$
son valores de una serie temporal cualquiera.\\

El EP de la primera capa, capa de entrada de datos, suministra el vector $X$ a
los EP's de la segunda capa, por lo tanto los pesos sin\'apticos de estas
conexiones ser\'an $1$:
\begin{itemize}
\item $w_{1,1}[2]=1$.
\item $w_{2,1}[2]=1$.
\item $w_{3,1}[2]=1$.
\end{itemize}
%
\newpage
%
Los EP's de la segunda capa funcionan de la siguiente manera:
\begin{itemize}
\item Calculan la distancia del vector $X$ al centro del grupo de datos que
representan $C_i=(c_{1,i},\dots,c_{n,i})$ para $i=1,2,3$.\\

Para el calculo de la distancia normalmente se emplea la distancia euclidea:
\begin{displaymath}
d(X,C_i) =\sqrt{\sum_{j=1}^n(x_j-c_{j,i})^2}\quad para\ i=1,2,3
\end{displaymath}
aunque tambi\'en puede usarse cualquier otra distancia.
\item La distancia $d(X,C_i)$ es transformada por alguna funci\'on y el
resultado obtenido es la salida del nodo.\\

Una funci\'on tipicamente usada para transformar las distancias $d(X,C_i)$ es
la siguiente:
\begin{displaymath}
\phi_i(x) = e^{-(\frac{x}{r_i})^2}
\end{displaymath}
donde $r_i$ es el radio del $i$-\'esimo EP de la segunda capa, el cual es
conocido. Estas funciones reciben el nombre de ``\emph{funciones gausianas}''.
\end{itemize}
Los pesos sin\'apticos asociados a las conexiones de esta capa con la capa de
salida se determinan de forma experimental mediante algoritmos como el de 
``backpropagation'' que ya vimos.\\ \\
%
El EP de la tercera capa funciona de la siguiente manera:
\begin{itemize}
\item Calcula la entrada total de red:
\begin{displaymath}
I_1[3] = I_1[3](X) = \sum_{i=1}^3 w_{1,i}[3]\cdot \phi_i(d(X,C_i))
\end{displaymath}
\item Se le suma un valor constante a la entrada total de red y el valor
obtenido es la predicci\'on sobre el comportamiento de la serie temporal:
\begin{displaymath}
y = w_0+I_1[3][X]
\end{displaymath}
\end{itemize}


%
% VENTAJAS DE LAS REDES RBF
%

%
% VENTAJAS DE LAS REDES RBF
%

\subsection{Ventajas de las redes RBF}

Los complejos sistemas no lineales, como son las series temporales, normalmente
son dificiles de modelizar usando t\'ecnicas de regresi\'on lineal est\'andar.
A diferencia de la regresi\'on las redes neuronales son no lineales. Sus
par\'ametros se determinan mediante entrenamiento con algoritmos como el de
``backpropagation''.\\

El mayor problema que tienen las redes neuronales est\'andar es el de determinar
su par\'ametros. Pero en las redes $RBF$ los \'unicos par\'ametros a determinar
son los pesos sin\'apticos de interconexi\'on entre las capas segunda y tercera.
Estos tres par\'ametros\footnote{$w_{1,1}[3]$, $w_{1,2}[3]$ y $w_{1,3}[3]$.}
son la soluci\'on de un sistema lineal y se pueden obtener mediante
interpolaci\'on\footnote{C. Bishop, Improving the generalization propierties of
radial basis function neural networks, Neural Computat. $3$ $(1.991)$
$579-588$.}. Por lo tanto estos par\'ametros se calculan de forma m\'as r\'apida
que para otro tipo de redes.



%
% INCONVENIENTES DE LAS REDES RBF
%

%
% INCONVENIENTES DE LAS REDES RBF
%

\subsection{Inconvenientes de las redes RBF}

Las redes $RBF$ dan predicciones poco fiables cuando el vector de entrada est\'a
lejos del conjunto de entrenamiento. Est\'a lejan\'{\i}a nos la da la distancia
que estemos utilizando, normalmente ser\'a la distancia euclidea.\\

En el caso de las funciones gausianas: 
\begin{displaymath}
\phi_i(x)=e^{-(\frac{x}{r_i})^2} = \frac{1}{e^{(\frac{x}{r_i})^2}}
\end{displaymath}
cuando el vector de entrada est\'a lejos del centro de los grupos de datos $C_i$
se tiene que:
\begin{displaymath}
\frac{1}{e^{(\frac{x}{r_i})^2}}\to 0\quad donde\ x=d(X,C_i)
\end{displaymath}
Luego las redes $RBF$ utilizando funciones gausianas son buenas solamente cuando
el vector de entrada est\'a proximo a los centros $C_i$. En caso contrario las
predicciones obtenidas son poco fiables.


%
\newpage
%

%
% REDES RBF CON FACTOR DE SEGURIDAD
%

%
% REDES RBF CON FACTOR DE SEGURIDAD
%

\section{Redes RBF con factor de seguridad}

En los apartados anteriores hemos visto como en algunos casos las predicciones
de una red $RBF$ no son fiables. Luego no todas las predicciones de una red de
este tipo son buenas predicciones, por lo tanto es muy importante identificar
que predicciones son buenas y cuales no lo son.\\

La forma m\'as directa de solucionar este problema es tener una red $RBF$ que
distinga las buenas predicciones de las malas.\\

Las funciones gausianas $\phi_i$ toman los valores entre $1.0$ y $0.0$. Para
distinguir las buenas de las malas predicciones utilizaremos un dato al que
denominaremos ``\emph{factor de seguridad}'', el cual vendr\'a dado por los
valores que toman las funciones $\phi_i$.\\

La salida de las neuronas de la segunda capa ser\'a de la forma:
\begin{displaymath}
\phi_i(d(X,C_i)) = e^{-(\frac{d(X,C_i)}{r_i})^2}
\end{displaymath}
sabemos que $\phi_i(x)\in [0,1]$ y el exponente de la funci\'on exponencial
de $\phi_i$ nos indica lo siguiente:
\begin{itemize}
\item Si $\phi_i(d(X,C_i))\approx 1$ entonces $X$ est\'a pr\'oximo al centro
$C_i$ y su aportaci\'on a la predicci\'on final es buena.
\item Si $\phi_i(d(X,C_i))\approx 0$ entonces $X$ no est\'a pr\'oximo al centro
$C_i$ y su aportaci\'on a la predicci\'on final puede estropear esta.
\end{itemize}
En el caso de que todos los EP's tengan valores de $\phi_i$ pr\'oximos a cero la
predicci\'on obtenida no ser\'a buena.\\

El problema es como obtener una medida de la precisi\'on de la predicci\'on
utilizando estos valores. A primera vista se le puede ocurrir a uno utilizar
el m\'aximo de los $\phi_i$, pero esto representa un problema y es que puede
haber varios EP's con $\phi_i \approx 1$ que hagan que la predicci\'on sea
buena y, por el hecho de existir un $\phi_i$ grande desechar esta
predicci\'on.
%
\newpage
%
Para dar una medida de la precisi\'on de la predicci\'on se utiliza la siguiente
f\'ormula recursiva:
\begin{equation}\label{eq:FormulaRecursiva}
CF_i=CF_{i-1}+(1-CF_{i-1})\cdot max_{i}(\{\phi_j(d(X,C_j))\}_{j=1}^N)
\end{equation}
donde:
\begin{itemize}
\item $N$ es el n\'umero de EP's en la capa oculta.
\item $max_i$ ser\'a el $i$-\'esimo mayor valor de los $\phi_j$.
\item $CF_0 = 0$ por definici\'on.
\item $CF_i$ es el $i$-\'esimo factor de seguridad.
\item $CF$ es el factor de seguridad.
\end{itemize}
%
\begin{figure}[!h]
\setlength{\unitlength}{3947sp}%
%
\begingroup\makeatletter\ifx\SetFigFont\undefined%
\gdef\SetFigFont#1#2#3#4#5{%
  \reset@font\fontsize{#1}{#2pt}%
  \fontfamily{#3}\fontseries{#4}\fontshape{#5}%
  \selectfont}%
\fi\endgroup%
\begin{picture}(6762,4347)(514,-4273)
\thinlines
\put(3976,-811){\circle{670}}
\put(6376,-1711){\circle{670}}
\put(3976,-2611){\circle{670}}
\put(1501,-1711){\circle{670}}
\put(3976,-1711){\circle{670}}
\put(6376,-811){\circle{670}}
\put(4351,-811){\vector( 2,-1){1650}}
\put(4351,-2611){\vector( 2, 1){1590}}
\put(6751,-1711){\vector( 1, 0){450}}
\put(526,-1711){\vector( 1, 0){600}}
\put(1876,-1711){\vector( 2, 1){1740}}
\put(1876,-1711){\vector( 1, 0){1725}}
\put(1876,-1711){\vector( 2,-1){1740}}
\put(4351,-1711){\vector( 1, 0){1650}}
\put(2401,-4261){\framebox(975,450){}}
\put(4351,-2611){\vector( 1, 1){1650}}
\put(4351,-1711){\vector( 2, 1){1680}}
\put(4351,-811){\vector( 1, 0){1650}}
\put(3376,-4036){\line( 1, 0){1500}}
\put(4876,-4036){\vector( 1, 3){997.500}}
\put(6751,-811){\vector( 1, 0){450}}
\put(3301,-61){\makebox(0,0)[lb]{\smash{\SetFigFont{12}{14.4}{\rmdefault}{\mddefault}{\updefault}EP's Ocultos}}}
\put(901,-61){\makebox(0,0)[lb]{\smash{\SetFigFont{12}{14.4}{\rmdefault}{\mddefault}{\updefault}EP de Entrada}}}
\put(1426,-1786){\makebox(0,0)[lb]{\smash{\SetFigFont{12}{14.4}{\rmdefault}{\mddefault}{\updefault}X}}}
\put(3826,-886){\makebox(0,0)[lb]{\smash{\SetFigFont{12}{14.4}{\rmdefault}{\mddefault}{\updefault}$C_1$}}}
\put(3826,-1786){\makebox(0,0)[lb]{\smash{\SetFigFont{12}{14.4}{\rmdefault}{\mddefault}{\updefault}$C_2$}}}
\put(3826,-2686){\makebox(0,0)[lb]{\smash{\SetFigFont{12}{14.4}{\rmdefault}{\mddefault}{\updefault}$C_3$}}}
\put(2800,-4080){\makebox(0,0)[lb]{\smash{\SetFigFont{12}{14.4}{\rmdefault}{\mddefault}{\updefault}$w_0$}}}
\put(3826,-3961){\makebox(0,0)[lb]{\smash{\SetFigFont{12}{14.4}{\rmdefault}{\mddefault}{\updefault}Constante}}}
\put(6176,-886){\makebox(0,0)[lb]{\smash{\SetFigFont{12}{14.4}{\rmdefault}{\mddefault}{\updefault}$I_1[3]$}}}
\put(6250,-1786){\makebox(0,0)[lb]{\smash{\SetFigFont{12}{14.4}{\rmdefault}{\mddefault}{\updefault}CF}}}
\put(7351,-836){\makebox(0,0)[lb]{\smash{\SetFigFont{12}{14.4}{\rmdefault}{\mddefault}{\updefault}y}}}
\put(7276,-1786){\makebox(0,0)[lb]{\smash{\SetFigFont{12}{14.4}{\rmdefault}{\mddefault}{\updefault}CF}}}
\put(5851,-61){\makebox(0,0)[lb]{\smash{\SetFigFont{12}{14.4}{\rmdefault}{\mddefault}{\updefault}EP's de Salida}}}
\end{picture}

\caption{Red neuronal del tipo $RBF$ con factor de seguridad.}
\end{figure}
%

Normalmente se hace que $i$ varie entre $1$ y $N$, pero tambi\'en se puede hacer
que varie entre $1$ y $M$ donde $M<N$. El factor de seguridad que mejor nos
indica la precisi\'on de las predicciones ser\'a $CF=CF_N$.\\

En este tipo de red el calculo de los pesos se realiza de igual forma que en la
red $RBF$ sin factor de seguridad:
\begin{itemize}
\item Los pesos de interconexi\'on entre la primera y segunda capa son todos
uno:
\begin{displaymath}
w_{1,1}[2]=w_{2,1}[2]=w_{3,1}[2]=1
\end{displaymath}
\item Los pesos de interconexi\'on entre la segunda y tercera capa se calculan
de la siguiente forma:
\begin{itemize}
\item Aquellos que van al EP de predicci\'on se calculan como en el caso
anterior. Estos pesos son:
\begin{displaymath}
w_{1,1}[3]\quad w_{1,2}[3]\quad w_{1,3}[3]
\end{displaymath}
\item Aquellos que van al EP que nos da el factor de seguridad son todos iguales
a uno:
\begin{displaymath}
w_{2,1}[3]=w_{2,2}[3]=w_{2,3}[3]=1
\end{displaymath}
\end{itemize}
\end{itemize}


%
% MODELO HIBRIDO
%

%
% MODELO HIBRIDO
%

\section{El modelo hibrido}

Hemos ido mejorando los modelos de redes.\\

En primer lugar teniamos una red $RBF$, la cual pod\'{\i}a producir predicciones
poco fiables y no teniamos ninguna forma de saber que predicciones eran fiables
y cuales no. Para mejorar esto modificamos la red e introdujimos un factor de
seguridad, el cual nos indicaba cuando una predicci\'on era fiable o no.\\

Aunque la red $RBF$ con factor de seguridad ofrece m\'as garantias que la red
$RBF$ sin factor de seguridad no tenemos resueltos todos los problemas. En el 
caso de obtener una predicci\'on no fiable la red $RBF$ no nos da una
predicci\'on fiable para ese caso, \'unicamente nos indica que la predicci\'on
no es fiable.\\

Para solucionar esto utilizamos un modelo hibrido, es decir un modelo que
utiliza una red neuronal y un m\'etodo de predicci\'on, en nuestro caso el de
$Box$ y $Jenkins$, y con ambos resultados nos ofrece una predicci\'on.
%
\begin{figure}[!h]
\setlength{\unitlength}{3947sp}%
%
\begingroup\makeatletter\ifx\SetFigFont\undefined%
\gdef\SetFigFont#1#2#3#4#5{%
  \reset@font\fontsize{#1}{#2pt}%
  \fontfamily{#3}\fontseries{#4}\fontshape{#5}%
  \selectfont}%
\fi\endgroup%
\begin{picture}(7437,5697)(514,-5623)
\thinlines
\put(3976,-811){\circle{670}}
\put(6376,-1711){\circle{670}}
\put(3976,-2611){\circle{670}}
\put(1501,-1711){\circle{670}}
\put(3976,-1711){\circle{670}}
\put(6376,-811){\circle{670}}
\put(4351,-811){\vector( 2,-1){1650}}
\put(4351,-2611){\vector( 2, 1){1590}}
\put(526,-1711){\vector( 1, 0){600}}
\put(1876,-1711){\vector( 2, 1){1740}}
\put(1876,-1711){\vector( 1, 0){1725}}
\put(1876,-1711){\vector( 2,-1){1740}}
\put(4351,-1711){\vector( 1, 0){1650}}
\put(2401,-4261){\framebox(975,450){}}
\put(4351,-2611){\vector( 1, 1){1650}}
\put(4351,-1711){\vector( 2, 1){1680}}
\put(4351,-811){\vector( 1, 0){1650}}
\put(3376,-4036){\line( 1, 0){1500}}
\put(4876,-4036){\vector( 1, 3){997.500}}
\put(826,-5161){\vector( 1, 0){900}}
\put(826,-5161){\line( 0, 1){3450}}
\put(1726,-5461){\framebox(1350,600){}}
\put(5176,-5611){\framebox(1950,900){}}
\put(6376,-2086){\vector( 0,-1){2625}}
\put(3076,-5161){\vector( 1, 0){2100}}
\put(6751,-811){\line( 1, 0){225}}
\put(6976,-811){\vector( 0,-1){3900}}
\put(7126,-5161){\vector( 1, 0){675}}
\put(3301,-61){\makebox(0,0)[lb]{\smash{\SetFigFont{12}{14.4}{\rmdefault}{\mddefault}{\updefault}EP's Ocultos}}}
\put(901,-61){\makebox(0,0)[lb]{\smash{\SetFigFont{12}{14.4}{\rmdefault}{\mddefault}{\updefault}EP de Entrada}}}
\put(1426,-1786){\makebox(0,0)[lb]{\smash{\SetFigFont{12}{14.4}{\rmdefault}{\mddefault}{\updefault}X}}}
\put(3826,-886){\makebox(0,0)[lb]{\smash{\SetFigFont{12}{14.4}{\rmdefault}{\mddefault}{\updefault}$C_1$}}}
\put(3826,-1786){\makebox(0,0)[lb]{\smash{\SetFigFont{12}{14.4}{\rmdefault}{\mddefault}{\updefault}$C_2$}}}
\put(3826,-2686){\makebox(0,0)[lb]{\smash{\SetFigFont{12}{14.4}{\rmdefault}{\mddefault}{\updefault}$C_3$}}}
\put(2800,-4080){\makebox(0,0)[lb]{\smash{\SetFigFont{12}{14.4}{\rmdefault}{\mddefault}{\updefault}$w_0$}}}
\put(3826,-3961){\makebox(0,0)[lb]{\smash{\SetFigFont{12}{14.4}{\rmdefault}{\mddefault}{\updefault}Constante}}}
\put(6176,-886){\makebox(0,0)[lb]{\smash{\SetFigFont{12}{14.4}{\rmdefault}{\mddefault}{\updefault}$I_1[3]$}}}
\put(6250,-1786){\makebox(0,0)[lb]{\smash{\SetFigFont{12}{14.4}{\rmdefault}{\mddefault}{\updefault}CF}}}
\put(5851,-61){\makebox(0,0)[lb]{\smash{\SetFigFont{12}{14.4}{\rmdefault}{\mddefault}{\updefault}EP's de Salida}}}
\put(6826,-3286){\makebox(0,0)[lb]{\smash{\SetFigFont{12}{14.4}{\rmdefault}{\mddefault}{\updefault}y}}}
\put(6076,-3286){\makebox(0,0)[lb]{\smash{\SetFigFont{12}{14.4}{\rmdefault}{\mddefault}{\updefault}CF}}}
\put(1900,-5236){\makebox(0,0)[lb]{\smash{\SetFigFont{12}{14.4}{\rmdefault}{\mddefault}{\updefault}B-J Modelo}}}
\put(5350,-5200){\makebox(0,0)[lb]{\smash{\SetFigFont{12}{14.4}{\rmdefault}{\mddefault}{\updefault}Predicci\'on Hibrida}}}
\put(7951,-5236){\makebox(0,0)[lb]{\smash{\SetFigFont{12}{14.4}{\rmdefault}{\mddefault}{\updefault}Y}}}
\end{picture}

\caption{Modelo de red hibrida utilizando $RBF$ y el modelo de $Box$ y
$Jenkins$.}
\end{figure}
%
\newpage
%
En este modelo tenemos tres datos para realizar la predicci\'on $Y$:
\begin{itemize}
\item La predicci\'on dada por la red $RBF$, $y_{RBF}$.
\item El factor de seguridad, $CF$.
\item La predicci\'on dada por el modelo de $Box$ y $Jenkins$, $y_{B-J}$.
\end{itemize}
Podemos combinar estos datos de varias formas para obtener la predicci\'on
deseada sobre el comportamiento de la serie temporal.

\subsection{M\'etodo de tolerancia}

En este m\'etodo el usuario dar\'a una tolerancia, $Tol$, para el factor de
seguridad. Esta tolerancia se determinar\'a experimentalmente para obtener un
grado alto de precisi\'on sobre las predicciones.\\ \\
%
El algoritmo para este m\'etodo es el siguiente:
\begin{quote}
if ( $CF\geq Tol$ ) then
\begin{quote}
$Y=y_{RBF}$
\end{quote}
else
\begin{quote}
$Y=y_{B-J}$
\end{quote}
\end{quote}

\subsection{M\'etodo de las aproximaciones de igual importancia}

Este m\'etodo es muy similar al anterior y en \'el tambi\'en se ha de
especificar una tolerancia, $Tol$, la cual se determina de igual forma que en el
m\'etodo anterior.\\

En el m\'etodo anterior despreciamos la predicci\'on dada por el modelo de
$Box$ y $Jenkins$ cuando $CF\geq Tol$, suponiendo esta predicci\'on como peor
que la obtenida por la red $RBF$, pero esto puede no ser cierto. Es por esta
raz\'on que en este m\'etodo se utiliza la media de ambas predicciones cuando
$CF\geq Tol$. El algoritmo para este m\'etodo es el siguiente:
\begin{quote}
if ( $CF\geq Tol$ ) then
\begin{quote}
$Y=\frac{y_{RBF}+y_{B-J}}{2}$
\end{quote}
else
\begin{quote}
$Y=y_{B-J}$
\end{quote}
\end{quote}
\newpage
\subsection{M\'etodo de la aproximaci\'on seg\'un el factor de seguridad}

Este m\'etodo combina las predicciones dadas tanto por la red $RBF$ como por el
modelo de $Box$ y $Jenkins$, pero da tanta importancia a la predicci\'on de la
red $RBF$ como seguridad nos indique el factor de seguridad. Es decir si el 
factor de seguridad es $0.80$ en la predicci\'on final el $80\ \%$ ser\'a
aportado por $y_{RBF}$ y el $20\ \%$ restante ser\'a aportado por $y_{B-J}$.\\

La predicci\'on por este m\'etodo vendr\'a dada por la siguiente f\'ormula:
\begin{displaymath}
Y=(CF\cdot y_{RBF})\cdot (1-CF)\cdot y_{B-J}
\end{displaymath}


%
% TEST DE PRECISION
%

%
% TEST DE PRECISION
%

\section{Test de precisi\'on para las predicciones}

El criterio por el cual elegiremos un m\'etodo de predicci\'on ser\'a el
grado de precisi\'on que tenga el m\'etodo. Entenderemos que un m\'etodo es
preciso cuando sus predicciones se ajusten a la realidad observada. \\

El m\'etodo m\'as utilizado para medir la precisi\'on en modelos para series
temporales es el denominado ``\emph{Desviaci\'on Absoluta Principal}'' al que
denominaremos $MAD$:
\begin{displaymath}
MAD = \frac{1}{T}\sum_{i=1}^T(Y_i-\widehat{Y}_i)
\end{displaymath}
donde:
\begin{itemize}
\item $T$ es un n\'umero de predicciones hechas por el modelo.
\item $Y_i=Z_i-\mu$ y $\mu = E[Z_i]$.
\item $\widehat{Y}_i$ es una predicci\'on para $Y_{i-1}$.
\end{itemize}

