%
% VENTAJAS DE LAS REDES RBF
%

\subsection{Ventajas de las redes RBF}

Los complejos sistemas no lineales, como son las series temporales, normalmente
son dificiles de modelizar usando t\'ecnicas de regresi\'on lineal est\'andar.
A diferencia de la regresi\'on las redes neuronales son no lineales. Sus
par\'ametros se determinan mediante entrenamiento con algoritmos como el de
``backpropagation''.\\

El mayor problema que tienen las redes neuronales est\'andar es el de determinar
su par\'ametros. Pero en las redes $RBF$ los \'unicos par\'ametros a determinar
son los pesos sin\'apticos de interconexi\'on entre las capas segunda y tercera.
Estos tres par\'ametros\footnote{$w_{1,1}[3]$, $w_{1,2}[3]$ y $w_{1,3}[3]$.}
son la soluci\'on de un sistema lineal y se pueden obtener mediante
interpolaci\'on\footnote{C. Bishop, Improving the generalization propierties of
radial basis function neural networks, Neural Computat. $3$ $(1.991)$
$579-588$.}. Por lo tanto estos par\'ametros se calculan de forma m\'as r\'apida
que para otro tipo de redes.
