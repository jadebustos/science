%
% CONCLUSIONES 
%

\chapter{Conclusiones}

Este trabajo esta orientado a la descripci\'on de \textbf{RNA} y a su
utilizaci\'on para la predicci\'on de series temporales. Por lo que la
descripci\'on de series temporales y modelos de regresi\'on no se ha tratado
en profundidad.

\section{Series temporales}

Las series temporales son procesos que siguen un comportamiento dificil de 
modelizar, de forma exacta, mediante una ecuaci\'on matem\'atica. Por lo tanto
para conocer el comportamiento de una serie temporal en el futuro se ha de
modelizar con funciones matem\'aticas ``simples'', las cuales nos indican, de
forma aproximada, el comportamiento de dicha serie en el futuro.\\

Para empresas, gobiernos, \dots\ es muy importante conocer estos comportamientos
con la mayor precisi\'on posible, por lo tanto se intenta resolver el problema
desde distintos puntos de vista y obtener m\'etodos cada vez m\'as precisos para
la predicci\'on de series temporales.

\section{Redes neuronales}

Las redes neuronales tienen un alto grado de paralelismo y son muy utilizadas 
para el reconocimiento de patrones. Gracias a esto tambi\'en se pueden utilizar
para el reconocimiento de pautas de comportamiento.\\

Es por esto que se utilizan para resolver aquellos problemas en los que no
podemos dar un modelo matem\'atico exacto o aquellos problemas que son de 
dificil resoluci\'on mediante programaci\'on secuencial.

\section{Modelos $ARIMA$}

Para la predicci\'on de series temporales hemos utilizado los modelos $ARIMA$ ya
que son los modelos que m\'as se ajustan a las series temporales, aunque no a
todas, y son un modelo facil de utilizar para predecir el comportamiento de una
serie temporal.\\

Estos modelos tienen inconvenientes derivados de ser modelos univariantes, es 
decir, \'unicamente utiliza los estados anteriores de la serie para determinar
el comportamiento futuro de la serie.\\

En la mayor\'{\i}a de las series temporales intervienen procesos ajenos al
comportamiento pasado de la serie, y es por esto que estos modelos tienen sus
inconvenientes. No pueden reconocer patrones sutiles de comportamiento en la
serie temporal.\\

La utilizaci\'on pr\'actica de modelos $ARIMA$ se ha descrito de una forma
breve, para una informaci\'on m\'as detallada ver \cite{Daniel},
\cite{econometria} y \cite{modeconometricos}.

\section{Predicci\'on con redes neuronales}

Debido al grado de paralelismo y las capacidades de las redes neuronales para
el reconocimiento de patrones solucionan el problema del reconicimiento de
patrones sutiles de comportamiento en las series temporales.\\

A pesar de esto hemos visto que algunas veces la predicci\'on de una red
neuronal puede no ser fiable. Para solucionar estos problemas se introdujeron
los factores de seguridad y los modelos hibridos, mediante los cuales se
obtienen valores m\'as fiables que en los modelos de redes neuronales puros.\\

Para obtener buenos resultados con estos m\'etodos tenemos que elegir el
m\'etodo m\'as preciso. El modelo que mejores resultados da es el hibrido,
pero para dar predicciones con este modelo tenemos tres posibles. La forma 
para elegir uno ser\'a comprobar los tres m\'etodos y elegir el m\'as preciso.\\

Antes de hacer esto hemos de entrenar la red con un conjunto de patrones que
hagan que la red se comporte de la ``mejor'' forma posible para nuestros
propositos.
%
\newpage
%
El modelo hibrido tiene la ventaja de poder ser utilizado con otro m\'etodo
de predicci\'on que no sea el de $Box$ y $Jenkins$ con muy pocos cambios. De
esta manera si no se obtienen resultados precisos con este modelo se puede 
cambiar el modelo de $Box$ y $Jenkins$ por otro modelo, ya sea univariante o
multivariante, con el cual se logren predicciones que se ajusten a nuestras
necesidades.\\

En \cite{RBF} se pueden ver ejemplos del uso de estos m\'etodos a casos reales.
