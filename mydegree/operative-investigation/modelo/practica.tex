%
% MODELOS ARIMA EN LA PRACTICA
%

\section{Modelos ARIMA en la pr\'actica}

Para identificar un modelo utilizaremos todos los datos que tengamos sobre la
serie temporal.

\subsection{Identificaci\'on de la estructura del modelo}

En esta fase tendremos que decidir las transformaciones necesarias para
transformar el proceso en estacionario.
\begin{itemize}
\item Tendremos que determinar el n\'umero de diferencias que se deben aplicar
para que la media sea constante. Este n\'umero, $d$, es normalmente uno o dos.
\item Tendremos que determinar si es necesario transformar la serie para que
tenga varianza constante.
\end{itemize}
%
\newpage
%
Una vez que hemos transformado el proceso en estacionario tendremos que
determinar un modelo para el proceso estacionario, es decir, los \'ordenes $p$ y
$q$ de su representaci\'on $ARMA(p,q)$, y si el proceso es estacional tendremos
que determinar los \'ordenes $P$ y $Q$ de la estructura $ARMA$ estacional.\\

Despu\'es tendremos que realizar un estudio acerca de la estacionalidad. En
caso de presentar estacionalidad con un per\'{\i}odo $s$ tendr\'{\i}amos que 
aplicar la diferencia $(1-B^s)$ para convertirla en estacionaria.\\ \\
%
Para ver si una serie temporal es estacional podemos utilizar:
\begin{itemize}
\item El gr\'afico de la serie, el cual presentar\'a una pauta repetida de
acuerdo con el per\'{\i}odo estacional $s$.
\item La funci\'on de autocorrelaci\'on simple, la cual presentar\'a
coeficientes positivos que decrecer\'an lentamente en los retardos $s$, $2s$,
$3s$, \dots
\end{itemize}

\subsection{Estimaci\'on de los par\'ametros del modelo}

Una vez identificado el modelo estimaremos sus par\'ametros mediante
la mi\-ni\-mi\-za\-ci\'on de la suma de los cuadrados de los errores $a_t$.\\

Para un proceso $AR(p)$ perderemos las $p$ primeras observaciones:
\begin{displaymath}
\sum_{t=p+1}^t a_t^2=S(\phi_1,\dots,\phi_p)
\end{displaymath}
este tipo de estimaciones recibe el nombre de ``estimaciones condicionadas'',
ya que depende de $z_1,\dots,z_p$, que son tomados como dato.\\

Las estimaciones condicionadas obtenidas minimizando $S$ se calculan con las
siguientes hip\'otesis:
\begin{itemize}
\item Modelos $AR$ valores iniciales: $z_1,\dots,z_p$.
\item Modelos $MA$ valores iniciales: $a_0,\dots,a_{-q}$.
\item Modelos $ARMA$ valores iniciales: $z_1,\dots,z_p;a_0,\dots,a_{-q}$.
\end{itemize}
%
\newpage
%
\subsection{Diagn\'ostico del modelo}

Una vez que tenemos construido el modelo hay que comprobar que dicho modelo
representa el comportamiento de la serie temporal, esto lo haremos mediante un
contraste de hip\'otesis. Un buen modelo debe cumplir:
\begin{itemize}
\item Los residuos del modelo estimado se aproximan al comportamiento de un
ruido blanco.
\item El modelo es estacionario e invertible.
\item Los coeficientes son estad\'{\i}sticamente significativos.
\item Los coeficientes del modelo est\'an poco correlacionados entre s\'{\i}.
\item El grado de ajuste es elevado en comparaci\'on al de otros modelos
alternativos.
\end{itemize}
El contraste que m\'as se suele utilizar para comprobar que las
autocorrelaciones en su conjunto no son significativas es el denominado
$Box$-$Pierce$.

\subsection{La esperanza condicionada como predictor \'optimo}

Si tenemos una observacion de longitud $T$ $Z_T=(z_1,\dots,z_T)$ de una serie
temporal y queremos obtener una predicci\'on del valor $z_{T+k}$, donde
$k\in \mathbb{N}$, minimizando el error cuadr\'atico medio. Podemos ver en
\cite{Daniel} que:
\begin{displaymath}
\hat{z}_{T+k}=E[z_{T+k} | Z_T]
\end{displaymath}
minimiza el error cuadr\'atico medio. Utilizando este resultado podemos obtener
el predictor cuando conozcamos las distribuciones condicionadas. Pero esto, en
general, no ocurre y buscaremos predictores, funciones lineales de las
observaciones, qeu minimizen el error de predicci\'on.
%
\newpage
%
\subsection{La ecuaci\'on de predicci\'on de un modelo ARIMA}

Supongamos que tenemos $T$ datos de un proceso $ARIMA(p,d,q)$ cuyos par\'ametros
son conocidos:
\begin{displaymath}
z_t=\phi_1\cdot z_{t-1}+\dots+\phi_h\cdot z_{t-h}+a_t-\theta_1\cdot a_{t-1}-
\dots - \theta_q\cdot a_{t-q}
\end{displaymath}
donde el operador $\Phi_h(B)=\Theta (B)\nabla^d$ puede tener raices en el
circulo unidad, luego $h=p+d$.\\

La predicci\'on \'optima de $z_{T+k}$ ser\'a la esperanza condicionada de esta
variable dada la realizaci\'on $Z_T=(z_1,\dots,z_T)$. Llamando:
\begin{eqnarray*}
\hat{z}_t(j)&=&E[z_{T+j} | Z_T]\\
\hat{a}_t(j)&=&E[a_{T+j} | Z_T]
\end{eqnarray*}
donde $T$ representa el origen de la predicci\'on y $j$ el horizonte.
Sustituyendo $z_t$ en la ecuaci\'on anterior por $z_{T+j}$ y tomando esperanzas
condicionadas a $Z_T$ tendremos:
\begin{equation}\label{eq:Esperanzas}
\hat{z}_T(j)=\phi_1\cdot \hat{z}_T(j-1)+\dots+\phi_h\cdot \hat{z}_t(j-h)-
\theta_1\cdot \hat{a}_T(j-1)-\dots-\hat{a}_T(j-q)
\end{equation}
donde:
\begin{itemize}
\item $\hat{z}_T(-i)=z_{T-i}$ para $i>0$.
\item $\hat{a}_T(i)=0$ para $i>0$.
\item $a_{T+i}=0$ para $i<0$.
\end{itemize}
Para $j=1$ tendremos:
\begin{displaymath}
a_{T+1}=z_{T+1}-\hat{z}_T(1)
\end{displaymath}
que indica que las perturbaciones del modelo son los errores de predicci\'on
de un per\'{\i}odo por delante.\\

La ecuaci\'on $(\ref{eq:Esperanzas})$ indica que despu\'es de unos valores
iniciales la predicci\'on queda determinada por la parte autoregresiva del
modelo. En efecto, para $j>q$ las predicciones de $(\ref{eq:Esperanzas})$
satisfacen la ecuaci\'on:
\begin{displaymath}
(1-\phi_1\cdot B-\dots-\phi_h\cdot B^h)\hat{z}_{T}(j) = 0
\end{displaymath}
siendo $B\cdot \hat{z}_T(j)=\hat{z}_t(j-1)$, es decir, el operador $B$ act\'ua
ahora sobre el horizonte de la previsi\'on $j$ ya que $T$, origen, el fijo.
