%
% PROCESO AR(p) 
%

\section{El proceso autoregresivo general AR(p)}

\subsection{Descripci\'on}

Diremos que un proceso $z_t$ es autoregresivo de orden $p$ si:
\begin{equation}\label{eq:AR(p)I}
\widetilde{z}_t = \phi_1\cdot \widetilde{z}_{t-1}+\dots+\phi_p\cdot
\widetilde{z}_{t-p}+a_t
\end{equation}
donde $\widetilde{z}_t=z_t-\mu$ y $a_t$ es un proceso de ruido blanco
independiente de $z_{t-h}$ para todo $h\ge 1$ y las $\phi_i$ son constantes.\\

Podemos expresar la ecuaci\'on $(\ref{eq:AR(p)I})$ utilizando este operador:
\begin{displaymath}
\widetilde{z}_t=(\phi_1\cdot B+\dots +\phi_p\cdot B^p)\widetilde{z}_t +a_t
\end{displaymath}
o lo que es lo mismo:
\begin{displaymath}
(1-\phi_1\cdot B - \phi_2\cdot B^2-\dots - \phi_p\cdot B^p)\widetilde{z}_t
= a_t
\end{displaymath}

\subsection{Ecuaci\'on caracter\'{\i}stica}

Para simplificar llamemos
$\phi_p(B)= 1-\phi_1\cdot B-\phi_2\cdot B^2-\dots -\phi_p\cdot B^p$, entonces
se tiene que $\phi_p(B)$ es un polinomio de grado $p$ en la variable $B$. 
Llamaremos ``\emph{ecuaci\'on caracter\'{\i}stica}'' del proceso a la
ecuaci\'on:
\begin{equation}\label{eq:ecucaracARp}
\phi_p(B) = 0
\end{equation}
Esta ecuaci\'on es una ecuaci\'on en $B$, la cual tiene $p$ raices, que
generalmente son distintas\footnote{Ver Ap\'endice $15.A$ de \cite{Daniel}.}.
Sean estas raices $G_1^{-1}$,\dots,$G_p^{-1}$, podemos expresar
$(\ref{eq:ecucaracARp})$ de la siguiente forma:
\begin{equation}
\phi_p(B)= \prod_{i=1}^p (1-G_i\cdot B) 
\end{equation}

Teniendo en cuenta la ecuaci\'on caracter\'{\i}stica de un proceso
\textbf{AR(p)} podemos expresar su ecuaci\'on como:
\begin{equation}\label{eq:AR(p)II}
\phi_p(B)\widetilde{z}_t = a_t
\end{equation}
Esta ecuaci\'on es la expresi\'on general de un proceso autoregresivo.\\ \\
%
El proceso es estacionario si $|G_i|<1$ para todo $i$.\\ \\
%
Para obtener informaci\'on m\'as detallada consultar \cite{Daniel}.
