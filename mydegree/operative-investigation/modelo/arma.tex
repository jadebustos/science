%
% PROCESO ARMA(p,q) 
%

\section{El proceso ARMA(p,q)}

Estos procesos son una generalizaci\'on que permite combinar los procesos
$AR(p)$ y $MA(q)$ en un \'unico proceso, $ARMA(p,q)$, que ser\'a $AR(p)$ y
$MA(q)$.

\subsection{Descripci\'on}

La ecuaci\'on de un \textbf{ARMA(p,q)} es:
\begin{displaymath}
(1-\phi_1\cdot B-\phi_2\cdot B^2-\dots -\phi_p\cdot B^p)\widetilde{z}_t=
(1-\theta_1\cdot B-\theta_2\cdot B^2-\dots -\theta_q\cdot B^q)a_t
\end{displaymath}
que se suele denotar:
\begin{equation}
\phi_p(B)\widetilde{z}_t = \theta_q(B)a_t
\end{equation}
El proceso es estacionario cuando las raices de $\phi_p(B)=0$  est\'an fuera del
c\'{\i}rculo unidad, e invertible si lo est\'an las de $\theta_q(B)=0$.
