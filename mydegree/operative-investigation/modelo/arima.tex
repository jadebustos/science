%
% PROCESO ARIMA(p,d,q) 
%

\section{El proceso ARIMA(p,d,q)}

El nombre $ARIMA$ viene de las siglas ``autoregresive integrated moving
average''\footnote{Procesos autoregresivos integrados de media m\'ovil.}.

\subsection{Descripci\'on}

Dado un proceso $ARMA(p,q)$ cualquiera y permitiendo que el operador $AR$
tenga raices unitarias obtenemos procesos no estacionarios del tipo:
\begin{displaymath}
(1-\phi_1\cdot B-\dots-\phi_p\cdot B^p)(1-B)^d\widetilde{z}_t =(1-\theta_1\cdot 
B-\dots-\theta_q\cdot B^q)a_t
\end{displaymath}
Este tipo de procesos recibe el nombre de \textbf{ARIMA($p$,$d$,$q$)}, donde:
\begin{itemize}
\item $p$ es el orden de la parte autoregresiva estacionaria.
\item $d$ es el n\'umero de raices unitarias del operador $AR$. Tambi\'en recibe
el nombre de ``\emph{orden de homogeneidad del proceso}''.
\item $q$ es el orden de la parte media m\'ovil.
\end{itemize}

Llamando $\nabla=1-B$ la ecuaci\'on que describe este proceso se puede
escribir como:
\begin{displaymath}
\phi_p(B)\nabla^d\widetilde{z}_t=\theta_q(B)a_t
\end{displaymath}

Todos los procesos $ARIMA(p,d,q)$ no estacionarios se caracterizan por decrecer
lentamente los coeficientes de la funci\'on de autocorrelaci\'on simple.
%
\newpage
%

%
% PROCESOS ARIMA ESTACIONALES
%

%
% PROCESOS ARIMA ESTACIONALES
%

\subsection{Procesos $ARIMA$ estacionales}

Si la estacionalidad fuese siempre exactamente peri\'odica podr\'{\i}a
eliminarse de la serie como un componente determinista. En general la
estacionalidad es s\'olo aproximadamente constante y la eliminaremos de la serie
tomando diferencias entre observaciones separadas por el per\'{\i}odo
estacional. A estas diferencias las llamaremos ``diferencias estacionales''.%
\\ \\
%
Sea $z_t$ una serie estacional de per\'{\i}odo $s$:
\begin{itemize}
\item $s=12$ para series mensuales.
\item $s=4$ para series cuatrimestrales.
\item \dots\dots
\end{itemize}
y con una estructura como la siguiente:
\begin{equation}\label{eq:serieestacional}
z_t=S_t^{(s)}+n_t
\end{equation}
donde:
\begin{itemize}
\item $S_t^{(s)}$ es el componente estacional determinista fijo:
\begin{displaymath}
S_t^{(s)}=S_{t-k\cdot s}^{(s)}\quad k\in \mathbb{Z}
\end{displaymath}
\item $n_t$ es un proceso estacionario.
\end{itemize}
Llamando $\nabla_s=1-B^s$ y al tomar diferencias de observaciones separadas
$s$ per\'{\i}odos tendremos:
\begin{displaymath}
\nabla_s z_t = \nabla_s n_t
\end{displaymath}
con lo que habremos convertido la serie en estacionaria.\\

Por el contrario, si la estacionalidad no es exactamente constante y se
verifica que:
\begin{displaymath}
S_t^{(s)}=S_{t-k\cdot s} + v_t^{(s)}\quad k\in \mathbb{Z}
\end{displaymath}
donde $v_t^{(s)}$ es un proceso estacionario.\\ \\
%
Si tomamos diferencias estacionales en $(\ref{eq:serieestacional})$:
\begin{displaymath}
\nabla_s z_t = v_t^{(s)}+\nabla_s n_t
\end{displaymath}
obtendremos, tambi\'en, un proceso estacionario.\\

Luego el operador $\nabla_s$ convierte procesos estacionales en procesos
estacionarios.

\subsubsection{Formulaci\'on general}

Sea $z_t$ una serie estacional con per\'{\i}odo $s$ conocido y suponiendo que
tenemos $n=s\cdot h$ observaciones. Entonces podremos dividir la serie total en
$s$ series de $h$ datos, tantas series como nos indique el periodo, que
llamaremos $y_{\tau}^1$,\dots,$y_{\tau}^s$ con $\tau=1,\dots,h$.\\ \\
%
La relaci\'on entre estas series y la original $z_t$ ser\'a:
\begin{equation}\label{eq:relacion}
y_{\tau}^{(j)}=z_{j+s\cdot (\tau-1)}\quad (\tau=1,\dots,h)\ y\ (j=1,\dots,s)
\end{equation}
La relaci\'on que obtenemos es $t=j+s\cdot(\tau-1)$.\\ \\
%
Por ejemplo para series mensuales tendremos:
\begin{itemize}
\item $s=12$ indicar\'{\i}a el mes.
\item $h$ nos indicar\'{\i}a de cuantos a\~nos tendr\'{\i}amos observaciones.
\end{itemize}
%
Construyamos ahora un modelo $ARIMA$ para cada una de estas series y supongamos
que este modelo es ``exactamente'' el mismo para todas. El modelo ser\'a el
siguiente:
\begin{equation}\label{eq:ModeloArima}
(1-\Phi_1\cdot B-\dots-\Phi_P\cdot B^P)(1-B)^Dy_{\tau}^{(j)}=
(1-\Theta_1\cdot B-\dots -\Theta_Q\cdot B^Q)u_{\tau}^{(j)}
\end{equation}
donde $j=1,\dots,s$ y $\tau=1,\dots,h$. Adem\'as como tenemos estacionalidad
$D\geq 1$ ya que si $D=0$ y las series fueran estacionarias, su modelo
podr\'{\i}a escribirse:
\begin{displaymath}
y_{\tau}^{(j)} = \mu_j+ \Psi_j(B)u_{\tau}^{(j)}\quad (\tau=1,\dots,h)\ y\ 
(j=1,\dots,s)
\end{displaymath}
donde $\mu_j=E[\{y_{\tau}^{(j)}\}_{\tau}]$ y como la serie es estacional
tendremos que $\mu_i\neq \mu_j$ para $i\neq j$, con lo que las $s$ series no
podr\'{\i}an tener un modelo com\'un.\\

Podemos escribir los modelos para las $s$ series utilizando
$(\ref{eq:relacion})$:
\begin{displaymath}
By_{\tau}^{(j)}= y_{\tau-1}^{(j)} = z_{j+s\cdot(\tau-2)}=
B^{s}z_{j+s\cdot(\tau-1)}
\end{displaymath}
Luego aplicar $B$ a $y_{\tau}$ es lo mismo que aplicar $B^s$ a $z_t$.\\

Definamos ahora una serie de ruido com\'un, $\alpha_t$, asignando a cada $t$ el
ruido del modelo $(\ref{eq:ModeloArima})$ correspondiente a dicho $t$:
\begin{displaymath}
\alpha_t = \alpha_{j+s\cdot (\tau-1)} = u_{\tau}^{(j)}
\end{displaymath}

Luego los $s$ modelos $(\ref{eq:ModeloArima})$ podemos escribirlos:
\begin{equation}\label{eq:sModelosArima}
(1-\Phi_1\cdot B^s-\dots-\Phi_P\cdot B^{P\cdot s})(1-B^s)^D\widetilde{z}_t=
(1-\Theta_1\cdot B^s-\dots-\Theta_Q\cdot B^{s\cdot Q})\alpha_t
\end{equation}
donde $t=1,\dots,n$.\\

Las series $u_{\tau}^{(j)}$ son, por hip\'otesis, ruido blanco, pero la serie
$\alpha_t$ para $t=1,\dots,n$ normalmente no lo ser\'a\footnote{Existir\'a en
general dependencia entre observaciones contiguas.}. Suponiendo que $\alpha_t$
sigue un proceso $ARIMA$ regular tendremos:
\begin{equation}\label{eq:ArimaAlpha}
\phi_p(B)\nabla^d\alpha_t=\theta_q(B)a_t
\end{equation}
Sustituyendo $(\ref{eq:ArimaAlpha})$ en $(\ref{eq:sModelosArima})$ obtenemos el
modelo completo para el proceso:
\begin{equation}\label{eq:ModeloCompleto}
\Phi_P(B^s)\phi_p(B)\nabla^d\nabla_s^Dz_t =\theta_q(B)\Theta_Q(B^s)a_t
\end{equation}

La ecuaci\'on $(\ref{eq:ModeloCompleto})$, que describe el proceso, recibe el nombre de modelo
$ARIMA\ (P,D,Q)_s\times (p,d,q)$ y fue dado por \emph{Box} y \emph{Jenkins} en
$1.976$. Representa de una forma simple muchos fen\'omenos reales que
encontramos en la pr\'actica. A pesar de su frecuencia no se da siempre.

