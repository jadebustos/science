%
% CONSTRUCCION DEL MODELO CLASICO 
%

\chapter{Construcci\'on del m\'odelo cl\'asico ARIMA}

En este cap\'{\i}tulo, y mientras no se diga lo contrario, supondremos que los
modelos son ``estacionarios'', con lo cual la media es constante.
Luego $\mu_t=\mu = E[z_t]$ para todo $t$.\\

De igual modo se tendr\'a que $\widetilde{z}_t = z_t -\mu$.

\begin{definicion}[Operador de retardo]\label{def:operadorderetardo}
Definiremos el operador de retardo, $B$, de la siguiente manera:
\begin{eqnarray*}
Bz_t=z_{t-1}\\
B^k z_t=\underbrace{B\dots B}_kz_t&=&z_{t-k}
\end{eqnarray*}
Este operador deja invariantes las constantes.
\end{definicion}
\newpage 
%
% MODELO AR(p)
%

%
% PROCESO AR(p) 
%

\section{El proceso autoregresivo general AR(p)}

\subsection{Descripci\'on}

Diremos que un proceso $z_t$ es autoregresivo de orden $p$ si:
\begin{equation}\label{eq:AR(p)I}
\widetilde{z}_t = \phi_1\cdot \widetilde{z}_{t-1}+\dots+\phi_p\cdot
\widetilde{z}_{t-p}+a_t
\end{equation}
donde $\widetilde{z}_t=z_t-\mu$ y $a_t$ es un proceso de ruido blanco
independiente de $z_{t-h}$ para todo $h\ge 1$ y las $\phi_i$ son constantes.\\

Podemos expresar la ecuaci\'on $(\ref{eq:AR(p)I})$ utilizando este operador:
\begin{displaymath}
\widetilde{z}_t=(\phi_1\cdot B+\dots +\phi_p\cdot B^p)\widetilde{z}_t +a_t
\end{displaymath}
o lo que es lo mismo:
\begin{displaymath}
(1-\phi_1\cdot B - \phi_2\cdot B^2-\dots - \phi_p\cdot B^p)\widetilde{z}_t
= a_t
\end{displaymath}

\subsection{Ecuaci\'on caracter\'{\i}stica}

Para simplificar llamemos
$\phi_p(B)= 1-\phi_1\cdot B-\phi_2\cdot B^2-\dots -\phi_p\cdot B^p$, entonces
se tiene que $\phi_p(B)$ es un polinomio de grado $p$ en la variable $B$. 
Llamaremos ``\emph{ecuaci\'on caracter\'{\i}stica}'' del proceso a la
ecuaci\'on:
\begin{equation}\label{eq:ecucaracARp}
\phi_p(B) = 0
\end{equation}
Esta ecuaci\'on es una ecuaci\'on en $B$, la cual tiene $p$ raices, que
generalmente son distintas\footnote{Ver Ap\'endice $15.A$ de \cite{Daniel}.}.
Sean estas raices $G_1^{-1}$,\dots,$G_p^{-1}$, podemos expresar
$(\ref{eq:ecucaracARp})$ de la siguiente forma:
\begin{equation}
\phi_p(B)= \prod_{i=1}^p (1-G_i\cdot B) 
\end{equation}

Teniendo en cuenta la ecuaci\'on caracter\'{\i}stica de un proceso
\textbf{AR(p)} podemos expresar su ecuaci\'on como:
\begin{equation}\label{eq:AR(p)II}
\phi_p(B)\widetilde{z}_t = a_t
\end{equation}
Esta ecuaci\'on es la expresi\'on general de un proceso autoregresivo.\\ \\
%
El proceso es estacionario si $|G_i|<1$ para todo $i$.\\ \\
%
Para obtener informaci\'on m\'as detallada consultar \cite{Daniel}.


%
% MODELO MA(q)
%

%
% PROCESO MA(q) 
%

\section{El proceso de media m\'ovil general MA(q)}

\subsection{Descripci\'on}

La ecuaci\'on de un \textbf{MA(q)} es:
\begin{displaymath}
\widetilde{z}_t=(1-\theta_1\cdot B-\theta_2\cdot B^2-\dots -\theta_q\cdot B^q)
a_t
\end{displaymath}
que se suele denotar:
\begin{equation}\label{eq:MA(q)}
\widetilde{z}_t = \theta_q(B)a_t
\end{equation}
El proceso es invertible cuando las raices de $\theta_q(B)=0$ son, en m\'odulo,
mayores que la unidad. S\'{\i} $q$ es finito entonces el proceso es
estacionario.


%
% MODELO ARMA(p,q)
%

%
% PROCESO ARMA(p,q) 
%

\section{El proceso ARMA(p,q)}

Estos procesos son una generalizaci\'on que permite combinar los procesos
$AR(p)$ y $MA(q)$ en un \'unico proceso, $ARMA(p,q)$, que ser\'a $AR(p)$ y
$MA(q)$.

\subsection{Descripci\'on}

La ecuaci\'on de un \textbf{ARMA(p,q)} es:
\begin{displaymath}
(1-\phi_1\cdot B-\phi_2\cdot B^2-\dots -\phi_p\cdot B^p)\widetilde{z}_t=
(1-\theta_1\cdot B-\theta_2\cdot B^2-\dots -\theta_q\cdot B^q)a_t
\end{displaymath}
que se suele denotar:
\begin{equation}
\phi_p(B)\widetilde{z}_t = \theta_q(B)a_t
\end{equation}
El proceso es estacionario cuando las raices de $\phi_p(B)=0$  est\'an fuera del
c\'{\i}rculo unidad, e invertible si lo est\'an las de $\theta_q(B)=0$.


%
\newpage
%

%
% MODELO ARIMA(p,d,q)
%

%
% PROCESO ARIMA(p,d,q) 
%

\section{El proceso ARIMA(p,d,q)}

El nombre $ARIMA$ viene de las siglas ``autoregresive integrated moving
average''\footnote{Procesos autoregresivos integrados de media m\'ovil.}.

\subsection{Descripci\'on}

Dado un proceso $ARMA(p,q)$ cualquiera y permitiendo que el operador $AR$
tenga raices unitarias obtenemos procesos no estacionarios del tipo:
\begin{displaymath}
(1-\phi_1\cdot B-\dots-\phi_p\cdot B^p)(1-B)^d\widetilde{z}_t =(1-\theta_1\cdot 
B-\dots-\theta_q\cdot B^q)a_t
\end{displaymath}
Este tipo de procesos recibe el nombre de \textbf{ARIMA($p$,$d$,$q$)}, donde:
\begin{itemize}
\item $p$ es el orden de la parte autoregresiva estacionaria.
\item $d$ es el n\'umero de raices unitarias del operador $AR$. Tambi\'en recibe
el nombre de ``\emph{orden de homogeneidad del proceso}''.
\item $q$ es el orden de la parte media m\'ovil.
\end{itemize}

Llamando $\nabla=1-B$ la ecuaci\'on que describe este proceso se puede
escribir como:
\begin{displaymath}
\phi_p(B)\nabla^d\widetilde{z}_t=\theta_q(B)a_t
\end{displaymath}

Todos los procesos $ARIMA(p,d,q)$ no estacionarios se caracterizan por decrecer
lentamente los coeficientes de la funci\'on de autocorrelaci\'on simple.
%
\newpage
%

%
% PROCESOS ARIMA ESTACIONALES
%

%
% PROCESOS ARIMA ESTACIONALES
%

\subsection{Procesos $ARIMA$ estacionales}

Si la estacionalidad fuese siempre exactamente peri\'odica podr\'{\i}a
eliminarse de la serie como un componente determinista. En general la
estacionalidad es s\'olo aproximadamente constante y la eliminaremos de la serie
tomando diferencias entre observaciones separadas por el per\'{\i}odo
estacional. A estas diferencias las llamaremos ``diferencias estacionales''.%
\\ \\
%
Sea $z_t$ una serie estacional de per\'{\i}odo $s$:
\begin{itemize}
\item $s=12$ para series mensuales.
\item $s=4$ para series cuatrimestrales.
\item \dots\dots
\end{itemize}
y con una estructura como la siguiente:
\begin{equation}\label{eq:serieestacional}
z_t=S_t^{(s)}+n_t
\end{equation}
donde:
\begin{itemize}
\item $S_t^{(s)}$ es el componente estacional determinista fijo:
\begin{displaymath}
S_t^{(s)}=S_{t-k\cdot s}^{(s)}\quad k\in \mathbb{Z}
\end{displaymath}
\item $n_t$ es un proceso estacionario.
\end{itemize}
Llamando $\nabla_s=1-B^s$ y al tomar diferencias de observaciones separadas
$s$ per\'{\i}odos tendremos:
\begin{displaymath}
\nabla_s z_t = \nabla_s n_t
\end{displaymath}
con lo que habremos convertido la serie en estacionaria.\\

Por el contrario, si la estacionalidad no es exactamente constante y se
verifica que:
\begin{displaymath}
S_t^{(s)}=S_{t-k\cdot s} + v_t^{(s)}\quad k\in \mathbb{Z}
\end{displaymath}
donde $v_t^{(s)}$ es un proceso estacionario.\\ \\
%
Si tomamos diferencias estacionales en $(\ref{eq:serieestacional})$:
\begin{displaymath}
\nabla_s z_t = v_t^{(s)}+\nabla_s n_t
\end{displaymath}
obtendremos, tambi\'en, un proceso estacionario.\\

Luego el operador $\nabla_s$ convierte procesos estacionales en procesos
estacionarios.

\subsubsection{Formulaci\'on general}

Sea $z_t$ una serie estacional con per\'{\i}odo $s$ conocido y suponiendo que
tenemos $n=s\cdot h$ observaciones. Entonces podremos dividir la serie total en
$s$ series de $h$ datos, tantas series como nos indique el periodo, que
llamaremos $y_{\tau}^1$,\dots,$y_{\tau}^s$ con $\tau=1,\dots,h$.\\ \\
%
La relaci\'on entre estas series y la original $z_t$ ser\'a:
\begin{equation}\label{eq:relacion}
y_{\tau}^{(j)}=z_{j+s\cdot (\tau-1)}\quad (\tau=1,\dots,h)\ y\ (j=1,\dots,s)
\end{equation}
La relaci\'on que obtenemos es $t=j+s\cdot(\tau-1)$.\\ \\
%
Por ejemplo para series mensuales tendremos:
\begin{itemize}
\item $s=12$ indicar\'{\i}a el mes.
\item $h$ nos indicar\'{\i}a de cuantos a\~nos tendr\'{\i}amos observaciones.
\end{itemize}
%
Construyamos ahora un modelo $ARIMA$ para cada una de estas series y supongamos
que este modelo es ``exactamente'' el mismo para todas. El modelo ser\'a el
siguiente:
\begin{equation}\label{eq:ModeloArima}
(1-\Phi_1\cdot B-\dots-\Phi_P\cdot B^P)(1-B)^Dy_{\tau}^{(j)}=
(1-\Theta_1\cdot B-\dots -\Theta_Q\cdot B^Q)u_{\tau}^{(j)}
\end{equation}
donde $j=1,\dots,s$ y $\tau=1,\dots,h$. Adem\'as como tenemos estacionalidad
$D\geq 1$ ya que si $D=0$ y las series fueran estacionarias, su modelo
podr\'{\i}a escribirse:
\begin{displaymath}
y_{\tau}^{(j)} = \mu_j+ \Psi_j(B)u_{\tau}^{(j)}\quad (\tau=1,\dots,h)\ y\ 
(j=1,\dots,s)
\end{displaymath}
donde $\mu_j=E[\{y_{\tau}^{(j)}\}_{\tau}]$ y como la serie es estacional
tendremos que $\mu_i\neq \mu_j$ para $i\neq j$, con lo que las $s$ series no
podr\'{\i}an tener un modelo com\'un.\\

Podemos escribir los modelos para las $s$ series utilizando
$(\ref{eq:relacion})$:
\begin{displaymath}
By_{\tau}^{(j)}= y_{\tau-1}^{(j)} = z_{j+s\cdot(\tau-2)}=
B^{s}z_{j+s\cdot(\tau-1)}
\end{displaymath}
Luego aplicar $B$ a $y_{\tau}$ es lo mismo que aplicar $B^s$ a $z_t$.\\

Definamos ahora una serie de ruido com\'un, $\alpha_t$, asignando a cada $t$ el
ruido del modelo $(\ref{eq:ModeloArima})$ correspondiente a dicho $t$:
\begin{displaymath}
\alpha_t = \alpha_{j+s\cdot (\tau-1)} = u_{\tau}^{(j)}
\end{displaymath}

Luego los $s$ modelos $(\ref{eq:ModeloArima})$ podemos escribirlos:
\begin{equation}\label{eq:sModelosArima}
(1-\Phi_1\cdot B^s-\dots-\Phi_P\cdot B^{P\cdot s})(1-B^s)^D\widetilde{z}_t=
(1-\Theta_1\cdot B^s-\dots-\Theta_Q\cdot B^{s\cdot Q})\alpha_t
\end{equation}
donde $t=1,\dots,n$.\\

Las series $u_{\tau}^{(j)}$ son, por hip\'otesis, ruido blanco, pero la serie
$\alpha_t$ para $t=1,\dots,n$ normalmente no lo ser\'a\footnote{Existir\'a en
general dependencia entre observaciones contiguas.}. Suponiendo que $\alpha_t$
sigue un proceso $ARIMA$ regular tendremos:
\begin{equation}\label{eq:ArimaAlpha}
\phi_p(B)\nabla^d\alpha_t=\theta_q(B)a_t
\end{equation}
Sustituyendo $(\ref{eq:ArimaAlpha})$ en $(\ref{eq:sModelosArima})$ obtenemos el
modelo completo para el proceso:
\begin{equation}\label{eq:ModeloCompleto}
\Phi_P(B^s)\phi_p(B)\nabla^d\nabla_s^Dz_t =\theta_q(B)\Theta_Q(B^s)a_t
\end{equation}

La ecuaci\'on $(\ref{eq:ModeloCompleto})$, que describe el proceso, recibe el nombre de modelo
$ARIMA\ (P,D,Q)_s\times (p,d,q)$ y fue dado por \emph{Box} y \emph{Jenkins} en
$1.976$. Representa de una forma simple muchos fen\'omenos reales que
encontramos en la pr\'actica. A pesar de su frecuencia no se da siempre.



%
% MODELOS ARIMA EN LA PRACTICA
%

%
% MODELOS ARIMA EN LA PRACTICA
%

\section{Modelos ARIMA en la pr\'actica}

Para identificar un modelo utilizaremos todos los datos que tengamos sobre la
serie temporal.

\subsection{Identificaci\'on de la estructura del modelo}

En esta fase tendremos que decidir las transformaciones necesarias para
transformar el proceso en estacionario.
\begin{itemize}
\item Tendremos que determinar el n\'umero de diferencias que se deben aplicar
para que la media sea constante. Este n\'umero, $d$, es normalmente uno o dos.
\item Tendremos que determinar si es necesario transformar la serie para que
tenga varianza constante.
\end{itemize}
%
\newpage
%
Una vez que hemos transformado el proceso en estacionario tendremos que
determinar un modelo para el proceso estacionario, es decir, los \'ordenes $p$ y
$q$ de su representaci\'on $ARMA(p,q)$, y si el proceso es estacional tendremos
que determinar los \'ordenes $P$ y $Q$ de la estructura $ARMA$ estacional.\\

Despu\'es tendremos que realizar un estudio acerca de la estacionalidad. En
caso de presentar estacionalidad con un per\'{\i}odo $s$ tendr\'{\i}amos que 
aplicar la diferencia $(1-B^s)$ para convertirla en estacionaria.\\ \\
%
Para ver si una serie temporal es estacional podemos utilizar:
\begin{itemize}
\item El gr\'afico de la serie, el cual presentar\'a una pauta repetida de
acuerdo con el per\'{\i}odo estacional $s$.
\item La funci\'on de autocorrelaci\'on simple, la cual presentar\'a
coeficientes positivos que decrecer\'an lentamente en los retardos $s$, $2s$,
$3s$, \dots
\end{itemize}

\subsection{Estimaci\'on de los par\'ametros del modelo}

Una vez identificado el modelo estimaremos sus par\'ametros mediante
la mi\-ni\-mi\-za\-ci\'on de la suma de los cuadrados de los errores $a_t$.\\

Para un proceso $AR(p)$ perderemos las $p$ primeras observaciones:
\begin{displaymath}
\sum_{t=p+1}^t a_t^2=S(\phi_1,\dots,\phi_p)
\end{displaymath}
este tipo de estimaciones recibe el nombre de ``estimaciones condicionadas'',
ya que depende de $z_1,\dots,z_p$, que son tomados como dato.\\

Las estimaciones condicionadas obtenidas minimizando $S$ se calculan con las
siguientes hip\'otesis:
\begin{itemize}
\item Modelos $AR$ valores iniciales: $z_1,\dots,z_p$.
\item Modelos $MA$ valores iniciales: $a_0,\dots,a_{-q}$.
\item Modelos $ARMA$ valores iniciales: $z_1,\dots,z_p;a_0,\dots,a_{-q}$.
\end{itemize}
%
\newpage
%
\subsection{Diagn\'ostico del modelo}

Una vez que tenemos construido el modelo hay que comprobar que dicho modelo
representa el comportamiento de la serie temporal, esto lo haremos mediante un
contraste de hip\'otesis. Un buen modelo debe cumplir:
\begin{itemize}
\item Los residuos del modelo estimado se aproximan al comportamiento de un
ruido blanco.
\item El modelo es estacionario e invertible.
\item Los coeficientes son estad\'{\i}sticamente significativos.
\item Los coeficientes del modelo est\'an poco correlacionados entre s\'{\i}.
\item El grado de ajuste es elevado en comparaci\'on al de otros modelos
alternativos.
\end{itemize}
El contraste que m\'as se suele utilizar para comprobar que las
autocorrelaciones en su conjunto no son significativas es el denominado
$Box$-$Pierce$.

\subsection{La esperanza condicionada como predictor \'optimo}

Si tenemos una observacion de longitud $T$ $Z_T=(z_1,\dots,z_T)$ de una serie
temporal y queremos obtener una predicci\'on del valor $z_{T+k}$, donde
$k\in \mathbb{N}$, minimizando el error cuadr\'atico medio. Podemos ver en
\cite{Daniel} que:
\begin{displaymath}
\hat{z}_{T+k}=E[z_{T+k} | Z_T]
\end{displaymath}
minimiza el error cuadr\'atico medio. Utilizando este resultado podemos obtener
el predictor cuando conozcamos las distribuciones condicionadas. Pero esto, en
general, no ocurre y buscaremos predictores, funciones lineales de las
observaciones, qeu minimizen el error de predicci\'on.
%
\newpage
%
\subsection{La ecuaci\'on de predicci\'on de un modelo ARIMA}

Supongamos que tenemos $T$ datos de un proceso $ARIMA(p,d,q)$ cuyos par\'ametros
son conocidos:
\begin{displaymath}
z_t=\phi_1\cdot z_{t-1}+\dots+\phi_h\cdot z_{t-h}+a_t-\theta_1\cdot a_{t-1}-
\dots - \theta_q\cdot a_{t-q}
\end{displaymath}
donde el operador $\Phi_h(B)=\Theta (B)\nabla^d$ puede tener raices en el
circulo unidad, luego $h=p+d$.\\

La predicci\'on \'optima de $z_{T+k}$ ser\'a la esperanza condicionada de esta
variable dada la realizaci\'on $Z_T=(z_1,\dots,z_T)$. Llamando:
\begin{eqnarray*}
\hat{z}_t(j)&=&E[z_{T+j} | Z_T]\\
\hat{a}_t(j)&=&E[a_{T+j} | Z_T]
\end{eqnarray*}
donde $T$ representa el origen de la predicci\'on y $j$ el horizonte.
Sustituyendo $z_t$ en la ecuaci\'on anterior por $z_{T+j}$ y tomando esperanzas
condicionadas a $Z_T$ tendremos:
\begin{equation}\label{eq:Esperanzas}
\hat{z}_T(j)=\phi_1\cdot \hat{z}_T(j-1)+\dots+\phi_h\cdot \hat{z}_t(j-h)-
\theta_1\cdot \hat{a}_T(j-1)-\dots-\hat{a}_T(j-q)
\end{equation}
donde:
\begin{itemize}
\item $\hat{z}_T(-i)=z_{T-i}$ para $i>0$.
\item $\hat{a}_T(i)=0$ para $i>0$.
\item $a_{T+i}=0$ para $i<0$.
\end{itemize}
Para $j=1$ tendremos:
\begin{displaymath}
a_{T+1}=z_{T+1}-\hat{z}_T(1)
\end{displaymath}
que indica que las perturbaciones del modelo son los errores de predicci\'on
de un per\'{\i}odo por delante.\\

La ecuaci\'on $(\ref{eq:Esperanzas})$ indica que despu\'es de unos valores
iniciales la predicci\'on queda determinada por la parte autoregresiva del
modelo. En efecto, para $j>q$ las predicciones de $(\ref{eq:Esperanzas})$
satisfacen la ecuaci\'on:
\begin{displaymath}
(1-\phi_1\cdot B-\dots-\phi_h\cdot B^h)\hat{z}_{T}(j) = 0
\end{displaymath}
siendo $B\cdot \hat{z}_T(j)=\hat{z}_t(j-1)$, es decir, el operador $B$ act\'ua
ahora sobre el horizonte de la previsi\'on $j$ ya que $T$, origen, el fijo.

