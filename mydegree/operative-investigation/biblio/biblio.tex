%
% BIBLIOGRAFIA
%

\begin{thebibliography}{99}
\bibitem{Daniel} Estad\'{\i}stica modelos y
m\'etodos $2$. Modelos lineales y series temporales. $2^a$ edici\'on revisada. 
Daniel Pe\~na S\'anchez de Rivera. Alianza Universidad Textos.
\bibitem{econometria} Modelos econom\'etricos y predicci\'on de series
temporales. Jos\'e M. Otero. Editorial AC.
\bibitem{quintin} Estad\'{\i}stica para ingenieros. Ram\'on Ardanuy Albajar.
Quint\'{\i}n Mart\'{\i}n Mart\'{\i}n. Editorial Hesp\'erides.
\bibitem{modeconometricos} Modelos econom\'etricos. Antonio Pulido. Editorial
Pir\'amide.
\bibitem{Pablo} Algoritmos gen\'eticos sobre redes neuronales. Proyecto fin de
carrera de Pablo Cabezas Mateos.
\bibitem{rnaeIO}Aplicaciones de las redes neuronales a la I.O. Estructura y
funcionamiento de las Redes Neuronales Artificiales. F. S. Wong. Articulo
proporcionado por D. Quint\'{\i}n Mart\'{\i}n Mart\'{\i}n.
\bibitem{RBF}Predicci\'on de series temporales combinando redes $RBF$,
factores de seguridad, y el
modelo de $Box$-$Jenkins$. Donald K. Wedding II, Krzysztof J. Cios. Articulo
proporcionado por D. Quint\'{\i}n Mart\'{\i}n Mart\'{\i}n.
\bibitem{snns} Simulador de Redes Neuronales Stuttgart. Articulo de Ed Petron, 
publicado en el n\'umero $3$ de la revista ``\emph{LiNUX Journal}'' (edici\'on
en Castellano).
\bibitem{jorge} Redes neuronales. Algoritmos, aplicaciones y t\'ecnicas de
programaci\'on. James A. Freeman y David M. Skapura. Addison-Wesley.
\end{thebibliography}
