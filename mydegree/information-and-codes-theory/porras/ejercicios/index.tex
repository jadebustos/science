%
% EJERCICIOS
%

\chapter{Ejercicios propuestos en clase}

\begin{ejercicio}
\ \\
Dados $n,m\in \mathbb{Z}$, $n>0$ y $m\neq 0$. Probar que la
condici\'on necesaria y suficiente para que exista un n\'umero entero $m'$
tal que $m\cdot m' \equiv 1$ mod $n$ es que $m$ y $n$ sean primos entre
s\'{\i}.
\end{ejercicio}
\underline{\textbf{Soluci\'on}}:
\begin{enumerate}
\item Supongamos que existe $m'\in \mathbb{Z}$ tal que $m\cdot m' \equiv 1$
mod $n$ y veamos que $m$ y $n$ son primos entre s\'{\i}.\\

Por hip\'otesis tendremos que:
\begin{equation}\label{eq:Ejer1}
m\cdot m' = \overline{m}\cdot n +1 \Longrightarrow m\cdot m' +(-\overline{m})
\cdot n = 1\quad \overline{m} \in \mathbb{Z}
\end{equation}
Por la \emph{Identidad de Bezout} y $(\ref{eq:Ejer1})$ tenemos que $m$ y $n$ son
primos entre s\'{\i}.

\item Supongamos que $m$ y $n$ son primos entre s\'{\i} y veamos que existe un
n\'umero entero, $m'$, tal que $m\cdot m'\equiv 1$ mod $n$.\\

Como $m$ y $n$ son primos por hip\'otesis aplicando la \emph{Identidad de
Bezout} tendremos que existen $a,b\in\mathbb{Z}$ tales que:
\begin{displaymath}
a\cdot m + b\cdot n = 1
\end{displaymath}
es decir:
\begin{displaymath}
a\cdot m = (-b)\cdot n +1 \Longrightarrow a\cdot m \equiv 1\ mod\ n
\end{displaymath}
Tomando $a$ como $m'$ hemos terminado.
\end{enumerate}
\begin{flushright}
$\blacksquare$
\end{flushright}

%
% COMPROBAR QUE LA DISTANCIA DE HAMMING ES UNA DISTANCIA
%

\begin{ejercicio}
Comprobar que la distancia de Hamming es una distancia.
\end{ejercicio}
\underline{\textbf{Soluci\'on}}:\\
Sea $\mathcal{C}$ un c\'odigo, entonces la distancia de Hamming ser\'a una
aplicaci\'on de la forma:
\begin{displaymath}
d:\mathcal{C}\times \mathcal{C} \longrightarrow \mathbb{Z}^{^+}
\end{displaymath}
Recordemos que la distancia de Hamming, entre dos palabras, es el n\'umero de
bits en el que no coinciden.
\begin{itemize}
\item $d(x,y)\geq 0$ y $d(x,y)=0$ $\Longleftrightarrow$ $x=y$\\
Como la distancia entre dos palabras es el n\'umero de bits en el que no
coinciden esta distancia siempre ser\'a mayor o igual que cero.\\

La distancia entre dos palabras ser\'a cero s\'{\i} y s\'olo s\'{\i} coinciden
en todos sus bits, luego ambas palabras ser\'an las mismas.
\item $d(x,y)=d(y,x)$ \\
El n\'umero de bits en el que no coincide $x$ con $y$ es el mismo que el de $y$
con $x$.
\item Desigualdad triangular: $d(x,z)\leq d(x,y)+d(y,z)$\\
Sean $x,y,z\in \mathcal{C}$ tres palabras cualesquiera del c\'odigo.\\ \\
%
Definamos los siguientes conjuntos:
\begin{displaymath}
U=\{i\ |\ x_i\neq z_i \}
\end{displaymath}
$U$ es el conjunto de indices en los que $x$ y $z$ no coinciden.
\begin{displaymath}
S=\{i\ |\ x_i\neq z_i\ y\ x_i=y_i \}
\end{displaymath}
$S$ es el conjunto de indices en los que $x$ y $z$ no coinciden y en los que
$x$ e $y$ coinciden.
\begin{displaymath}
T=\{i\ |\ x_i\neq z_y\ y\ x_i\neq y_i\}
\end{displaymath}
$T$ es el conjunto de indices en los que $x$ y $z$ no coinciden y en los que
$x$ e $y$ tampoco coinciden.\\ \\
%
Entenderemos por $\#U$, $\#S$ y $\#T$ al n\'umero de elementos de los conjuntos
$U$, $S$ y $T$ respectivamente.\\ \\
%
De las definiciones anteriores se deduce que:
\begin{equation}\label{eq:Primera}
\#U=d(x,z)=\#S+\#T
\end{equation}
Como $d(x,y)$ es el n\'umero de indices en los que no coinciden $x$ e $y$
tenemos entonces:
\begin{equation}\label{eq:Segunda}
d(x,y)\geq \#T
\end{equation}
De la definici\'on de $d(y,z)$ y de $S$ se tiene que:
\begin{equation}\label{eq:Tercera}
d(y,z)\geq \#S
\end{equation}
Si sumamos miembro a miembro $(\ref{eq:Segunda})$ y $(\ref{eq:Tercera})$ y
tenemos en cuenta $(\ref{eq:Primera})$ tenemos: $$d(x,z)\leq d(x,y)+d(y,z)$$
que es lo que queriamos demostrar.
\end{itemize}
\begin{flushright}
$\blacksquare$
\end{flushright}

%
% PROPIEDAD QUE SE DEDUCE DE LA DESIGUALDAD TRIANGULAR
%

\begin{ejercicio}
Dado un c\'odigo $\mathcal{C}\subset \mathbb{F}^{^n}_q$ tal que
$d_{min} = 2\cdot t + 1$ comprobar que si dado
$z\in \mathbb{F}^{^n}_q$ tal que existe $c\in \mathcal{C}$ verificando que
$d(z,c)\leq t$ entonces la distancia de $z$ a cualquier otra palabra del
c\'odigo es estrictamente mayor que $t$.
\end{ejercicio}
\underline{\textbf{Soluci\'on}}:\\
Dado $z\in \mathbb{F}^{^n}_q$ supongamos que existe $c\in \mathcal{C}$ tal que
$d(z,c)\leq t$. Ahora bien como $d_{min} = 2\cdot t+1$ tendremos que
$d(c,c')\geq 2\cdot t+1$ $\forall \ c'\in \mathcal{C}$ con $c'\neq c$.
Por la desigualdad triangular tendremos:
\begin{displaymath}
2\cdot t+1 \leq d(c,c')\leq d(c,z)+d(z,c')
\end{displaymath}
para todo $z\in \mathbb{F}^{^n}_q$, en particular para nuestro $z$. Como por
hip\'otesis se tiene que $d(z,c)\leq t$:
\begin{displaymath}
2\cdot t +1 \leq t + d(z,c')
\end{displaymath}
o lo que es lo mismo:
\begin{displaymath}
t + 1 \leq d(z,c') \Longrightarrow d(z,c') > t\quad \forall \ c'\neq c \in \mathcal{C}
\end{displaymath}
\begin{flushright}
$\blacksquare$
\end{flushright}

Este ejercicio nos indica que si conocemos la distancia m\'{\i}nima, $d_{min}$,
de un c\'odigo $\mathcal{C}$ y calculamos $t$ de tal manera que verifique
$d_{min} = 2\cdot t +1$ entonces en el caso de recibir una transmisi\'on 
incorrecta, $z$, tal que $d(z,c)\leq t$ se tiene que cualquier otra palabra del
c\'odigo, $c'$, verifica que $d(z,c')> t$. Esto quiere decir que podemos
corregir el error cometido ya que s\'olo existe una palabra del c\'odigo
a distancia $t$ de $w$, entonces esa palabra del c\'odigo es la que se
transmiti\'o originalmente, luego $c$ es la palabra transmitida originalmente.

%
% CALCULO DE UN CODIGO A PARTIR DE SU MATRIZ DE CHEQUEO
% 

\begin{ejercicio}
\ \\
Dada la siguiente matriz de control de un c\'odigo lineal binario:
\begin{displaymath}
H=\left( \begin{array}{ccccccc}
1&0&1&1&1&0&0\\
1&1&1&0&0&1&0\\
0&1&1&1&0&0&1
\end{array} \right)
\end{displaymath}
calcular el c\'odigo al que corresponde. Dicho c\'odigo es el c\'odigo de
Hamming $Ham(3)$, el cual es $1$-perfecto.
\end{ejercicio}
\underline{\textbf{Soluci\'on}}:\\
Dado que $H$ es la matriz de control de un c\'odigo lineal por definici\'on
se tiene que cualquier palabra del c\'odigo aplicada a la matriz ser\'a cero:
\begin{displaymath}
\left( \begin{array}{ccccccc}
1&0&1&1&1&0&0\\
1&1&1&0&0&1&0\\
0&1&1&1&0&0&1
\end{array} \right)\cdot
\left( \begin{array}{c}
x_1\\
x_2\\
x_3\\
x_4\\
x_5\\
x_6\\
x_7
\end{array} \right)=
\left( \begin{array}{c}
0\\
0\\
0
\end{array} \right)
\end{displaymath}
De donde obtenemos las ecuaciones impl\'{\i}citas del c\'odigo:
\begin{eqnarray*}
x_5&=&x_1+x_3+x_4\\
x_6&=&x_1+x_2+x_3\\
x_7&=&x_2+x_3+x_4
\end{eqnarray*}
Recordar que estamos trabajando sobre $\mathbb{F}_2$, donde tenemos que
$-1 \equiv 1$.     
%
\newpage
%
Luego tenemos un c\'odigo lineal con longitud de palabra $7$ en el que los
tres \'ultimos bits vienen determinados por los cuatro primeros, son funci\'on
lineal\footnote{La suma de elementos es una funci\'on lineal.}, luego son
combinaci\'on lineal de los cuatro primeros bits. Es decir el c\'odigo lineal
es de dimensi\'on cuatro. Luego el c\'odigo buscado es un c\'odigo
$\mathcal{C}[7,4]$.\\

Para calcular todas las palabras del c\'odigo basta con encontrar cuatro
vectores linealmente independientes, ya que al ser el c\'odigo de dimensi\'on
cuatro vectores linealmente independientes, que pertenezcan al c\'odigo, 
generar\'an el c\'odigo. Con el fin de dar la matriz generadora en forma
est\'andar cogeremos los vectores $\{e_i\}_{i=1}^4$ donde de los cuatro
primeros bits unicamente sea no nulo el $i$-esimo y los bits del quinto al
septimo estar\'an determinados por las ecuaciones impl\'{\i}citas del c\'odigo.
\begin{eqnarray*}
e_1&=&(1,0,0,0,1,1,0)\\
e_2&=&(0,1,0,0,0,1,1)\\
e_3&=&(0,0,1,0,1,1,1)\\
e_4&=&(0,0,0,1,1,0,1)
\end{eqnarray*}
Luego la matriz generadora, en forma est\'andar, del c\'odigo ser\'a:
\begin{displaymath}
\left( \begin{array}{cccc}
1&0&0&0\\
0&1&0&0\\
0&0&1&0\\
0&0&0&1\\
1&0&1&1\\
1&1&1&0\\
0&1&1&1
\end{array} \right)
\end{displaymath}
Como el c\'odigo es de dimensi\'on cuatro y estamos trabajando sobre un
cuerpo finito de dos elementos, $\mathbb{F}_2$, tendremos que el c\'odigo
tendr\'a $2^4=16$ palabras. La forma de calcularlas ser\'a coger todas las
posibles combinaciones, $16$, de cuatro elementos de $\mathbb{F}_2$ y 
a\~nadirle otros tres bits, los determinados por las ecuaciones impl\'{\i}citas
del c\'odigo.

\begin{flushright}
$\blacksquare$
\end{flushright}

%
\newpage
%

%
% ESTRUCTURA DE LOS ANILLOS DE POLINOMIOS
%

\begin{ejercicio}\label{ejer:Anillo}
\ \\
Sea $P(x)$ un polinomio en $x$ de grado $n$ y con coeficientes en
$\mathbb{F}_q$.
\begin{enumerate}
\item Probar que $\mathbb{F}_q[x]/(P(x))$ es un espacio vectorial y que:
\begin{displaymath}
\mathbb{F}_q[x]/(P(x))=<\overline{1},\overline{x},\dots,\overline{x}^{n-1}>
\end{displaymath}
\item Calcular los ideales del anillo $\mathbb{F}_q[x]/(x^n-1)$.
\end{enumerate}
Los ideales del anillo $\mathbb{F}_q[x]/(x^n-1)$ clasifican los c\'odigos
c\'{\i}clicos.
\end{ejercicio}
\underline{\textbf{Soluci\'on}}:\\
\begin{enumerate}
\item Sea $P(x)$ un polinomio en $x$ de grado $n$ y con coeficientes en
$\mathbb{F}_q$.\\

$\mathbb{F}_q[x]/(P(x))$ son los restos al dividir polinomios en $x$ con
coeficientes en $\mathbb{F}_q$ por $P(x)$. Luego ser\'an polinomios en $x$
con coeficientes en $\mathbb{F}_q$ y de grado menor o igual que $n-1$, al
tener $P(x)$ grado $n$.\\

Denotaremos a los elementos de $\mathbb{F}_q[x]/(P(x))$ como 
$\overline{p_i(x)}$. Entonces por la definic\'on de $\mathbb{F}_q[x]/(P(x))$ y
por el \emph{algoritmo de Euclides} tendremos que:
\begin{displaymath}
h(x) = h_i(x)\cdot P(x) + p_i(x)
\end{displaymath}
Luego $\overline{p_i(x)}$ representar\'a, en $\mathbb{F}_q[x]/(P(x))$, a todos
aquellos polinomios $h(x)\in \mathbb{F}_q[x]$ que tengan resto $p_i(x)$ al
dividir por $P(x)$.\\

Las operaciones de suma de polinomios, en $\mathbb{F}_q[x]/(P(x))$, as\'{\i}
como multiplicar polinomios, de $\mathbb{F}_q[x]/(P(x))$, por escalares est\'an
bien definidas. Es de inmediata comprobaci\'on.\newpage
\begin{enumerate}
\item $\mathbb{F}_q[x]/(P(x))$ es un $\mathbb{F}_q$-espacio vectorial.\\
\begin{enumerate}
\item $(\mathbb{F}_q[x]/(P(x)),+)$ es un grupo conmutativo.\\

Es inmediato, se deduce de la estructura de grupo conmutativo de
$(\mathbb{F}_q,+)$ y de la definici\'on de $\mathbb{F}_q[x]/(P(x))$.\\

\item $\lambda\cdot (\overline{p_1(x)}+\overline{p_2(x)})= \lambda\cdot
\overline{p_1(x)}+\lambda\cdot \overline{p_2(x)}$ para cualesquiera que sean
$\lambda\in\mathbb{F}_q$, $\overline{p_1(x)},\overline{p_2(x)}\in
\mathbb{F}_q[x]/(P(x))$.\\

Inmediato a partir de las definiciones.\\
\item $(\lambda+\mu)\cdot\overline{p(x)}=\lambda\cdot\overline{p(x)}+\mu\cdot
\overline{p(x)}$ para cualesquiera que sean $\lambda,\mu\in \mathbb{F}_q$ y
$\overline{p(x)}\in \mathbb{F}_q[x]/(P(x))$.\\

Inmediato a partir de las definiciones.\\
\item $(\lambda\cdot\mu)\cdot \overline{p(x)}=\lambda\cdot(\mu\cdot
\overline{p(x)})$ para cualesquiera $\lambda,\mu\in \mathbb{F}_q$ y
$\overline{p(x)}\in \mathbb{F}_q[x]/(P(x))$.\\

Inmediato a partir de las definiciones.\\
\item $1\cdot \overline{p(x)}= \overline{p(x)}$ para todo
$\overline{p(x)}\in\mathbb{F}_q[x]/(P(x))$.\\

Inmediato a partir de las definiciones.\\
\end{enumerate}
%
\newpage
%
\item $\mathbb{F}_q[x]/(P(x))=<\overline{1},\overline{x},\dots,
\overline{x}^{n-1}>$.\\
\begin{enumerate}
\item $<\overline{1},\overline{x},\dots,\overline{x}^{n-1}>$ son un sistema de
generadores.\\

Sea $\overline{p(x)}\in \mathbb{F}_q[x]/(P(x))$, luego su grado es menor que
el grado de $P(x)$, que es $n$, luego es de grado menor o igual que $n-1$.
Es obvio que $\overline{p(x)}$ es combinaci\'on lineal de
$\{\overline{1},\overline{x},\dots,\overline{x}^{n-1}\}$, ya que s\'{\i}
$\overline{p(x)}=a_0+a_1\cdot\overline{x}+\dots+a_{n-1}\cdot\overline{x}^{n-1}$
tendremos que:
\begin{displaymath}
\overline{p(x)} = a_0\cdot \overline{1}+a_1\cdot \overline{x}+\dots+a_{n-1}\cdot
\overline{x}^{n-1}
\end{displaymath}
con $a_0,a_1,\dots,a_{n-1}\in \mathbb{F}_q$.\\

Luego $<\overline{1},\overline{x},\dots,\overline{x}^{n-1}>$ son un sistema
de generadores.\\
\item  $<\overline{1},\overline{x},\dots,\overline{x}^{n-1}>$ son linealmente
independientes.\\

Sean $\lambda_0,\dots,\lambda_{n-1}\in \mathbb{F}_q$ tales que:
\begin{displaymath}
\lambda_0\cdot \overline{1}+\lambda_1\cdot \overline{x}+\dots+\lambda_{n-1}
\cdot \overline{x}^{n-1} = 0
\end{displaymath}
Como la combinaci\'on lineal es cero entonces debe ser un m\'ultiplo de $P(x)$,
pero como la combinaci\'on lineal es un polinomio de grado $n-1$, tiene menor
grado que $P(x)$, entonces la combinaci\'on lineal es el polinomio nulo, o lo
que es lo mismo:
\begin{displaymath}
\lambda_0=\lambda_1=\dots=\lambda_{n-1}=0
\end{displaymath}
luego $<\overline{1},\overline{x},\dots,\overline{x}^{n-1}>$ son linealmente
independientes.\\ 
\end{enumerate}
\end{enumerate}
\item Calcular los ideales del anillo $\mathbb{F}_q[x]/(x^n-1)$.\\

Para calcular los ideales del anillo $\mathbb{F}_q[x]/(x^n-1)$ utilizaremos el
siguiente teorema del albebra:
\begin{teorema}\label{the:EstructuraCociente}
\ \\
Sea $\phi:A\longrightarrow B$ un morfismo epiyectivo de anillos. Entonces existe
la siguiente correspondencia biun\'{\i}voca:
\begin{eqnarray*}
\left[ \begin{array}{c}
Ideales \\
de \ B\\
\end{array} \right]
&\stackrel{\sim }\longrightarrow &
\left[ \begin{array}{c}
Ideales\ de\ A\ que\\
continen\ a\ \ker \phi
\end{array} \right]\\
I&\longmapsto&\phi^{-1}(I)
\end{eqnarray*}
\end{teorema}
Consideremos ahora la aplicaci\'on de paso al cociente:
\begin{displaymath}
\pi:\mathbb{F}_q[x]\longrightarrow \mathbb{F}_q[x]/(x^n-1)
\end{displaymath}
Es inmediato demostrar que $\pi$ es un morfismo epiyectivo de anillos y que
$\ker \pi = (x^n-1)$. Luego por el teorema $\ref{the:EstructuraCociente}$
tendremos que los ideales de $\mathbb{F}_q[x]/(x^n-1)$ se identificar\'an con
los ideales de $\mathbb{F}_q[x]$ que contienen a $(x^n-1)$.\\

Sea $x^n-1=p_1(x)\cdot \dots \cdot p_r(x)$ la descomposici\'on en
irreducibles de $x^n-1$
en $\mathbb{F}_q[x]$. Luego los ideales que contienen a $x^n-1$ ser\'an los
ideales generados por los productos de los factores irreducibles en los
que descompone $x^n-1$.\\

Los ideales de $\mathbb{F}_q[x]/(x^n-1)$ son los ideales generados en dicho
anillo por los polinomios divisores de $x^n-1$.
\end{enumerate}
\begin{flushright}
$\blacksquare$
\end{flushright}
%
\newpage
%
%
% CALCULAR LA MATRIZ GENERADORA DE UN CODIGO CICLICO
%
\begin{ejercicio}
\ \\
Sea $\mathbb{F}_2[x]/(x^7-1)$, donde la descomposici\'on de $x^7-1$ en 
irreducibles en $\mathbb{F}_2[x]$ es la siguiente:
\begin{displaymath}
x^7-1=(x+1)\cdot(x^3+x+1)\cdot(x^3+x^2+1)
\end{displaymath}
Calcular la matriz generadora para el c\'odigo c\'{\i}clico que tiene por
polinomio generador:
\begin{displaymath}
g(x)=(x+1)\cdot(x^3+x+1)=x^4+x^3+x^2+1
\end{displaymath}
\end{ejercicio}
\underline{\textbf{Soluci\'on}}:\\
La dimensi\'on del c\'odigo ser\'a el grado de $x^3+x^2+1$, luego ser\'a
de dimensi\'on tres.\\

Una base del c\'odigo ser\'a:
\begin{eqnarray*}
g(x)&=&x^4+x^3+x^2+1\\
x\cdot g(x)&=& x^5+x^4+x^3+x\\
x^2\cdot g(x)&=& x^6+x^5+x^4+x^2
\end{eqnarray*}
Luego la matriz generadora del c\'odigo ser\'a:
\begin{displaymath}
\left( \begin{array}{ccc}
1&0&0\\
0&1&0\\
1&0&1\\
1&1&0\\
1&1&1\\
0&1&1\\
0&0&1
\end{array} \right)
\end{displaymath}
\begin{flushright}
$\blacksquare$
\end{flushright}
%
\newpage 
%
\begin{ejercicio}
\ \\
Sea $\mathbb{F}_2[x]/(x^7-1)$, donde la descomposici\'on de $x^7-1$ en 
irreducibles en $\mathbb{F}_2[x]$ es la siguiente:
\begin{displaymath}
x^7-1=(x+1)\cdot(x^3+x+1)\cdot(x^3+x^2+1)
\end{displaymath}
Calcular las matrices generadoras y de control para todos los c\'odigos que
podemos definir.
\end{ejercicio}
\underline{\textbf{Soluci\'on}}:\\
Los c\'odigos que podemos definir son los definidos por los divisores de
$x^7-1$ luego ser\'an:
\begin{itemize}
\item $g(x)=x+1$, la dimensi\'on del c\'odigo ser\'a $6$.\\

Una base del c\'odigo ser\'a:
\begin{eqnarray*}
g(x)&=&x+1\\
x\cdot g(x)&=&x^2+x\\
x^2\cdot g(x)&=&x^3+x^2\\
x^3\cdot g(x)&=&x^4+x^3\\
x^4\cdot g(x)&=&x^5+x^4\\
x^5\cdot g(x)&=&x^6+x^5
\end{eqnarray*}
Luego la matriz del c\'odigo ser\'a:
\begin{displaymath}
\left( \begin{array}{cccccc}
1&0&0&0&0&0\\
1&1&0&0&0&0\\
0&1&1&0&0&0\\
0&0&1&1&0&0\\
0&0&0&1&1&0\\
0&0&0&0&1&1\\
0&0&0&0&0&1
\end{array} \right)
\end{displaymath}
La matriz de control ser\'a la traspuesta de la matriz generadora de su c\'odigo
dual. Su c\'odigo dual estar\'a generado por:
\begin{displaymath}
p(x)=x^6\cdot h(\frac{1}{x}) = 1+x+x^2+x^3+x^4+x^5+x^6
\end{displaymath}
con $h(x)=(x^3+x+1)\cdot (x^3+x^2+1)$. Luego una base del c\'odigo dual ser\'a:
\begin{displaymath}
p(x)=1+x+x^2+x^3+x^4+x^5+x^6
\end{displaymath}
Luego la matriz de control de este c\'odigo ser\'a:
\begin{displaymath}
\left( \begin{array}{ccccccc}
1&1&1&1&1&1&1
\end{array} \right)
\end{displaymath}
\item $g(x)=x^3+x+1$, la dimensi\'on del c\'odigo ser\'a $4$.\\

Una base del c\'odigo ser\'a:
\begin{eqnarray*}
g(x)&=&x^3+x+1\\
x\cdot g(x)&=&x^4+x^2+x\\
x^2\cdot g(x)&=&x^5+x^3+x^2\\
x^3\cdot g(x)&=&x^6+x^4+x^4
\end{eqnarray*}
Luego la matriz del c\'odigo ser\'a:
\begin{displaymath}
\left( \begin{array}{cccc}
1&0&0&0\\
1&1&0&0\\
0&1&1&0\\
1&0&1&1\\
0&1&0&1\\
0&0&1&0\\
0&0&0&1
\end{array} \right)
\end{displaymath}
La matriz de control ser\'a la matriz traspuesta de la matriz de control de su
c\'odigo dual. Su c\'odigo dual estar\'a generado por:
\begin{displaymath}
p(x) = x^4\cdot h(\frac{1}{x}) = 1+x^2+x^3+x^4
\end{displaymath}
con $h(x)=(x+1)\cdot(x^3+x^2+1)$. Una base del c\'odigo dual ser\'a:
\begin{eqnarray*}
p(x)&=&x^4+x^3+x^2+1\\
x\cdot p(x)&=&x^5+x^4+x^3+x\\
x^2\cdot p(x)&=&x^6+x^5+x^5+x^2
\end{eqnarray*}
Luego la matriz de control ser\'a:
\begin{displaymath}
\left( \begin{array}{ccccccc}
1&0&1&1&1&0&0\\
0&1&0&1&1&1&0\\
0&0&1&0&1&1&1
\end{array} \right)
\end{displaymath}
\item $g(x)=x^3+x^2+1$, la dimensi\'on del c\'odigo ser\'a $4$.\\

Una base del c\'odigo ser\'a:
\begin{eqnarray*}
g(x)&=&x^3+x^2+1\\
x\cdot g(x)&=&x^4+x^3+x\\
x^2\cdot g(x)&=&x^5+x^4+x^2\\
x^3\cdot g(x)&=&x^6+x^5+x^3
\end{eqnarray*}
Luego la matriz del c\'odigo ser\'a:
\begin{displaymath}
\left( \begin{array}{cccc}
1&0&0&0\\
0&1&0&0\\
1&0&1&0\\
1&1&0&1\\
0&1&1&0\\
0&0&1&1\\
0&0&0&1
\end{array} \right)
\end{displaymath}
La matriz de control del c\'odigo ser\'a la traspuesta de la matriz generadora
de su c\'odigo dual. Su c\'odigo dual estar\'a generado por:
\begin{displaymath}
p(x) = x^4\cdot h(\frac{1}{x}) = 1+x+x^2+x^4
\end{displaymath}
con $h(x)=(x+1)\cdot (x^3+x+1)$. Una base del c\'odigo dual ser\'a:
\begin{eqnarray*}
p(x)&=&x^4+x^2+x+1\\
x\cdot p(x)&=&x^5+x^3+x^2+x\\
x^2\cdot p(x)&=&x^6+x^4+x^3+x^2
\end{eqnarray*}
Luego la matriz de control ser\'a:
\begin{displaymath}
\left( \begin{array}{ccccccc}
1&1&1&0&1&0&0\\
0&1&1&1&0&1&0\\
0&0&1&1&1&0&1
\end{array} \right)
\end{displaymath}
\item $g(x)=(x+1)\cdot (x^3+x+1)$, la dimensi\'on del c\'odigo ser\'a $3$.
\begin{displaymath}
g(x)=(x+1)\cdot (x^3+x+1)=x^4+x^3+x^2+1
\end{displaymath}
Una base del c\'odigo ser\'a:
\begin{eqnarray*}
g(x)&=&x^4+x^3+x^2+1\\
x\cdot g(x)&=&x^5+x^4+x^3+x\\
x^2\cdot g(x)&=&x^6+x^5+x^4+x^2
\end{eqnarray*}
Luego la matriz del c\'odigo ser\'a:
\begin{displaymath}
\left( \begin{array}{ccc}
1&0&0\\
0&1&0\\
1&0&1\\
1&1&0\\
1&1&1\\
0&1&1\\
0&0&1
\end{array} \right)
\end{displaymath}
La matriz de control ser\'a la traspuesta de la matriz generadora de su 
c\'odigo dual. Su c\'odigo dual estar\'a generado por:
\begin{displaymath}
p(x) = x^3\cdot h(\frac{1}{x}) = 1+x+x^3
\end{displaymath}
con $h(x)=x^3+x^2+1$. Una base del c\'odigo dual ser\'a:
\begin{eqnarray*}
p(x)&=&x^3+x+1\\
x\cdot p(x)&=&x^4+x^2+x\\
x^2\cdot p(x)&=&x^5+x^3+x^2\\
x^3\cdot p(x)&=&x^6+x^4+x^3
\end{eqnarray*}
Luego la matriz de control ser\'a:
\begin{displaymath}
\left( \begin{array}{ccccccc}
1&1&0&1&0&0&0\\
0&1&1&0&1&0&0\\
0&0&1&1&0&1&0\\
0&0&0&1&1&0&1
\end{array} \right)
\end{displaymath}

\item $g(x)=(x+1)\cdot (x^3+x^2+1)$, la dimensi\'on del c\'odigo ser\'a $3$.
\begin{displaymath}
g(x)=(x+1)\cdot (x^3+x^2+1)= 1+x+x^2+x^4
\end{displaymath}
Una base del c\'odigo ser\'a:
\begin{eqnarray*}
g(x)&=&x^4+x^2+x+1\\
x\cdot g(x)&=&x^5+x^3+x^2+x\\
x^2\cdot g(x)&=&x^6+x^4+x^3+x^2
\end{eqnarray*}
Luego la matriz del c\'odigo ser\'a:
\begin{displaymath}
\left( \begin{array}{ccc}
1&0&0\\
1&1&0\\
1&1&1\\
0&1&1\\
1&0&1\\
0&1&0\\
0&0&1
\end{array} \right)
\end{displaymath}
La matriz de control ser\'a la traspuesta de la matriz generadora de su
c\'odigo dual. Su c\'odigo dual estar\'a generado por:
\begin{displaymath}
p(x) = x^3\cdot h(\frac{1}{x}) = 1+x^2+x^3
\end{displaymath}
con $h(x)=x^3+x+1$. Una base del c\'odigo dual ser\'a:
\begin{eqnarray*}
p(x)&=& x^3+x^2+1\\
x\cdot p(x)&=& x^4+x^3+x\\
x^2\cdot p(x)&=& x^5+x^4+x^2\\
x^3\cdot p(x)&=& x^6+x^5+x^3
\end{eqnarray*}
Luego la matriz de control del c\'odigo ser\'a:
\begin{displaymath}
\left( \begin{array}{ccccccc}
1&0&1&1&0&0&0\\
0&1&0&1&1&0&0\\
0&0&1&0&1&1&0\\
0&0&0&1&0&1&1
\end{array} \right)
\end{displaymath}

\item $g(x)=(x^3+x+1)\cdot (x^3+x^2+1)$, la dimensi\'on del c\'odigo ser\'a $1$.
\begin{displaymath}
g(x)=(x^3+x+1)\cdot(x^3+x^2+1)=x^6+x^5+x^4+x^3+x^2+x+1
\end{displaymath}
Una base del c\'odigo ser\'a:
\begin{eqnarray*}
g(x)&=&x^6+x^5+x^4+x^3+x^2+x+1
\end{eqnarray*}
Luego la matriz del c\'odigo ser\'a:
\begin{displaymath}
\left( \begin{array}{c}
1\\
1\\
1\\
1\\
1\\
1\\
1
\end{array} \right)
\end{displaymath}
La matriz de control ser\'a la traspuesta de la matriz generadora de su c\'odigo
dual. Su c\'odigo dual estar\'a generado por:
\begin{displaymath}
p(x)=x\cdot h(\frac{1}{x}) = 1+x
\end{displaymath}
con $h(x)=x+1$. Una base del c\'odigo dual ser\'a:
\begin{eqnarray*}
p(x)&=&x+1\\
x\cdot p(x)&=&x^2+x\\
x^2\cdot p(x)&=& x^3+x^2\\
x^3\cdot p(x)&=& x^4+x^3\\
x^4\cdot p(x)&=& x^5+x^4\\
x^5\cdot p(x)&=& x^6+x^5
\end{eqnarray*}
Luego la matriz de control del c\'odigo ser\'a:
\begin{displaymath}
\left( \begin{array}{ccccccc}
1&1&0&0&0&0&0\\
0&1&1&0&0&0&0\\
0&0&1&1&0&0&0\\
0&0&0&1&1&0&0\\
0&0&0&0&1&1&0\\
0&0&0&0&0&1&1
\end{array} \right)
\end{displaymath}
\end{itemize}
\begin{flushright}
$\blacksquare$
\end{flushright}
