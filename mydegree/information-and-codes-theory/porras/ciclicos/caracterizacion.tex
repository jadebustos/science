%
% CARACTERIZACION DE LOS CODIGOS CICLICOS
%

\section{Caracterizaci\'on de los c\'odigos c\'{\i}clicos}

Los c\'odigos c\'{\i}clicos son ideales del anillo $\mathbb{F}_q[x]/(x^n-1)$.

\begin{proposicion}\label{pro:Ideal}
\ \\
Sea $\mathcal{C}[n,m]$ un c\'odigo c\'{\i}clico. Para todo $\overline{p(x)}\in
\mathcal{C}[n,m]$ y para cualquier $\overline{Q(x)}\in \mathbb{F}_q[x]/(x^n-1)$
se tiene que $\overline{p(x)}\cdot \overline{Q(x)}\in \mathcal{C}[n,m]$.
\end{proposicion}
\underline{\textbf{Demostraci\'on}}:\\
Sean:
\begin{eqnarray*}
\overline{p(x)}=p_0+p_1\cdot \overline{x}+\dots+p_{n-1}\cdot
\overline{x}^{n-1}&\in& \mathcal{C}[n,m]\\
\overline{Q(x)}=q_0+q_1\cdot \overline{x}+\dots+q_{n-1}\cdot
\overline{x}^{n-1}&\in& \mathbb{F}_q[x]/(x^n-1)
\end{eqnarray*}
entonces se tiene que:
\begin{displaymath}
\overline{p(x)}\cdot \overline{Q(x)} = q_0\cdot \overline{p(x)}+
q_1\cdot\overline{x}\cdot\overline{p(x)}+\dots+q_{n-1}\cdot\overline{x}^{n-1}
\cdot\overline{p(x)}
\end{displaymath}
como el c\'odigo es c\'{\i}clico se tiene que:
\begin{displaymath}
\overline{x}^i\cdot\overline{p(x)}\in \mathcal{C}[n,m]\quad i=0,1,\dots
\end{displaymath}
y como el c\'odigo es lineal se tiene que:
\begin{displaymath}
\lambda\cdot \overline{h(x)}\in\mathcal{C}[n,m]
\end{displaymath}
para $\lambda\in\mathbb{F}_q$ y $\overline{h(x)}\in \mathcal{C}[n,m]$. Luego
$\overline{p(x)}\cdot\overline{Q(x)}$ es la suma de polinomios que pertenecen
al c\'odigo y por la linealidad de este la suma de elementos del c\'odigo
pertenece al c\'odigo.
\begin{flushright}
$\blacksquare$
\end{flushright}
%
%
\begin{teorema}[Caracterizaci\'on de los c\'odigos c\'{\i}clicos]\label{the:CarCiclicos}
\ \\
Un c\'odigo $\mathcal{C}[n,m]$ es c\'{\i}clico s\'{\i} y s\'olo s\'{\i} como
subconjunto es un ideal de $\mathbb{F}_q[x]/(x^n-1)$.
\end{teorema}
\underline{\textbf{Demostraci\'on}}:\\
Es inmediato a partir de la definici\'on de ideal y de la proposici\'on
$\ref{pro:Ideal}$.
\begin{flushright}
$\blacksquare$
\end{flushright}
Seg\'un este teorema todos los c\'odigos c\'{\i}clicos son ideales del anillo,
de ideales principales, $\mathbb{F}_q[x]/(x^n-1)$. Luego todos
sus ideales estar\'an generados por un elemento, es decir ser\'an de la forma
$(\overline{h(x)})$.\\

En el ejercicio $\ref{ejer:Anillo}$, en la p\'agina $\pageref{ejer:Anillo}$, se
demuestra que todos los ideales del anillo $\mathbb{F}_q[x]/(x^n-1)$ est\'an
generados por los productos de los factores irreducibles en los que descompone
$x^n-1$ en $\mathbb{F}_q[x]$.
%
\begin{definicion}[Polinomio generador del c\'odigo c\'{\i}clico]
\ \\
Se llama \textbf{``polinomio generador del c\'odigo c\'{\i}clico''} al
polinomio que genera el ideal al que corresponde el c\'odigo.
\end{definicion}
