%
% CODIGOS DE REED-SOLOMON
%

\section{C\'odigos de Reed-Solomon}

Sea $\beta$ una raiz primitiva $n$-\'esima de la unidad\footnote{Cualquier otra
raiz primitiva $n$-\'esima de la unidad es una potencia suya.} en 
$\mathbb{F}_q$ y $f(x)$ su polinomio m\'{\i}nimo, entonces:
\begin{displaymath}
x^n-1=(x-1)\cdot f_1(x)\cdot f_k(x)
\end{displaymath}
definamos el polinomio $g(x)$ como el m\'{\i}nimo com\'um multiplo de los
polinomios m\'{\i}nimos $f_1,\dots,f_k$. Entonces el c\'odigo generado por
dicho polinomio es un c\'odigo \emph{BCH}.\\

Si tomamos $n=q-1$, $q=2^k$, el c\'odigo resultante se llama c\'odigo de
\emph{Reed-Solomon}.
