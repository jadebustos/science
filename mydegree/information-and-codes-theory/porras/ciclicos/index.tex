%
% CODIGOS CICLICOS
%

\chapter{C\'odigos c\'{\i}clicos}

\begin{definicion}[C\'odigos c\'{\i}clicos]
\ \\
Dado un c\'odigo lineal $\mathcal{C}[n,m]$ diremos que es un \textbf{``c\'odigo 
c\'{\i}clico''} si verifica que para toda palabra $u\in\mathcal{C}[n,m]$:
\begin{displaymath}
u=(u_1,u_2\dots,u_{n-1},u_n)\Longrightarrow (u_n,u_1,u_2,\dots,u_{n-2},u_{n-1})
\in \mathcal{C}[n,m]
\end{displaymath}
\end{definicion}

\begin{definicion}[Permutaci\'on c\'{\i}clica de orden uno]
\ \\
Una \textbf{``permutaci\'on c\'{\i}clica de orden uno''} es una aplicaci\'on
que desplaza un lugar hac\'{\i}a la derecha todos los elementos de una
sucesi\'on finita poniendo como primer elemento de la sucesi\'on el \'ultimo.
S\'{\i} $\sigma$ es una permutaci\'on c\'{\i}clica de orden uno entonces:
\begin{displaymath}
\sigma(\ (a_1,a_2,a_3,a_4)\ ) =(a_4,a_1,a_2,a_3)
\end{displaymath}
\end{definicion}

\begin{definicion}[Permutaci\'on c\'{\i}clica de orden k]
\ \\
Una \textbf{``permutaci\'on c\'{\i}clica de orden $k$''} consiste en aplicar
$k$ veces una permutaci\'on c\'{\i}clica de orden uno.
\begin{displaymath}
\sigma^{k}(\ (a_1,a_2,a_3,a_4)\ ) = (\sigma \circ \sigma \circ \stackrel{k}\dots
\circ \sigma)(\ (a_1,a_2,a_3,a_4)\ )
\end{displaymath}
\end{definicion}
%
\newpage
%
\begin{proposicion}
\ \\
Los c\'odigos c\'{\i}clicos son invariantes por permutaciones c\'{\i}clicas.
\end{proposicion}
\underline{\textbf{Demostraci\'on}}:\\
Para ver que son invariantes por permutaciones c\'{\i}clicas tenemos que ver
aplicar una permutaci\'on c\'{\i}clica de orden $k$, para cualquier $k$, a
una palabra cualquiera del c\'odigo produce otra palabra del c\'odigo.\\

Aplicar una permutaci\'on c\'{\i}clica de orden $k$ equivale a aplicar $k$
veces la permutaci\'on c\'{\i}clica de orden uno, por definici\'on.\\

Al aplicar la permutaci\'on c\'{\i}clica de orden uno a un elemento de
$\mathcal{C}[n,m]$ obtenemos otro elemento del c\'odigo, ya que el c\'odigo
es c\'{\i}clico. S\'{\i} a este elemento le volvemos a aplicar la permutaci\'on
c\'{\i}clica volvemos a obtener otra palabra del c\'odigo y asi sucesivamente.
Luego un c\'odigo c\'{\i}clico es invariante por permutaciones c\'{\i}clicas.
\begin{flushright}
$\blacksquare$
\end{flushright}

%
% UN ANILLO ESPECIAL
%

%
% ESTRUCTURA DEL ANILLO Z/q[x]/(x^n-1)
%

\section{El anillo $\mathbb{F}_q[x]/(x^n-1)$}

$\mathbb{F}_q[x]/(x^n-1)$ es un anillo conmutativo con unidad y adem\'as es
un $\mathbb{F}_q$-espacio vectorial\footnote{Ejercicio $\ref{ejer:Anillo}$ en la
p\'agina $\pageref{ejer:Anillo}$.}.
\begin{displaymath}
\mathbb{F}_q[x]/(x^n-1)=<\overline{1},\overline{x},\dots,\overline{x}^{n-1}>
\end{displaymath}
como $\mathbb{F}_q$-espacio vectorial.\\

Sean $P(x),Q(x)\in \mathbb{F}_q[x]$ entonces son equivalentes m\'odulo $x^n-1$
cuando su diferencia es un m\'ultiplo de $x^n-1$.
\begin{displaymath}
P(x)\equiv Q(x)\ mod\ x^n-1 \Longleftrightarrow P(x)-Q(x)=
\stackrel{\cdot }{(x^n-1)}
\end{displaymath}

Los c\'odigos que estamos utilizando son siempre subespacios vectoriales del
$\mathbb{F}_q$-espacio vectorial $\mathbb{F}^{^n}_q$. Podemos establecer un
isomorfismo entre $\mathbb{F}^{^n}_q$ y $\mathbb{F}_q[x]/(x^n-1)$:
\begin{displaymath}
\begin{array}{rcl}
\mathbb{F}_q[x]/(P(x))&\stackrel{\sim }\longrightarrow&\mathbb{F}^{^n}_q\\
a_0+a_1\cdot \overline{x}+\dots+a_{n-1}\cdot\overline{x}^{n-1}&\longrightarrow&
(a_0,a_1,\dots,a_{n-1})
\end{array}
\end{displaymath}
\begin{proposicion}\label{pro:EquivPCicl}
\ \\
Multiplicar por $\overline{x}$ en $\mathbb{F}_q[x]/(x^n-1)$ equivale a aplicar
una permutaci\'on c\'{\i}clica de orden uno en $\mathbb{F}^{^n}_q$.
\end{proposicion}
\underline{\textbf{Demostraci\'on}}:\\
Sea $\overline{p(x)}=a_0+a_1\cdot \overline{x}+\dots+a_{n-1}\cdot
\overline{x}^{n-1}$ entonces por el isomorfismo anterior se corresponder\'a, de
forma \'unica, con $(a_0,a_1,\dots,a_{n-1})\in \mathbb{F}^{^n}_q$.
Multipliquemos $\overline{p(x)}$ por $\overline{x}$:
\begin{displaymath}
\overline{x}\cdot \overline{p(x)} = a_0\cdot \overline{x}+a_1\cdot
\overline{x}^2+\dots+a_{n-2}\cdot \overline{x}^{n-1}+a_{n-1}\cdot
\overline{x}^n
\end{displaymath}
pero como $x^n-1=0$ en $\mathbb{F}_q[x]/(x^n-1)$ tenemos que $x^n=1$ en
$\mathbb{F}_q[x]/(x^n-1)$, luego:
\begin{displaymath}
\overline{x}\cdot \overline{p(x)} = a_{n-1}+a_0\cdot \overline{x}+a_1\cdot
\overline{x}^2+\dots+a_{n-2}\cdot\overline{x}^{n-1}
\end{displaymath}
el cual se corresponde, por el isomorfismo anterior, con:
\begin{displaymath}
(a_{n-1},a_0,a_1,\dots,a_{n-2})\in\mathbb{F}^{^n}_q
\end{displaymath}
elemento que se obtiene al aplicar una permutaci\'on c\'{\i}clica a
$(a_0,a_1,\dots,a_{n-1})$, elemento que representa a
$\overline{p(x)}$ en $\mathbb{F}^{^n}_q$.
\begin{flushright}
$\blacksquare$
\end{flushright}
%
\begin{corolario}
\ \\
Multiplicar por $\overline{x}^i$ con $i=0,\dots,n-1$ en
$\mathbb{F}_q[x]/(x^n-1)$ equivale a aplicar una permutaci\'on c\'{\i}clica de
orden $i$ en $\mathbb{F}^{^n}_q$.
\end{corolario}
\underline{\textbf{Demostraci\'on}}:\\
Inmediata a partir de la proposici\'on $\ref{pro:EquivPCicl}$.
\begin{flushright}
$\blacksquare$
\end{flushright}
%
\begin{teorema}\label{the:Multiplicativo}
\ \\
Sea $\mathcal{C}[n,m]\subset \mathbb{F}_q[x]/(x^n-1)$ un c\'odigo.
$\mathcal{C}[n,m]$ es c\'{\i}clico s\'{\i} y s\'olo s\'{\i} para todo
$\overline{p(x)}\in \mathcal{C}[n,m]$ se verifica que
$\overline{x}\cdot\overline{p(x)}\in \mathcal{C}[n,m]$.
\end{teorema}
\underline{\textbf{Demostraci\'on}}:\\
Inmediata a partir de las definiciones y la proposici\'on
$\ref{pro:EquivPCicl}$.
\begin{flushright}
$\blacksquare$
\end{flushright}
%
\begin{corolario}
\ \\
S\'{\i} $\overline{p(x)}\in \mathbb{F}_q[x]/(x^n-1)$
entonces para $i\in \mathbb{Z}$ se tiene que:
\begin{displaymath}
\overline{x}^i\cdot \overline{p(x)}\in \mathbb{F}_q[x]/(x^n-1)
\end{displaymath}
\end{corolario}
\underline{\textbf{Demostraci\'on}}:\\
Es inmediato a partir del teorema $\ref{the:Multiplicativo}$ y
$\overline{x}^i=\overline{x}^{(i-1)}\cdot \overline{x}$.
\begin{flushright}
$\blacksquare$
\end{flushright}


%
% CARACTERIZACION DE LOS CODIGOS CICLICOS
%

%
% CARACTERIZACION DE LOS CODIGOS CICLICOS
%

\section{Caracterizaci\'on de los c\'odigos c\'{\i}clicos}

Los c\'odigos c\'{\i}clicos son ideales del anillo $\mathbb{F}_q[x]/(x^n-1)$.

\begin{proposicion}\label{pro:Ideal}
\ \\
Sea $\mathcal{C}[n,m]$ un c\'odigo c\'{\i}clico. Para todo $\overline{p(x)}\in
\mathcal{C}[n,m]$ y para cualquier $\overline{Q(x)}\in \mathbb{F}_q[x]/(x^n-1)$
se tiene que $\overline{p(x)}\cdot \overline{Q(x)}\in \mathcal{C}[n,m]$.
\end{proposicion}
\underline{\textbf{Demostraci\'on}}:\\
Sean:
\begin{eqnarray*}
\overline{p(x)}=p_0+p_1\cdot \overline{x}+\dots+p_{n-1}\cdot
\overline{x}^{n-1}&\in& \mathcal{C}[n,m]\\
\overline{Q(x)}=q_0+q_1\cdot \overline{x}+\dots+q_{n-1}\cdot
\overline{x}^{n-1}&\in& \mathbb{F}_q[x]/(x^n-1)
\end{eqnarray*}
entonces se tiene que:
\begin{displaymath}
\overline{p(x)}\cdot \overline{Q(x)} = q_0\cdot \overline{p(x)}+
q_1\cdot\overline{x}\cdot\overline{p(x)}+\dots+q_{n-1}\cdot\overline{x}^{n-1}
\cdot\overline{p(x)}
\end{displaymath}
como el c\'odigo es c\'{\i}clico se tiene que:
\begin{displaymath}
\overline{x}^i\cdot\overline{p(x)}\in \mathcal{C}[n,m]\quad i=0,1,\dots
\end{displaymath}
y como el c\'odigo es lineal se tiene que:
\begin{displaymath}
\lambda\cdot \overline{h(x)}\in\mathcal{C}[n,m]
\end{displaymath}
para $\lambda\in\mathbb{F}_q$ y $\overline{h(x)}\in \mathcal{C}[n,m]$. Luego
$\overline{p(x)}\cdot\overline{Q(x)}$ es la suma de polinomios que pertenecen
al c\'odigo y por la linealidad de este la suma de elementos del c\'odigo
pertenece al c\'odigo.
\begin{flushright}
$\blacksquare$
\end{flushright}
%
%
\begin{teorema}[Caracterizaci\'on de los c\'odigos c\'{\i}clicos]\label{the:CarCiclicos}
\ \\
Un c\'odigo $\mathcal{C}[n,m]$ es c\'{\i}clico s\'{\i} y s\'olo s\'{\i} como
subconjunto es un ideal de $\mathbb{F}_q[x]/(x^n-1)$.
\end{teorema}
\underline{\textbf{Demostraci\'on}}:\\
Es inmediato a partir de la definici\'on de ideal y de la proposici\'on
$\ref{pro:Ideal}$.
\begin{flushright}
$\blacksquare$
\end{flushright}
Seg\'un este teorema todos los c\'odigos c\'{\i}clicos son ideales del anillo,
de ideales principales, $\mathbb{F}_q[x]/(x^n-1)$. Luego todos
sus ideales estar\'an generados por un elemento, es decir ser\'an de la forma
$(\overline{h(x)})$.\\

En el ejercicio $\ref{ejer:Anillo}$, en la p\'agina $\pageref{ejer:Anillo}$, se
demuestra que todos los ideales del anillo $\mathbb{F}_q[x]/(x^n-1)$ est\'an
generados por los productos de los factores irreducibles en los que descompone
$x^n-1$ en $\mathbb{F}_q[x]$.
%
\begin{definicion}[Polinomio generador del c\'odigo c\'{\i}clico]
\ \\
Se llama \textbf{``polinomio generador del c\'odigo c\'{\i}clico''} al
polinomio que genera el ideal al que corresponde el c\'odigo.
\end{definicion}


%
% CONSTRUCCION DE CODIGOS CICLICOS
%

%
% CONSTRUCCION DE CODIGOS CICLICOS
% 

\section{Construcci\'on de c\'odigos c\'{\i}clicos}

Sea $\mathcal{C}[n,m]$ un c\'odigo lineal c\'{\i}clico. Tendremos entonces que:
\begin{displaymath}
\mathcal{C}[n,m]\subset \mathbb{F}^{^n}_q\quad y\quad
\mathcal{C}[n,m]\simeq \mathbb{F}^{^m}_q
\end{displaymath}
Adem\'as como el c\'odigo es c\'{\i}clico se corresponde con un ideal del 
anillo $\mathbb{F}_q[x]/(x^n-1)$:
\begin{displaymath}
\mathcal{C}[n,m] = (g(x))\quad con\ g(x)=a_0+a_1\cdot \overline{x}+\dots+
a_{m-1}\cdot \overline{x}^{m-1}
\end{displaymath}

\subsection{Matriz generadora de un c\'odigo c\'{\i}clico}

La matriz generadora del c\'odigo c\'{\i}clico $\mathcal{C}[n,m]$ ser\'a la
matriz de la aplicaci\'on lineal:
\begin{displaymath}
\mathbb{F}^{^m}_q\stackrel{\sim}\longrightarrow \mathcal{C}[n,m]
\end{displaymath}
Observar que $\mathbb{F}_q[x]/(x^n-1)$ es un $\mathbb{F}_q$-espacio vectorial,
adem\'as de anillo, por lo cual cualquier ideal suyo es tambi\'en subespacio
vectorial.\\

Por el teorema $\ref{the:CarCiclicos}$ sabemos que los c\'odigos c\'{\i}clicos
son los ideales del anillo $\mathbb{F}_q[x]/(x^n-1)$, o lo que es lo mismo su
polinonio generador divide a $x^n-1$, sobre $\mathbb{F}_q$.\\

Sea $x^n-1=p_1(x)\cdot p_r(x)$ la descomposici\'on en irreducibles sobre
$\mathbb{F}_q[x]$, y sea $g(x)=p_1(x)\cdot p_i(x)$ con $i<r$, entonces
tendremos que $x^n-1= g(x)\cdot h(x)$ con:
\begin{eqnarray*}
g(x)&=&g_0+g_1\cdot x+\dots+g_{n-k}\cdot x^{n-m}\\
h(x)&=&h_0+h_1\cdot x+\dots+h_k\cdot x^m
\end{eqnarray*}
Sea $\mathcal{C}=(g(x))$ y llamemos $A=\mathbb{F}_q[x]/(x^n-1)$, entonces:
\begin{displaymath}
A/\mathcal{C} = \mathbb{F}_q[x]/(g(x))
\end{displaymath}
$\mathbb{F}_q[x]/(g(x))$ es de dimensi\'on $n-m$, luego el c\'odigo ser\'a
de dimensi\'on $m$. Para encontrar una base nos bastar\'{\i}a con encontrar
$m$ vectores linealmente independientes entonces formar\'{\i}an base al estar
en un subespacio de dimensi\'on $m$.
\begin{displaymath}
\mathcal{C} = <g(x),x\cdot g(x),\dots.x^{m-1}\cdot g(x)>
\end{displaymath}
Que son base es inmediato, ya que al ser $\mathcal{C}$ un c\'odigo c\'{\i}clico
y $g(x)\in \mathcal{C}$ entonces $x^i\cdot g(x)\in \mathcal{C}$. Dichos
polinomios expresados en la base can\'onica de $A$ ser\'an:
\begin{eqnarray*}
g(x)&=& (g_0,\dots,g_{n-m},0,\stackrel{m-1)}\dots,0)\\
x\cdot g(x)&=&(0,g_0,\dots,g_{n-m},0,\stackrel{m-2)}\dots,0)\\
\dots& &\dots \dots \dots\\
x^{m-1}\cdot g(x)&=&(0,\stackrel{m-1)}\dots,0,g_0,\dots,g_{n-m})
\end{eqnarray*}
que son $m$ vectores linealmente independientes, luego generan. La matriz
generadora del c\'odigo ser\'a:
\begin{displaymath}
\left( \begin{array}{cccc}
g_0&0&\dots & 0 \\
g_1&g_0&\dots& 0 \\
g_2&g_1&\dots& \vdots \\
\vdots &\vdots &\vdots & g_0 \\
g_{n-m}&\vdots & \vdots &\vdots \\
0&g_{n-m}& \vdots &\vdots\\
\vdots &\vdots &\vdots & \vdots\\
0&0& \vdots & g_{n-m} 
\end{array}\right)
\end{displaymath}
Luego tenemos un c\'odigo de dimensi\'on $m$ y con palabras de longitud $n$,
luego nuestro c\'odigo ser\'a del tipo $\mathcal{C}[n,m]$.

\subsection{C\'odigo dual de un c\'odigo c\'{\i}clico}

Sea $x^n-1=g(x)\cdot h(x)$ y $\mathcal{C}[n,m]=(g(x))$ entonces definimos el
c\'odigo dual de $\mathcal{C}[n,m]$, y lo denotamos como
$\mathcal{C}^{\perp}[n,m]$, al c\'odigo c\'{\i}clico generado por el
polinomio $x^k\cdot h(x^{-1})$.\\

La matriz generadora de $\mathcal{C}^{\perp}[n,m]$ es la matriz de control de
$\mathcal{C}[n,m]$.


%
% CODIGOS DE REED SOLOMON
%

%
% CODIGOS DE REED-SOLOMON
%

\section{C\'odigos de Reed-Solomon}

Sea $\beta$ una raiz primitiva $n$-\'esima de la unidad\footnote{Cualquier otra
raiz primitiva $n$-\'esima de la unidad es una potencia suya.} en 
$\mathbb{F}_q$ y $f(x)$ su polinomio m\'{\i}nimo, entonces:
\begin{displaymath}
x^n-1=(x-1)\cdot f_1(x)\cdot f_k(x)
\end{displaymath}
definamos el polinomio $g(x)$ como el m\'{\i}nimo com\'um multiplo de los
polinomios m\'{\i}nimos $f_1,\dots,f_k$. Entonces el c\'odigo generado por
dicho polinomio es un c\'odigo \emph{BCH}.\\

Si tomamos $n=q-1$, $q=2^k$, el c\'odigo resultante se llama c\'odigo de
\emph{Reed-Solomon}.

