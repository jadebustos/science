%
% CODIGOS LINEALES
%

\chapter{C\'odigos lineales}

Hemos visto que los c\'odigos de bloques, con sus codificadores,
decodificadores, procesadores de error se definen sobre un alfabeto
$\mathbb{F}_q$, el cual es un conjunto cualquiera. Pero si sobre este
conjunto podemos establecer alguna estructura algebraica, la cual nos permita
realizar operaciones sobre los elementos del conjunto, podremos entonces
simplificar el c\'odigo. Esta simplificaci\'on nos permitir\'a trabajar de
una forma m\'as comoda.

\section{Requisitos para c\'odigos lineales}

Cuando trabajemos con c\'odigos lineales tendremos que $(\mathbb{F}_q,+,\cdot)$
es cuerpo, por ejemplo $\mathbb{F}_2=\mathbb{Z}/2=\mathbb{Z}_2=\{0,1\}$.\\

Dado un cuerpo cualquiera $A$ se tiene que $A^{^n}=A\times \stackrel{n)}\cdots
\times A$ es un \mbox{$A$-espacio vectorial} con las siguientes operaciones:
\begin{itemize}
\item $(x_1,\cdots,x_n)+(x_1',\cdots,x_n')=(x_1+x_1',\cdots,x_n+x_n')$, donde
tenemos que $(x_1,\cdots,x_n)$ y $(x_1',\cdots,x_n')$ son dos palabras
cualesquiera de $A^{^n}$.
\item $x\odot (x_1,\cdots,x_n)=(x\cdot x_1,\cdots, x\cdot x_n)$, donde
tenemos que $x\in A$, $(x_1,\cdots,x_n)\in A^{^n}$ y $\odot$ es el producto
por escalares definido en $A^{^n}$.
\end{itemize}
Nosotros trabajaremos con cuerpos de la forma $\mathbb{F}_q$ y con
$\mathbb{F}^{^n}_q$ que, por lo visto antes, ser\'an
\mbox{$\mathbb{F}_q$-espacios vectoriales}.

%
% CUANDO UN CODIGO ES LINEAL
%

%
% CUANDO UN CODIGO ES LINEAL
%

%
\newpage
%
\section{?`Cuando un c\'odigo es lineal?}

Podemos distinguir unos c\'odigos de bloques especiales que son los c\'odigos
lineales.
\begin{definicion}[C\'odigos Lineales]
\ \\
Dado un cuerpo cualquiera, $\mathbb{K}$, y un $\mathbb{K}$-espacio vectorial de
la forma $\mathbb{K}^{^n}$ diremos que un subconjunto $\mathcal{C}\subset
\mathbb{K}^{^n}$ es un \textbf{``c\'odigo lineal''} si $\mathcal{C}$ es un
subespacio vectorial de $\mathbb{K}^{^n}$.
\end{definicion}
Nosotros consideraremos $\mathbb{K}=\mathbb{F}_q$ y
$\mathbb{K}^{^n}=\mathbb{F}^{^n}_q$.

\subsection{Propiedades de los c\'odigos lineales}

Dado que los c\'odigos lineales son subespacios vectoriales poseen unas
propiedades especiales.\\ \\ 
%
Sea $\mathcal{C}\subset \mathbb{F}^{^n}_2$ un c\'odigo lineal, entonces se
verifican las siguientes propiedades:
\begin{itemize}
\item La suma de dos palabras del c\'odigo es otra palabra del c\'odigo.
\begin{displaymath}
(x_1,\cdots,x_n)+(x_1',\cdots,x_n') = (x_1+x_1',\cdots,x_n+x_n')\in \mathcal{C}
\end{displaymath}
\item Multiplicar una palabra del c\'odigo por un elemento del cuerpo es otra
palabra del c\'odigo.
\begin{displaymath}
x\odot (x_1,\cdots,x_n) = (x\cdot x_1,\cdots, x\cdot x_n)\in \mathcal{C}
\end{displaymath}
\item El $0$ siempre es una palabra del c\'odigo.
\begin{displaymath}
(0,\stackrel{n)}\cdots,0)\in \mathcal{C}
\end{displaymath}
\end{itemize}


%
% DISTANCIA MINIMA
%

%
% DISTANCIA MINIMA 
%

\section{Calculo de la distancia m\'{\i}nima}

Debido a la estructura de espacios vectoriales que poseen estos c\'odigos se
puede simplificar el calculo de la distancia m\'{\i}nima, $d_{min}$. Seg\'un
la definici\'on de distancia m\'{\i}nima el n\'umero de distancias que tendremos
que calcular para encontrala ser\'a de $\frac{|\mathcal{C}|^2}{2}$, es decir,
tendr\'{\i}amos que calcular tantas distancias como parejas de dos elementos
distintos del c\'odigo podamos tener dividido por dos, ya que la distancia es
sim\'etrica. \\

\begin{proposicion}[Distancia m\'{\i}nima para c\'odigos lineales]
\ \\
Sea $\mathcal{C}\subset \mathbb{F}^{^n}_q$ un c\'odigo lineal, entonces:
$$d_{min} = \min_{u \in \mathcal{C}\atop u\neq 0} \{\ d(u,0)\ \}$$
\end{proposicion}
\underline{\textbf{Demostraci\'on}}:\ \\
Supongamos que la distancia m\'{\i}nima es:
$$d_{min} = d(u,v)\quad u,v\in \mathcal{C}$$
entonces por la definici\'on de la distancia de Hamming se tiene que:
$$d(u,v)=d(u-v,0)$$ y como el c\'odigo $\mathcal{C}$ es un c\'odigo lineal se
tiene entonces que $u-v\in \mathcal{C}$. Es decir:
$$d_{min}= \min_{u\in \mathcal{C} \atop u\neq 0} \{\ d(u,0)\ \}$$
\begin{flushright}
$\blacksquare$
\end{flushright}
\begin{observacion}
\ \\
De esta proposici\'on se deduce que para calcular la distancia m\'{\i}nima de 
un c\'odigo lineal basta con calcular la distancia de todas las palabras, salvo
la palabra cero, al cero y tomar el m\'{\i}nimo de esas distancias. Con lo cual
unicamente tendremos que calcular $|\mathcal{C}|-1$ distancias en lugar de las
$\frac{|\mathcal{C}|^2}{2}$ distancias que deberiamos calcular para un c\'odigo
no lineal.
\end{observacion}


%
% CODIFICADORES
%

%
% CODIFICADORES
%

\section{Codificadores}

Seg\'un el apartado $\ref{sec:Codificadores}$, en la p\'agina
$\pageref{sec:Codificadores}$, un codificador es una aplicaci\'on biyectiva.

\begin{definicion}[Codificador para c\'odigos lineales]
\ \\
Sea $\mathcal{C}\subset \mathbb{F}^{^n}_q$ un c\'odigo lineal. Un
\textbf{``codificador''} para un c\'odigo lineal es un isomorfismo%
\footnote{Aplicaci\'on lineal biyectiva.} del siguiente tipo:
\begin{eqnarray*}
C:\mathbb{F}^{^m}_q&\stackrel{\sim}\longrightarrow&\mathcal{C}\\
x&\longrightarrow& u
\end{eqnarray*}
tal que codifica palabras de longitud $m$ con palabras de longitud $n$.
\end{definicion}
\begin{observacion}
\ \\
\begin{itemize}
\item Al ser el codificador una aplicaci\'on biyectiva cada palabra tiene una, y
s\'olo una, forma de codificarse. 
\item Todas las posibles palabras que puedan formar un mensaje se pueden
codificar con una palabra del c\'odigo. Esto es gracias a que el codificador es
epiyectivo, y, por lo dicho en el punto anterior dicha palabra es \'unica.
\item Como el codificador es una aplicaci\'on lineal podemos representar el
codificador por la matriz que representa a la aplicaci\'on lineal una vez
fijadas las correspondientes bases. A est\'a matriz se la conoce como
``\textbf{matriz generadora}''.
\item Como la matriz depende de las bases escogidas, entonces la matriz del
codificador no es \'unica.
\item La matriz de un codificador de un c\'odigo lineal estar\'a formada por
tantas columnas como dimensi\'on tenga el c\'odigo. Y la columna $i$-\'esima
del codificador ser\'a el vector $i$-esimo de la base, del c\'odigo, 
codificado seg\'un el c\'odigo.
\item Como el codificador es un isomorfismo, aplicaci\'on lineal biyectiva, la
imagen de una base es otra base, entonces las columnas de la matriz generadora
son base del c\'odigo $\mathcal{C}$.
\end{itemize}
\end{observacion}

\subsection{Matriz generadora}

\begin{definicion}[Matriz generadora]
\ \\
Dado un c\'odigo lineal y su codificador llamaremos \textbf{``matriz
generadora''} de c\'odigo a la matriz de la aplicaci\'on lineal que representa
al codificador respecto de unas bases fijadas.
\end{definicion}
Para codificar una palabra basta con multiplicarla por la matriz generadora del
c\'odigo y obtendremos la palabra codificada. Por ejemplo, sea $C$ la matriz
generadora de un c\'odigo y queremos codificar la palabra $(1,1,0)$:
\begin{displaymath}
\left( \begin{array}{ccc}
1&0&0\\
0&1&0\\
0&0&1\\
1&1&0\\
1&0&1\\
0&1&1
\end{array} \right) \cdot
\left( \begin{array}{c}
1\\
1\\
0
\end{array} \right) =
\left( \begin{array}{c}
1\\
1\\
0\\
0\\
1\\
1
\end{array} \right)
\end{displaymath}
Luego la palabra $(1,1,0)$ codificada es $(1,1,0,0,1,1)$.\\ 

La siguiente proposici\'on nos dice cuando una matriz es una matriz generadora
de un c\'odigo, y en caso afirmativo, de que tipo de c\'odigo se trata.

\begin{proposicion}
\ \\
Sea $\mathcal{C}[n,m]$ un c\'odigo lineal y sea $C$ una matriz de orden
$n\times m$. $C$ es una matriz generadora para el c\'odigo $\mathcal{C}[n,m]$
s\'{\i} y s\'olo s\'{\i} tiene rango $m$ y sus columnas son palabras del
c\'odigo.
\end{proposicion}
\underline{\textbf{Demostraci\'on}}:\\
$\Rightarrow |$ Sea $C$ una matriz generadora para el c\'odigo
$\mathcal{C}[n,m]$. \\

Por definici\'on de matriz generadora sus columnas ser\'an palabras del
c\'odigo, y como el c\'odigo lineal es de dimensi\'on $m$, dimensi\'on
del subespacio imagen de la aplicaci\'on lineal que determina $C$, entonces
el rango de $C$ es $m$.\\ \\
%
$\Leftarrow |$ Sean $C_1$ , \dots, $C_m$ las columnas de la matriz $C$, las
cuales son, por hip\'otesis, palabras del c\'odigo.\\

Sea $a=(a_1,\dots,a_m )$ una palabra entonces $C\cdot a^t = a_1\cdot
C_1^t+\dots +a_m\cdot C_m^t$ que es una combinaci\'on lineal de
$\{C_i \}_{i=1}^m$ y como el c\'odigo $\mathcal{C}[n,m]$ es lineal se tiene
que las combinaciones lineales de palabras del c\'odigo son palabras del 
c\'odigo. Luego la matriz $C$ transforma cualquier palabra, de longitud $m$, 
en una palabra del c\'odigo.\\

La dimensi\'on del subespacio imagen es el rango de la matriz $C$, luego la
dimensi\'on del subespacio imagen es $m$.\\

Como la matriz $C$ codifica palabras de longitud $m$ en palabras de longitud
$n$ en un subespacio de dimensi\'on $m$ entonces $C$ es la matriz generadora
de un c\'odigo $\mathcal{C}[n,m]$.
\begin{flushright}
$\blacksquare$
\end{flushright}
%
\newpage
%
\subsection{Forma est\'andar de la matriz generadora}

La expresi\'on de la ``\textbf{matriz generadora}'' no es unica, sino que
depende de las bases escogidas. Luego eligiremos un convenio para elegir
las bases en las que expresaremos dicha matriz.
\begin{definicion}[Forma est\'andar de la matriz generadora]
\ \\
Sea $\mathcal{C}\subset \mathbb{F}^{^n}_q$ un c\'odigo lineal que utilizaremos
para codificar palabras de longitud $m$, $\mathbb{F}^{^m}_q$. Sea $C$ la
matriz del codificador, una vez fijadas las bases correspondientes. Diremos
que $C$ est\'a en \textbf{``forma est\'andar''} cuando los $m$ primeros elementos
de la palabra codificada $C(x)\in \mathcal{C}\subset \mathbb{F}^{^n}_q$
coincidan con $x\in \mathbb{F}^{^m}_q$.
\end{definicion}
\begin{proposicion}
\ \\
Sea $\mathcal{C}[n,m]$ un c\'odigo lineal. La matriz generadora est\'a en 
forma est\'andar s\'{\i} y s\'olo s\'{\i} en las $m$ primeras columnas
aparece la matriz identidad de orden $m$.
\end{proposicion}
\underline{\textbf{Demostraci\'on}}:\\
Tanto el directo como el rec\'{\i}proco son una mera comprobaci\'on.
\begin{flushright}
$\blacksquare$
\end{flushright}


%
% ECUACIONES DE LOS CODIGOS LINEALES
%

%
% ECUACIONES DE LOS CODIGOS LINEALES
%

\section{Ecuaciones de los c\'odigos lineales}

Como hemos visto un c\'odigo lineal $\mathcal{C}$ es un subespacio vectorial.
Utilizando la teor\'{\i}a de espacios vectoriales podemos calcular las
ecuaciones de un subespacio vectorial cualquiera. La utilidad de estas
ecuaciones radica en que nos indican las condiciones que ha de cumplir un
vector para pertenecer al c\'odigo.\\

Sea $\mathcal{C}\subset \mathbb{F}^{^n}_q$ un c\'odigo lineal, donde
$\mathbb{F}^{^n}_q$ es un $\mathbb{F}_q$-espacio vectorial tal que
$\dim_{\mathbb{F}_q} \mathbb{F}^{^n}_q=n$. Supongamos que
$\dim_{\mathbb{F}_q} \mathcal{C} = m$, con $m<n$ y que el espacio de palabras
a codificar es $\mathbb{F}^{^m}_q$.
%
\newpage
%
\subsection{Ecuaciones param\'etricas}\label{sec:EcuParametricas}

Una vez conocida la matriz generadora del c\'odigo se pueden calcular las
ecuaciones param\'etricas del c\'odigo. Dichas ecuaciones nos permitir\'an
calcular todas las palabras del c\'odigo.\\

Sea $C$ la matriz generadora del c\'odigo, para calcular las ecuaciones
pa\-ra\-m\'etricas del c\'odigo bastar\'a con aplicar la matriz generadora a un
vector gen\'erico del espacio de palabras a codificar.
\begin{displaymath}
C\cdot \left( \begin{array}{c}
\lambda_1\\
\lambda_2\\
\vdots\\
\lambda_m
\end{array} \right) =
\left( \begin{array}{c}
\theta_1\\
\theta_2\\
\vdots\\
\theta_n
\end{array} \right)
\end{displaymath}
luego las ecuaciones param\'etricas ser\'an:
\begin{displaymath}
\left\{ \begin{array}{ccc}
x_1 &=&\theta_1\\
x_2 &=&\theta_2\\
\cdots&\cdots&\cdots\\
x_n &=&\theta_n
\end{array} \right.
\end{displaymath}
Tendremos que $\theta_i = f_i (\lambda_1,\dots,\lambda_m)$ para $i=1,\dots,n$ y
dando valores a $\{\lambda_j \}_{j=1}^m$ en el cuerpo $\mathbb{F}_q$ tendremos
todos los elementos del c\'odigo.

\subsection{Ecuaciones impl\'{\i}citas}\label{sec:EcuImplicitas}

Para calcular las ecuaciones impl\'{\i}citas de $\mathcal{C}$ calcularemos el
subespacio incidente a $\mathcal{C}$, $\mathcal{C}^{^{\circ}}\subset
(\mathbb{F}^{^n}_q)^{^*}$. El subespacio incidente est\'a formado por las
formas lineales que se anulan sobre $\mathcal{C}$.\\

El n\'umero de ecuaciones impl\'{\i}citas que tendr\'a nuestro c\'odigo
$\mathcal{C}$ ser\'a:
$$dim_{\mathbb{F}_q} \mathbb{F}^{^n}_q - dim_{\mathbb{F}_q} \mathcal{C}=n-m$$
Como $\dim_{\mathbb{F}_q} \mathcal{C} = m$ tendremos que
$\mathcal{C}=<c_1,\ldots,c_m>$. Para calcular las ecuaciones tendremos que
resolver un sistema de ecuaciones que tendr\'a m\'as de una soluci\'on.
%
\newpage
%
Sean $\{x_i^j\}_{j=1}^m$ con $i=1,\dots,n$ la $i$-\'esima coordenada del
vector $j$-\'esimo de la base de $\mathcal{C}$. El sistema que nos da las
ecuaciones ser\'a:
\begin{displaymath}
\left( \begin{array}{cccc}
x_1^1 & \ldots & \ldots & x_n^1 \\
x_1^2 & \ddots & & x_n^2 \\
\vdots & &\ddots & \vdots \\
x_1^m &\ldots &\ldots & x_n^m
\end{array} \right) \cdot
\left( \begin{array}{c}
x_1 \\
\vdots \\
\vdots \\
x_n
\end{array} \right) =
\left( \begin{array}{c}
0 \\
\vdots \\
\vdots \\
0
\end{array} \right)
\end{displaymath}

Cualquier $x=(x_1,\ldots,x_n)$ que verifique este sistema de ecuaciones
ser\'a un elemento de $\mathcal{C}$, o lo que es lo mismo cualquier punto que no
verifique el sistema de ecuaciones anterior NO pertenecer\'a a $\mathcal{C}$.


%
% DETECCION DE ERRORES
%

%
% DETECCION DE ERRORES
%

\section{Detecci\'on de errores}

Supondremos que tenemos un c\'odigo lineal $\mathcal{C}\subset
\mathbb{F}^{^n}_q$ tal que $\dim_{\mathbb{F}_q} \mathcal{C}=m$, con $m<n$. De
esto se deduce que utilizaremos el c\'odigo para codificar palabras de longitud
$m$ luego es un c\'odigo $\mathcal{C}[n,m]$.\\

Se detecta un error cuando una de las palabras recibidas no pertenece al
c\'odigo. En el caso de los c\'odigos lineales es facil comprobar cuando una
palabra pertenece o no al c\'odigo.\\

Dado que los c\'odigos lineales son subespacios vectoriales tienen asociadas
unas ecuaciones:
\begin{itemize}
\item Ecuaciones param\'etricas\footnote{Apartado $(\ref{sec:EcuParametricas})$,
en la p\'agina $\pageref{sec:EcuParametricas}$.}.
\item Ecuaciones impl\'{\i}citas\footnote{Apartado $(\ref{sec:EcuImplicitas})$,
en la p\'agina $\pageref{sec:EcuImplicitas}$.}.
\end{itemize}
Para comprobar s\'{\i} una palabra pertenece o no al c\'odigo basta con ver
s\'{\i} verifica las ecuaciones que definen al c\'odigo, param\'etricas o
impl\'{\i}citas.
%
\newpage
%
\subsection{Matriz de control}

\begin{definicion}[Matriz de control]
\ \\
Para un c\'odigo lineal $\mathcal{C}$ diremos que una matriz $M$ de orden
$k\times n$ es una \textbf{``matriz de control''} del c\'odigo si verifica que
$M\cdot u^t = 0$ para todo $u\in \mathbb{F}^{^n}_q$ s\'{\i} y
s\'olo s\'{\i} $u\in \mathcal{C}$.
\end{definicion}
$k$ puede ser cualquiera, pero normalmente $k=n-m$ ya que esa es la dimensi\'on
del subespacio incidente a $\mathcal{C}$.\\ \\
%
La matriz de control la podemos obtener de:
\begin{itemize}
\item Las ecuaciones impl\'{\i}citas de $\mathcal{C}$.
\item La base del subespacio incidente a $\mathcal{C}$.
\end{itemize}
La matriz de control es una matriz cuyas filas son una base de ``vectores''
incidentes a $\mathcal{C}$.

\subsection{Forma est\'andar de la matriz de control}

Al igual que en el caso de la matriz generadora la expresi\'on de la 
\textbf{``matriz de control''} depende de la base escogida en el subespacio
incidente a $\mathcal{C}$.

\begin{definicion}[Forma est\'andar de la matriz de control]
\ \\
Sea $M$ una matriz de control de orden $(n-m)\times n$, diremos que est\'a en
\textbf{``forma est\'andar''} si $M=(M_1,M_2)$, donde $M_2$ es la matriz
identidad de orden $(n-m)\times (n-m)$.
\end{definicion}


%
% CODIGOS DUALES
%

%
% CODIGOS DUALES
%

%
\newpage
%
\section{C\'odigos duales}

\begin{definicion}[Producto escalar en $\mathbb{F}^{^n}_q$]
\ \\
En $\mathbb{F}^{^n}_q$ podemos definir el siguiente producto escalar:
\begin{displaymath}
\begin{array}{cccl}
<\ ,\ >:&\mathbb{F}^{^n}_q\times\mathbb{F}^{^n}_q&\longrightarrow&\mathbb{Z}^+\\
&(x,y)&\longrightarrow& x_1\cdot y_1+\dots+x_n\cdot y_n
\end{array}
\end{displaymath}
\end{definicion}
Se comprueba de forma inmediata que $<\ ,\ >$ verifica las condiciones para ser
producto escalar:
\begin{itemize}
\item $<x,y>\geq 0$ $\forall \ x,y\in \mathbb{F}^{^n}_q$.
\item $<x,x> > 0$ $\forall \ x\in \mathbb{F}^{^n}_q$ y $u\neq 0$.
\end{itemize}

Una vez definido un producto escalar podemos definir una relaci\'on de
ortoganilidad.
\begin{definicion}[Ortogonalidad entre vectores]
\ \\
$x$ e $y$ son ortogonales s\'{\i} y s\'olo s\'{\i} $<x,y>=0$.
\end{definicion}
Mediante esta definici\'on de ortogonalidad podemos calcular el ortogonal a
un subespacio dado.\\

Sea $\mathcal{C}[n,m]$ un c\'odigo lineal. Entonces es un subespacio vectorial
de $\mathbb{F}^{^n}_q$ tal que $\dim_{\mathbb{F}_q} \mathcal{C}[n,m]=m$.
\begin{definicion}[C\'odigo dual]
\ \\
Se define el \textbf{``c\'odigo dual''} de $\mathcal{C}[n,m]$ como el subespacio
vectorial ortogonal a $\mathcal{C}[n,m]$ y lo denotaremos como
$\mathcal{C}[n,m]^{\perp}$.
\begin{displaymath}
\mathcal{C}[n,m]^{\perp} = \{\ x\in \mathbb{F}^{^n}_q\ |\
<x,\mathcal{C}[n,m]> = 0\ \}
\end{displaymath}
\end{definicion}
\begin{observacion}
\ \\
\begin{itemize}
\item $\mathcal{C}[n,m]^{\perp}=\mathcal{C}'[n,n-m]$ ya que:
\begin{displaymath}
\dim_{\mathbb{F}_q} \mathcal{C}[n,m]^{\perp} =\dim_{\mathbb{F}_q}
\mathbb{F}^{^n}_q - \dim_{\mathbb{F}_q} \mathcal{C}[n,m] = n-m
\end{displaymath}
\item Sea $G$ la matriz generadora del c\'odigo $\mathcal{C}[n,m]$ y $H$ la
matriz generadora del c\'odigo $\mathcal{C}[n,m]^{\perp}$ entonces se tiene que
$H^t$ es una matriz de control para el c\'odigo $\mathcal{C}[n,m]$. O lo que es
lo mismo:
\begin{displaymath}
\left( \mathcal{C}[n,m]^{\perp} \right)^{\perp} = \mathcal{C}[n,m]
\end{displaymath}
\end{itemize}
\end{observacion}
De las observaciones se deduce que el c\'odigo dual determina al c\'odigo:
\begin{displaymath}
\mathcal{C}[n,m] = \{\ x\in \mathbb{F}^{^n}_q\ |\ H^t \cdot x^t=0\ \}
\end{displaymath}
donde $H^t$ es la matriz traspuesta de la matriz generadora del c\'odigo
$\mathcal{C}[n,m]^{\perp}$.

