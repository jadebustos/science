%
% DETECCION DE ERRORES
%

\section{Detecci\'on de errores}

Supondremos que tenemos un c\'odigo lineal $\mathcal{C}\subset
\mathbb{F}^{^n}_q$ tal que $\dim_{\mathbb{F}_q} \mathcal{C}=m$, con $m<n$. De
esto se deduce que utilizaremos el c\'odigo para codificar palabras de longitud
$m$ luego es un c\'odigo $\mathcal{C}[n,m]$.\\

Se detecta un error cuando una de las palabras recibidas no pertenece al
c\'odigo. En el caso de los c\'odigos lineales es facil comprobar cuando una
palabra pertenece o no al c\'odigo.\\

Dado que los c\'odigos lineales son subespacios vectoriales tienen asociadas
unas ecuaciones:
\begin{itemize}
\item Ecuaciones param\'etricas\footnote{Apartado $(\ref{sec:EcuParametricas})$,
en la p\'agina $\pageref{sec:EcuParametricas}$.}.
\item Ecuaciones impl\'{\i}citas\footnote{Apartado $(\ref{sec:EcuImplicitas})$,
en la p\'agina $\pageref{sec:EcuImplicitas}$.}.
\end{itemize}
Para comprobar s\'{\i} una palabra pertenece o no al c\'odigo basta con ver
s\'{\i} verifica las ecuaciones que definen al c\'odigo, param\'etricas o
impl\'{\i}citas.
%
\newpage
%
\subsection{Matriz de control}

\begin{definicion}[Matriz de control]
\ \\
Para un c\'odigo lineal $\mathcal{C}$ diremos que una matriz $M$ de orden
$k\times n$ es una \textbf{``matriz de control''} del c\'odigo si verifica que
$M\cdot u^t = 0$ para todo $u\in \mathbb{F}^{^n}_q$ s\'{\i} y
s\'olo s\'{\i} $u\in \mathcal{C}$.
\end{definicion}
$k$ puede ser cualquiera, pero normalmente $k=n-m$ ya que esa es la dimensi\'on
del subespacio incidente a $\mathcal{C}$.\\ \\
%
La matriz de control la podemos obtener de:
\begin{itemize}
\item Las ecuaciones impl\'{\i}citas de $\mathcal{C}$.
\item La base del subespacio incidente a $\mathcal{C}$.
\end{itemize}
La matriz de control es una matriz cuyas filas son una base de ``vectores''
incidentes a $\mathcal{C}$.

\subsection{Forma est\'andar de la matriz de control}

Al igual que en el caso de la matriz generadora la expresi\'on de la 
\textbf{``matriz de control''} depende de la base escogida en el subespacio
incidente a $\mathcal{C}$.

\begin{definicion}[Forma est\'andar de la matriz de control]
\ \\
Sea $M$ una matriz de control de orden $(n-m)\times n$, diremos que est\'a en
\textbf{``forma est\'andar''} si $M=(M_1,M_2)$, donde $M_2$ es la matriz
identidad de orden $(n-m)\times (n-m)$.
\end{definicion}
