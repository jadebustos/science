%
% ECUACIONES DE LOS CODIGOS LINEALES
%

\section{Ecuaciones de los c\'odigos lineales}

Como hemos visto un c\'odigo lineal $\mathcal{C}$ es un subespacio vectorial.
Utilizando la teor\'{\i}a de espacios vectoriales podemos calcular las
ecuaciones de un subespacio vectorial cualquiera. La utilidad de estas
ecuaciones radica en que nos indican las condiciones que ha de cumplir un
vector para pertenecer al c\'odigo.\\

Sea $\mathcal{C}\subset \mathbb{F}^{^n}_q$ un c\'odigo lineal, donde
$\mathbb{F}^{^n}_q$ es un $\mathbb{F}_q$-espacio vectorial tal que
$\dim_{\mathbb{F}_q} \mathbb{F}^{^n}_q=n$. Supongamos que
$\dim_{\mathbb{F}_q} \mathcal{C} = m$, con $m<n$ y que el espacio de palabras
a codificar es $\mathbb{F}^{^m}_q$.
%
\newpage
%
\subsection{Ecuaciones param\'etricas}\label{sec:EcuParametricas}

Una vez conocida la matriz generadora del c\'odigo se pueden calcular las
ecuaciones param\'etricas del c\'odigo. Dichas ecuaciones nos permitir\'an
calcular todas las palabras del c\'odigo.\\

Sea $C$ la matriz generadora del c\'odigo, para calcular las ecuaciones
pa\-ra\-m\'etricas del c\'odigo bastar\'a con aplicar la matriz generadora a un
vector gen\'erico del espacio de palabras a codificar.
\begin{displaymath}
C\cdot \left( \begin{array}{c}
\lambda_1\\
\lambda_2\\
\vdots\\
\lambda_m
\end{array} \right) =
\left( \begin{array}{c}
\theta_1\\
\theta_2\\
\vdots\\
\theta_n
\end{array} \right)
\end{displaymath}
luego las ecuaciones param\'etricas ser\'an:
\begin{displaymath}
\left\{ \begin{array}{ccc}
x_1 &=&\theta_1\\
x_2 &=&\theta_2\\
\cdots&\cdots&\cdots\\
x_n &=&\theta_n
\end{array} \right.
\end{displaymath}
Tendremos que $\theta_i = f_i (\lambda_1,\dots,\lambda_m)$ para $i=1,\dots,n$ y
dando valores a $\{\lambda_j \}_{j=1}^m$ en el cuerpo $\mathbb{F}_q$ tendremos
todos los elementos del c\'odigo.

\subsection{Ecuaciones impl\'{\i}citas}\label{sec:EcuImplicitas}

Para calcular las ecuaciones impl\'{\i}citas de $\mathcal{C}$ calcularemos el
subespacio incidente a $\mathcal{C}$, $\mathcal{C}^{^{\circ}}\subset
(\mathbb{F}^{^n}_q)^{^*}$. El subespacio incidente est\'a formado por las
formas lineales que se anulan sobre $\mathcal{C}$.\\

El n\'umero de ecuaciones impl\'{\i}citas que tendr\'a nuestro c\'odigo
$\mathcal{C}$ ser\'a:
$$dim_{\mathbb{F}_q} \mathbb{F}^{^n}_q - dim_{\mathbb{F}_q} \mathcal{C}=n-m$$
Como $\dim_{\mathbb{F}_q} \mathcal{C} = m$ tendremos que
$\mathcal{C}=<c_1,\ldots,c_m>$. Para calcular las ecuaciones tendremos que
resolver un sistema de ecuaciones que tendr\'a m\'as de una soluci\'on.
%
\newpage
%
Sean $\{x_i^j\}_{j=1}^m$ con $i=1,\dots,n$ la $i$-\'esima coordenada del
vector $j$-\'esimo de la base de $\mathcal{C}$. El sistema que nos da las
ecuaciones ser\'a:
\begin{displaymath}
\left( \begin{array}{cccc}
x_1^1 & \ldots & \ldots & x_n^1 \\
x_1^2 & \ddots & & x_n^2 \\
\vdots & &\ddots & \vdots \\
x_1^m &\ldots &\ldots & x_n^m
\end{array} \right) \cdot
\left( \begin{array}{c}
x_1 \\
\vdots \\
\vdots \\
x_n
\end{array} \right) =
\left( \begin{array}{c}
0 \\
\vdots \\
\vdots \\
0
\end{array} \right)
\end{displaymath}

Cualquier $x=(x_1,\ldots,x_n)$ que verifique este sistema de ecuaciones
ser\'a un elemento de $\mathcal{C}$, o lo que es lo mismo cualquier punto que no
verifique el sistema de ecuaciones anterior NO pertenecer\'a a $\mathcal{C}$.
