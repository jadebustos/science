%
% LENGUJES Y SIMBOLOS
%

\section{Lenguajes y simbolos}

Para emitir un mensaje hay que hacerlo utilizando un determinado lenguaje,
com\'un al transmisor y al receptor. Todo lenguaje estar\'a formado por un
conjunto de simbolos, los cuales formar\'an las palabras del lenguaje.\\

Los simbolos tambi\'en recibiran el nombre de ``bits''.

\begin{ejemplo}
\ \\ \\
Cuando queremos comunicarnos con una persona, transmitir un mensaje, hablamos
con esa persona en un idioma conocido por ambas partes, lenguaje, y los
simbolos que utilizamos son las letras del abecedario, n\'umeros, \dots y el
medio o canal utilizado puede ser el habla o la escritura, por ejemplo.
\end{ejemplo} 

\begin{ejemplo}[I.S.B.N.]
\ \\ \\
Sistema internacional de libros. \\

Este sistema se utiliza para catalogar y numerar los libros. Est\'a formado por
palabras de $10$ cifras y los simbolos que utiliza este lenguaje son:

\begin{displaymath}
Smb_{I.S.B.N.} = \{0, 1, 2, 3, 4, 5, 6, 7, 8, 9 \}
\end{displaymath}

Una palabra de este lenguaje ser\'a:

\begin{displaymath}
a=(a_1,a_2,a_3,a_4,a_5,a_6,a_7,a_8,a_9,a_{10})\quad a_i \in Smb_{I.S.B.N.}
\quad \forall \quad i=1,\ldots,10
\end{displaymath}
\end{ejemplo}
%
\newpage
%
\section{Objetivos de la teor\'{\i}a de c\'odigos}

Los objetivos de la teor\'{\i}a de c\'odigos son los siguientes:
\begin{itemize}
\item Construir c\'odigos para la transmisi\'on de informaci\'on.
\item Dichos c\'odigos han de detectar cuando ha ocurrido un error en la
transmisi\'on de la informaci\'on.
\item Dichos c\'odigos deben corregir el mayor n\'umero posible de errores.
\end{itemize}

Con el fin de detectar y corregir los posibles errores ocurridos durante la
transmisi\'on del mensaje original introduciremos informaci\'on redundante
acerca del mensaje original, con el fin de cotejar esta informaci\'on con el 
mensaje original y poder comprobar de esta forma si se produjo alg\'un error en
la transmisi\'on del mensaje.\\

No podemos abusar de la informaci\'on redundante que introducimos. Al
introducir informaci\'on redundante en el mensaje a transmitir obtenemos un
mensaje m\'as largo de transmitir, por lo tanto ser\'a m\'as costoso y lento
el poder transmitirlo. Por lo tanto si introducimos demasiada informaci\'on
redundante tenemos la ventaja de poder detectar y corregir bastantes errores,
pero tambi\'en tenemos el inconveniente de un mayor costo al transmitir la
informaci\'on.\\

Otro objetivo que buscaremos ser\'a que cuando se origine un error en la
transmisi\'on de una palabra se produzca una palabra NO perteneciente al
c\'odigo. Por ejemplo si transmitimos una palabra ``A'', ocurre un error y,
fruto de ese error la palabra que llega al receptor es ``B'', la situaci\'on
ideal ser\'{\i}a que ``B'' NO perteneciera a nuestro c\'odigo. La explicaci\'on
de esto es que si nos llega una palabra que NO es del c\'odigo que estamos
utilizando entonces se ha producido un error, mientras que s\'{\i} la palabra
es del c\'odigo la daremos como buena y no nos percataremos de que se ha 
cometido un error en la transmisi\'on.
%
\newpage
%
\subsection{Ejemplos de la utilizaci\'on de c\'odigos}

Los c\'odigos se usan de una manera continua en muchos campos, como por ejemplo:

\begin{itemize}
\item En las comunicaciones:
\begin{itemize}
\item Radio.
\item Televisi\'on.
\item Sat\'elites.
\item Ordenadores.
\end{itemize}
\item Sistemas de grabaci\'on de datos:
\begin{itemize}
\item Voz.
\item Video.
\item CD-Rom.
\end{itemize}
\item Comunicaci\'on escrita.
\item Comunicaci\'on oral(sonora).
\end{itemize}

\section{Notaciones}

Trabajaremos siempre, salvo que no se diga lo contrario, sobre el conjunto
$\mathbb{F}_q$. Este conjunto es un conjunto finito de $q$ elementos y
supondremos siempre que $q=p^n$, donde $p$ es un n\'umero primo y $n\in
\mathbb{N}$.\\

Nuestros c\'odigos estar\'an formados por palabras de $n$ ``bits'' o simbolos,
donde cada palabra pertenece a $\mathbb{F}_q$.\\

El conjunto de todas las palabras de $n$ simbolos donde cada simbolo es un
elemento de $\mathbb{F}_q$ lo denotaremos como $\mathbb{F}^{^n}_q$.\\

Luego tendremos que:
\begin{displaymath}
\mathbb{F}^{^n}_q = \{(a_1,a_2,\dots,a_n)\ |\ a_i\in \mathbb{F}_q\quad \forall \
i=1,\dots,n\}
\end{displaymath}

Como $\mathbb{F}_q$ tiene un n\'umero finito de elementos, $q$, dado un n\'umero
entero $n<\infty$ podemos calcular la cantidad de elementos que tiene
$\mathbb{F}^{^n}_q$ al cual denotaremos como $|\mathbb{F}^{^n}_q|$.\\

El n\'umero de elementos de $\mathbb{F}^{^n}_q$, al que llamaremos orden de
$\mathbb{F}^{^n}_q$, ser\'a el n\'umero de combinaciones, distintas, que podemos
hacer de $n$ elementos de $\mathbb{F}^{^n}_q$, es decir:
\begin{displaymath}
|\mathbb{F}^{^n}_q| = q^n\ donde\ n\in \mathbb{N}
\end{displaymath}
%
Los elementos de $\mathbb{F}^{^n}_q$ los denotaremos de dos formas:
\begin{itemize}
\item $(a_1,\dots,a_n)$ fundamentalmente cuando consideremos dicho elemento como
elemento de una estructura algebraica.
\item $a_1\dots a_n$ fundamentalmente cuando consideremos dicho elemento como
elemento de un c\'odigo.
\end{itemize}
pero ambas notaciones sirven para hacer referencia al mismo elemento.

\section{Los ejemplos fundamentales}

Todo lo expuesto en estos apuntes ser\'a aplicado a los siguientes ejemplos:
\begin{itemize}
\item Bit de control de paridad.
\item Triple repetici\'on.
\item Triple control.
\end{itemize}
De igual modo los c\'odigos que utilizaremos estar\'an formados por palabras
compuestas de $0$ y $1$, es decir los simbolos ser\'an $\{0,1\}$.\\

En la pr\'actica la informaci\'on redundante que introduciremos en los c\'odigos
se introduce al principio de la palabra, pero nosotros la introduciremos
al final\footnote{Por comodidad}.

\subsection{C\'odigo binario del ``bit de control de paridad''}

Este c\'odigo estar\'a formado por palabras de $8$ simbolos. 
\begin{displaymath}
\mathcal{C}=\{(a_1,a_2,a_3,a_4,a_5,a_6,a_7,a_8)\ |\ a_i\in \mathbb{F}_2\quad
\forall \ i=1,\dots,8\}
\end{displaymath}
Obviamente se tiene que $\mathcal{C} \subset \mathbb{F}^{^8}_2$, ya que s\'{\i}
se diera la igualdad unicamente estar\'{\i}amos codificando la informaci\'on sin
introducir informaci\'on redundante\footnote{Lo cual no ser\'{\i}a efectivo pues
no podr\'{\i}amos conocer cuando se ha cometido un error en la transmisi\'on.}.\\

Ahora bien, como hemos comentado antes introduciremos en el c\'odigo
informaci\'on redundante. Esto quiere decir que en los ocho digitos que componen
cada palabra del c\'odigo que vamos a utilizar tendremos unos digitos que no
aportan nada al mensaje que queremos transmitir, su \'unica utilidad ser\'a la
de darnos informaci\'on sobre la palabra que hemos transmitido y poder comprobar
si hubo alg\'un error en la transmisi\'on de dicha palabra.\\

En este c\'odigo se utiliza un \'unico ``bit'' para transmitir esa informaci\'on
redundate. Con lo cual \'unicamente utilizaremos siete ``bits'' para codificar
informaci\'on y otro ``bit'' adicional para dar informaci\'on sobre la palabra
transmitida. Seg\'un esto el n\'umero de palabras de nuestro c\'odigo ser\'a de
\mbox{$2^7=128$}, ya que son siete los ``bits'' que se emplean para codificar
informaci\'on:
\begin{displaymath}
\mathcal{C}=\{(a_1,a_2,a_3,a_4,a_5,a_6,a_7,c_1)\ |\ a_i,c_1\in\mathbb{F}_2\quad \forall \
i=1,\dots,7\}
\end{displaymath}
El ``bit'' $c_1$ de las palabras de este c\'odigo recibe el nombre de ``bit de
control'' y aporta informaci\'on sobre los siete ``bits'' anteriores.\\

Una vez conocidos $a_1a_2a_3a_4a_5a_6a_7$ la forma de determinar $c_1$ es la 
siguiente:
\begin{quote}
$c_1$ ser\'a tal que en la palabra haya un n\'umero par de $1$, es decir, si
en los ``bits'' anteriores hay un n\'umero impar de $1$ $c_1=1$, en caso
contrario $c_1=0$.
\end{quote}
\begin{table}[!h]
\begin{center}
\begin{tabular}{|c|c|c|c|}
\hline
Palabra & Bits de informaci\'on & Bit de control & ?`Palabra correcta?\\
\hline
$01101111$ & $0110111$ & $1$ & S\'{\i} \\
\hline
$11101000$ & $1110100$ & $0$ & S\'{\i} \\
\hline
$00011000$ & $0001100$ & $0$ & S\'{\i} \\
\hline
$11100011$ & $1110001$ & $1$ & No \\
\hline 
$00010000$ & $0001000$ & $0$ & No \\
\hline
$01010100$ & $0101010$ & $0$ & No \\
\hline
\end{tabular}
\end{center}
\caption{Ejemplos del c\'odigo del bit de control de paridad.}
\end{table}

Podemos ver que la informaci\'on redundante, ``bit'' de control, lo hemos
situado al final de la palabra, es decir, es el ``bit'' menos significativo. En
la pr\'actica esto no ocurre ya que los ``bits'' de control se situan al 
principio de la palabra, son los ``bits'' m\'as significativos.

\subsubsection{Ventajas del c\'odigo}

Este c\'odigo tiene las siguientes ventajas:
\begin{itemize}
\item A\~nade poca informaci\'on redundate, unicamente $\frac{1}{8}$ de la
informaci\'on transmitida es redundante.
\item Detecta un error. Si en la transmisi\'on se transmite un ``bit'' de
forma erronea\footnote{Un $1$ pasa a ser un $0$ y viceversa.} entonces el
``bit'' de control no concuerda con los anteriores, se produjo un error.
\end{itemize}
En realidad si se produce un n\'umero impar de errores el c\'odigo lo detecta,
pero no hay forma de saber cuantos errores se han cometido, por lo cual se 
supone que se ha cometido un error.

\subsubsection{Inconvenientes del c\'odigo}

Este c\'odigo tiene los siguientes inconvenientes:
\begin{itemize}
\item Aunque se detecte un error no hay forma de saber en cual de los
siete ``bits'' de informaci\'on se ha producido, luego no corrige errores.
\item Si se produce un n\'umero par de errores la palabra resultante es
otra palabra del c\'odigo, con lo cual no se detecta el error ya que se da
por buena. 
\end{itemize}

\subsection{C\'odigo binario de ``triple repetici\'on''}

Este c\'odigo consiste en repetir cada ``bit'' de informaci\'on tres veces.
\begin{table}[!h]
\begin{center}
\begin{tabular}{|c|c|}
\hline
Bit de informaci\'on & Palabra del c\'odigo \\
\hline
$0$ & $000$ \\
\hline
$1$ & $111$ \\
\hline
\end{tabular}
\end{center}
\caption{Palabras del c\'odigo de triple repetici\'on.}
\end{table}
\begin{displaymath}
\mathcal{C} = \{000,111\} \subset \mathbb{F}^{^3}_2
\end{displaymath}

\subsubsection{Ventajas del c\'odigo}

Este c\'odigo tiene las siguientes ventajas:
\begin{itemize}
\item Detecta hasta dos errores.
\item Corrige un error.
\end{itemize}
Estamos suponiendo que en cada ``bit'' unicamente se produce un error,
observar que si en un ``bit'' se produce un n\'umero par de errores el
``bit'' queda inalterado.\\

Para poder corregir, con este c\'odigo, necesitamos una hip\'otesis adicional
sobre la probabilidad de errores en el canal. Es decir necesitamos conocer que
error es m\'as probable que se cometa en la transmisi\'on en el canal,
dedicandonos a corregir el error que, con m\'as probabilidad, se cometa.\\

Supongamos que se realiza una transmisi\'on utilizando este c\'odigo y se
recibe $010$. Sabiendo que en el canal es m\'as probable que se cometan dos
errores que uno entonces tendremos que la palabra que, con m\'as probabilidad,
se transmiti\'o es $111$. Esto no significa que la palabra transmitida fuera
$111$, ya que existe una probabilidad $1-P$, donde $P$ es la probabilidad de
que se cometieran dos errores en la transmisi\'on, de que se produzca un error
en la transmisi\'on. La palabra transmitida pudiera haber sido $000$ y haberse
cometido ``un solo'' error. Por eso en este c\'odigo lo que se hace es corregir
el error que tiene mayor probabilidad de cometerse, existiendo siempre un margen
de error en la correcci\'on. Dicho margen viene dado por la probabilidad de 
cometerse el error que no corregimos.\\

Si se produjeran tres errores, uno en cada ``bit'', obtendr\'{\i}amos la
otra palabra del c\'odigo, es decir no detectariamos el error.

\subsubsection{Inconvenientes del c\'odigo}

Este c\'odigo tiene los siguientes inconvenientes:
\begin{itemize}
\item Mucha informaci\'on redundante. Este c\'odigo es lento transmitiendo,
raz\'on por la cual no es efectivo ni pr\'actico.
\end{itemize}
%
\newpage
%
\subsection{C\'odigo binario de ``triple control''}

De este c\'odigo se podr\'{\i}a decir que es una mezcla de los dos c\'odigos
anteriores. Consiste en dividir la informaci\'on a transmitir en bloques de
tres ``bits'' y a\~nadirle otros tres ``bits'' de control. Seg\'un lo dicho
nuestro c\'odigo ser\'a $\mathcal{C}\subset \mathbb{F}^{^6}_2$. Como la
informaci\'on se transmite en bloques de tres ``bits''\footnote{Los otros tres
``bits'' son informaci\'on redundante.} el c\'odigo tendr\'a $2^3=8$ palabras.\\

El c\'odigo ser\'a:
\begin{displaymath}
\mathcal{C} = \{(a_1,a_2,a_3,c_1,c_2,c_3)\ donde\ a_i,c_i\in \mathbb{F}_2 \}
\end{displaymath}
Los ``bits'' de control $c_1$ ,$c_2$ y $c_3$ est\'an determinados de la
siguiente manera:
\begin{itemize}
\item $c_1$ es tal que el n\'umero de $1$ en $a_1a_2c_1$ sea par.
\item $c_2$ es tal que el n\'umero de $1$ en $a_1a_3c_2$ sea par.
\item $c_3$ es tal que el n\'umero de $1$ en $a_2a_3c_3$ sea par.
\end{itemize}
\begin{table}[!h]
\begin{center}
\begin{tabular}{|c|c|}
\hline
Palabra & Bits de control \\
\hline
$000000$ & $000$ \\
\hline
$001011$ & $011$ \\
\hline 
$010101$ & $101$ \\
\hline
$100110$ & $110$ \\
\hline
$011110$ & $110$ \\
\hline
$101101$ & $101$ \\
\hline
$110011$ & $011$ \\
\hline 
$111000$ & $000$ \\
\hline
\end{tabular}
\end{center}
\caption{Palabras del c\'odigo de triple control.}
\end{table}

\subsubsection{Ventajas del c\'odigo}

Este c\'odigo tiene las siguientes ventajas:
\begin{itemize}
\item Detecta un error y lo corrige.
\end{itemize}

\subsubsection{Inconvenientes del c\'odigo}

Este c\'odigo tiene los siguientes inconvenientes:
\begin{itemize}
\item La mitad de la informaci\'on que se transmite es redundante.
\end{itemize}

\section{?`Como modelizar la situaci\'on?}

Para poder elaborar una teor\'{\i}a que nos permita construir ``buenos
c\'odigos''\footnote{Que detecten y corrijan errores, es decir que sean
efectivos} con los que podamos transmitir informaci\'on necesitaremos
introducir una serie de hip\'otesis sobre el modelo.

\subsection{Primera hip\'otesis: ``Canal discreto''}\label{sec:CanalDiscreto}

Supondremos que en el canal por el que se transmite la informaci\'on se
transmite ``a golpes'' discretos. Es decir con ``paquetes'' a los que 
llamaremos \emph{BLOQUES} o \emph{PALABRAS}.\\

Resumiendo la informaci\'on se divide en \emph{PALABRAS} que se transmiten de
forma separada, es decir no se transmiten todas juntas. Las \emph{PALABRAS} de
nuestro mensaje estar\'an formadas por \emph{SIMBOLOS}, \emph{BITS} o
\emph{LETRAS}.

\subsection{Segunda hip\'otesis: ``Errores aleatorios''}\label{sec:ErrorAleato}

El tipo de canal por el que se efectua la transmisi\'on introduce
``\emph{errores aleatorios}''.\\

En la realidad los errores no son asi, son ``\emph{a rafagas}'', es decir se
deja de enterder desde que empieza el ``\emph{ruido}'' hasta que termina.\\

Nuestro modelo supondr\'a que, dadas dos palabras $A$ y $B$ \textbf{existe una
probabilidad (fija)} de que se transmita $A$ y se reciba $B$. Esta probabilidad
la denotaremos como $P_{A,B}$.
\begin{definicion}[Canal Sim\'etrico]
\ \\
Diremos que un canal es \textbf{``sim\'etrico''} cuando $P_{A,B}=P_{B,A}$, es
decir, la probabilidad de que $A$ pase a ser $B$ es la misma probabilidad de
que $B$ pase a ser $A$.
\end{definicion}
%
\newpage
%
Siempre supondremos que nuestro canal ser\'a:
\begin{itemize}
\item De errores aleatorios.
\item Sim\'etrico.
\item $P_{A,B}$ no depende ni de $A$ ni de $B$, por lo tanto utilizaremos la
siguiente notaci\'on $P_{A,B}=P$.
\end{itemize}

\subsubsection{Caso en el que $P<\frac{1}{2}$}

Siempre supondremos esta hip\'otesis: $P<\frac{1}{2}$.

\subsubsection{Caso en el que $P>\frac{1}{2}$}

En este caso cogemos la palabra que nos ha llegado, la damos la vuelta y ya
tenemos que $P<\frac{1}{2}$.

\subsubsection{Caso en el que $P=\frac{1}{2}$}

En este caso tenemos que si transmitimos un $0$ puede llegar, indistintamente,
un $0$ o un $1$, con lo cual el canal considerado no es nada fiable y no es
util desde un punto de vista pr\'actico. Luego no consideraremos nunca
este caso.

\section{Resumen}

Para transmitir informaci\'on lo que haremos ser\'a:
\begin{itemize}
\item Dividir la informaci\'on en bloques o palabras.
\item Codificar cada palabra del mensaje utilizando un c\'odigo. Este paso
supone a\~nadir a cada palabra informaci\'on redundante para poder controlar
cuando se ha producido un error en la transmisi\'on y, en algunos casos,
corregir dicho error.
\item Enviar el mensaje.
\end{itemize}
%
\newpage
%
Una vez recibido el mensaje para poder conocer el mensaje original:
\begin{itemize}
\item Comprobar cada palabra del mensaje recibido, utilizando la informaci\'on
redundante\footnote{``Bits'' de control.}, para comprobar que no hubo ning\'un
error en la transmisi\'on del mensaje. En caso de haberlo se tienen dos
opciones:
\begin{itemize}
\item Intentar corregir el error, en el caso que el c\'odigo lo permita.
Corregir el error debe enterderse como calcular la palabra del c\'odigo que,
con mayor probabilidad, fue transmitida originalmente.
\item Solicitar, de nuevo, la informaci\'on erronea.
\end{itemize}
\item Eliminar la informaci\'on redundante introducida\footnote{Decodificar el
mensaje.}.
\end{itemize}

\subsection{Alfabeto}

\begin{definicion}[Alfabeto]
\ \\
Un \textbf{``alfabeto''} ser\'a el conjunto de si\'{\i}mbolos utilizaremos
para codificar el mensaje.
\end{definicion}
En todo alfabeto existe un s\'{\i}mbolo distinguido, que es el ``\emph{espacio
en blanco}''. Este s\'{\i}mbolo nos permite distiguir cuando termina una palabra
y comienza la siguiente.\\

Por ejemplo si utilizamos como alfabeto $\mathbb{F}_2$ los s\'{\i}mbolos 
serian $\{0,1\}$.

\subsection{Palabras}

\begin{definicion}[Palabra]
\ \\
Entenderemos por \textbf{``palabra''} a una sucesi\'on ordenada de simbolos de
un alfabeto.
\end{definicion}
En los casos que vamos a considerar las palabras ser\'an elementos de
$\mathbb{F}^{^n}_q$, puesto que vamos a utilizar palabras de longitud fija.\\ \\
%
El n\'umero de elementos que posee $\mathbb{F}^{^n}_q$ es $|\mathbb{F}_q|^{^n}$.


\subsection{C\'odigos}

\begin{definicion}[C\'odigos]
\ \\
Llamaremos \textbf{``c\'odigo''} al conjunto de palabras que utilizaremos para
codificar informaci\'on.
\end{definicion}
En los casos que vamos a considerar los c\'odigos ser\'an subconjuntos de
$\mathbb{F}^{^n}_q$, es decir $\mathcal{C}\subset \mathbb{F}^{^n}_q$.

\subsection{Canales}

El canal es el medio por el cual se transmite la informaci\'on y sobre este
canal supondremos siempre que:
\begin{itemize}
\item Cumple la hip\'otesis de ``\emph{canal discreto}''\footnote{Apartado
$\ref{sec:CanalDiscreto}$ en la p\'agina $\pageref{sec:CanalDiscreto}$.}. 
\item Cumple la hip\'otesis de ''\emph{errores aleatorios}''\footnote{Apartado
$\ref{sec:ErrorAleato}$ en la p\'agina $\pageref{sec:ErrorAleato}$.}.
\item El canal es ``\emph{sim\'etrico}''.
\end{itemize}

%
% EJERCICIOS
%

%
% EJERCICIOS
%

\section{Ejercicios}
%
% SINDROMES
%
\begin{ejercicio}
\ \\
Programar la correcci\'on de errores de un c\'odigo lineal utilizando
sindromes.
\end{ejercicio}
\underline{\textbf{Soluci\'on}}:\\
Se ha programado el funcionamiento de un procesador de error para el 
c\'odigo de triple control utilizando la tabla de sindromes.\\

El programa est\'a escrito en lenguaje $C$ y tanto el c\'odigo como los
binarios para \emph{MS-DOS} estan el el disco adjunto.\\ \\
%
El funcionamiento del programa es el siguiente:
\begin{itemize}
\item El programa pide, por teclado, una palabra de seis digitos bin\'arios. 
Para prevenir el funcionamiento an\'omalo del programa se ha recurrido a 
admitir una cadena, de hasta $80$ caracteres, como entrada. El programa
seleccionar\'a los seis primeros caracteres num\'ericos y los convertir\'a a
$\mathbb{F}_2$, considerando a estos como la palabra recibida.
\item El programa conoce, a priori, las palabras del c\'odigo, su matriz de
control y los patrones de error.
\item Seguidamente calcular\'a el sindrome de la palabra recibida, as\'{\i}
como los sindromes de todas palabras de peso uno, errores, y el de una palabra
de peso dos, para detectar cuando ocurre un error de peso dos.
\item S\'{\i} el error es peso menor o igual que uno el programa da la palabra
transmitida originalmente.
\item S\'{\i} el error es de peso dos el programa indica que ocurrio un error
de peso dos, y no nos da la palabra que se transmiti\'o originalmente. Para
poder calcular esta palabra ser\'{\i}a necesario tener alguna hip\'otesis
adicional sobre que error de peso dos es el que m\'as se da en el canal de
transmisi\'on.
\end{itemize}

\begin{flushright}
$\blacksquare$
\end{flushright}
%
% TABLA ESTANDAR
%
\begin{ejercicio}
\ \\
Programar la la correcci\'on de errores de un c\'odigo lineal utilizando una
tabla est\'andar.
\end{ejercicio}
\underline{\textbf{Soluci\'on}}:\\
Se ha programado el funcionamiento de un procesador de error para el 
c\'odigo de triple control utilizando una tabla est\'andar.\\

El programa est\'a escrito en lenguaje $C$ y tanto el c\'odigo como los
binarios para \emph{MS-DOS} estan el el disco adjunto.\\ \\
%
El funcionamiento del programa es el siguiente:
\begin{itemize}
\item El programa pide, por teclado, una palabra de seis digitos bin\'arios. 
Para prevenir el funcionamiento an\'omalo del programa se ha recurrido a 
admitir una cadena, de hasta $80$ caracteres, como entrada. El programa
seleccionar\'a los seis primeros caracteres num\'ericos y los convertir\'a a
$\mathbb{F}_2$, considerando a estos como la palabra recibida.
\item El programa conoce, a priori, las palabras del c\'odigo y los patrones
de error.
\item Seguidamente calcular\'a la tabla est\'andar a partir de las palabras
del c\'odigo y los patrones de error.
\item S\'{\i} el error es peso menor o igual que uno el programa da la palabra
transmitida originalmente.
\item S\'{\i} el error es de peso dos el programa indica que ocurrio un error
de peso dos, y no nos da la palabra que se transmiti\'o originalmente. Para
poder calcular esta palabra ser\'{\i}a necesario tener alguna hip\'otesis
adicional sobre que error de peso dos es el que m\'as se da en el canal de
transmisi\'on.
\end{itemize}

\begin{flushright}
$\blacksquare$
\end{flushright}

