%
% CODIGOS R-PERFECTOS
%

\chapter{C\'odigos $r$-perfectos}

\begin{definicion}[C\'odigos $r$-perfectos]
\ \\
Sea $\mathcal{C}[n,m]$ un c\'odigo. Se dice que es ``\textbf{$r$-perfecto}'' si
para cada palabra $v\in\mathbb{F}^{^n}_q$ existe una \'unica palabra 
$u\in \mathcal{C}[n,m]$ tal que $d(u,v)\leq r$.
\end{definicion}
\begin{observacion}
\ \\
\begin{itemize}
\item Como detecta y corrige errores de peso menor o igual que $r$ entonces su
distancia m\'{\i}nima verifica que $d_{min} \geq 2\cdot r + 1$.
\end{itemize}
\end{observacion}

\begin{definicion}[Bola o disco, abiertos, de centro $u$ y radio $r$]
\ \\
Llamaremos bola o disco, abiertos, de centro $u$ y radio $r$ y los denotaremos
por $D(u,r)$ al siguiente conjunto:
\begin{displaymath}
D(u,r) =\{\ v\in \mathbb{F}^{^n}_q\quad verificando \ d(u,v)< r\ \}
\end{displaymath}
\end{definicion}
\begin{definicion}[Bola o disco, cerrados, de centro $u$ y radio $r$]
\ \\
Llamaremos bola o disco, cerrados, de centro $u$ y radio $r$ y los denotaremos
por $\overline{D}(u,r)$ al siguiente conjunto:
\begin{displaymath}
\overline{D}(u,r)=\{\ v\in \mathbb{F}^{^n}_q\quad verificando\ d(u,v)\leq r\ \}
\end{displaymath}
\end{definicion}
En las dos definiciones anteriores $u\in \mathbb{F}^{^n}_q$.
%
\newpage
%

\section{C\'odigos binarios $r$-perfectos}

Los siguientes resultados pueden ser generalizados para c\'odigos no binarios.

%
% NUMERO DE PALABRAS EN UN DISCO CERRADO
%

% 
% NUMERO DE PALABRAS EN UN DISCO CERRADO
%

\subsection{N\'umero de palabras en un disco cerrado}

\begin{lema}[N\'umero de palabras en un disco cerrado]\label{lem:Cantidad}
\ \\
Dado un disco cerrado $\overline{D}(u,r)$ donde $u\in\mathbb{F}^{^n}_2$ y
$r\in \mathbb{Z}$ se tiene que el n\'umero de palabras que hay en dicho
disco, $|\overline{D}(u,r)\ |$, es:
\begin{displaymath}
|\overline{D}(u,r)\ | = {n \choose 0}+{n \choose 1}+\cdots+{n\choose r}
\end{displaymath}
\end{lema}
\underline{\textbf{Desmostraci\'on}}:\\
El n\'umero de palabras en $\overline{D}(u,r)$ es el n\'umero de palabras que
podemos construir variando hasta $r$ bits.\\

Dada $u$ hay ${n \choose i}$ palabras $v$ tales que $d(u,v) = i$, luego por
la definici\'on de disco cerrado y distancia tendremos que:
\begin{displaymath}
|\overline{D}(u,r)\ | = {n\choose 0}+{n\choose 1}+\cdots+{n\choose r}
\end{displaymath}
\begin{flushright}
$\blacksquare$
\end{flushright}
\begin{observacion}\ \\
\begin{itemize}
\item De este lema se deduce que el n\'umero de palabras que hay en un disco
cerrado\footnote{Con la distancia de Hamming.} \'unicamente depende de su
radio $r$ y nunca de su centro. Luego utilaremos la siguiente notaci\'on:
\begin{displaymath}
|\overline{D}(u,r)\ | = |\overline{D}_r\ |
\end{displaymath}
\end{itemize}
\end{observacion}


%
\newpage
%

%
% CANTIDAD DE CODIGOS r-PERFECTOS
%

%
% CANTIDAD DE CODIGOS r-PERFETOS
%

\subsection{?`Cuantos c\'odigos $r$-perfectos existen?}

\begin{lema}\label{lem:Disjuntos}
\ \\
Un c\'odigo $r$-perfecto, $\mathcal{C}[n,m]$ es uni\'on disjunta de discos
cerrados centrados en palabras del c\'odigo y radio $r$. Para c\'odigos
bin\'arios.
\end{lema}
\underline{\textbf{Demostraci\'on}}:\\
Como $\mathcal{C}[n,m]$ es un c\'odigo $r$-perfecto se tiene que:
\begin{equation}\label{eq:Recubrimiento}
\mathbb{F}^{n}_2 = \bigcup_{u\in \mathcal{C}[n,m]} \overline{D}(u,r)
\end{equation}
donde las $\overline{D}(u,r)$ para $u\in \mathcal{C}[n,m]$ son disjuntas entre
s\'{\i}, es decir:
\begin{displaymath}
\overline{D}(u_i,r)\bigcap \overline{D}(u_j,r) = \emptyset\quad para\ i\neq j
\end{displaymath}
La ecuaci\'on $(\ref{eq:Recubrimiento})$ se deduce de la definici\'on de 
c\'odigo $r$-perfecto:
\begin{itemize}
\item Que $\mathbb{F}^{^n}_2$ es la uni\'on de discos cerrados se deduce del
hecho de que al ser $\mathcal{C}[n,m]$ un c\'odigo $r$-perfecto dada
$v\in\mathbb{F}^{^n}_2$, una palabra cualquiera, siempre existe una palabra
$u\in\mathcal{C}[n,m]$ tal que $d(u,v)\leq r$.
\item Y el hecho de que los discos sean disjuntos se deduce del hecho de que
las palabras $u\in \mathcal{C}[n,m]$ anteriores son unicas\footnote{Por
definici\'on de c\'odigo $r$-perfecto.}.
\end{itemize}
\begin{flushright}
$\blacksquare$
\end{flushright}

En $\mathbb{F}^{^n}_2$ tenemos $2^n$ palabras y en $\mathcal{C}[n,m]$ tenemos
$2^m$ palabras. De esto y de los lemas $\ref{lem:Cantidad}$ y
$\ref{lem:Disjuntos}$ se deduce que un c\'odigo $r$-perfecto verifica:
\begin{displaymath}
2^n=2^m\cdot\left({n\choose 0}+{n\choose 1}+\cdots+{n\choose r} \right)
\end{displaymath}
Luego tendremos tantos c\'odigos $r$-perfectos como $n$ y $m$ verifiquen la
condici\'on anterior.
%
\newpage
%
\begin{proposicion}
\ \\
Si existe un c\'odigo $\mathcal{C}[n,m]$ $r$-perfecto entonces no existe
ning\'un c\'odigo $\mathcal{C}[n,m']$ con $m'>m$ y que tenga distancia 
m\'{\i}nima mayor que $2\cdot r$.
\end{proposicion}
\underline{\textbf{Demostraci\'on}}:\\
Sea $\mathcal{C}[n,m]$ un c\'odigo $r$-perfecto. Entonces por los lemas
$\ref{lem:Cantidad}$ y $\ref{lem:Disjuntos}$ tendremos que:
\begin{equation}\label{eq:Primer}
2^n=2^m\cdot |\overline{D}_r\ |
\end{equation}
Sea $\mathcal{C}'[n,m']$ con $m'>m$ tal que $d_{min}>2\cdot r$ es decir,
$d_{min}\geq2\cdot r+1$ entonces el c\'odigo corrige errores de peso menor o
igual que $r$, luego $\mathcal{C}'[n,m]$ es $r$-perfecto. Entonces:
\begin{displaymath}
2^n=2^{m'}\cdot |\overline{D}_r\ |
\end{displaymath}
Teniendo en cuenta esta \'ultima ecuaci\'on y $(\ref{eq:Primer})$ llegamos a
la conclusi\'on de que $m'=m$.
\begin{flushright}
$\blacksquare$
\end{flushright}
Luego para una distancia m\'{\i}nima fijada, $2\cdot r+1$, los c\'odigos
$r$-perfectos tienen los bits de informaci\'on, $m$, m\'aximos. Para c\'odigos
con igual longitud de palabra.


%
% CASOS POSIBLES
%

%
% CASOS POSIBLES
%

\subsection{Casos posibles de c\'odigos binarios perfectos}

\subsubsection{Los c\'odigos de repetici\'on}

Estos c\'odigos tienen longitud $n$ y repiten cada bit de informaci\'on $n-1$
veces. Estos c\'odigos son de la forma $\mathcal{C}[n,1]$, donde $n=2\cdot r+1$,
con $r$ arbitrario.
\begin{eqnarray*} 
2^n&=&2^{2\cdot r+1}= (1+1)^{2\cdot r+1}=\sum_{i=0}^{2\cdot r+1}
{2\cdot r+1 \choose i}1^{2\cdot r+1-i}\cdot 1^i=\\
&=&\sum_{i=0}^{2\cdot r+1}{2\cdot r+1 \choose i} = {2\cdot r+1\choose 0}+
{2\cdot r+1\choose 1}+\cdots+{2\cdot r+1\choose 2\cdot r+1}=\\
&=&2^1\cdot \sum_{i=0}^{r}{2\cdot r+1\choose i}
\end{eqnarray*}
Hemos utilizado el binomio de \emph{Newton} y la siguiente propiedad de los
n\'umeros combinatorios:
\begin{displaymath}
{n\choose r} = \frac{n!}{r!\cdot (n-r)!} = {n \choose n-r}
\end{displaymath}
con la cual se tiene que:
\begin{displaymath}
{2\cdot r +1 \choose i}={2\cdot r+1 \choose 2\cdot r +1-i}
\end{displaymath}
Estos c\'odigos son $r$-perfectos.

\subsubsection{Los c\'odigos de Hamming}

Los c\'odigos de \emph{Hamming} verifican:
\begin{displaymath}
2^{2^k-1}=2^{2^k-k-1}\cdot \sum_{i=0}^1{2^k-1 \choose i}
\end{displaymath}
Estos c\'odigos son $1$-perfectos.

\subsubsection{El c\'odigo de Golay $G_{23}$}

Este c\'odigo verifica:
\begin{displaymath}
2^{23}=2^{12}\cdot \sum_{i=0}^3{23\choose i}
\end{displaymath}
Este c\'odigo es $3$-perfecto.

