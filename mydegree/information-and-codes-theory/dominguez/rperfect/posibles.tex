%
% CASOS POSIBLES
%

\subsection{Casos posibles de c\'odigos binarios perfectos}

\subsubsection{Los c\'odigos de repetici\'on}

Estos c\'odigos tienen longitud $n$ y repiten cada bit de informaci\'on $n-1$
veces. Estos c\'odigos son de la forma $\mathcal{C}[n,1]$, donde $n=2\cdot r+1$,
con $r$ arbitrario.
\begin{eqnarray*} 
2^n&=&2^{2\cdot r+1}= (1+1)^{2\cdot r+1}=\sum_{i=0}^{2\cdot r+1}
{2\cdot r+1 \choose i}1^{2\cdot r+1-i}\cdot 1^i=\\
&=&\sum_{i=0}^{2\cdot r+1}{2\cdot r+1 \choose i} = {2\cdot r+1\choose 0}+
{2\cdot r+1\choose 1}+\cdots+{2\cdot r+1\choose 2\cdot r+1}=\\
&=&2^1\cdot \sum_{i=0}^{r}{2\cdot r+1\choose i}
\end{eqnarray*}
Hemos utilizado el binomio de \emph{Newton} y la siguiente propiedad de los
n\'umeros combinatorios:
\begin{displaymath}
{n\choose r} = \frac{n!}{r!\cdot (n-r)!} = {n \choose n-r}
\end{displaymath}
con la cual se tiene que:
\begin{displaymath}
{2\cdot r +1 \choose i}={2\cdot r+1 \choose 2\cdot r +1-i}
\end{displaymath}
Estos c\'odigos son $r$-perfectos.

\subsubsection{Los c\'odigos de Hamming}

Los c\'odigos de \emph{Hamming} verifican:
\begin{displaymath}
2^{2^k-1}=2^{2^k-k-1}\cdot \sum_{i=0}^1{2^k-1 \choose i}
\end{displaymath}
Estos c\'odigos son $1$-perfectos.

\subsubsection{El c\'odigo de Golay $G_{23}$}

Este c\'odigo verifica:
\begin{displaymath}
2^{23}=2^{12}\cdot \sum_{i=0}^3{23\choose i}
\end{displaymath}
Este c\'odigo es $3$-perfecto.
