%
% CONSTRUCCION DEL PRIMER CODIGO BCH
%

\section{Construcci\'on de un c\'odigo BCH}

Veamos la construcci\'on de un c\'odigo BCH en concreto.\\

Consideremos la matriz de control del c\'odigo de Hamming $Ham(4)$, $H_4$. Las
columnas de esta matriz son todas las palabras no nulas de $\mathbb{F}^{^4}_2$,
por el isomorfismo visto antes podemos asignar a cada columna de $H_4$ un
elemento de $\mathbb{Z}/2^4$.\\

Podemos ordenar las columnas de $H_4$ de la forma que m\'as nos guste, lo 
\'unico que hay que tener en cuenta es que habr\'a que permutar los bits de
las palabras del c\'odigo de la misma manera en que permutemos las columnas de
la matriz.\\

Antes de reordenar las columnas eligiremos un elemento primitivo de
$\mathbb{Z}/2^n$, $\alpha$. En nuestro caso elegiremos $\alpha=2$.\\

Recordar que $H_4$ tendr\'a $2^4-1$ columnas y la columna $i$-\'esima se
co\-rres\-pon\-de\-r\'a con un elemento $a_i\in \mathbb{Z}/2^4$. Pero al ser
$\alpha$ un elemento primitivo de $\mathbb{Z}/2^4$ tendremos que $a_i=\alpha^j$
con $j=0,1,\dots,14$. Luego reordenaremos las columnas en orden descendente de
potencias del elemento primitivo.
%
\newpage
%
La reordenaci\'on de columnas ser\'a la siguiente:
\begin{displaymath}
\left( \begin{array}{ccccccccccccccc}
2^{14}&2^{13}&2^{12}&2^{11}&2^{10}&2^9&2^8&2^7&2^6&2^5&2^4&2^3&2^2&2^1&2^0
\end{array} \right)
\end{displaymath}
o lo que es lo mismo\footnote{Este c\'odigo es un c\'odigo polinomial, cuyo
generador es el polinomio $h(x)=x^4+x^3+1$, el cual es un divisor de $x^6-1$
sobre $\mathbb{F}_2[x]$. Como $2$ es un elemento primitivo de $\mathbb{F}_{2^4}$
entonces sabemos que $2^4+2^3+1=0$ en $\mathbb{F}_{2^4}$, lo cual nos servir\'a
para calcular todos los elementos del cuerpo finito de $16$ elementos como
potencias de $2$.}:
\begin{displaymath}
\left( \begin{array}{ccccccccccccccc}
12&6&3&13&10&5&14&7&15&11&9&8&4&2&1
\end{array} \right)
\end{displaymath}
que expresado en binario ser\'a:
\begin{displaymath}
\left( \begin{array}{ccccccccccccccc}
1&0&0&1&1&0&1&0&1&1&1&1&0&0&0\\
1&1&0&1&0&1&1&1&1&0&0&0&1&0&0\\
0&1&1&0&1&0&1&1&1&1&0&0&0&1&0\\
0&0&1&1&0&1&0&1&1&1&1&0&0&0&1
\end{array} \right)
\end{displaymath}
esta es la matriz de control de $Ham(4)$, reordenadas las columnas. Ahora para
crear el c\'odigo \emph{BCH} hemos de eliminar palabras del c\'odigo a\~nadiendo
nuevas condiciones a la matriz de control, mediante funciones no lineales.\\

Las funciones \emph{no lineales} que utilizaremos ser\'an de la forma
$f_i(x)=x^i$ con $i=2,3,\dots$. Por ejemplo si a\~nadimos tres nuevas
condiciones al c\'odigo tendremos que utilizar las funciones \emph{no lineales}:
\begin{eqnarray*}
f_2(x)&=&x^2\\
f_3(x)&=&x^3\\
f_4(x)&=&x^4
\end{eqnarray*}
Por lo tanto la matriz de control de nuestro c\'odigo \emph{BCH} ser\'a:
\begin{displaymath}
\left( \begin{array}{ccccccccccccccc}
12\ &6\ &3\ &13\ &10\ &5\ &14\ &7\ &15\ &11\ &9\ &8\ &4\ &2\ &1\ \\
12^2&6^2&3^2&13^2&10^2&5^2&14^2&7^2&15^2&11^2&9^2&8^2&4^2&2^2&1^2\\
12^3&6^3&3^3&13^3&10^3&5^3&14^3&7^3&15^3&11^3&9^3&8^3&4^3&2^3&1^3\\
12^4&6^4&3^4&13^4&10^4&5^4&14^4&7^4&15^4&11^4&9^4&8^4&4^4&2^4&1^4
\end{array} \right)
\end{displaymath}
o lo que es lo mismo:
\begin{displaymath}
\left( \begin{array}{ccccccccccccccc}
12&6&3&13&10&5&14&7&15&11&9&8&4&2&1\\
6&13&5&7&11&8&2&12&3&10&14&15&9&4&1\\
3&5&15&8&1&3&5&15&8&1&3&5&15&8&1\\
13&7&8&12&10&15&4&6&5&11&2&3&14&9&1
\end{array} \right)
\end{displaymath}
en la figura $\ref{fig:PrimerBCH}$, p\'agina $\pageref{fig:PrimerBCH}$, podemos
ver la matriz de control en binario.
\begin{figure}[!h]
\begin{displaymath}
\left( \begin{array}{ccccccccccccccc}
1&0&0&1&1&0&1&0&1&1&1&1&0&0&0\\
1&1&0&1&0&1&1&1&1&0&0&0&1&0&0\\
0&1&1&0&1&0&1&1&1&1&0&0&0&1&0\\
0&0&1&1&0&1&0&1&1&1&1&0&0&0&1\\
\\
0&1&0&0&1&1&0&1&0&1&1&1&1&0&0\\
1&1&1&1&0&0&0&1&0&0&1&1&0&1&0\\
1&0&0&1&1&0&1&0&1&1&1&1&0&0&0\\
0&1&1&1&1&0&0&0&1&0&0&1&1&0&1\\
\\
0&0&1&1&0&0&0&1&1&0&0&0&1&1&0\\
0&1&1&0&0&0&1&1&0&0&0&1&1&0&0\\
1&0&1&0&0&1&0&1&0&0&1&0&1&0&0\\
1&1&1&0&1&1&1&1&0&1&1&1&1&0&1\\
\\
1&0&1&1&1&1&0&0&0&1&0&0&1&1&0\\
1&1&0&1&0&1&1&1&1&0&0&0&1&0&0\\
0&1&0&0&1&1&0&1&0&1&1&1&1&0&0\\
1&1&0&0&0&1&0&0&1&1&0&1&0&1&1\\
\\
1&0&1&1&1&1&0&0&0&1&0&0&1&1&0\\
1&1&0&1&0&1&1&1&1&0&0&0&1&0&0\\
0&1&0&0&1&1&0&1&0&1&1&1&1&0&0\\
1&1&0&0&0&1&0&0&1&1&0&1&0&1&1
\end{array} \right)
\end{displaymath}
\caption{Matriz de control del c\'odigo BCH(4,2).}\label{fig:PrimerBCH}
\end{figure}
