%
% CONSTRUCCION DE LOS CODIGOS BCH
%

\section{Preliminares}

Como hemos dicho antes estos c\'odigos ser\'an subc\'odigos de los c\'odigos
de Hamming. Para ello eliminaremos palabras de los c\'odigos de Hamming con el
objetivo de obtener un nuevo c\'odigo con una distancia m\'{\i}nima mayor.\\

La forma de eliminar palabras ser\'a a\~nadiendo nuevas condiciones al
c\'odigo, o lo que es lo mimo a\~nadir nuevas filas a la matriz de control.
Estas nuevas filas que a\~nadiremos no ser\'an combinaciones lineales de las
filas de la matriz de control, ya que no estariamos haciendo nada.
Matem\'aticamente 
estaremos a\~nadiendo nuevas ecuaciones impl\'{\i}citas al c\'odigo, con lo
cual reduciremos su dimensi\'on, y como consecuencia el n\'umero de palabras
del c\'odigo.\\

Las nuevas filas que a\~nadiremos ser\'an una funci\'on \textbf{no lineal} de
la filas de la matriz de control. Al utilizar funciones \textbf{no lineales}
tenemos un
cierto control sobre la distancia m\'{\i}nima, ya que esta variar\'a seg\'un
la funci\'on que estemos utilizando.
%
\newpage
%
\subsection{Funciones no lineales}

Ejemplos de funciones \emph{no lineales} son aquellas en las que aparecen
potencias de las variables:
\begin{eqnarray*}
f(x)&=&x^2\\
f(x)&=&x^3\\
&\dots& \\
f(x)&=&x^n\quad n=2,3,\dots
\end{eqnarray*}

Sobre $\mathbb{F}_2$ es dificil encontrar funciones \emph{no lineales}, las
del tipo anterior son lineales en este cuerpo:
\begin{eqnarray*}
\mathbb{F}_2&\longrightarrow&\mathbb{F}_2\\
x&\longrightarrow&x^n = x
\end{eqnarray*}
La funci\'on $x^n$ es \emph{no lineal} para valores de $n$ distintos de $0$ y
$1$, pero en $\mathbb{F}_2$ la funci\'on \emph{no lineal} $x^n$, $n\neq 0$, es
equivalente a la funci\'on $x$, la cual es lineal.\\

Es por este motivo que vamos a considerar funciones \emph{no lineales} sobre
potencias de $\mathbb{F}_2$, es decir sobre $\mathbb{F}^{^n}_2$ con
$n=2,3,\dots.$.

\subsection{Funciones no lineales sobre $\mathbb{F}^{^n}_2$}

Dado $\mathbb{F}^{^n}_2$ podemos dotarle de estructura de grupo definiendo la
operaci\'on suma componente a componente.\\

El n\'umero de elementos que tiene $\mathbb{F}^{^n}_2$ es $2^n$. Luego como
tiene estructura de grupo podemos establecer un isomorfismo con un grupo que
tenga orden $2^n$, este grupo ser\'a $\mathbb{Z}/2^n$. El isomorfismo ser\'a
el siguiente:
\begin{eqnarray*}
\mathbb{F}^{^n}_2&\stackrel{\sim}\longrightarrow&\mathbb{Z}/2^n\simeq
\mathbb{F}_{2^n}\\
(a_{0},\dots,a_{n-1})&\longrightarrow& a_0\cdot 2^0+a_1\cdot 2^i+\dots+a_{n-1}
\cdot 2^{n-1}
\end{eqnarray*}
Es decir consideramos los elementos de $\mathbb{F}^{^n}_2$ como la
representaci\'on binaria de los $2^n$ primeros n\'umeros enteros, incluido el
cero. Luego podremos representar los n\'umeros enteros comprendidos entre $0$
y $2^n-1$, ambos inclusive.
%
\newpage
%
Ahora podemos definir la siguiente familia de funciones \emph{no lineales}:
\begin{eqnarray*}
f_i:\mathbb{F}_{2^n}&\longrightarrow&\mathbb{F}_{2^n}\\
x&\longrightarrow & x^i
\end{eqnarray*}
para $i=2,3,\dots$.\\

Por ejemplo para $i=2$ y $n=4$ tendremos:
\begin{eqnarray*}
f_2(0)&=&0^2=0\\
f_2(1)&=&1^2=1\\
f_2(2)&=&2^2=4\\
f_2(3)&=&3^2=5
\end{eqnarray*}
