%
% CREACION DE CODIGOS BCH
%

%
\newpage
%
\section{C\'odigos BCH(k,t)}

\begin{definicion}[C\'odigos BCH(k,t)]
\ \\
El c\'odigo $BCH(k,t)$ sobre un cuerpo de orden $2^k$, basado en el elemento
primitivo $\alpha$, tiene las siguientes columnas en su matriz de control: 
\begin{displaymath}
\left( \begin{array}{c}
\alpha^i\\
\alpha^{2\cdot i}\\
\vdots\\
\alpha^{2\cdot t\cdot i}
\end{array} \right)
\end{displaymath}
en la $j$-\'esima columna $i=(2^k-1)-j$. A la matriz de control del c\'odigo
$BCH(k,t)$ la denotaremos como $V_{k,t}$.
\end{definicion}
De esta definici\'on se deduce que la longitud de palabra de un c\'odigo
$BCH(k,t)$ es $2^k-1$, la misma longitud que para el c\'odigo $Ham(k)$. Algo 
que era de esperar ya que es un subcodigo de $Ham(k)$.

\subsection{Distancia m\'{\i}nima para BCH(k,t)}

Los c\'odigos $BCH(k,t)$ son $t$-perfectos, es decir corrigen error de peso
menor o igual que $t$.

\begin{teorema}[Los c\'odigos BCH(k,t) son t-perfectos]
\ \\
Los c\'odigos $BCH(k,t)$ corrigen errores de peso $t$.
\end{teorema}
\underline{\textbf{Demostraci\'on}}:\\
Por el teorema $\ref{the:DistMin}$, en la p\'agina $\pageref{the:DistMin}$,
para que los c\'odigos $BCH(k,t)$ corrijan errores de peso $t$ su distancia
m\'{\i}nima ha de verificar $d_{min}> 2\cdot t$ o lo que es lo mismo que en
su matriz de control no existan $2\cdot t$ columnas linealmente dependientes.\\

Sean $i_{1},\dots,i_{2\cdot t}$ un conjunto de $2\cdot t$ \'{\i}ndices,
distintos, cualesquiera del siguiente conjunto: $$2^k-2,\dots,0$$
Dicho conjunto son los exponentes de las potencias del elemento primitivo que
forman la primera fila de la matriz de control. Podemos formar la siguiente
matriz con las columnas determinadas por dichos elementos en la matriz de 
control:
\begin{displaymath}
\left( \begin{array}{cccc}
\alpha^{i_1}&\alpha^{i_2}&\cdots&\alpha^{i_{2\cdot t}}\\
\alpha^{2\cdot i_1}&\alpha^{2\cdot i_2}&\cdots&\alpha^{2\cdot i_{2\cdot t}}\\
\alpha^{3\cdot i_1}&\alpha^{3\cdot i_2}&\cdots&\alpha^{3\cdot i_{2\cdot t}}\\
\vdots& & &\vdots\\
\alpha^{(2\cdot t)\cdot i_1}&\alpha^{(2\cdot t)\cdot i_2}&\cdots&
\alpha^{(2\cdot t)\cdot i_{2\cdot t}}
\end{array} \right)
\end{displaymath}
Es inmediato que la matriz que acabamos de construir es una matriz de
Vandermonde $V(\alpha^{i_1},\dots,\alpha^{i_{2\cdot t}})$. Entonces por el
teorema $\ref{the:Vandermonde}$, en la p\'agina $\pageref{the:Vandermonde}$,
sus columnas son linealmente independientes. Sus columnas son $2\cdot t$
columnas cualesquiera, distintas, de la matriz de control. Entonces
se tiene que los c\'odigos $BCH(k,t)$ tienen $d_{min}> 2\cdot t$ o lo que es lo
mismo que corrigen errores de peso $t$. Son $t$-perfectos.
\begin{flushright}
$\blacksquare$
\end{flushright}
%
\newpage
%
\begin{observacion}
\ \\
\begin{itemize}
\item Gracias a las funciones \emph{no lineales} $f_i(x)=x^i$ para $i=2,3,\dots$
podemos controlar la distancia m\'{\i}nima, ya que gracias a la estructura de
la matrices de \emph{Vandermonde} podemos calcular cuantas restricciones
a\~nadir a un c\'odigo de \emph{Hamming} para que el c\'odigo resultante nos
corrija errores de peso $t$.
\item Como los c\'odigos $BCH(k,t)$ son $t$-perfectos, es decir corrigen todos
los errores de peso menor o igual que $t$, tendremos que su distancia
m\'{\i}nima ser\'a mayor o igual que $2\cdot t+1$.
\begin{displaymath}
d_{min} \geq 2\cdot t+1
\end{displaymath}
\end{itemize}
\end{observacion}
