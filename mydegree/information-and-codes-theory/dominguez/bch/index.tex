%
% INTRODUCCION A LOS CODIGOS BCH
%

\chapter{Introducci\'on a los c\'odigos BCH}

Estos c\'odigos fueron descubiertos por \emph{Hocquenghem} en $1.959$ y
tambi\'en por \emph{Bose} y \emph{Ray-Chaudhuri} en $1.960$. De las
iniciales de sus nombres es de donde viene el nombre que se les da a estos
c\'odigos.\\

Hemos visto anteriormente que los c\'odigos de Hamming corrigen los errores de
peso menor o igual que uno. Podemos modificar los c\'odigos de Hamming para
corregir dos errores, pero para eso tenemos que aumentar la distancia
m\'{\i}nima de tres a cinco. Esto lo podemos hacer de dos formas:
\begin{enumerate}
\item Aumentando la longitud de las palabras.\\

En este caso corregir un error en una palabra de longitud cinco o corregir dos
errores en una palabra de longitud diez es m\'as o menos equivalente y no
introduce mejoras sustanciales en el c\'odigo.
\item Separando m\'as las palabras del c\'odigo.\\

Este m\'etodo ser\'a el que utilizaremos, y para conseguirlo eliminaremos 
palabras del c\'odigo, con lo que obtendremos un subc\'odigo de un c\'odigo
de Hamming con una distancia m\'{\i}nima mayor, lo que nos permitir\'a
corregir hasta errores de peso dos.
\end{enumerate}

%
% MATRICES DE VANDERMONDE
%

%
% MATRICES DE VANDERMONDE
%

\section{Matrices de Vandermonde}

Para poder desarrollar la teor\'{\i}a de estos c\'odigos vamos a utilizar
algunas propiedades de un cierto tipo de matrices, ``las matrices de
Vandermonde''.  
\begin{definicion}[Matriz de Vandermonde]
\ \\
Sea $\mathbb{K}$ un cuerpo cualquiera y sean $\lambda_1,\dots,\lambda_n$
elementos distintos y no nulos de $\mathbb{K}$. Denotaremos la \textbf{``matriz
de Vandermonde''} asociada a estos elementos como
$V(\lambda_1,\dots,\lambda_n)$ y dicha matriz ser\'a:
\begin{displaymath}
V(\lambda_1,\dots,\lambda_n) =\left( \begin{array}{cccc}
\lambda_1&\lambda_2&\cdots&\lambda_n\\
\lambda_1^2&\lambda_2^2&\cdots&\lambda_n^2\\
\vdots&\vdots& &\vdots\\
\lambda_1^n&\lambda_2^n&\cdots &\lambda_n^n
\end{array} \right)
\end{displaymath}
\end{definicion}
\begin{teorema}\label{the:Vandermonde}
\ \\
Sean $\lambda_1,\dots,\lambda_n$ elementos distintos y no nulos del cuerpo
$\mathbb{K}$. Entonces las columnas de la matriz de Vandermonde
$V(\lambda_1,\dots,\lambda_n)$ son linealmente independientes sobre el cuerpo
$\mathbb{K}$.
\end{teorema}


%
% PRELIMINARES
%

%
% PRELIMINARES
%

\section{Preliminares}

Consideremos el c\'odigo $Ham(3)$, si rotamos c\'{\i}clicamente sus palabras:
\begin{displaymath}
a_1a_2a_3a_4a_5a_6a_7\rightarrow a_7a_1a_2a_3a_4a_5a_6
\end{displaymath}
obtenemos otro c\'odigo lineal, $K$ con las mismas propiedades que $Ham(3)$:
\begin{itemize}
\item $n=7$
\item $m=4$
\item $d_{min}=3$ 
\end{itemize}
y verificando que las \'unicas palabras que tienen en com\'un ambos c\'odigos
son el $0$ y el $1$.
%
\newpage
%
Adem\'as podemos a\~nadir en los dos c\'odigos un octavo bit de ``control de
paridad'' y hacer que mantengan las propiedades. Luego tendremos unos nuevos
c\'odigos $Ham(3)'$ y $K'$ de modo que:
\begin{itemize}
\item $n=8$
\item $m=4$
\item $d_{min}=3$
\end{itemize}

Estos nuevos c\'odigos lineales que hemos construidos nos van a servir para
construir otros c\'odigos lineales entendiendolos como subespacios. Esto lo
podemos hacer aprovechando la estructura de espacios vectoriales que tienen
los c\'odigos $Ham(3)'$ y $K'$, la cual nos permite realizar operaciones con
dichos espacios:
\begin{itemize}
\item Sumas.
\item Productos directos.
\item Intersecciones.
\item Uniones.
\end{itemize}

Las propiedades que tienen estos c\'odigos se obtienen de las propiedades de las
operaciones con subespacios:
\begin{eqnarray*}
\dim (Ham(3)'\times K')&=&\dim Ham(3)'+\dim K'\\
\dim (Ham(3)'+K')&=&\dim Ham(3)'+\dim K' -\dim (Ham(3)'\bigcap K')
\end{eqnarray*}


%
% CONSTRUCCION DE UN CODIGO BCH
%

%
% CONSTRUCCION DEL PRIMER CODIGO BCH
%

\section{Construcci\'on de un c\'odigo BCH}

Veamos la construcci\'on de un c\'odigo BCH en concreto.\\

Consideremos la matriz de control del c\'odigo de Hamming $Ham(4)$, $H_4$. Las
columnas de esta matriz son todas las palabras no nulas de $\mathbb{F}^{^4}_2$,
por el isomorfismo visto antes podemos asignar a cada columna de $H_4$ un
elemento de $\mathbb{Z}/2^4$.\\

Podemos ordenar las columnas de $H_4$ de la forma que m\'as nos guste, lo 
\'unico que hay que tener en cuenta es que habr\'a que permutar los bits de
las palabras del c\'odigo de la misma manera en que permutemos las columnas de
la matriz.\\

Antes de reordenar las columnas eligiremos un elemento primitivo de
$\mathbb{Z}/2^n$, $\alpha$. En nuestro caso elegiremos $\alpha=2$.\\

Recordar que $H_4$ tendr\'a $2^4-1$ columnas y la columna $i$-\'esima se
co\-rres\-pon\-de\-r\'a con un elemento $a_i\in \mathbb{Z}/2^4$. Pero al ser
$\alpha$ un elemento primitivo de $\mathbb{Z}/2^4$ tendremos que $a_i=\alpha^j$
con $j=0,1,\dots,14$. Luego reordenaremos las columnas en orden descendente de
potencias del elemento primitivo.
%
\newpage
%
La reordenaci\'on de columnas ser\'a la siguiente:
\begin{displaymath}
\left( \begin{array}{ccccccccccccccc}
2^{14}&2^{13}&2^{12}&2^{11}&2^{10}&2^9&2^8&2^7&2^6&2^5&2^4&2^3&2^2&2^1&2^0
\end{array} \right)
\end{displaymath}
o lo que es lo mismo\footnote{Este c\'odigo es un c\'odigo polinomial, cuyo
generador es el polinomio $h(x)=x^4+x^3+1$, el cual es un divisor de $x^6-1$
sobre $\mathbb{F}_2[x]$. Como $2$ es un elemento primitivo de $\mathbb{F}_{2^4}$
entonces sabemos que $2^4+2^3+1=0$ en $\mathbb{F}_{2^4}$, lo cual nos servir\'a
para calcular todos los elementos del cuerpo finito de $16$ elementos como
potencias de $2$.}:
\begin{displaymath}
\left( \begin{array}{ccccccccccccccc}
12&6&3&13&10&5&14&7&15&11&9&8&4&2&1
\end{array} \right)
\end{displaymath}
que expresado en binario ser\'a:
\begin{displaymath}
\left( \begin{array}{ccccccccccccccc}
1&0&0&1&1&0&1&0&1&1&1&1&0&0&0\\
1&1&0&1&0&1&1&1&1&0&0&0&1&0&0\\
0&1&1&0&1&0&1&1&1&1&0&0&0&1&0\\
0&0&1&1&0&1&0&1&1&1&1&0&0&0&1
\end{array} \right)
\end{displaymath}
esta es la matriz de control de $Ham(4)$, reordenadas las columnas. Ahora para
crear el c\'odigo \emph{BCH} hemos de eliminar palabras del c\'odigo a\~nadiendo
nuevas condiciones a la matriz de control, mediante funciones no lineales.\\

Las funciones \emph{no lineales} que utilizaremos ser\'an de la forma
$f_i(x)=x^i$ con $i=2,3,\dots$. Por ejemplo si a\~nadimos tres nuevas
condiciones al c\'odigo tendremos que utilizar las funciones \emph{no lineales}:
\begin{eqnarray*}
f_2(x)&=&x^2\\
f_3(x)&=&x^3\\
f_4(x)&=&x^4
\end{eqnarray*}
Por lo tanto la matriz de control de nuestro c\'odigo \emph{BCH} ser\'a:
\begin{displaymath}
\left( \begin{array}{ccccccccccccccc}
12\ &6\ &3\ &13\ &10\ &5\ &14\ &7\ &15\ &11\ &9\ &8\ &4\ &2\ &1\ \\
12^2&6^2&3^2&13^2&10^2&5^2&14^2&7^2&15^2&11^2&9^2&8^2&4^2&2^2&1^2\\
12^3&6^3&3^3&13^3&10^3&5^3&14^3&7^3&15^3&11^3&9^3&8^3&4^3&2^3&1^3\\
12^4&6^4&3^4&13^4&10^4&5^4&14^4&7^4&15^4&11^4&9^4&8^4&4^4&2^4&1^4
\end{array} \right)
\end{displaymath}
o lo que es lo mismo:
\begin{displaymath}
\left( \begin{array}{ccccccccccccccc}
12&6&3&13&10&5&14&7&15&11&9&8&4&2&1\\
6&13&5&7&11&8&2&12&3&10&14&15&9&4&1\\
3&5&15&8&1&3&5&15&8&1&3&5&15&8&1\\
13&7&8&12&10&15&4&6&5&11&2&3&14&9&1
\end{array} \right)
\end{displaymath}
en la figura $\ref{fig:PrimerBCH}$, p\'agina $\pageref{fig:PrimerBCH}$, podemos
ver la matriz de control en binario.
\begin{figure}[!h]
\begin{displaymath}
\left( \begin{array}{ccccccccccccccc}
1&0&0&1&1&0&1&0&1&1&1&1&0&0&0\\
1&1&0&1&0&1&1&1&1&0&0&0&1&0&0\\
0&1&1&0&1&0&1&1&1&1&0&0&0&1&0\\
0&0&1&1&0&1&0&1&1&1&1&0&0&0&1\\
\\
0&1&0&0&1&1&0&1&0&1&1&1&1&0&0\\
1&1&1&1&0&0&0&1&0&0&1&1&0&1&0\\
1&0&0&1&1&0&1&0&1&1&1&1&0&0&0\\
0&1&1&1&1&0&0&0&1&0&0&1&1&0&1\\
\\
0&0&1&1&0&0&0&1&1&0&0&0&1&1&0\\
0&1&1&0&0&0&1&1&0&0&0&1&1&0&0\\
1&0&1&0&0&1&0&1&0&0&1&0&1&0&0\\
1&1&1&0&1&1&1&1&0&1&1&1&1&0&1\\
\\
1&0&1&1&1&1&0&0&0&1&0&0&1&1&0\\
1&1&0&1&0&1&1&1&1&0&0&0&1&0&0\\
0&1&0&0&1&1&0&1&0&1&1&1&1&0&0\\
1&1&0&0&0&1&0&0&1&1&0&1&0&1&1\\
\\
1&0&1&1&1&1&0&0&0&1&0&0&1&1&0\\
1&1&0&1&0&1&1&1&1&0&0&0&1&0&0\\
0&1&0&0&1&1&0&1&0&1&1&1&1&0&0\\
1&1&0&0&0&1&0&0&1&1&0&1&0&1&1
\end{array} \right)
\end{displaymath}
\caption{Matriz de control del c\'odigo BCH(4,2).}\label{fig:PrimerBCH}
\end{figure}


%
% CREACION DE CODIGOS BCH
%

%
% CREACION DE CODIGOS BCH
%

%
\newpage
%
\section{C\'odigos BCH(k,t)}

\begin{definicion}[C\'odigos BCH(k,t)]
\ \\
El c\'odigo $BCH(k,t)$ sobre un cuerpo de orden $2^k$, basado en el elemento
primitivo $\alpha$, tiene las siguientes columnas en su matriz de control: 
\begin{displaymath}
\left( \begin{array}{c}
\alpha^i\\
\alpha^{2\cdot i}\\
\vdots\\
\alpha^{2\cdot t\cdot i}
\end{array} \right)
\end{displaymath}
en la $j$-\'esima columna $i=(2^k-1)-j$. A la matriz de control del c\'odigo
$BCH(k,t)$ la denotaremos como $V_{k,t}$.
\end{definicion}
De esta definici\'on se deduce que la longitud de palabra de un c\'odigo
$BCH(k,t)$ es $2^k-1$, la misma longitud que para el c\'odigo $Ham(k)$. Algo 
que era de esperar ya que es un subcodigo de $Ham(k)$.

\subsection{Distancia m\'{\i}nima para BCH(k,t)}

Los c\'odigos $BCH(k,t)$ son $t$-perfectos, es decir corrigen error de peso
menor o igual que $t$.

\begin{teorema}[Los c\'odigos BCH(k,t) son t-perfectos]
\ \\
Los c\'odigos $BCH(k,t)$ corrigen errores de peso $t$.
\end{teorema}
\underline{\textbf{Demostraci\'on}}:\\
Por el teorema $\ref{the:DistMin}$, en la p\'agina $\pageref{the:DistMin}$,
para que los c\'odigos $BCH(k,t)$ corrijan errores de peso $t$ su distancia
m\'{\i}nima ha de verificar $d_{min}> 2\cdot t$ o lo que es lo mismo que en
su matriz de control no existan $2\cdot t$ columnas linealmente dependientes.\\

Sean $i_{1},\dots,i_{2\cdot t}$ un conjunto de $2\cdot t$ \'{\i}ndices,
distintos, cualesquiera del siguiente conjunto: $$2^k-2,\dots,0$$
Dicho conjunto son los exponentes de las potencias del elemento primitivo que
forman la primera fila de la matriz de control. Podemos formar la siguiente
matriz con las columnas determinadas por dichos elementos en la matriz de 
control:
\begin{displaymath}
\left( \begin{array}{cccc}
\alpha^{i_1}&\alpha^{i_2}&\cdots&\alpha^{i_{2\cdot t}}\\
\alpha^{2\cdot i_1}&\alpha^{2\cdot i_2}&\cdots&\alpha^{2\cdot i_{2\cdot t}}\\
\alpha^{3\cdot i_1}&\alpha^{3\cdot i_2}&\cdots&\alpha^{3\cdot i_{2\cdot t}}\\
\vdots& & &\vdots\\
\alpha^{(2\cdot t)\cdot i_1}&\alpha^{(2\cdot t)\cdot i_2}&\cdots&
\alpha^{(2\cdot t)\cdot i_{2\cdot t}}
\end{array} \right)
\end{displaymath}
Es inmediato que la matriz que acabamos de construir es una matriz de
Vandermonde $V(\alpha^{i_1},\dots,\alpha^{i_{2\cdot t}})$. Entonces por el
teorema $\ref{the:Vandermonde}$, en la p\'agina $\pageref{the:Vandermonde}$,
sus columnas son linealmente independientes. Sus columnas son $2\cdot t$
columnas cualesquiera, distintas, de la matriz de control. Entonces
se tiene que los c\'odigos $BCH(k,t)$ tienen $d_{min}> 2\cdot t$ o lo que es lo
mismo que corrigen errores de peso $t$. Son $t$-perfectos.
\begin{flushright}
$\blacksquare$
\end{flushright}
%
\newpage
%
\begin{observacion}
\ \\
\begin{itemize}
\item Gracias a las funciones \emph{no lineales} $f_i(x)=x^i$ para $i=2,3,\dots$
podemos controlar la distancia m\'{\i}nima, ya que gracias a la estructura de
la matrices de \emph{Vandermonde} podemos calcular cuantas restricciones
a\~nadir a un c\'odigo de \emph{Hamming} para que el c\'odigo resultante nos
corrija errores de peso $t$.
\item Como los c\'odigos $BCH(k,t)$ son $t$-perfectos, es decir corrigen todos
los errores de peso menor o igual que $t$, tendremos que su distancia
m\'{\i}nima ser\'a mayor o igual que $2\cdot t+1$.
\begin{displaymath}
d_{min} \geq 2\cdot t+1
\end{displaymath}
\end{itemize}
\end{observacion}


%
% MATRIZ DE CONTROL REDUCIDA
%

%
% MATRIZ DE CONTROL REDUCIDA
%

\subsection{Matriz de control reducida}

Las matrices de control para c\'odigos $BCH(k,t)$ vistas hasta ahora poseen
filas innecesarias, que son combinaci\'on lineal del resto. Para formar estas
matrices hemos ampliado la matriz, de control, de un c\'odigo de Hamming con
filas que no fueran linealmente dependientes con las filas de la matriz, de
control, del c\'odigo de Hamming. Pero no hemos impuesto ninguna condici\'on
de no linealidad entre las nuevas filas que hemos a\~nadido. Debido a esto
las matrices de control que hemos visto no son pr\'acticas.\\

Por algebra lineal sabemos que las filas de una matriz que son combinaci\'on
lineal del resto no son necesarias para encontrar la soluci\'on de un
sistema de ecuaciones, que es lo que hacemos con la matriz de control de un
c\'odigo lineal. Por lo tanto podemos eliminarlar y de esta forma obtenemos
una m\'atriz m\'as peque\~na y manejable que hace la misma funci\'on que la
anterior, servir de matriz de control. A esta matriz la llamaremos 
matriz de control reducida y la denotaremos por $H_{k,t}$\\

Por ejemplo la matriz de control, sin reducir, del c\'odigo $BCH(4,3)$
ser\'a:
\begin{displaymath}
V_{4,3} =
\left( \begin{array}{ccccccccccccccccc}
12&6&3&13&10&5&14&7&15&11&9&8&4&2&1\\
6&13&5&7&11&8&2&12&3&10&14&15&9&4&1\\
3&5&15&8&1&3&5&15&8&1&3&5&15&8&1\\
13&7&8&12&10&15&4&6&5&11&2&3&14&9&1\\
10&11&1&10&11&1&10&11&1&10&11&1&10&11&1\\
5&8&3&15&1&5&8&3&15&1&5&8&3&15&1
\end{array} \right)
\end{displaymath}
mientras que la matriz de control reducida para el mismo c\'odigo ser\'a:
\begin{displaymath}
H_{4,3} =
\left( \begin{array}{ccccccccccccccccc}
12&6&3&13&10&5&14&7&15&11&9&8&4&2&1\\
3&5&15&8&1&3&5&15&8&1&3&5&15&8&1\\
10&11&1&10&11&1&10&11&1&10&11&1&10&11&1
\end{array} \right)
\end{displaymath}
Si ponemos en notaci\'on binaria la matriz de control reducida observaremos
que podemos seguir eliminando filas que son combinaci\'on lineal de otras
filas presentes en la matriz, pero esto traer\'{\i}a como consecuencia que
no podr\'{\i}amos seguir utilizando la notaci\'on que hemos utilizado hasta
ahora con $\mathbb{Z}/2^4$.\\

La matriz de control reducida, $H_{k,t}$, es m\'as sencilla y manejable
que la matriz de control original, $V_{k,t}$. Pero la matriz $V_{k,t}$ tiene
como ventaja que nos permite calcular la distancia m\'{\i}nima del c\'odigo,
mediante las matrices de \emph{Vandermonde}, este argumento no es valido
para la matriz $H_{k,t}$.


%
% MATRIZ DE CONTROL Y LOS PATRONES DE ERROR
%

%
% LA MATRIZ DE CONTROL Y LOS PATRONES DE ERROR
%

\section{Errores que corrige BCH(k,t)}

Hemos visto anteriormente que el c\'odigo $BCH(k,t)$ detecta y corrige todos
los errores de peso menor o igual que $t$.\\

La siguiente proposici\'on nos demuestra que todos los errores que detecta y
corrige el c\'odigo $BCH(k,t)$ est\'an determinados de forma \'unica.
%
%
\begin{proposicion}
\ \\
Sea $u\in BCH(k,t)$ y $v,e\in\mathbb{F}^{^n}_q$ con $n=2^k-1$ tales que $v=u+e$.
Siendo $e$ un patr\'on de error de peso $t$, a lo sumo. Entonces $e$ esta 
determinado de forma \'unica por el sindrome $V_{k,t}\cdot v^t$.
\end{proposicion}
\underline{\textbf{Demostraci\'on}}:\\
Supongamos que existe un patr\'on, $e'$, de error de peso $t$, a lo sumo,
tal que produce una palabra, $v'$, con el mismo sindrome que $v$.
\begin{displaymath}
v'=u'+e'\quad u'\in BCH(k,t)
\end{displaymath}
Como $v$ y $v'$ tienen el mismo sindrome tendremos que $V_{k,t}\cdot v^t =
V_{k,t}\cdot v'^t$, y como $u'\in BCH(k,t)$ entonces $V_{k,t}\cdot u'^t=0$. De
donde se deduce:
\begin{displaymath}
V_{k,t}\cdot v^t = V_{k,t}\cdot (u'+e')^t = V_{k,t}\cdot e'^t \Longrightarrow
V_{k,t}\cdot (v-e')^t = 0
\end{displaymath}
Luego $v-e'\in BCH(k,t)$. Como $v=u+e$ entonces tendremos que:
\begin{displaymath}
v-e'=u+e-e'
\end{displaymath}
$u$ y $v-e'$ son palabras de $BCH(k,t)$. Calculemos su distancia:
\begin{displaymath}
d(u,v-e') = d(u,u+e-e') = w(e-e')
\end{displaymath}
Como $e$ y $e'$ son dos patrones de error de peso $t$, a lo sumo tendremos:
\begin{displaymath}
w(e)\leq t\ y\ w(e')\leq t\qquad \Longrightarrow \qquad w(e-e')\leq 2\cdot t
\end{displaymath}
Como $BCH(k,t)$ es $t$-perfecto, corrige todos los errores de peso menor o
igual que $t$, su distancia m\'{\i}nima ser\'a:
\begin{displaymath}
d_{min}\geq 2\cdot t +1
\end{displaymath}
es decir, las palabras del c\'odigo estar\'an, como m\'{\i}nimo, a una distancia
de $2\cdot t+1$ unas de otras. Luego no puede ser que $d(u,v-e')\leq 2\cdot t$
con $u$ y $v-e'$ en $BCH(k,t)$. Luego no puede existir un patr\'on de error
con peso $t$, a lo sumo, tal que produzca una palabra con el mismo sindrome que
$v$.
\begin{flushright}
$\blacksquare$
\end{flushright}
Este teorema nos indica que los errores que detecta y corrige el c\'odigo
$BCH(k,t)$ tienen sindromes distintos. Podemos utilizar el algoritmo de los
sindromes que vimos en los c\'odigos lineales para corregir errores. Estos
c\'odigos son lineales, ya que son subc\'odigos de los c\'odigos de Hamming,
que son lineales, y adem\'as los hemos definido como aquellos c\'odigos que
tienen una determinada matriz de control, matriz del subespacio incidente a
un subespacio vectorial que ser\'a el c\'odigo $BCH(k,t)$.

