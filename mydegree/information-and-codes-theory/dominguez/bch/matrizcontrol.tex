%
% LA MATRIZ DE CONTROL Y LOS PATRONES DE ERROR
%

\section{Errores que corrige BCH(k,t)}

Hemos visto anteriormente que el c\'odigo $BCH(k,t)$ detecta y corrige todos
los errores de peso menor o igual que $t$.\\

La siguiente proposici\'on nos demuestra que todos los errores que detecta y
corrige el c\'odigo $BCH(k,t)$ est\'an determinados de forma \'unica.
%
%
\begin{proposicion}
\ \\
Sea $u\in BCH(k,t)$ y $v,e\in\mathbb{F}^{^n}_q$ con $n=2^k-1$ tales que $v=u+e$.
Siendo $e$ un patr\'on de error de peso $t$, a lo sumo. Entonces $e$ esta 
determinado de forma \'unica por el sindrome $V_{k,t}\cdot v^t$.
\end{proposicion}
\underline{\textbf{Demostraci\'on}}:\\
Supongamos que existe un patr\'on, $e'$, de error de peso $t$, a lo sumo,
tal que produce una palabra, $v'$, con el mismo sindrome que $v$.
\begin{displaymath}
v'=u'+e'\quad u'\in BCH(k,t)
\end{displaymath}
Como $v$ y $v'$ tienen el mismo sindrome tendremos que $V_{k,t}\cdot v^t =
V_{k,t}\cdot v'^t$, y como $u'\in BCH(k,t)$ entonces $V_{k,t}\cdot u'^t=0$. De
donde se deduce:
\begin{displaymath}
V_{k,t}\cdot v^t = V_{k,t}\cdot (u'+e')^t = V_{k,t}\cdot e'^t \Longrightarrow
V_{k,t}\cdot (v-e')^t = 0
\end{displaymath}
Luego $v-e'\in BCH(k,t)$. Como $v=u+e$ entonces tendremos que:
\begin{displaymath}
v-e'=u+e-e'
\end{displaymath}
$u$ y $v-e'$ son palabras de $BCH(k,t)$. Calculemos su distancia:
\begin{displaymath}
d(u,v-e') = d(u,u+e-e') = w(e-e')
\end{displaymath}
Como $e$ y $e'$ son dos patrones de error de peso $t$, a lo sumo tendremos:
\begin{displaymath}
w(e)\leq t\ y\ w(e')\leq t\qquad \Longrightarrow \qquad w(e-e')\leq 2\cdot t
\end{displaymath}
Como $BCH(k,t)$ es $t$-perfecto, corrige todos los errores de peso menor o
igual que $t$, su distancia m\'{\i}nima ser\'a:
\begin{displaymath}
d_{min}\geq 2\cdot t +1
\end{displaymath}
es decir, las palabras del c\'odigo estar\'an, como m\'{\i}nimo, a una distancia
de $2\cdot t+1$ unas de otras. Luego no puede ser que $d(u,v-e')\leq 2\cdot t$
con $u$ y $v-e'$ en $BCH(k,t)$. Luego no puede existir un patr\'on de error
con peso $t$, a lo sumo, tal que produzca una palabra con el mismo sindrome que
$v$.
\begin{flushright}
$\blacksquare$
\end{flushright}
Este teorema nos indica que los errores que detecta y corrige el c\'odigo
$BCH(k,t)$ tienen sindromes distintos. Podemos utilizar el algoritmo de los
sindromes que vimos en los c\'odigos lineales para corregir errores. Estos
c\'odigos son lineales, ya que son subc\'odigos de los c\'odigos de Hamming,
que son lineales, y adem\'as los hemos definido como aquellos c\'odigos que
tienen una determinada matriz de control, matriz del subespacio incidente a
un subespacio vectorial que ser\'a el c\'odigo $BCH(k,t)$.
