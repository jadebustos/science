%
% MATRICES DE VANDERMONDE
%

\section{Matrices de Vandermonde}

Para poder desarrollar la teor\'{\i}a de estos c\'odigos vamos a utilizar
algunas propiedades de un cierto tipo de matrices, ``las matrices de
Vandermonde''.  
\begin{definicion}[Matriz de Vandermonde]
\ \\
Sea $\mathbb{K}$ un cuerpo cualquiera y sean $\lambda_1,\dots,\lambda_n$
elementos distintos y no nulos de $\mathbb{K}$. Denotaremos la \textbf{``matriz
de Vandermonde''} asociada a estos elementos como
$V(\lambda_1,\dots,\lambda_n)$ y dicha matriz ser\'a:
\begin{displaymath}
V(\lambda_1,\dots,\lambda_n) =\left( \begin{array}{cccc}
\lambda_1&\lambda_2&\cdots&\lambda_n\\
\lambda_1^2&\lambda_2^2&\cdots&\lambda_n^2\\
\vdots&\vdots& &\vdots\\
\lambda_1^n&\lambda_2^n&\cdots &\lambda_n^n
\end{array} \right)
\end{displaymath}
\end{definicion}
\begin{teorema}\label{the:Vandermonde}
\ \\
Sean $\lambda_1,\dots,\lambda_n$ elementos distintos y no nulos del cuerpo
$\mathbb{K}$. Entonces las columnas de la matriz de Vandermonde
$V(\lambda_1,\dots,\lambda_n)$ son linealmente independientes sobre el cuerpo
$\mathbb{K}$.
\end{teorema}
