%
% MATRIZ DE CONTROL REDUCIDA
%

\subsection{Matriz de control reducida}

Las matrices de control para c\'odigos $BCH(k,t)$ vistas hasta ahora poseen
filas innecesarias, que son combinaci\'on lineal del resto. Para formar estas
matrices hemos ampliado la matriz, de control, de un c\'odigo de Hamming con
filas que no fueran linealmente dependientes con las filas de la matriz, de
control, del c\'odigo de Hamming. Pero no hemos impuesto ninguna condici\'on
de no linealidad entre las nuevas filas que hemos a\~nadido. Debido a esto
las matrices de control que hemos visto no son pr\'acticas.\\

Por algebra lineal sabemos que las filas de una matriz que son combinaci\'on
lineal del resto no son necesarias para encontrar la soluci\'on de un
sistema de ecuaciones, que es lo que hacemos con la matriz de control de un
c\'odigo lineal. Por lo tanto podemos eliminarlar y de esta forma obtenemos
una m\'atriz m\'as peque\~na y manejable que hace la misma funci\'on que la
anterior, servir de matriz de control. A esta matriz la llamaremos 
matriz de control reducida y la denotaremos por $H_{k,t}$\\

Por ejemplo la matriz de control, sin reducir, del c\'odigo $BCH(4,3)$
ser\'a:
\begin{displaymath}
V_{4,3} =
\left( \begin{array}{ccccccccccccccccc}
12&6&3&13&10&5&14&7&15&11&9&8&4&2&1\\
6&13&5&7&11&8&2&12&3&10&14&15&9&4&1\\
3&5&15&8&1&3&5&15&8&1&3&5&15&8&1\\
13&7&8&12&10&15&4&6&5&11&2&3&14&9&1\\
10&11&1&10&11&1&10&11&1&10&11&1&10&11&1\\
5&8&3&15&1&5&8&3&15&1&5&8&3&15&1
\end{array} \right)
\end{displaymath}
mientras que la matriz de control reducida para el mismo c\'odigo ser\'a:
\begin{displaymath}
H_{4,3} =
\left( \begin{array}{ccccccccccccccccc}
12&6&3&13&10&5&14&7&15&11&9&8&4&2&1\\
3&5&15&8&1&3&5&15&8&1&3&5&15&8&1\\
10&11&1&10&11&1&10&11&1&10&11&1&10&11&1
\end{array} \right)
\end{displaymath}
Si ponemos en notaci\'on binaria la matriz de control reducida observaremos
que podemos seguir eliminando filas que son combinaci\'on lineal de otras
filas presentes en la matriz, pero esto traer\'{\i}a como consecuencia que
no podr\'{\i}amos seguir utilizando la notaci\'on que hemos utilizado hasta
ahora con $\mathbb{Z}/2^4$.\\

La matriz de control reducida, $H_{k,t}$, es m\'as sencilla y manejable
que la matriz de control original, $V_{k,t}$. Pero la matriz $V_{k,t}$ tiene
como ventaja que nos permite calcular la distancia m\'{\i}nima del c\'odigo,
mediante las matrices de \emph{Vandermonde}, este argumento no es valido
para la matriz $H_{k,t}$.
