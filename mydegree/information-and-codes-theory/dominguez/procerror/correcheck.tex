%
% DETECCION Y CORRECCION DE ERRORES MEDIANTE LA MATRIZ DE CHEKEO
%

\section{Detecci\'on y correcci\'on de errores mediante la matriz de control}

Seg\'un el m\'etodo que hemos visto antes necesitariamos almacenar una tabla
de $q^n$ elementos y realizar, a lo sumo, $q^n$ comparaciones.\\

Esto no es pr\'actico ya que si utilizamos c\'odigos con palabras de longitud
algo grande la cantidad de memoria necesaria para almacenar la tabla ser\'a
bastante grande, as\'{\i} como el n\'umero de comparaciones necesarias. Por
ejemplo para $q=2$ y $n=25$ tendremos que almacenar $2^{25}=33.554.432$
palabras, lo cual requiere una gran inversi\'on en memoria y gran cantidad
de tiempo para realizar una comparaci\'on.\\

Para evitar las $q^n$ comparaciones y el almacenamiento de toda la tabla
est\'andar podemos utilizar la matriz de control del c\'odigo lineal.
\begin{proposicion}
\ \\
Sea $\mathcal{C}[n,m]$ un c\'odigo lineal y $H$ su matriz de control. 
$u,v\in \mathbb{F}_q^{^n}$ est\'an en la misma fila s\'{\i} y s\'olo s\'{\i}
$H\cdot u^t = H\cdot v^t$.
\end{proposicion}
\underline{\textbf{Demostraci\'on}}:\\
Como ya hemos visto antes $u$ y $v$ est\'an en la misma fila s\'{\i} y s\'olo 
s\'{\i} se verifica que $u-v\in \mathcal{C}[n,m]$ entonces se tiene que:
\begin{displaymath}
H\cdot (u-v)^t = 0 \Longleftrightarrow H\cdot u^t=H\cdot v^t
\end{displaymath}
\begin{flushright}
$\blacksquare$
\end{flushright}
Seg\'un esta proposici\'on podemos asignar a cada fila de una tabla est\'andar
un vector constante, el cual unicamente ser\'a nulo para la primera fila de
dicha tabla.
\begin{definicion}[Sindrome de una palabra]
\ \\
Dado un c\'odigo lineal $\mathcal{C}[n,m]$ con matriz de control $H$ llamaremos
\textbf{``sindrome de una palabra''}, $u\in \mathbb{F}_q^{^n}$, al siguiente
vector $(H\cdot u^t)^t$.
\end{definicion}
Como todas las palabras de una fila tienen el mismo sindrome y el error que
corregiremos ser\'a el primer elemento de cada fila entonces unicamente
almacenaremos el primer elemento de cada fila junto con su sindrome. De esta
forma necesitaremos menos memoria para almacenar la tabla.\\ \\
%
El algoritmo para corregir un error es el siguiente:
\begin{itemize}
\item Recibimos la palabra $v$.
\item Calculamos su sindrome $(H\cdot v^t)^t$.
\item Buscamos en la nueva tabla el sindrome anterior y vemos cual de los
elementos tiene dicho sidrome, supongamos que $e_i$ es tal que:
$$(H\cdot e_i^t)^t=(H\cdot v^t)^t$$ 
\item La palabra correcta ser\'a $u=v-e_i$.
\end{itemize}
Ademas de necesitar menos memoria para almacenar la tabla de los sindromes
tambi\'en necesitamos menos tiempo para corregir un error ya que unicamente
tendremos que hacer $q^{n-m}$ comparaciones, en lugar de las $q^n$ requeridas
por la utilizaci\'on de la tabla est\'andar.
