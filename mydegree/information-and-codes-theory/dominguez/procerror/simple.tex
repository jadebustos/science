%
% CONDICION PARA LA CORRECCION DE ERRORES SIMPLES
%

\section{Condici\'on para la correcci\'on de errores de peso uno}

Los errores de peso uno son una base del $\mathbb{F}_q$-espacio vectorial
$\mathbb{F}^{^n}_q$. Por ejemplo para $n=3$ y $q=2$:
\begin{eqnarray*}
e_1&=&(1,0,0)\\
e_2&=&(0,1,0)\\
e_3&=&(0,0,1)
\end{eqnarray*}
Sea $H$ la matriz de control de un c\'odigo lineal:
\begin{displaymath}
\mathcal{C}[n,m]\subset \mathbb{F}^{^n}_2 =<e_1,\dots,e_n>
\end{displaymath}
donde los $\{e_i\}_{i=1}^n$ son los errores de peso uno. Seg\'un el teorema
$\ref{the:DistintaFila}$ un procesador de error $P$ distingue dos errores $e_i$
y $e_j$ $\Longleftrightarrow$ est\'an en diferente fila $\Longleftrightarrow$
$H \cdot e_i^t \neq H\cdot e_j^t$. Ahora bien $H\cdot e_i^t=i$-\'esima columna
de la matriz de control $H$, luego de esto se deduce:
\begin{corolario}
\ \\
Un c\'odigo lineal binario puede distinguir todos los errores de peso uno
s\'{\i} y s\'olo s\'{\i} la matriz de control tiene todas las columnas distintas
y no nulas.
\begin{flushright}
$\blacksquare$
\end{flushright}
\end{corolario}
%
Este corolario lo podemos generalizar a canales no binarios:
\begin{teorema}[Condici\'on necesaria y suficiente]
\ \\
Sea $\mathcal{C}[n,m]$ un c\'odigo lineal con matriz de control $H$. El c\'odigo
puede corregir todos los errores de peso uno s\'{\i} y s\'olo s\'{\i} todas las
columnas de la matriz de control son no nulas y ninguna columna es un multiplo
de las restantes columnas\footnote{Son linealmente independientes dos a dos.}.
\end{teorema}
\underline{\textbf{Demostraci\'on}}:\\
Por el teorema $\ref{the:DistintaFila}$ los errores de peso uno se distinguen y
corrigen $\Longleftrightarrow$ estan en diferente fila en una tabla est\'andar
del c\'odigo $\Longleftrightarrow$ tienen distintos sindromes.
%
\newpage
%
Sea $e$ un error de peso uno, es decir una palabra con todas sus componentes
nulas excepto una. Entonces podemos distiguir dos casos:
\begin{itemize}
\item Canal binario, sobre $\mathbb{F}_2$.\\ \\
%
Tendremos que $e=e_i$ para alg\'un $i$, luego $H\cdot e_i^t=i$-\'esima columna
de la matriz de control $H$.
\item Canal no binario, sobre $\mathbb{F}_q$, con $q\neq 2$.\\ \\
%
Sea $a_i\in \mathbb{F}_q$ la componente no nula de $e$, entonces tendremos que
$H\cdot e^t=a_i\cdot c_i$, donde $c_i$ es la $i$-\'esima columna de la matriz
de control del c\'odigo $H$.
\end{itemize}
Luego los sindromes de los errores de peso uno son multiplos de las columnas
de la matriz de control, $H$.\\ \\
%
Como el c\'odigo distingue y corrige todos los errores de peso uno
$\Longleftrightarrow$ tienen distinto sindrome, y los sindromes%
\footnote{De los errores de peso uno.} son multiplos de las columnas de la
matriz de control se tiene que:
\begin{quote}
El c\'odigo distingue y corrige todos los errores de peso uno
$\Longleftrightarrow$ no hay ninguna columna que sea multiplo de las
dem\'as.
\end{quote}
Recordar que la palabra cero es multiplo de todas, basta con multiplicar
cualquier palabra por el escalar cero.
\begin{flushright}
$\blacksquare$
\end{flushright}
Seg\'un este teorema podemos ver si un c\'odigo distingue y corrige todos
los errores de peso uno s\'olo con mirar las columnas de la matriz de control
y comprobar s\'{\i} son linealmente independientes dos a dos.
