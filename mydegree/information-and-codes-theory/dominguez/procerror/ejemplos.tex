%
% EJEMPLOS
%

\section{Ejemplos}

\subsection{C\'odigo del bit de control de paridad}

Este c\'odigo es un c\'odigo de $2^7=128$ palabra palabras. Es un subconjunto de
$\mathbb{F}^{^8}_2$ el cual tiene $2^8=256$ palabras, con lo cual el n\'umero
de patrones de error que tendremos ser\'a $2^8-2^7=128$.

\subsubsection{Tabla est\'andar del c\'odigo}

Como el c\'odigo tiene $2^7=128$ palabras la tabla tendr\'a $2^7=128$ columnas
y el n\'umero de filas ser\'a $2^{8-7}=2$.\\

Debido a la gran cantidad de palabras de este c\'odigo no construiremos su
tabla est\'andar.

\subsubsection{Sindromes}

Sea $H$ la matriz de control, en forma est\'andar, del c\'odigo:
\begin{displaymath}
\left( \begin{array}{cccccccc}
1&1&1&1&1&1&1&1
\end{array} \right)
\end{displaymath}

\begin{table}[!h]
\begin{displaymath}
\begin{array}{|c|c|}
\hline
Errores & Sindromes \\
\hline
00000000 & 0 \\
\hline
10000000 & 1 \\
\hline
\end{array}
\end{displaymath}
\caption{Tabla de sindromes del c\'odigo del bit de control de paridad.}
\end{table}

\subsection{C\'odigo de triple repetici\'on}

Este c\'odigo es un c\'odigo de $2^1=2$ palabras. Es un subconjunto de
$\mathbb{F}^{^3}_2$ el cual tiene $2^3=8$ palabras, con lo cual el n\'umero
de patrones de error que tendremos ser\'a $2^3-2^1=6$.

\subsubsection{Tabla est\'andar del c\'odigo}

Como el c\'odigo tiene $2^1=2$ palabras la tabla tendr\'a $2^1=2$ columnas y
el n\'umero de filas ser\'a $2^{3-1}=4$.\\

La primera fila estar\'a formada por las palabras del c\'odigo con la
condici\'on de que el $0$ sea el primer elemento. Luego la primera fila ser\'a:
\begin{displaymath}
\begin{array}{cc}
000&111
\end{array}
\end{displaymath}
Para la segunda fila cogeremos un elemento de $\mathbb{F}^{^3}_2$ que no este en
la fila cero, y el primer elemento de cada fila ha de ser el de m\'{\i}nimo
peso de dicha fial. El elemento $100$ no esta en la fila cero, luego elegimos
ese elemento como primer elemento de la fila uno ya que es de peso uno y no es
posible encontrar otro de peso menor. El resto de elementos de la fila ser\'an:
\begin{displaymath}
Tb(\mathcal{C}[3,1])_{1,i}=Tb(\mathcal{C}[3,1])_{1,0}+Tb(\mathcal{C}[3,1])_{0,i}
\quad i=0,1
\end{displaymath}
Luego la segunda fila ser\'a:
\begin{displaymath}
\begin{array}{cc}
100&011
\end{array}
\end{displaymath}
Siguiendo el mismo razonamiento elegiremos como primer elemento de la fila tres
un elemento que no haya aparecido en las filas anteriores y de peso m\'{\i}nimo,
por ejemplo $010$ y siguiendo el razonamiento anterior completaremos la tabla.
\begin{eqnarray*}
Tb(\mathcal{C}[3,1])_{2,0}&=&010\\
Tb(\mathcal{C}[3,1])_{3,0}&=&001
\end{eqnarray*}
\begin{table}[!h]
\begin{displaymath}
\begin{array}{|c|c|}
\hline
000&111\\
\hline
100&011\\
\hline
010&101\\
\hline
001&110\\
\hline
\end{array}
\end{displaymath}
\caption{Tabla est\'andar del c\'odigo de triple repetici\'on.}\label{tab:TablaII}
\end{table}

\subsubsection{Correcci\'on de errores}

Supongamos que recibimos la palabra $011$, que no pertenece al c\'odigo. Para
corregir el c\'odigo haremos:
\begin{itemize}
\item Localizamos el lugar de dicha palabra en la tabla.
\begin{displaymath}
Tb(\mathcal{C}[3,1])_{1,1}=011
\end{displaymath}
\item Elegimos como palabra correcta la palabra que est\'e en la misma columna y
en la fila cero.
\begin{displaymath}
Tb(\mathcal{C}[3,1])_{0,1}=111
\end{displaymath}
\end{itemize}
El error cometido vendr\'a dado por el primer elemento de la fila en la que se
encuentre la palabra recibida:
\begin{displaymath}
Tb(\mathcal{C}[3,1])_{1,0}=100
\end{displaymath}
Luego en la transmisi\'on se ha cometido un error de peso $w(100)=1$ en el
primer bit.

\subsubsection{Sindromes}

Supondremos que queremos corregir los mismos errores que se corrigen con la
tabla $\ref{tab:TablaII}$.\\ \\
%
Sea $H$ la matriz de control, en forma est\'andar, del c\'odigo:
\begin{displaymath}
H=\left( \begin{array}{ccc}
1&1&0\\
1&0&1
\end{array} \right)
\end{displaymath}

\begin{table}[!h]
\begin{displaymath}
\begin{array}{|c|c|}
\hline
Errores & Sindromes \\
\hline
000 & 00 \\
\hline
100 & 11 \\
\hline
010 & 10 \\
\hline
001 & 01 \\
\hline
\end{array}
\end{displaymath}
\caption{Tabla de sindromes del c\'odigo de triple repetici\'on.}\label{tab:TabSindromes}
\end{table}

Para corregir errores supongamos que recibimos la palabra $101$, calculamos
su sindrome que ser\'a $10$. Entonces utilizando la tabla
$\ref{tab:TabSindromes}$ buscamos el error que tiene sindrome $10$, dicho
error es $010$. Luego la palabra transmitida ser\'a $101-010=111$. Recordar
que $-1=1$ en $\mathbb{F}_2$.

\subsection{C\'odigo de triple control}

Este c\'odigo es un c\'odigo de $2^3=8$ palabras. Es un subconjunto de
$\mathbb{F}^{^6}_2$ el cual tiene $2^6=64$ palabras, con lo cual el n\'umero
de patrones de error que tendremos ser\'a $2^6-2^3=56$.

\subsubsection{Tabla est\'andar del c\'odigo}

Como el c\'odigo tiene $2^3=8$ palabras la tabla tendr\'a $2^3=8$ columnas y
el n\'umero de filas ser\'a $2^{6-3}=8$.\\ 

La primera fila estar\'a formada por las palabras del c\'odigo con la
condici\'on de que el $0$ sea el primer elemento. Luego la primera fila
ser\'a:
\begin{displaymath}
\begin{array}{cccccccc}
000000&100110&010101&001011&111000&011110&101101&110011
\end{array}
\end{displaymath}
Para la segunda fila cogeremos un elemento de $\mathbb{F}^{^6}_2$ que no este
en la fila cero, y el primer elemento de cada fila ha de ser el de
m\'{\i}nimo peso de dicha fila. El elemento $100000$ no est\'a en la fila 
cero,
luego elegimos ese elemento como primer elemento de la fila uno ya que es de
peso uno y no es posible encontrar otro elemento de peso menor. El resto de
elementos de la fila ser\'an:
\begin{displaymath}
Tb(\mathcal{C}[6,3])_{1,i}=Tb(\mathcal{C}[6,3])_{1,0}+Tb(\mathcal{C}[6,3])_{0,
i}
\quad i=1,\dots,7
\end{displaymath}
Luego la segunda fila ser\'a:
\begin{displaymath}
\begin{array}{cccccccc}
100000&000110&110101&101011&011000&111110&001101&010011
\end{array}
\end{displaymath}
%
\newpage
%
Siguiendo el mismo razonamiento escogeremos como primer elemento de la fila 
dos un elemento de $\mathbb{F}^{^6}_2$ que no aparezca en las filas cero y uno. 
Como hay elementos de peso uno que no aparecen en dichas filas elegiremos como
primer elemento de la fila dos $010000$. Y siguiendo el mismo razonamiento
construiremos la fila y eligiremos los siguientes elementos:
\begin{eqnarray*}
Tb(\mathcal{C}[6,3])_{3,0}&=& 001000\\
Tb(\mathcal{C}[6,3])_{4,0}&=& 000100\\
Tb(\mathcal{C}[6,3])_{5,0}&=& 000010\\
Tb(\mathcal{C}[6,3])_{6,0}&=& 000001
\end{eqnarray*}
Para elegir el primer elemento de la fila siete observaremos que todas las
palabras de $\mathbb{F}^{^6}_2$ de peso uno estan en alguna de las filas
anteriores, con lo cual el elemento de la fila siete de menor peso tendr\'a un
peso mayor o igual que dos. El elemento $100001$ no esta en ninguna de las  
filas anteriores con lo cual lo elegiremos como primer elemento de la fila
siete.\\ \\ 
%
%
\begin{table}[!h]
\begin{displaymath}
\begin{array}{|c|c|c|c|c|c|c|c|}
\hline
000000&100110&010101&001011&111000&011110&101101&110011\\
\hline
100000&000110&110101&101011&011000&111110&001101&010011\\
\hline
010000&110110&000101&011011&101000&001110&111101&100011\\
\hline
001000&101110&011101&000011&110000&010110&100101&111011\\
\hline
000100&100010&010001&001111&111100&011010&101001&110111\\
\hline
000010&100100&010111&001001&111010&011100&101111&110001\\
\hline
000001&100111&010100&001010&111001&011111&101100&110010\\
\hline
100001&000111&110100&101010&011001&111111&001100&010010\\
\hline
\end{array}
\end{displaymath}
\caption{Tabla est\'andar del c\'odigo de triple control.}\label{tab:Tabla}
\end{table}
%
%
En la tabla $\ref{tab:Tabla}$ podemos ver una tabla est\'andar para el c\'odigo
de triple repetici\'on. No es la \'unica tabla est\'andar. Para obtener otras
tablas est\'andar bastar\'a con elegir distintos elementos como primer elemento
de cada fila obteniendo las mismas filas, pero en ordenadas de distinta forma.
%
\newpage
%
\subsubsection{Correcci\'on de errores}

Supongamos que recibimos la palabra $111100$, la cual no pertenece al c\'odigo.
Para corregir el error haremos:
\begin{itemize}
\item Localizamos el lugar de dicha palabra en la tabla.
\begin{displaymath}
Tb(\mathcal{C}[6,3])_{4,4}=111100
\end{displaymath}
\item Elegimos como palabra correcta la palabra que est\'e en la misma columna y
en la fila cero.
\begin{displaymath}
Tb(\mathcal{C}[6,3])_{0,4}=111000
\end{displaymath}
\end{itemize}
El error cometido vendr\'a dado por el primer elemento de la fila en la que se
encuentre la palabra recibida:
\begin{displaymath}
Tb(\mathcal{C}[6,3])_{4,0}=000100
\end{displaymath}
Luego en la transmisi\'on se ha cometido un error de peso $w(000100)=1$ en el
cuarto bit.

\subsubsection{Sindromes}

Supondremos que queremos corregir los mismos errores que se corrigen con la
tabla $\ref{tab:Tabla}$.\\ \\
%
Sea $H$ la matriz de control, en forma est\'andar, del c\'odigo:
\begin{displaymath}
H=\left( \begin{array}{cccccc}
1&1&0&1&0&0\\
1&0&1&0&1&0\\
0&1&1&0&0&1
\end{array} \right)
\end{displaymath}

\begin{table}[!h]
\begin{displaymath}
\begin{array}{|c|c|}
\hline
Error  & Sindrome \\
\hline
000000 & 000 \\
\hline
100000 & 110 \\
\hline
010000 & 101 \\
\hline
001000 & 011 \\
\hline
000100 & 100 \\
\hline
000010 & 010 \\
\hline
000001 & 001 \\
\hline
100001 & 111 \\
\hline
\end{array}
\end{displaymath}
\caption{Tabla de sindromes del c\'odigo de triple control.}\label{tab:TablaSindromes}
\end{table}
Obviamente requiere menor costo almacenar la tabla $\ref{tab:TablaSindromes}$
que almacenar la tabla $\ref{tab:Tabla}$.\\

Para corregir errores supongamos que recibimos la palabra $100011$, calculamos
su sindrome que ser\'a $101$. Entonces utilizando la tabla
$\ref{tab:TablaSindromes}$ buscamos que error tiene sindrome $101$, dicho error
es $010000$. Luego la palabra transmitida ser\'a $100011-010000=110011$.
Recordar que $-1=1$ en $\mathbb{F}_2$.
