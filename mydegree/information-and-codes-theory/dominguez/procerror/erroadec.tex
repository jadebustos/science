%
% ERRORES QUE SE CORRIGEN ADECUADAMENTE
%

%
\newpage
%
\section{Errores que se corrigen adecuadamente}

S\'{\i} tenemos un c\'odigo lineal $\mathcal{C}[n,m]\subset \mathbb{F}^{^n}_q$
con una tabla est\'andar $Tb(\mathcal{C}[n,m])$, en la cual hay $q^{n-m}$ filas
y $q^m$ columnas, ya hemos visto como corregir errores. Los errores los
corregiremos sumando el primer elemento, de la fila en la que est\'a la palabra
recibida, a dicha palabra. Luego los patrones de error que se corrigen vienen
dados por los primeros elementos de cada fila.\\ 

Debido al hecho de que siempre corregiremos por la palabra del c\'odigo m\'as
cercana a la palabra recibida se deduce que ``s\'olo debe haber una palabra
del c\'odigo'' que tenga distancia m\'{\i}nima a la palabra recibida. En caso
contrario no sabriamos cual de las palabras elegir como correcta, se dice 
entonces que no distinguiriamos el error.\\

La proposici\'on $\ref{pro:Cercania}$ nos indica que utilizando tablas
est\'andar tenemos solucionado este problema, pero no podemos corregir
adecuadamente todos los errores mediante una tabla est\'andar.\\ \\
%
El siguiente teorema nos indica cuando se distinguen dos errores.
%
%
\begin{teorema}\label{the:DistintaFila}
\ \\
Sea $\mathcal{C}[n,m]$ un c\'odigo lineal y 
sea $P$ un procesador de error para dicho c\'odigo. Sea $S=\{e_1,\dots,e_s\}$
un conjunto de patrones de error. El procesador de error $P$ puede distinguir
estos patrones de error s\'{\i} y s\'olo s\'{\i} cada patr\'on est\'a en una
fila distinta de la de los otros patrones (en una tabla del c\'odigo).
\end{teorema}
\underline{\textbf{Demostraci\'on}}:\\
$\Rightarrow |$ Sea $e$ un patr\'on de error y $v\neq e$ y que est\'e en la
misma fila que $e$. Entonces tendremos que $u=v-e$, con $u\in \mathcal{C}[n,m]$
y tal que $u\neq 0$. S\'{\i} el procesador de error corrige el error $e$
entonces corregir\'a $v$ a $u$.\\ \\
%
Supongamos que el procesador $P$ corrige el error $v$, entonces corregir\'a $v$ 
a $0$, lo cual no es posible. Luego no existir\'a un procesador de error que
distinga dos errores situados en la misma fila.\\ \\
%
$\Leftarrow |$ Sean $e_i$ y $e_j$ con $i\neq j$, entonces podemos construir una
tabla est\'andar con $e_i$ y $e_j$ como primeros elementos de dos filas.\\ \\
%
Con dicha tabla se tiene que para toda palabra $u$ del c\'odigo el procesador
corregir\'a las palabras $u+e_i$ y $u+e_j$ a $u$.\\ \\
%
Luego el procesador de error $P$ puede distinguir, corregir, dos patrones de
error cuando esten en distinta fila, $i\neq j$.
\begin{flushright}
$\blacksquare$
\end{flushright}
Seg\'un todo lo que hemos visto podremos corregir tantos errores como clases
de equivalencia tenga el $\mathbb{F}_q$-espacio vectorial
$\mathbb{F}_q^{^n}/\mathcal{C}[n,m]$, ya que un procesador de error, $P$, 
distingue dos errores s\'{\i} y s\'olo s\'{\i} estan en distinta fila, es decir,
s\'{\i} y s\'olo s\'{\i} pertenecen a distinta clase de equivalencia.

\subsection{M\'etodo pr\'actico}

Con lo que hemos visto un m\'etodo ``pr\'actico'' para detectar y corregir
errores ser\'a el siguiente:\\ \\
%
Dado un c\'odigo lineal $\mathcal{C}[n,m]$:
\begin{itemize}
\item Construir una tabla est\'andar, utilizando como primer elemento de cada
fila el tipo de error que queremos detectar y corregir.
\item Una vez recibida una palabra comparar con las $q^n$ palabras de la tabla.
\item Seleccionar como palabra correcta la palabra que se encuentre en la fila
cero y en la columna de la palabra recibida.
\end{itemize}
Este m\'etodo aunque pr\'actico es poco util, ya que hay que realizar $q^n$
comparaciones y, a medida que aumenta $n$ $q^n$ crece aun m\'as rapidamente.
