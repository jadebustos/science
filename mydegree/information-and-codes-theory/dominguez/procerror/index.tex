%
% PROCESADORES DE ERROR PARA CODIGOS LINEALES
%

\chapter{Procesamiento de error en c\'odigos lineales}\label{cap:PError}

En este cap\'{\i}tulo supondremos que estamos utilizando un c\'odigo lineal
$\mathcal{C}[n,m]$, es decir, $\mathcal{C}[n,m]\subset \mathbb{F}^{^n}_q$.\\

Ya vimos en el apartado $(\ref{sec:ProcErr})$ lo que era un procesador de error
para c\'odigos de bloques.\\

En los c\'odigos lineales nos es posible sumar palabras y multiplicarlas por
constantes, de modo que utilizaremos estas propiedades para simplificar el
procesamiento de errores. \\

Utilizando la linealidad de estos c\'odigos construiremos una tabla con todas
las palabras de $\mathbb{F}^{^n}_q$ situadas por proximidad con las palabras
del c\'odigo $\mathcal{C}[n,m]$. La proximidad vendr\'a dada por la distancia
de Hamming.

% 
% CONSTRUCCION DE LA TABLA DE UN CODIGO
%

%
% CONSTRUCCION DE LA TABLA DE UN CODIGO
%

%
\newpage
%
\section{Construcci\'on de la tabla de un c\'odigo lineal}

En la tabla estar\'an los $q^n$ elementos de $\mathbb{F}^{^n}_q$. La tabla
la organizaremos por filas y columnas.\\ \\
%
Como $\mathcal{C}$ tiene $q^m$ elementos, es de dimensi\'on $m$, organizaremos
la tabla con $q^{n-m}$ filas y $q^m$ columnas. De esta forma en la tabla
tendremos: $$q^{n-m}\cdot q^m=q^{n-m+m}=q^n$$ elementos.\\ \\
%
La justificaci\'on de elegir esta estructura para la tabla es que de esta manera
pondremos como primera fila a todos los elementos del c\'odigo.\\

Para referirnos individualmente a cada elemento de la tabla lo haremos
utilizando la siguiente notaci\'on $Tb(\mathcal{C}[n,m])_{i,j}$ donde 
$i=0,\dots,q^{n-m}-1$ nos indica la fila y $j=0,\dots,q^m-1$ nos
indica la columna.
\begin{definicion}[Fila $0$ de la tabla]
\ \\
La fila $0$ de la tabla estar\'a formada por todos los elementos del c\'odigo
$\mathcal{C}[n,m]$ en cualquier orden, pero con la condici\'on de que el primer
elemento sea el $0$.
\end{definicion}
%
%
\begin{definicion}[Fila $i$ de la tabla]
\ \\
Supondremos que ya hemos construido todas las columnas hasta la $i-1$ inclusive.
Entonces la construcci\'on de la fila $i$ la haremos en dos pasos:
\begin{enumerate}
\item Eligiremos una palabra de $\mathbb{F}^{^n}_q$ que no este en ninguna de
las filas anteriores y la situaremos en $Tb(\mathcal{C}[n,m])_{i,0}$.
\item Para calcular el resto de los elementos de la fila lo haremos de la
siguiente forma:
\begin{displaymath}
Tb(\mathcal{C}[n,m])_{i,j} = Tb(\mathcal{C}[n,m])_{i,0}+
Tb(\mathcal{C}[n,m])_{0,j}\qquad j=1,\dots,q^m-1
\end{displaymath}
\end{enumerate}
\end{definicion}
%
%
\begin{definicion}[Tabla est\'andar de un c\'odigo]
\ \\
Llamaremos \textbf{``tabla est\'andar''} de un c\'odigo a una tabla construida
de la manera anterior.
\end{definicion}


%
% ESTRUCTURA DE UNA TABLA ESTANDAR
%

%
% CONSTRUCCION DE LA TABLA DE UN CODIGO
%

\section{Estructura de una tabla est\'andar}

Ya hemos visto como se construye la tabla de un c\'odigo, pero la 
cons\-tru\-cci\'on que hemos hecho no nos asegura que en la tabla esten todos 
los elementos de $\mathbb{F}^{^n}_q$.
%
%
\begin{teorema}
\ \\
En una tabla est\'andar para un c\'odigo $\mathcal{C}[n,m]$ aparecen todas las
palabras de $\mathbb{F}^{^n}_q$ una sola vez.
\end{teorema}
\underline{\textbf{Demostraci\'on}}:\\
Dados dos elementos cualesquiera de la tabla:
\begin{eqnarray*}
Tb(\mathcal{C}[n,m])_{i,k}&=&Tb(\mathcal{C}[n,m])_{i,0}+
Tb(\mathcal{C}[n,m])_{0,k}\\
Tb(\mathcal{C}[n,m])_{j,l}&=&Tb(\mathcal{C}[n,m])_{j,0}+
Tb(\mathcal{C}[n.m])_{0,l}
\end{eqnarray*}
S\'{\i} ambos elementos son iguales entonces su diferencia es $0$, y como
$\mathcal{C}[n,m]$ es lineal se tiene que $0\in \mathcal{C}[n,m]$ entonces,
y debido tambi\'en a la linealidad del c\'odigo, ambos elementos deben
pertenecer al c\'odigo, con lo cual $i=j$. Debido a la construcci\'on de la
tabla tendremos que $i=j=0$ ya que los elementos del c\'odigo est\'an en la
primera fila. Teniendo en cuenta esto tendremos que:
\begin{displaymath}
Tb(\mathcal{C}[n,m])_{i,k}-Tb(\mathcal{C}[n,m])_{j,l} =
Tb(\mathcal{C}[n,m])_{0,k}-Tb(\mathcal{C}[n,m])_{0,l}
\end{displaymath}
Tenemos dos posibilidades:
\begin{itemize}
\item $k=l$ entonces como $i=j$ tendremos que ambos elementos son el mismo. Es
decir ambos elementos ocupan el mismo lugar en la tabla, son el mismo. Luego no
aparece repetido en la tabla.
\item $k\neq l$ no se puede dar ya que estamos en un c\'odigo lineal y si la
diferencia de dos palabras del c\'odigo es cero entonces ambas palabras han de
ser la misma, con lo cual $k=l$.
\end{itemize}
El n\'umero de elementos de $\mathbb{F}^{^n}_q$ es $q^n$.\\ \\
% 
En la tabla tenemos $q^{n-m}$ filas y $q^m$ columnas entonces el n\'umero de
elementos de la tabla es $q^{n-m}\cdot q^m=q^{n-m+m}=q^n$. Es decir en la tabla
hay tantos elementos como en $\mathbb{F}^{^n}_q$ y como en la tabla no se
repiten elementos entonces tenemos que est\'an todos los elementos.
\begin{flushright}
$\blacksquare$
\end{flushright}
%
%
\begin{lema}[Diferencias horizontales]\label{lem:DifHorizontales}
\ \\
Dado un c\'odigo lineal $\mathcal{C}[n,m]$, con tabla $Tb(\mathcal{C}[n,m])$,
dos elementos de la misma fila difieren en un elemento del c\'odigo.
\end{lema}
\underline{\textbf{Demostraci\'on}}:\\
Dos elementos cualesquiera de la fila $i$-\'esima ser\'an de la forma:
\begin{eqnarray*}
Tb(\mathcal{C}[n,m])_{i,k}&=&Tb(\mathcal{C}[n,m])_{i,0}+
Tb(\mathcal{C}[n,m])_{0,k}\\
Tb(\mathcal{C}[n,m])_{i,l}&=&Tb(\mathcal{C}[n,m])_{i,0}+
Tb(\mathcal{C}[n,m])_{0,l}
\end{eqnarray*}
La diferencia entre ambos elementos es:
\begin{displaymath}
Tb(\mathcal{C}[n,m])_{i,k}-Tb(\mathcal{C}[n,m])_{i,l}=
Tb(\mathcal{C}[n,m])_{0,k}-Tb(\mathcal{C}[n,m])_{0,l}
\end{displaymath}
Por construcci\'on los elementos de la fila cero son elementos del c\'odigo, y,
como este es lineal la diferencia de dos elementos del c\'odigo es otro 
elemento del c\'odigo.
\begin{flushright}
$\blacksquare$
\end{flushright}
%
%
\begin{lema}[Diferencias verticales]\label{lem:DifVerticales}
\ \\
Dado un c\'odigo lineal $\mathcal{C}[n,m]$, con tabla $Tb(\mathcal{C}[n,m])$,
dos elementos distintos de la misma columna difieren en una palabra que no
pertenece al c\'odigo.
\end{lema}
\underline{\textbf{Demostraci\'on}}:\\
Dos elementos cualesquiera, $i\neq j$, de la columna $k$-\'esima ser\'an de
la forma:
\begin{eqnarray*}
Tb(\mathcal{C}[n,m])_{i,k}&=&Tb(\mathcal{C}[n,m])_{i,0}+
Tb(\mathcal{C}[n,m])_{0,k}\\
Tb(\mathcal{C}[n,m])_{j,k}&=&Tb(\mathcal{C}[n,m])_{j,0}+
Tb(\mathcal{C}[n,m])_{0,k}
\end{eqnarray*}
La diferencia entre ambos elementos es:
\begin{displaymath}
Tb(\mathcal{C}[n,m])_{i,k}-Tb(\mathcal{C}[n,m])_{j,k}=
Tb(\mathcal{C}[n,m])_{i,0}-Tb(\mathcal{C}[n,m])_{j,0}
\end{displaymath}
Los elementos del c\'odigo est\'an en la fila cero y
$Tb(\mathcal{C}[n,m])_{i,0}$, $Tb(\mathcal{C}[n,m])_{j,0}$ no pueden, ambos,
estar en la fila cero. Luego al menos uno de ellos no pertenecer\'a al c\'odigo,
y al ser un c\'odigo lineal cualquier operaci\'on que realizemos entre dos
palabras, con una que no pertenezca al c\'odigo, ser\'a otra palabra que no
estar\'a en el c\'odigo.
\begin{flushright}
$\blacksquare$
\end{flushright}
%
%
\begin{proposicion}[Propiedades de la tabla est\'andar]\label{pro:Propiedades}
\ \\
Sea $\mathcal{C}[n,m]$ un c\'odigo lineal, entonces se tiene que:
\begin{eqnarray*}
Tb(\mathcal{C}[n,m])_{i,k}-Tb(\mathcal{C}[n,m])_{i,l}&=&
Tb(\mathcal{C}[n,m])_{j,k}-Tb(\mathcal{C}[n,m])_{j,l}\quad \forall \ i,j,k,l\\
Tb(\mathcal{C}[n,m])_{i,k}-Tb(\mathcal{C}[n,m])_{j,k}&=&
Tb(\mathcal{C}[n,m])_{i,l}-Tb(\mathcal{C}[n,m])_{j,l}\quad \forall \ i,j,k,l
\end{eqnarray*}
donde $Tb(\mathcal{C}[n,m])$ es la tabla est\'andar del c\'odigo.
\end{proposicion}
\underline{\textbf{Demostraci\'on}}:\\
Por construcci\'on de la tabla tenemos:
\begin{eqnarray*}
Tb(\mathcal{C}[n,m])_{i,k}&=& Tb(\mathcal{C}[n,m])_{i,0}+
Tb(\mathcal{C}[n,m])_{0,k}\\
Tb(\mathcal{C}[n,m])_{i,l}&=& Tb(\mathcal{C}[n,m])_{i,0}+
Tb(\mathcal{C}[n,m])_{0,l}\\
Tb(\mathcal{C}[n,m])_{j,k}&=& Tb(\mathcal{C}[n,m])_{j,0}+
Tb(\mathcal{C}[n,m])_{0,k}\\
Tb(\mathcal{C}[n,m])_{j,l}&=& Tb(\mathcal{C}[n,m])_{j,0}+
Tb(\mathcal{C}[n,m])_{0,l}
\end{eqnarray*}
Basta con calcular las diferencias:
\begin{eqnarray*}
Tb(\mathcal{C}[n,m])_{i,k}-Tb(\mathcal{C}[n,m])_{i,l}&=&
Tb(\mathcal{C}[n,m])_{0,k}-Tb(\mathcal{C}[n,m])_{0,l}\\
Tb(\mathcal{C}[n,m])_{j,k}-Tb(\mathcal{C}[n,m])_{j,l}&=&
Tb(\mathcal{C}[n,m])_{0,k}-Tb(\mathcal{C}[n,m])_{0,l}
\end{eqnarray*}
Ambas diferencias son iguales. Y:
\begin{eqnarray*}
Tb(\mathcal{C}[n,m])_{i,k}-Tb(\mathcal{C}[n,m])_{j,k}&=&
Tb(\mathcal{C}[n,m])_{i,0}-Tb(\mathcal{C}[n,m])_{j,0}\\
Tb(\mathcal{C}[n,m])_{i,l}-Tb(\mathcal{C}[n,m])_{j,l}&=&
Tb(\mathcal{C}[n,m])_{i,0}-Tb(\mathcal{C}[n,m])_{j,0}
\end{eqnarray*}
Ambas diferencias son iguales.
\begin{flushright}
$\blacksquare$
\end{flushright}
%
%
Ya hemos visto como se construye la tabla est\'andar de un c\'odigo lineal 
$\mathcal{C}[n,m]$, y que en dicha tabla estan todos los elementos de
$\mathbb{F}^{^n}_q$.\\

Adem\'as de esto podemos ver la relaci\'on que guardan entre s\'{\i} los 
elementos de cada fila.
%
\newpage
%
\begin{teorema}\label{the:ElementosFila}
\ \\
Sea $\mathcal{C}[n,m]$ un c\'odigo lineal y $Tb(\mathcal{C}[n,m])$ su tabla en
forma est\'andar. Sean $Tb(\mathcal{C}[n,m])_{i,k}$ y
$Tb(\mathcal{C}[n,m])_{j,l}$ dos elementos de la tabla, entonces est\'an en la
misma fila s\'{\i} y s\'olo s\'{\i} su diferencia es una palabra del c\'odigo.
\end{teorema}
\underline{\textbf{Demostraci\'on}}:\\
$\Rightarrow |$ Es el lema $\ref{lem:DifHorizontales}$ en la p\'agina
$\pageref{lem:DifHorizontales}$.\\ \\
%
$\Leftarrow |$ Sean dos elementos cualesquiera de $Tb(\mathcal{C}[n,m])$:
\begin{eqnarray*}
Tb(\mathcal{C}[n,m])_{i,k}&=&Tb(\mathcal{C}[n,m])_{i,0}+
Tb(\mathcal{C}[n,m])_{0,k}\\
Tb(\mathcal{C}[n,m])_{j,l}&=&Tb(\mathcal{C}[n,m])_{j,0}+
Tb(\mathcal{C}[n,m])_{0,l}
\end{eqnarray*}
tal que su diferencia sea una palabra del c\'odigo.\\ \\
%
Su diferencia ser\'a:
\begin{equation} \label{eq:Diferencia}
Tb(\mathcal{C}[n,m])_{i,k}-Tb(\mathcal{C}[n,m])_{j,l} = x + y
\end{equation}
donde:
\begin{displaymath}
x=Tb(\mathcal{C}[n,m])_{0,k}-Tb(\mathcal{C}[n,m])_{0,l}\quad e\quad
y=Tb(\mathcal{C}[n,m])_{i,0}-Tb(\mathcal{C}[n,m])_{j,0}
\end{displaymath}
$x$ es una palabra del c\'odigo ya que es la diferencia de dos palabras
del c\'odigo. Es la diferencia de dos palabras de la fila cero.\\ \\
%
Despejando $y$ en $(\ref{eq:Diferencia})$ tenemos:
\begin{displaymath}
y=(Tb(\mathcal{C}[n,m])_{i,k}-Tb(\mathcal{C}[n,m])_{j,l})-x
\end{displaymath}
luego como $\mathcal{C}[n,m]$ es un c\'odigo lineal e $y$ es diferencia de dos
palabras del c\'odigo se tiene que $y\in \mathcal{C}[n,m]$.\\

Supongamos ahora que $i\neq j$ entonces como $Tb(\mathcal{C}[n,m])_{i,0}$ y
$Tb(\mathcal{C}[n,m])_{j,0}$ pertenecen a la misma columna y distinta fila
su diferencia no pertenecer\'a al c\'odigo seg\'un el lema
$\ref{lem:DifVerticales}$. Luego $y\notin \mathcal{C}[n,m]$ por ser la
diferencia de una palabra del c\'odigo y otra que no lo es, llegamos a 
contradicci\'on ya que $y$ s\'{\i} que pertenece al c\'odigo. Entoces $i=j$
luego ambas palabras est\'an en la misma fila.
\begin{flushright}
$\blacksquare$
\end{flushright}
%
\newpage
%
La interpretaci\'on matem\'atica de este teorema es:
\begin{corolario}
\ \\
Sea un c\'odigo lineal $\mathcal{C}[n,m]\subset \mathbb{F}^{^n}_q$. El
 $\mathbb{F}_q$-espacio vectorial $\mathbb{F}^{^n}_q/\mathcal{C}[n,m]$ tiene
tantas clases de equivalencia como filas tiene su tabla est\'andar,
$Tb(\mathcal{C}[n,m])$. Cada clase de equivalencia estar\'a formada por los
elementos de una fila. 
\end{corolario}
\underline{\textbf{Demostraci\'on}}:\\
Es inmediata a partir del teorema $\ref{the:ElementosFila}$ y del hecho de que
dos elementos de la misma clase de equivalencia m\'odulo $\mathcal{C}[n,m]$ 
difieren en un elemento de $\mathcal{C}[n,m]$.
\begin{flushright}
$\blacksquare$
\end{flushright}
\begin{observacion}\ \\
\begin{itemize}
\item Dos filas tienen un elemento en com\'un s\'{\i} y s\'olo s\'{\i} son la
misma fila. Se deduce del hecho de que las clases de equivalencia son disjuntas
entre s\'{\i} y que cada fila forma una clase de equivalencia.
\end{itemize}
\end{observacion}
A partir de ahora en la tabla de un c\'odigo est\'andar eligiremos como primer
elemento de cada fila el elemento que menor peso tenga de la fila.


%
% PROCESAMIENTO DE ERROR
%

%
% PROCESAMIENTO DE ERROR
%

\section{Procesamiento de errores}

Un error es la diferencia entre la palabra recibida y la transmitida. En los
c\'odigos lineales, gracias a su estructrura algebraica, ese error lo podemos
expresar como la diferencia entre dos palabras. Si se transmiti\'o la palabra
$u$ y se recibio la palabra $v$ el error cometido en la transmisi\'on del
mensaje lo podemos expresar como $e=v-u$.
\begin{definicion}[Error cometido para c\'odigos lineales]
\ \\
Sea $\mathcal{C}[n,m]$ un c\'odigo lineal. S\'{\i} se transmiti\'o la palabra
$u$ y se recibi\'o la palabra $v$ diremos que el error que ocurri\'o en la
transmisi\'on es $e=v-u$.
\end{definicion}
Seg\'un esta definici\'on tendremos:
\begin{itemize}
\item La palabra recibida ser\'a: $v=u+e$.
\item La palabra transmitida ser\'a: $u=v-e$.
\end{itemize}
%
%
\begin{proposicion} \label{pro:ProcesarError}
\ \\
Sea $\mathcal{C}[n,m]$ un c\'odigo lineal y sea $P$ un procesador de error para
dicho c\'odigo. S\'{\i} se recibe una palabra $w$ entonces $P$ corregir\'a $w$
como $u=w-e$, donde $e$ es el primer elemento de la fila en la que se encuentra
$w$.
\end{proposicion}
\underline{\textbf{Desmostraci\'on}}:\\
$w=Tb(\mathcal{C}[n,m])_{i,k}$ entonces $e=Tb(\mathcal{C}[n,m])_{i,0}$.\\ \\
%
Por la proposici\'on $\ref{pro:Propiedades}$, en la p\'agina
$\pageref{pro:Propiedades}$, tendremos que:
\begin{displaymath}
Tb(\mathcal{C}[n,m])_{i,k}-Tb(\mathcal{C}[n,m])_{0,k}=
Tb(\mathcal{C}[n,m])_{i,0}-Tb(\mathcal{C}[n,m])_{0,0}
\end{displaymath}
donde $Tb(\mathcal{C}[n,m])_{0,k} = u$ y por contrucci\'on de la tabla tenemos
que: $$Tb(\mathcal{C}[n,m])_{0,0} = (0,\stackrel{n)}\dots,0)$$
De donde se deduce que $w-u=e-0$ entoces $u=w-e$.
\begin{flushright}
$\blacksquare$
\end{flushright}
Seg\'un esta proposici\'on un procesador de error corrige errores en funci\'on
del primer elemento de cada fila, luego los errores que corrige un procesador
de errores son aquellos que est\'an en primer lugar en cada fila.
%
\newpage
%
\begin{ejemplo}
\ \\
Sea $\mathcal{C}[m,m]$ un c\'odigo lineal binario con tabla
$Tb(\mathcal{C}[n,m])$:
\begin{itemize}
\item $Tb(\mathcal{C}[n,m])_{1,0}=100000$ nos indica que con esta tabla se
pueden corregir errores de peso uno en los que el error est\'e en el primer bit.
\item $Tb(\mathcal{C}[n,m])_{2,0}=010000$ nos indica que con esta tabla se
pueden corregir errores de peso uno en los que el error est\'e en el segundo
bit.
\item $Tb(\mathcal{C}[n,m])_{3,0}=001000$ nos indica que con esta tabla se
pueden corregir errores de peso uno en los que el error est\'e en el tercer bit.
\item $Tb(\mathcal{C}[n,m])_{4,0}=000100$ nos indica que con esta tabla se
pueden corregir errores de peso uno en los que el error est\'e en el cuarto bit.
\item $Tb(\mathcal{C}[n,m])_{5,0}=000010$ nos indica que con esta tabla se
pueden corregir errores de peso uno en los que el error est\'e en el quinto
bit.
\item $Tb(\mathcal{C}[n,m])_{6,0}=000001$ nos indica que con esta tabla se
pueden corregir errores de peso uno en los que el error est\'e en el \'ultimo
bit.
\item $Tb(\mathcal{C}[n,m])_{7,0}=100001$ nos indica que con esta tabla se
pueden corregir errores de peso dos en los que el error est\'e en el primer y
\'ultimo bit.
\end{itemize}
\begin{flushright}
$\blacksquare$
\end{flushright}
\end{ejemplo}
%
%
\begin{proposicion}\label{pro:Cercania}
\ \\
Sea $\mathcal{C}[n,m]\subset \mathbb{F}^{^n}_q$ un c\'odigo lineal
con una tabla est\'andar
$Tb(\mathcal{C}[n,m])$ en la que el primer elemento de cada fila es el de
menor peso de dicha fila. Sea $Tb(\mathcal{C}[n,m])_{i,k}$ una palabra
cualquiera de la tabla entonces se tiene que:
\begin{displaymath}
d(Tb(\mathcal{C}[n,m])_{i,k},Tb(\mathcal{C}[n,m])_{0,k})\leq
d(Tb(\mathcal{C}[n,m])_{i,k},Tb(\mathcal{C}[n,m])_{0,j}) 
\end{displaymath}
donde $j=0,\dots,q^{m}-1$.
\end{proposicion}
\underline{\textbf{Demostraci\'on}}:\\
Tenemos por definici\'on de distancia que:
\begin{displaymath}
d(Tb(\mathcal{C}[n,m])_{i,k},Tb(\mathcal{C}[n,m])_{0,j}) =
w(Tb(\mathcal{C}[n,m])_{i,k}-Tb(\mathcal{C}[n,m]_{0,j}))
\end{displaymath}
Seg\'un varia $j$ tenemos que $T(\mathcal{C}[n,m])_{0,j}$ varia en
$\mathcal{C}[n,m]$, luego por las propiedades de la tabla tendremos que
$Tb(\mathcal{C}[n,m])_{i,k}-Tb(\mathcal{C}[n,m])_{0,j}$ varia en el
conjunto formado por los elementos de la fila $i$. Luego el problema de
encontrar el m\'{\i}nimo de:
\begin{displaymath}
w(Tb(\mathcal{C}[n,m])_{i,k}-Tb(\mathcal{C}[n,m])_{0,j})
\end{displaymath}
equivale a encontrar la palabra de m\'{\i}nimo peso de la fila $i$, pero por
cons\-trucci\'on dicha palabra es $Tb(\mathcal{C}[n,m])_{i,0}$.
\begin{displaymath}
\min_{j=0,\dots,q^{m}-1}\{\ w(Tb(\mathcal{C}[n,m])_{i,k}-
Tb(\mathcal{C}[n,m])_{0,j})\ \} = Tb(\mathcal{C}[n,m]_{i,0})
\end{displaymath}
Luego:
\begin{displaymath}
d(Tb(\mathcal{C}[n,m])_{i,k},Tb(\mathcal{C}[n,m])_{0,j})\geq
w(Tb(\mathcal{C}[n,m])_{i,0})\quad j=0,\dots,q^{m}-1
\end{displaymath}
Pero como $Tb(\mathcal{C}[n,m])_{i,k}=Tb(\mathcal{C}[n,m])_{i,0}+
Tb(\mathcal{C}[n,m])_{0,k}$, o lo que es lo mismo $Tb(\mathcal{C}[n,m])_{i,0}=
Tb(\mathcal{C}[n,m])_{i,k}-Tb(\mathcal{C}[n,m])_{0,k}$ tenemos entonces:
\begin{displaymath}
d(Tb(\mathcal{C}[n,m])_{i,k},Tb(\mathcal{C}[n,m])_{0,j})\geq
d(Tb(\mathcal{C}[n,m])_{i,k},Tb(\mathcal{C}[n,m]_{0,k}))
\end{displaymath}
para $j=0,\dots,q^{m}-1$.
\begin{flushright}
$\blacksquare$
\end{flushright}

\subsection{Correcci\'on de errores}

Seg\'un la proposici\'on $\ref{pro:ProcesarError}$ para corregir un error
tendremos que realizar los siguientes pasos:
\begin{itemize}
\item Localizar la palabra recibida en una tabla est\'andar.
\item Suponiendo que la palabra recibida es $Tb(\mathcal{C}[6,3])_{i,k}$ 
entonces tomaremos como palabra transmitida la palabra que est\'e en la misma
columna pero en la fila cero, $Tb(\mathcal{C}[6,3])_{0,k}$. Siendo el error
cometido el primer elemento de la fila en la que se encuentre la palabra
recibida, $Tb(\mathcal{C}[6,3])_{i,0}$.
\item El n\'umero de errores que corrige viene dado por el n\'umero de filas
que posee la tabla del c\'odigo, y los patrones de error que corrige son los
primeros elementos de cada fila.
\item La proposici\'on $\ref{pro:Cercania}$ nos indica que corregiremos siempre
por la palabra m\'as cercana del c\'odigo, seg\'un la distancia de Hamming.
\end{itemize}


%
% ERRORES QUE SE CORRIGEN ADECUADAMENTE
%

%
% ERRORES QUE SE CORRIGEN ADECUADAMENTE
%

%
\newpage
%
\section{Errores que se corrigen adecuadamente}

S\'{\i} tenemos un c\'odigo lineal $\mathcal{C}[n,m]\subset \mathbb{F}^{^n}_q$
con una tabla est\'andar $Tb(\mathcal{C}[n,m])$, en la cual hay $q^{n-m}$ filas
y $q^m$ columnas, ya hemos visto como corregir errores. Los errores los
corregiremos sumando el primer elemento, de la fila en la que est\'a la palabra
recibida, a dicha palabra. Luego los patrones de error que se corrigen vienen
dados por los primeros elementos de cada fila.\\ 

Debido al hecho de que siempre corregiremos por la palabra del c\'odigo m\'as
cercana a la palabra recibida se deduce que ``s\'olo debe haber una palabra
del c\'odigo'' que tenga distancia m\'{\i}nima a la palabra recibida. En caso
contrario no sabriamos cual de las palabras elegir como correcta, se dice 
entonces que no distinguiriamos el error.\\

La proposici\'on $\ref{pro:Cercania}$ nos indica que utilizando tablas
est\'andar tenemos solucionado este problema, pero no podemos corregir
adecuadamente todos los errores mediante una tabla est\'andar.\\ \\
%
El siguiente teorema nos indica cuando se distinguen dos errores.
%
%
\begin{teorema}\label{the:DistintaFila}
\ \\
Sea $\mathcal{C}[n,m]$ un c\'odigo lineal y 
sea $P$ un procesador de error para dicho c\'odigo. Sea $S=\{e_1,\dots,e_s\}$
un conjunto de patrones de error. El procesador de error $P$ puede distinguir
estos patrones de error s\'{\i} y s\'olo s\'{\i} cada patr\'on est\'a en una
fila distinta de la de los otros patrones (en una tabla del c\'odigo).
\end{teorema}
\underline{\textbf{Demostraci\'on}}:\\
$\Rightarrow |$ Sea $e$ un patr\'on de error y $v\neq e$ y que est\'e en la
misma fila que $e$. Entonces tendremos que $u=v-e$, con $u\in \mathcal{C}[n,m]$
y tal que $u\neq 0$. S\'{\i} el procesador de error corrige el error $e$
entonces corregir\'a $v$ a $u$.\\ \\
%
Supongamos que el procesador $P$ corrige el error $v$, entonces corregir\'a $v$ 
a $0$, lo cual no es posible. Luego no existir\'a un procesador de error que
distinga dos errores situados en la misma fila.\\ \\
%
$\Leftarrow |$ Sean $e_i$ y $e_j$ con $i\neq j$, entonces podemos construir una
tabla est\'andar con $e_i$ y $e_j$ como primeros elementos de dos filas.\\ \\
%
Con dicha tabla se tiene que para toda palabra $u$ del c\'odigo el procesador
corregir\'a las palabras $u+e_i$ y $u+e_j$ a $u$.\\ \\
%
Luego el procesador de error $P$ puede distinguir, corregir, dos patrones de
error cuando esten en distinta fila, $i\neq j$.
\begin{flushright}
$\blacksquare$
\end{flushright}
Seg\'un todo lo que hemos visto podremos corregir tantos errores como clases
de equivalencia tenga el $\mathbb{F}_q$-espacio vectorial
$\mathbb{F}_q^{^n}/\mathcal{C}[n,m]$, ya que un procesador de error, $P$, 
distingue dos errores s\'{\i} y s\'olo s\'{\i} estan en distinta fila, es decir,
s\'{\i} y s\'olo s\'{\i} pertenecen a distinta clase de equivalencia.

\subsection{M\'etodo pr\'actico}

Con lo que hemos visto un m\'etodo ``pr\'actico'' para detectar y corregir
errores ser\'a el siguiente:\\ \\
%
Dado un c\'odigo lineal $\mathcal{C}[n,m]$:
\begin{itemize}
\item Construir una tabla est\'andar, utilizando como primer elemento de cada
fila el tipo de error que queremos detectar y corregir.
\item Una vez recibida una palabra comparar con las $q^n$ palabras de la tabla.
\item Seleccionar como palabra correcta la palabra que se encuentre en la fila
cero y en la columna de la palabra recibida.
\end{itemize}
Este m\'etodo aunque pr\'actico es poco util, ya que hay que realizar $q^n$
comparaciones y, a medida que aumenta $n$ $q^n$ crece aun m\'as rapidamente.


%
% CORRECCION Y DETECCION DE ERRORES MEDIANTE LA MATRIZ DE CHEQUEO
%

%
% DETECCION Y CORRECCION DE ERRORES MEDIANTE LA MATRIZ DE CHEKEO
%

\section{Detecci\'on y correcci\'on de errores mediante la matriz de control}

Seg\'un el m\'etodo que hemos visto antes necesitariamos almacenar una tabla
de $q^n$ elementos y realizar, a lo sumo, $q^n$ comparaciones.\\

Esto no es pr\'actico ya que si utilizamos c\'odigos con palabras de longitud
algo grande la cantidad de memoria necesaria para almacenar la tabla ser\'a
bastante grande, as\'{\i} como el n\'umero de comparaciones necesarias. Por
ejemplo para $q=2$ y $n=25$ tendremos que almacenar $2^{25}=33.554.432$
palabras, lo cual requiere una gran inversi\'on en memoria y gran cantidad
de tiempo para realizar una comparaci\'on.\\

Para evitar las $q^n$ comparaciones y el almacenamiento de toda la tabla
est\'andar podemos utilizar la matriz de control del c\'odigo lineal.
\begin{proposicion}
\ \\
Sea $\mathcal{C}[n,m]$ un c\'odigo lineal y $H$ su matriz de control. 
$u,v\in \mathbb{F}_q^{^n}$ est\'an en la misma fila s\'{\i} y s\'olo s\'{\i}
$H\cdot u^t = H\cdot v^t$.
\end{proposicion}
\underline{\textbf{Demostraci\'on}}:\\
Como ya hemos visto antes $u$ y $v$ est\'an en la misma fila s\'{\i} y s\'olo 
s\'{\i} se verifica que $u-v\in \mathcal{C}[n,m]$ entonces se tiene que:
\begin{displaymath}
H\cdot (u-v)^t = 0 \Longleftrightarrow H\cdot u^t=H\cdot v^t
\end{displaymath}
\begin{flushright}
$\blacksquare$
\end{flushright}
Seg\'un esta proposici\'on podemos asignar a cada fila de una tabla est\'andar
un vector constante, el cual unicamente ser\'a nulo para la primera fila de
dicha tabla.
\begin{definicion}[Sindrome de una palabra]
\ \\
Dado un c\'odigo lineal $\mathcal{C}[n,m]$ con matriz de control $H$ llamaremos
\textbf{``sindrome de una palabra''}, $u\in \mathbb{F}_q^{^n}$, al siguiente
vector $(H\cdot u^t)^t$.
\end{definicion}
Como todas las palabras de una fila tienen el mismo sindrome y el error que
corregiremos ser\'a el primer elemento de cada fila entonces unicamente
almacenaremos el primer elemento de cada fila junto con su sindrome. De esta
forma necesitaremos menos memoria para almacenar la tabla.\\ \\
%
El algoritmo para corregir un error es el siguiente:
\begin{itemize}
\item Recibimos la palabra $v$.
\item Calculamos su sindrome $(H\cdot v^t)^t$.
\item Buscamos en la nueva tabla el sindrome anterior y vemos cual de los
elementos tiene dicho sidrome, supongamos que $e_i$ es tal que:
$$(H\cdot e_i^t)^t=(H\cdot v^t)^t$$ 
\item La palabra correcta ser\'a $u=v-e_i$.
\end{itemize}
Ademas de necesitar menos memoria para almacenar la tabla de los sindromes
tambi\'en necesitamos menos tiempo para corregir un error ya que unicamente
tendremos que hacer $q^{n-m}$ comparaciones, en lugar de las $q^n$ requeridas
por la utilizaci\'on de la tabla est\'andar.


%
% CONDICION PARA LA CORRECCION DE ERRORES SIMPLES
%

%
% CONDICION PARA LA CORRECCION DE ERRORES SIMPLES
%

\section{Condici\'on para la correcci\'on de errores de peso uno}

Los errores de peso uno son una base del $\mathbb{F}_q$-espacio vectorial
$\mathbb{F}^{^n}_q$. Por ejemplo para $n=3$ y $q=2$:
\begin{eqnarray*}
e_1&=&(1,0,0)\\
e_2&=&(0,1,0)\\
e_3&=&(0,0,1)
\end{eqnarray*}
Sea $H$ la matriz de control de un c\'odigo lineal:
\begin{displaymath}
\mathcal{C}[n,m]\subset \mathbb{F}^{^n}_2 =<e_1,\dots,e_n>
\end{displaymath}
donde los $\{e_i\}_{i=1}^n$ son los errores de peso uno. Seg\'un el teorema
$\ref{the:DistintaFila}$ un procesador de error $P$ distingue dos errores $e_i$
y $e_j$ $\Longleftrightarrow$ est\'an en diferente fila $\Longleftrightarrow$
$H \cdot e_i^t \neq H\cdot e_j^t$. Ahora bien $H\cdot e_i^t=i$-\'esima columna
de la matriz de control $H$, luego de esto se deduce:
\begin{corolario}
\ \\
Un c\'odigo lineal binario puede distinguir todos los errores de peso uno
s\'{\i} y s\'olo s\'{\i} la matriz de control tiene todas las columnas distintas
y no nulas.
\begin{flushright}
$\blacksquare$
\end{flushright}
\end{corolario}
%
Este corolario lo podemos generalizar a canales no binarios:
\begin{teorema}[Condici\'on necesaria y suficiente]
\ \\
Sea $\mathcal{C}[n,m]$ un c\'odigo lineal con matriz de control $H$. El c\'odigo
puede corregir todos los errores de peso uno s\'{\i} y s\'olo s\'{\i} todas las
columnas de la matriz de control son no nulas y ninguna columna es un multiplo
de las restantes columnas\footnote{Son linealmente independientes dos a dos.}.
\end{teorema}
\underline{\textbf{Demostraci\'on}}:\\
Por el teorema $\ref{the:DistintaFila}$ los errores de peso uno se distinguen y
corrigen $\Longleftrightarrow$ estan en diferente fila en una tabla est\'andar
del c\'odigo $\Longleftrightarrow$ tienen distintos sindromes.
%
\newpage
%
Sea $e$ un error de peso uno, es decir una palabra con todas sus componentes
nulas excepto una. Entonces podemos distiguir dos casos:
\begin{itemize}
\item Canal binario, sobre $\mathbb{F}_2$.\\ \\
%
Tendremos que $e=e_i$ para alg\'un $i$, luego $H\cdot e_i^t=i$-\'esima columna
de la matriz de control $H$.
\item Canal no binario, sobre $\mathbb{F}_q$, con $q\neq 2$.\\ \\
%
Sea $a_i\in \mathbb{F}_q$ la componente no nula de $e$, entonces tendremos que
$H\cdot e^t=a_i\cdot c_i$, donde $c_i$ es la $i$-\'esima columna de la matriz
de control del c\'odigo $H$.
\end{itemize}
Luego los sindromes de los errores de peso uno son multiplos de las columnas
de la matriz de control, $H$.\\ \\
%
Como el c\'odigo distingue y corrige todos los errores de peso uno
$\Longleftrightarrow$ tienen distinto sindrome, y los sindromes%
\footnote{De los errores de peso uno.} son multiplos de las columnas de la
matriz de control se tiene que:
\begin{quote}
El c\'odigo distingue y corrige todos los errores de peso uno
$\Longleftrightarrow$ no hay ninguna columna que sea multiplo de las
dem\'as.
\end{quote}
Recordar que la palabra cero es multiplo de todas, basta con multiplicar
cualquier palabra por el escalar cero.
\begin{flushright}
$\blacksquare$
\end{flushright}
Seg\'un este teorema podemos ver si un c\'odigo distingue y corrige todos
los errores de peso uno s\'olo con mirar las columnas de la matriz de control
y comprobar s\'{\i} son linealmente independientes dos a dos.


%
% DISTANCIA MINIMA Y MATRIZ DE CONTROL
%

%
% DISTANCIA MINIMA Y MATRIZ DE CONTROL
%

\section{Relaci\'on entre la distancia m\'{\i}nima y la matriz de control}

En el teorema $\ref{the:Correccion}$, en la p\'agina $\pageref{the:Correccion}$,
vimos que un c\'odigo puede corregir todos los errores de peso $t$ s\'{\i} y
s\'olo s\'{\i} $d_{min}\geq 2\cdot t+1$.
%
\begin{teorema}\label{the:DistMin}
\ \\
Sea $\mathcal{C}[n,m]$ un c\'odigo lineal con matriz de control $H$. El c\'odigo
tiene distancia m\'{\i}nima, $d_{min}>d$ s\'{\i} y s\'olo s\'{\i} no existen
$d$ columnas de la matriz $H$ que sean linealmente dependientes. Es decir
cualquier conjunto de $d$ columnas es linealmente independiente.
\end{teorema}
\underline{\textbf{Demostraci\'on}}:\\
En la demostraci\'on utilizaremos los siguientes resultados:
\begin{enumerate}
\item Como el c\'odigo es lineal su distancia m\'{\i}nima es el m\'{\i}nimo de 
los pesos de las palabras, no nulas, del c\'odigo.
\item Una palabra $u$ pertenece al c\'odigo s\'{\i} y s\'olo s\'{\i}
$H\cdot u^t=0$.
\item Sea $u=(u_1,\dots,u_n)$ verificando $H\cdot u^t=0$ entonces las columnas
de $H$, $H_i$, donde $i$ es tal que $u_i\neq 0$ son linealmente dependientes.
\end{enumerate}
$\Rightarrow |$ Supongamos que existe una palabra no nula del c\'odigo,
$u=(u_1,\dots,u_n)$, tal que sea de peso menor o igual que $d$. Dado que 
$u\in \mathcal{C}[n,m]$ tendremos que $H\cdot u^t=0$, es decir:
\begin{equation}\label{eq:DepenI}
H\cdot u^t = u_1\cdot H_1+\cdots +u_n\cdot H_n = 0
\end{equation}
Como $u$ es de peso, a lo sumo, $d$ entonces tendr\'a, como mucho, $d$
componentes no nulas, por comodidad supondremos que son las $d$ primeras. De
esta forma la ecuaci\'on $(\ref{eq:DepenI})$ nos queda:
\begin{displaymath}
H\cdot u^t = u_1\cdot H_1+\cdots + u_d\cdot H_d = 0
\end{displaymath}
Como $\{u_i\}_{i=1}^d$ son no nulos entonces tenemos $d$ columnas linealmente
dependientes en $H$.\\

Hemos llegado a esta conclusi\'on suponiendo la existencia de un elemento de
peso menor o igual que $d$ en el c\'odigo, pero por hip\'otesis esto no se puede
dar ya que $d_{min}>d$, entonces en la matriz de control $H$ no existen $d$ 
columnas linealmente dependientes.\\ \\
%
$\Leftarrow |$ Supongamos que $H$ tiene $d$ columnas linealmente dependientes, y
supongamos que son las $d$ primeras, por comodidad. Luego existen
$\lambda_1,\dots,\lambda_d \in \mathbb{F}_q$, no todos nulos, tales que:
\begin{displaymath}
\lambda_1 \cdot H_1+\dots+\lambda_d \cdot H_d = 0
\end{displaymath}
%
\newpage
%
Tomemos $u_i = \lambda_i$ para $i=1,\dots,d$ y $u_i=0$ para $i=d+1,\dots,n$ y
construyamos $u=(u_1,\dots,u_n)$. Por construcci\'on tendremos que
$H\cdot u^t=0$, luego $u\in \mathcal{C}[n,m]$ y $u$ es no nula ya que en sus
$d$ primeras componentes hay, por lo menos, una que no es nula. Es decir $u$
es una palabra de peso, a lo sumo, $d$.\\

Luego si $H$ tiene $d$ columnas linealmente dependientes entonces el c\'odigo
tiene palabras de, a lo sumo, peso $d$. Como por hip\'otesis $H$ no tiene
$d$ columnas linealmente dependientes entonces el c\'odigo tiene palabras de
peso mayor que $d$, es decir su distancia m\'{\i}nima es mayor que $d$,
$d_{min}>d$.
\begin{flushright}
$\blacksquare$
\end{flushright}
Este teorema nos indica lo bueno que es un c\'odigo lineal en virtud de las
columnas de su matriz de control. El sistema para elegir buenos c\'odigos
consiste en elegir bien las columnas de su matriz de control y tomar como
c\'odigo el determinado por su matriz de control.


%
% EJEMPLOS
%

%
% EJEMPLOS
%

\section{Ejemplos}

\subsection{C\'odigo del bit de control de paridad}

Este c\'odigo es un c\'odigo de $2^7=128$ palabra palabras. Es un subconjunto de
$\mathbb{F}^{^8}_2$ el cual tiene $2^8=256$ palabras, con lo cual el n\'umero
de patrones de error que tendremos ser\'a $2^8-2^7=128$.

\subsubsection{Tabla est\'andar del c\'odigo}

Como el c\'odigo tiene $2^7=128$ palabras la tabla tendr\'a $2^7=128$ columnas
y el n\'umero de filas ser\'a $2^{8-7}=2$.\\

Debido a la gran cantidad de palabras de este c\'odigo no construiremos su
tabla est\'andar.

\subsubsection{Sindromes}

Sea $H$ la matriz de control, en forma est\'andar, del c\'odigo:
\begin{displaymath}
\left( \begin{array}{cccccccc}
1&1&1&1&1&1&1&1
\end{array} \right)
\end{displaymath}

\begin{table}[!h]
\begin{displaymath}
\begin{array}{|c|c|}
\hline
Errores & Sindromes \\
\hline
00000000 & 0 \\
\hline
10000000 & 1 \\
\hline
\end{array}
\end{displaymath}
\caption{Tabla de sindromes del c\'odigo del bit de control de paridad.}
\end{table}

\subsection{C\'odigo de triple repetici\'on}

Este c\'odigo es un c\'odigo de $2^1=2$ palabras. Es un subconjunto de
$\mathbb{F}^{^3}_2$ el cual tiene $2^3=8$ palabras, con lo cual el n\'umero
de patrones de error que tendremos ser\'a $2^3-2^1=6$.

\subsubsection{Tabla est\'andar del c\'odigo}

Como el c\'odigo tiene $2^1=2$ palabras la tabla tendr\'a $2^1=2$ columnas y
el n\'umero de filas ser\'a $2^{3-1}=4$.\\

La primera fila estar\'a formada por las palabras del c\'odigo con la
condici\'on de que el $0$ sea el primer elemento. Luego la primera fila ser\'a:
\begin{displaymath}
\begin{array}{cc}
000&111
\end{array}
\end{displaymath}
Para la segunda fila cogeremos un elemento de $\mathbb{F}^{^3}_2$ que no este en
la fila cero, y el primer elemento de cada fila ha de ser el de m\'{\i}nimo
peso de dicha fial. El elemento $100$ no esta en la fila cero, luego elegimos
ese elemento como primer elemento de la fila uno ya que es de peso uno y no es
posible encontrar otro de peso menor. El resto de elementos de la fila ser\'an:
\begin{displaymath}
Tb(\mathcal{C}[3,1])_{1,i}=Tb(\mathcal{C}[3,1])_{1,0}+Tb(\mathcal{C}[3,1])_{0,i}
\quad i=0,1
\end{displaymath}
Luego la segunda fila ser\'a:
\begin{displaymath}
\begin{array}{cc}
100&011
\end{array}
\end{displaymath}
Siguiendo el mismo razonamiento elegiremos como primer elemento de la fila tres
un elemento que no haya aparecido en las filas anteriores y de peso m\'{\i}nimo,
por ejemplo $010$ y siguiendo el razonamiento anterior completaremos la tabla.
\begin{eqnarray*}
Tb(\mathcal{C}[3,1])_{2,0}&=&010\\
Tb(\mathcal{C}[3,1])_{3,0}&=&001
\end{eqnarray*}
\begin{table}[!h]
\begin{displaymath}
\begin{array}{|c|c|}
\hline
000&111\\
\hline
100&011\\
\hline
010&101\\
\hline
001&110\\
\hline
\end{array}
\end{displaymath}
\caption{Tabla est\'andar del c\'odigo de triple repetici\'on.}\label{tab:TablaII}
\end{table}

\subsubsection{Correcci\'on de errores}

Supongamos que recibimos la palabra $011$, que no pertenece al c\'odigo. Para
corregir el c\'odigo haremos:
\begin{itemize}
\item Localizamos el lugar de dicha palabra en la tabla.
\begin{displaymath}
Tb(\mathcal{C}[3,1])_{1,1}=011
\end{displaymath}
\item Elegimos como palabra correcta la palabra que est\'e en la misma columna y
en la fila cero.
\begin{displaymath}
Tb(\mathcal{C}[3,1])_{0,1}=111
\end{displaymath}
\end{itemize}
El error cometido vendr\'a dado por el primer elemento de la fila en la que se
encuentre la palabra recibida:
\begin{displaymath}
Tb(\mathcal{C}[3,1])_{1,0}=100
\end{displaymath}
Luego en la transmisi\'on se ha cometido un error de peso $w(100)=1$ en el
primer bit.

\subsubsection{Sindromes}

Supondremos que queremos corregir los mismos errores que se corrigen con la
tabla $\ref{tab:TablaII}$.\\ \\
%
Sea $H$ la matriz de control, en forma est\'andar, del c\'odigo:
\begin{displaymath}
H=\left( \begin{array}{ccc}
1&1&0\\
1&0&1
\end{array} \right)
\end{displaymath}

\begin{table}[!h]
\begin{displaymath}
\begin{array}{|c|c|}
\hline
Errores & Sindromes \\
\hline
000 & 00 \\
\hline
100 & 11 \\
\hline
010 & 10 \\
\hline
001 & 01 \\
\hline
\end{array}
\end{displaymath}
\caption{Tabla de sindromes del c\'odigo de triple repetici\'on.}\label{tab:TabSindromes}
\end{table}

Para corregir errores supongamos que recibimos la palabra $101$, calculamos
su sindrome que ser\'a $10$. Entonces utilizando la tabla
$\ref{tab:TabSindromes}$ buscamos el error que tiene sindrome $10$, dicho
error es $010$. Luego la palabra transmitida ser\'a $101-010=111$. Recordar
que $-1=1$ en $\mathbb{F}_2$.

\subsection{C\'odigo de triple control}

Este c\'odigo es un c\'odigo de $2^3=8$ palabras. Es un subconjunto de
$\mathbb{F}^{^6}_2$ el cual tiene $2^6=64$ palabras, con lo cual el n\'umero
de patrones de error que tendremos ser\'a $2^6-2^3=56$.

\subsubsection{Tabla est\'andar del c\'odigo}

Como el c\'odigo tiene $2^3=8$ palabras la tabla tendr\'a $2^3=8$ columnas y
el n\'umero de filas ser\'a $2^{6-3}=8$.\\ 

La primera fila estar\'a formada por las palabras del c\'odigo con la
condici\'on de que el $0$ sea el primer elemento. Luego la primera fila
ser\'a:
\begin{displaymath}
\begin{array}{cccccccc}
000000&100110&010101&001011&111000&011110&101101&110011
\end{array}
\end{displaymath}
Para la segunda fila cogeremos un elemento de $\mathbb{F}^{^6}_2$ que no este
en la fila cero, y el primer elemento de cada fila ha de ser el de
m\'{\i}nimo peso de dicha fila. El elemento $100000$ no est\'a en la fila 
cero,
luego elegimos ese elemento como primer elemento de la fila uno ya que es de
peso uno y no es posible encontrar otro elemento de peso menor. El resto de
elementos de la fila ser\'an:
\begin{displaymath}
Tb(\mathcal{C}[6,3])_{1,i}=Tb(\mathcal{C}[6,3])_{1,0}+Tb(\mathcal{C}[6,3])_{0,
i}
\quad i=1,\dots,7
\end{displaymath}
Luego la segunda fila ser\'a:
\begin{displaymath}
\begin{array}{cccccccc}
100000&000110&110101&101011&011000&111110&001101&010011
\end{array}
\end{displaymath}
%
\newpage
%
Siguiendo el mismo razonamiento escogeremos como primer elemento de la fila 
dos un elemento de $\mathbb{F}^{^6}_2$ que no aparezca en las filas cero y uno. 
Como hay elementos de peso uno que no aparecen en dichas filas elegiremos como
primer elemento de la fila dos $010000$. Y siguiendo el mismo razonamiento
construiremos la fila y eligiremos los siguientes elementos:
\begin{eqnarray*}
Tb(\mathcal{C}[6,3])_{3,0}&=& 001000\\
Tb(\mathcal{C}[6,3])_{4,0}&=& 000100\\
Tb(\mathcal{C}[6,3])_{5,0}&=& 000010\\
Tb(\mathcal{C}[6,3])_{6,0}&=& 000001
\end{eqnarray*}
Para elegir el primer elemento de la fila siete observaremos que todas las
palabras de $\mathbb{F}^{^6}_2$ de peso uno estan en alguna de las filas
anteriores, con lo cual el elemento de la fila siete de menor peso tendr\'a un
peso mayor o igual que dos. El elemento $100001$ no esta en ninguna de las  
filas anteriores con lo cual lo elegiremos como primer elemento de la fila
siete.\\ \\ 
%
%
\begin{table}[!h]
\begin{displaymath}
\begin{array}{|c|c|c|c|c|c|c|c|}
\hline
000000&100110&010101&001011&111000&011110&101101&110011\\
\hline
100000&000110&110101&101011&011000&111110&001101&010011\\
\hline
010000&110110&000101&011011&101000&001110&111101&100011\\
\hline
001000&101110&011101&000011&110000&010110&100101&111011\\
\hline
000100&100010&010001&001111&111100&011010&101001&110111\\
\hline
000010&100100&010111&001001&111010&011100&101111&110001\\
\hline
000001&100111&010100&001010&111001&011111&101100&110010\\
\hline
100001&000111&110100&101010&011001&111111&001100&010010\\
\hline
\end{array}
\end{displaymath}
\caption{Tabla est\'andar del c\'odigo de triple control.}\label{tab:Tabla}
\end{table}
%
%
En la tabla $\ref{tab:Tabla}$ podemos ver una tabla est\'andar para el c\'odigo
de triple repetici\'on. No es la \'unica tabla est\'andar. Para obtener otras
tablas est\'andar bastar\'a con elegir distintos elementos como primer elemento
de cada fila obteniendo las mismas filas, pero en ordenadas de distinta forma.
%
\newpage
%
\subsubsection{Correcci\'on de errores}

Supongamos que recibimos la palabra $111100$, la cual no pertenece al c\'odigo.
Para corregir el error haremos:
\begin{itemize}
\item Localizamos el lugar de dicha palabra en la tabla.
\begin{displaymath}
Tb(\mathcal{C}[6,3])_{4,4}=111100
\end{displaymath}
\item Elegimos como palabra correcta la palabra que est\'e en la misma columna y
en la fila cero.
\begin{displaymath}
Tb(\mathcal{C}[6,3])_{0,4}=111000
\end{displaymath}
\end{itemize}
El error cometido vendr\'a dado por el primer elemento de la fila en la que se
encuentre la palabra recibida:
\begin{displaymath}
Tb(\mathcal{C}[6,3])_{4,0}=000100
\end{displaymath}
Luego en la transmisi\'on se ha cometido un error de peso $w(000100)=1$ en el
cuarto bit.

\subsubsection{Sindromes}

Supondremos que queremos corregir los mismos errores que se corrigen con la
tabla $\ref{tab:Tabla}$.\\ \\
%
Sea $H$ la matriz de control, en forma est\'andar, del c\'odigo:
\begin{displaymath}
H=\left( \begin{array}{cccccc}
1&1&0&1&0&0\\
1&0&1&0&1&0\\
0&1&1&0&0&1
\end{array} \right)
\end{displaymath}

\begin{table}[!h]
\begin{displaymath}
\begin{array}{|c|c|}
\hline
Error  & Sindrome \\
\hline
000000 & 000 \\
\hline
100000 & 110 \\
\hline
010000 & 101 \\
\hline
001000 & 011 \\
\hline
000100 & 100 \\
\hline
000010 & 010 \\
\hline
000001 & 001 \\
\hline
100001 & 111 \\
\hline
\end{array}
\end{displaymath}
\caption{Tabla de sindromes del c\'odigo de triple control.}\label{tab:TablaSindromes}
\end{table}
Obviamente requiere menor costo almacenar la tabla $\ref{tab:TablaSindromes}$
que almacenar la tabla $\ref{tab:Tabla}$.\\

Para corregir errores supongamos que recibimos la palabra $100011$, calculamos
su sindrome que ser\'a $101$. Entonces utilizando la tabla
$\ref{tab:TablaSindromes}$ buscamos que error tiene sindrome $101$, dicho error
es $010000$. Luego la palabra transmitida ser\'a $100011-010000=110011$.
Recordar que $-1=1$ en $\mathbb{F}_2$.


%
% EJERCICIOS
%

%
% EJERCICIOS
%

\section{Ejercicios}

\begin{ejercicio}
\ \\
Dada una palabra cualquiera del c\'odigo de triple control, comprobar que es lo
que falla s\'{\i} se produce un fallo en cualquiera de los seis bits.
\end{ejercicio}
\textbf{\underline{Soluci\'on:}}\\
Sea $110011$ una palabra perteneciente al c\'odigo de triple control.\\ \\
%
Todas las palabras del c\'odigo son de la forma:
\begin{displaymath}
\mathcal{C}[6,3]=\{\ (u_1,u_2,u_3,u_1+u_2,u_1+u_3,u_2+u_3)\ u_i \in \mathbb{F}_2
\ \}
\end{displaymath}
Llamaremos a los bit de control:
\begin{displaymath}
c_1 = u_1+u_2\quad c_2=u_1+u_3\quad c_3=u_2+u_3
\end{displaymath}
Introduciremos un error en cada bit y veremos que bits de control fallan,
suponiendo siempre que los bits de informaci\'on son correctos.\\ 
\begin{center}
\begin{tabular}{|c|c|c|c|}
\hline
Bit de error & Palabra erronea & Bits de control & Bits correctos\\
\hline
Primer bit & $010011$ & $011$ & $101$ \\
\hline
Segundo bit & $100011$ & $011$ & $110$ \\
\hline
Tercer bit & $111011$ & $011$ & $000$ \\
\hline
Cuarto bit & $110111$ & $111$ & $011$ \\
\hline
Quinto bit & $110001$ & $001$ & $011$ \\
\hline
Sexto bit & $110010$ & $010$ & $011$ \\
\hline
\end{tabular}
\end{center}
\begin{flushright}
$\blacksquare$
\end{flushright}

