%
% DISTANCIA MINIMA Y MATRIZ DE CONTROL
%

\section{Relaci\'on entre la distancia m\'{\i}nima y la matriz de control}

En el teorema $\ref{the:Correccion}$, en la p\'agina $\pageref{the:Correccion}$,
vimos que un c\'odigo puede corregir todos los errores de peso $t$ s\'{\i} y
s\'olo s\'{\i} $d_{min}\geq 2\cdot t+1$.
%
\begin{teorema}\label{the:DistMin}
\ \\
Sea $\mathcal{C}[n,m]$ un c\'odigo lineal con matriz de control $H$. El c\'odigo
tiene distancia m\'{\i}nima, $d_{min}>d$ s\'{\i} y s\'olo s\'{\i} no existen
$d$ columnas de la matriz $H$ que sean linealmente dependientes. Es decir
cualquier conjunto de $d$ columnas es linealmente independiente.
\end{teorema}
\underline{\textbf{Demostraci\'on}}:\\
En la demostraci\'on utilizaremos los siguientes resultados:
\begin{enumerate}
\item Como el c\'odigo es lineal su distancia m\'{\i}nima es el m\'{\i}nimo de 
los pesos de las palabras, no nulas, del c\'odigo.
\item Una palabra $u$ pertenece al c\'odigo s\'{\i} y s\'olo s\'{\i}
$H\cdot u^t=0$.
\item Sea $u=(u_1,\dots,u_n)$ verificando $H\cdot u^t=0$ entonces las columnas
de $H$, $H_i$, donde $i$ es tal que $u_i\neq 0$ son linealmente dependientes.
\end{enumerate}
$\Rightarrow |$ Supongamos que existe una palabra no nula del c\'odigo,
$u=(u_1,\dots,u_n)$, tal que sea de peso menor o igual que $d$. Dado que 
$u\in \mathcal{C}[n,m]$ tendremos que $H\cdot u^t=0$, es decir:
\begin{equation}\label{eq:DepenI}
H\cdot u^t = u_1\cdot H_1+\cdots +u_n\cdot H_n = 0
\end{equation}
Como $u$ es de peso, a lo sumo, $d$ entonces tendr\'a, como mucho, $d$
componentes no nulas, por comodidad supondremos que son las $d$ primeras. De
esta forma la ecuaci\'on $(\ref{eq:DepenI})$ nos queda:
\begin{displaymath}
H\cdot u^t = u_1\cdot H_1+\cdots + u_d\cdot H_d = 0
\end{displaymath}
Como $\{u_i\}_{i=1}^d$ son no nulos entonces tenemos $d$ columnas linealmente
dependientes en $H$.\\

Hemos llegado a esta conclusi\'on suponiendo la existencia de un elemento de
peso menor o igual que $d$ en el c\'odigo, pero por hip\'otesis esto no se puede
dar ya que $d_{min}>d$, entonces en la matriz de control $H$ no existen $d$ 
columnas linealmente dependientes.\\ \\
%
$\Leftarrow |$ Supongamos que $H$ tiene $d$ columnas linealmente dependientes, y
supongamos que son las $d$ primeras, por comodidad. Luego existen
$\lambda_1,\dots,\lambda_d \in \mathbb{F}_q$, no todos nulos, tales que:
\begin{displaymath}
\lambda_1 \cdot H_1+\dots+\lambda_d \cdot H_d = 0
\end{displaymath}
%
\newpage
%
Tomemos $u_i = \lambda_i$ para $i=1,\dots,d$ y $u_i=0$ para $i=d+1,\dots,n$ y
construyamos $u=(u_1,\dots,u_n)$. Por construcci\'on tendremos que
$H\cdot u^t=0$, luego $u\in \mathcal{C}[n,m]$ y $u$ es no nula ya que en sus
$d$ primeras componentes hay, por lo menos, una que no es nula. Es decir $u$
es una palabra de peso, a lo sumo, $d$.\\

Luego si $H$ tiene $d$ columnas linealmente dependientes entonces el c\'odigo
tiene palabras de, a lo sumo, peso $d$. Como por hip\'otesis $H$ no tiene
$d$ columnas linealmente dependientes entonces el c\'odigo tiene palabras de
peso mayor que $d$, es decir su distancia m\'{\i}nima es mayor que $d$,
$d_{min}>d$.
\begin{flushright}
$\blacksquare$
\end{flushright}
Este teorema nos indica lo bueno que es un c\'odigo lineal en virtud de las
columnas de su matriz de control. El sistema para elegir buenos c\'odigos
consiste en elegir bien las columnas de su matriz de control y tomar como
c\'odigo el determinado por su matriz de control.
