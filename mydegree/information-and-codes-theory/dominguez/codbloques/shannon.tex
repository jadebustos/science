%
% EL TEOREMA DE SHANNON
%

%
\newpage
%
\section{El teorema de Shannon}

\emph{Claude Shannon} demostr\'o en $1.948$ que existe una constante llamada
``\emph{capacidad del canal}'', $C(P)$, para cualquier canal sim\'etrico,
binario y con probabilidad $P$ tal que siempre existen c\'odigos de bloques
donde la probabilidad de transmisi\'on correcta est\'a arbitrariamente
pr\'oxima a $C(P)$.
\begin{teorema}[de Shannon, $1.948$]
\ \\
Dado un canal sim\'etrico, binario y con probabilidad $P$ siempre existen
c\'odigos de bloques donde la probabilidad de transmisi\'on correcta est\'a
arbitrariamente pr\'oxima a:
\begin{displaymath}
C(P) = 1 + P \cdot \log_2 P+(1-P)\cdot \log_2 (1-P)
\end{displaymath}
La constante $C(P)$ recibe el nombre de ``\emph{capacidad del canal}''.
\end{teorema}

Supongamos que la probabilidad de error del canal es $P=\frac{1}{2}$ entonces
se tiene que $C(\frac{1}{2})=0$, es decir, la probabilidad de transmisi\'on 
correcta en un canal de este tipo es practicamente nula. Con lo cual no es 
aconsejable trabajar con canales de este tipo.
