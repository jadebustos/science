%
% NOCION DE DISTANCIA
%

\section{Noci\'on de distancia}

Dar una distancia en un c\'odigo equivale a decir cuando dos palabras en el
c\'odigo son diferentes y en ``\emph{cuanto}'' difieren.\\

Dadas dos palabras $a=(a_1,\cdots,a_n)$ y $b=(b_1,\cdots,b_n)$, donde $a,b
\in \mathcal{C}[n,m]\subset \mathbb{F}^{^n}_q$, se tienen las siguientes
definiciones:
\begin{definicion}[Igualdad de palabras]
\ \\
Se dice que dos palabras $a$ y $b$, del mismo c\'odigo, son \textbf{``iguales''}
cuando se verifica que $\forall \ i$ se tiene que $a_i=b_i$. En caso contrario
ambas palabras son \textbf{``distintas''}.
\end{definicion}

\begin{definicion}[Error en el lugar $i$-\'esimo]
\ \\
Dadas dos palabras $a$ y $b$, del mismo c\'odigo, diremos que hay un
\textbf{``error en el lugar $i$-\'esimo''} cuando $a_i\neq b_i$.
\end{definicion}
Medir la diferencia de dos palabras dadas, de un mismo c\'odigo, equivale a
dar una noci\'on de distancia, o lo que es lo mismo una ``\emph{m\'etrica}'',
en el c\'odigo. Esta diferencia la mediremos utilizando la ``\textbf{distancia
de Hamming}''.
\begin{definicion}[Distancia de Hamming]
\ \\
Dadas dos palabras $a$ y $b$, del mismo c\'odigo, definiremos su distancia como
el numero de $i$ tales que $a_i\neq b_i$, o lo que es lo mismo, como el n\'umero
de $i$ tales que $a_i-b_i \neq 0$. La distancia entre dos palabras la
denotaremos como $d(a,b)$.
\end{definicion}
Matem\'aticamente hablando tendremos, que la ``distancia de Hamming'' vendr\'a
dada por una aplicaci\'on:
\begin{displaymath}
\begin{array}{cccc}
d:&\mathbb{F}^{^n}_q\times \mathbb{F}^{^n}_q&\longrightarrow & \mathbb{Z}^{^+}\\
&(a,b)&\longrightarrow & d(a,b)
\end{array}
\end{displaymath}
Como observaci\'on diremos que esta ``\emph{distancia}'' tambi\'en se verifica
para $\mathbb{F}^{^n}_q$, con lo cual la distancia no s\'olo est\'a definida
en el c\'odigo, sino que tambi\'en est\'a definida en el total.

\begin{definicion}[Error de peso $k$]
\ \\
Dadas dos palabras $a$ y $b$, de un mismo c\'odigo, diremos que existe un
\textbf{``error de peso $k$''} si $d(a,b) = k$.
\end{definicion}
\begin{definicion}[Peso de una palabra]
\ \\
Llamaremos \textbf{``peso de una palabra''} a su distancia con el $0$. El peso
de una palabra de $n$ bits ser\'a:
\begin{displaymath}
d((a_1,\dots ,a_n),(0,\stackrel{n)} \ldots ,0))
\end{displaymath}
y lo denotaremos como $w(u)$, con $u=(a_1,\dots,a_n)$.
\end{definicion}

\subsection{Algunos ejemplos}

Consideremos un c\'odigo $\mathcal{C}[4,1]\subset{\mathbb{F}^{^4}_2}$. Cada
palabra de $\mathbb{F}^{^4}_2$ ser\'a de la forma $a=(a_1,a_2,a_3,a_4)$.
\begin{itemize}
\item $a=(0,0,1,0)$ y $b=(1,1,1,1)$ entonces $d(a,b) = 3$ ya que $a_1\neq b_1$,
$a_2\neq b_2$ y $a_4\neq b_4$. Es decir hay $3$ bits que no coinciden. Tenemos
un error de peso $3$.
\item $a=(0,1,0,1)$ y $b=(0,1,0,1)$ entonces $d(a,b) = 0$ ya que $a_i=b_i$
$\forall \ i$. Tenemos un error de peso $0$, ambas palabras son iguales.
\item $a=(1,0,0,1)$ y $b=(0,1,1,0)$ entonces $d(a,b) = 4$  ya que $a_i\neq b_i$
$\forall \ i$. Es decir ambas palabras no coinciden en ning\'un bit. Tenemos
un error de peso $4$.
\end{itemize}

\subsection{Propiedades de las distancias}

Hemos definido antes una aplicaci\'on a la que hemos llamado ``\emph{distancia
de Hamming}''. Una ``\emph{distancia}'' es una aplicaci\'on que cumple una
serie de condiciones.\\

Dado un conjunto cualquiera $A$, diremos que una aplicaci\'on, $d$, es una
distancia si es de la forma:
\begin{eqnarray*}
d: A\times A&\longrightarrow& \mathbb{R}^{^+}\\
(a,b)&\longrightarrow& d(a,b)
\end{eqnarray*}
y verifica las siguientes condiciones:
\begin{itemize}
\item $d(a,b)\geq 0$ $\forall \ a,b \in A$ y $d(a,b)=0$ $\Longleftrightarrow$
$a=b$.
\item $d(a,b)=d(b,a)$ $\forall \ a,b\in A$.
\item ``\emph{Desigualdad triangular}''
\begin{displaymath}
d(a,c)\leq d(a,b) + d(b,c)\quad \forall \ a,b,c\in A
\end{displaymath}
\end{itemize}
En nuestros casos $A=\mathbb{F}^{^n}_q$ y en lugar de considerar
$\mathbb{R}^{^+}$ consideraremos $\mathbb{Z}^{^+}$.

\subsection{Detecci\'on de errores mediante las distancias}\label{sec:DistanErr}

Para detectar errores en una transmisi\'on utilizaremos una propiedad de las
distancias:
\begin{quote}
La distancia entre dos palabras es nula $\Longleftrightarrow$ s\'{\i} ambas
palabras son la misma. $d(a,b)=0$ $\Longleftrightarrow$ $a=b$.
\end{quote}
Cuando hayamos recibido una palabra en una transmisi\'on lo que haremos para
detectar si ocurrio alg\'un error en la transmisi\'on ser\'a:
\begin{itemize}
\item Calcular la distancia de la palabra recibida, $u$, a todas las del
c\'odigo.
\item Si alguna de las distancias es nula entonces se tiene que la palabra
recibida $u$ pertenece al c\'odigo. En caso contrario la palabra recibida $u$
no pertenece al c\'odigo, con lo cual habr\'a ocurrido alg\'un error en la
transmisi\'on.
\end{itemize}
