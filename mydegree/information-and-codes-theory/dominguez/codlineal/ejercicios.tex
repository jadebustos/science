%
% EJERCICIOS DEL TEMA DE CODIGOS LINEALES
%

\section{Ejercicios}

\begin{ejercicio}
\ \\
Demostrar que pasar de la matriz generadora a la matriz de control equivale a
pasar de las ecuaciones param\'etricas a las impl\'{\i}citas.
\end{ejercicio}
\underline{\textbf{Soluci\'on}}:\\
Las ecuaciones param\'etricas se obtienen a partir de la matriz generadora,
mientras que las ecuaciones impl\'{\i}citas se obtienen a partir de la matriz
de control.\\

Ambas ecuaciones nos dan la estructura del c\'odigo y se pueden utilizar
tanto para construir las palabras del c\'odigo como para verificar que una
palabra dada pertenece al c\'odigo.\\

Debido a la definici\'on de dichas ecuaciones el paso de la matriz generadora
a la de control equivale al paso de ecuaciones param\'etricas a impl\'{\i}citas,
ya que las ecuaciones param\'etricas se obtienen de la matriz generadora y las
impl\'{\i}citas de la matriz de control.
\begin{flushright}
$\blacksquare$
\end{flushright}
