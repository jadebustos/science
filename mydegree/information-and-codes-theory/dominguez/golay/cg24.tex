%
% CONSTRUCCION GOLAY 24
%

\section{Construcci\'on del c\'odigo Golay $G_{24}$}

Este c\'odigo es un c\'odigo lineal $3$-perfecto, es del tipo
$\mathcal{C}[24,12]$ y lo denotaremos como $G_{24}$.\\

\begin{definicion}[C\'odigo de Golay $G_{24}$]
\ \\
El c\'odigo de Golay $G_{24}$ consiste en todas las palabras de longitud $24$
que son de la forma:
\begin{displaymath}
a+x,b+x,a+b+x
\end{displaymath}
donde $a,b\in Ham(3)'$ y $x\in K'$.
\end{definicion}
Por ejemplo, si $a=11010001$, $b=10010110$ y $x=01011001$ la palabra del
c\'odigo $G_{24}$ que definen es:
\begin{displaymath}
\underbrace{10001000}_{a+x}\underbrace{11001111}_{b+x}\underbrace{00011110}_{a+b+x}
\end{displaymath}

\begin{proposicion}
\ \\
Sea $u\in G_{24}$ entonces est\'a determinada de forma \'unica por
$a,b\in Ham(3)'$ y $x\in K'$.
\end{proposicion}
\underline{\textbf{Demostraci\'on}}:\\
Supongamos que existen $c,d\in Ham(3)'$ e $y\in K'$ tales que:
\begin{displaymath}
a+x,b+x,a+b+x=c+y,d+y,c+d+y
\end{displaymath}
Como $a+x=c+y$ y $a,c\in Ham(3)'$ y $x,y\in K'$ tendremos que $a+c=x+y$,
recordar que son palabras binarias y que $-1\equiv 1$ en $\mathbb{F}_2$.
Luego $a+c$ pertenece a $Ham(3)'$ y a $K'$, entonces tenemos dos posibilidades:
\begin{itemize}
\item $a+c=00000000$\\

En este caso tendremos que $a=c$ de donde se deduce que $x=y$. Como tenemos, por
hip\'otesis, que $a+b+x=c+d+y$ entonces $a+b+x=a+d+x$ o lo que es lo mismo
$a+a+b=x+x+d$. La suma de una palabra binaria consigo misma es nula entonces
$b=d$.\\

Con lo cual hemos demostrado que, en este caso, la descomposici\'on de una
palabra de $G_{24}$ es \'unica.
\item $a+c=11111111$\\

En este caso tendremos que $a$ es la palabra complementaria de $c$ y $x$ lo
ser\'a de $y$. Como tenemos, por hip\'otesis, que $a+b+x=c+d+y$ entonces
tendremos que:
\begin{displaymath}
\underbrace{a+c}_{11111111}+\underbrace{x+y}_{11111111}+b=d
\end{displaymath}
de donde se tiene que $b=d$.\\

Con lo cual hemos demostrado que, en este caso, la descomposici\'on de una
palabra de $G_{24}$ es \'unica.
\end{itemize}
\begin{flushright}
$\blacksquare$
\end{flushright}

\subsection{Propiedades de $G_{24}$}

\begin{proposicion}
\ \\
La longitud de las palabras de $G_{24}$ es $24$ y su rango es $12$.
\end{proposicion}
\underline{\textbf{Demostraci\'on}}:\\
La longitud de las palabras de $G_{24}$ se deduce de su construcci\'on y es
$24$.\\

$G_{24}$ es un c\'odigo lineal, donde cada palabra se construye a partir de
tres palabras, cada una de las cuales pertenece a un c\'odigo de rango $4$.
De esto se deduce que $G_{24}$ es un c\'odigo con $2^4\cdot 2^4\cdot 2^4=2^{12}$
palabras, luego su rango es $12$.
\begin{flushright}
$\blacksquare$
\end{flushright}

\begin{proposicion}
\ \\
La distancia m\'{\i}nima de $G_{24}$ es $8$.
\end{proposicion}
