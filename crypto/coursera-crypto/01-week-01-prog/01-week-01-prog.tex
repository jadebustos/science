\documentclass[a4paper,12pt]{article}

%\addtolength{\voffset}{-4cm}
%\usepackage[spanish]{babel}
\usepackage{amsmath}
\usepackage{mycmds}
\usepackage{graphicx}
\pagestyle{headings}

\author{Jos\'e Angel de Bustos P\'erez}

\frenchspacing

\hyphenation{si-guien-te apro-xi-ma-cio-nes FORTRAN double intGrado 
dblDerivada dblResultado intNMI intOrden dblMatriz Newton intResultado
cabezera coor-de-na-das de-pen-dien-do di-men-sio-na-do re-pre-sen-tan-do}

\begin{document}

\thispagestyle{empty}

\begin{center}
\Huge{Cryptography I - Coursera} \\[.75cm]
\large{Week 1 - Programming assignment}
\end{center}

\large 
\begin{flushright}
\yo \\
$<$jadebustos@gmail.com$>$\\ \ \\ 
Versi\'on $1.0$, \today .\\
\textbf{\LaTeXe}
\end{flushright}

\normalsize

\textbf{Question 1} \\

Many Time Pad 

Let us see what goes wrong when a stream cipher key is used more than once. Below are eleven hex-encoded ciphertexts that are the result of encrypting eleven plaintexts with a stream cipher, all with the same stream cipher key. Your goal is to decrypt the last ciphertext, and submit the secret message within it as solution. \\

Hint: XOR the ciphertexts together, and consider what happens when a space is XORed with a character in [a-zA-Z].\\

\textbf{ciphertext \#1}:
\begin{verbatim}
315c4eeaa8b5f8aaf9174145bf43e1784b8fa00dc71d885a804e5ee9fa40b163
49c146fb778cdf2d3aff021dfff5b403b510d0d0455468aeb98622b137dae857
553ccd8883a7bc37520e06e515d22c954eba5025b8cc57ee59418ce7dc6bc415
56bdb36bbca3e8774301fbcaa3b83b220809560987815f65286764703de0f3d5
24400a19b159610b11ef3e
\end{verbatim}

\textbf{ciphertext \#2}:
\begin{verbatim}
234c02ecbbfbafa3ed18510abd11fa724fcda2018a1a8342cf064bbde548b12b
07df44ba7191d9606ef4081ffde5ad46a5069d9f7f543bedb9c861bf29c7e205
132eda9382b0bc2c5c4b45f919cf3a9f1cb74151f6d551f4480c82b2cb24cc5b
028aa76eb7b4ab24171ab3cdadb8356f
\end{verbatim}

\textbf{ciphertext \#3}:
\begin{verbatim}
32510ba9a7b2bba9b8005d43a304b5714cc0bb0c8a34884dd91304b8ad40b62b
07df44ba6e9d8a2368e51d04e0e7b207b70b9b8261112bacb6c866a232dfe257
527dc29398f5f3251a0d47e503c66e935de81230b59b7afb5f41afa8d661cb
\end{verbatim}

\textbf{ciphertext \#4}:
\begin{verbatim}
32510ba9aab2a8a4fd06414fb517b5605cc0aa0dc91a8908c2064ba8ad5ea06a
029056f47a8ad3306ef5021eafe1ac01a81197847a5c68a1b78769a37bc8f457
5432c198ccb4ef63590256e305cd3a9544ee4160ead45aef520489e7da7d8354
02bca670bda8eb775200b8dabbba246b130f040d8ec6447e2c767f3d30ed81ea
2e4c1404e1315a1010e7229be6636aaa
\end{verbatim}

\textbf{ciphertext \#5:}
\begin{verbatim}
3f561ba9adb4b6ebec54424ba317b564418fac0dd35f8c08d31a1fe9e24fe568
08c213f17c81d9607cee021dafe1e001b21ade877a5e68bea88d61b93ac5ee0d
562e8e9582f5ef375f0a4ae20ed86e935de81230b59b73fb4302cd95d770c65b
40aaa065f2a5e33a5a0bb5dcaba43722130f042f8ec85b7c2070
\end{verbatim}

\textbf{ciphertext \#6:}
\begin{verbatim}
32510bfbacfbb9befd54415da243e1695ecabd58c519cd4bd2061bbde24eb76a
19d84aba34d8de287be84d07e7e9a30ee714979c7e1123a8bd9822a33ecaf512
472e8e8f8db3f9635c1949e640c621854eba0d79eccf52ff111284b4cc61d119
02aebc66f2b2e436434eacc0aba938220b084800c2ca4e693522643573b2c4ce
35050b0cf774201f0fe52ac9f26d71b6cf61a711cc229f77ace7aa88a2f19983
122b11be87a59c355d25f8e4
\end{verbatim}

\textbf{ciphertext \#7:}
\begin{verbatim}
32510bfbacfbb9befd54415da243e1695ecabd58c519cd4bd90f1fa6ea5ba47b
01c909ba7696cf606ef40c04afe1ac0aa8148dd066592ded9f8774b529c7ea12
5d298e8883f5e9305f4b44f915cb2bd05af51373fd9b4af511039fa2d96f8341
4aaaf261bda2e97b170fb5cce2a53e675c154c0d9681596934777e2275b381ce
2e40582afe67650b13e72287ff2270abcf73bb028932836fbdecfecee0a3b894
473c1bbeb6b4913a536ce4f9b13f1efff71ea313c8661dd9a4ce
\end{verbatim}

\textbf{ciphertext \#8:}
\begin{verbatim}
315c4eeaa8b5f8bffd11155ea506b56041c6a00c8a08854dd21a4bbde54ce568
01d943ba708b8a3574f40c00fff9e00fa1439fd0654327a3bfc860b92f89ee04
132ecb9298f5fd2d5e4b45e40ecc3b9d59e9417df7c95bba410e9aa2ca24c547
4da2f276baa3ac325918b2daada43d6712150441c2e04f6565517f317da9d3
\end{verbatim}

\textbf{ciphertext \#9:}
\begin{verbatim}
271946f9bbb2aeadec111841a81abc300ecaa01bd8069d5cc91005e9fe4aad6e
04d513e96d99de2569bc5e50eeeca709b50a8a987f4264edb6896fb537d0a716
132ddc938fb0f836480e06ed0fcd6e9759f40462f9cf57f4564186a2c1778f15
43efa270bda5e933421cbe88a4a52222190f471e9bd15f652b653b7071aec59a
2705081ffe72651d08f822c9ed6d76e48b63ab15d0208573a7eef027
\end{verbatim}

\textbf{ciphertext \#10:}
\begin{verbatim}
466d06ece998b7a2fb1d464fed2ced7641ddaa3cc31c9941cf110abbf409ed39
598005b3399ccfafb61d0315fca0a314be138a9f32503bedac8067f03adbf357
5c3b8edc9ba7f537530541ab0f9f3cd04ff50d66f1d559ba520e89a2cb2a83
\end{verbatim}

\textbf{target ciphertext (decrypt this one)}:
\begin{verbatim}
32510ba9babebbbefd001547a810e67149caee11d945cd7fc81a05e9f85aac65
0e9052ba6a8cd8257bf14d13e6f0a803b54fde9e77472dbff89d71b57bddef12
1336cb85ccb8f3315f4b52e301d16e9f52f904
\end{verbatim}

For completeness, here is the python script used to generate the ciphertexts. (it doesn't matter if you can't read this):
%
\begin{verbatim}
import sys

MSGS = ( ---  11 secret messages  --- )

def strxor(a, b):     # xor two strings of different lengths
    if len(a) > len(b):
        return "".join([chr(ord(x) ^ ord(y)) for (x, y) in zip(a[:len(b)], b)])
    else:
        return "".join([chr(ord(x) ^ ord(y)) for (x, y) in zip(a, b[:len(a)])])

def random(size=16):
    return open("/dev/urandom").read(size)

def encrypt(key, msg):
    c = strxor(key, msg)
    print
    print c.encode('hex')
    return c

def main():
    key = random(1024)
    ciphertexts = [encrypt(key, msg) for msg in MSGS]  
\end{verbatim}

\textbf{Solution}\\

\end{document}
