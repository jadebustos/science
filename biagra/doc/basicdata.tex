%
% basicdata.h
%

\chapter{basicdata.h}

\section{Introduction}

Common data structures are defined in this file.

\section{Polynomial data structures}

\subsection{Polynomials}

\begin{equation}
p(x) = a_0 + a_1 \cdot x + \cdots + a_n \cdot x^n = \sum_{i=0}^n a_i \cdot x^i
\end{equation}

\subsection{biaRealPol data structure} \label{sec:biaRealPol}

This data structure is used to handle polinomials $p(x) \in \mathbb{R}[x]$. \textbf{biaPol} data structure is defined in figure \ref{fig:biaRealPol} where:

\begin{description}
\item[intDegree] polynomial degree.
\item[intRealRoots] number of real roots (if any).
\item[intCompRoots] number of complex roots (if any).
\item[*dblCoef] pointer to store polynomial coeficients.
\end{description}

\begin{figure}[!h]
\begin{verbatim}
typedef struct {
  int  intDegree    = 0,
       intRealRoots = 0,
       intCompRoots = 0;

  double  *dblCoefs;
  } biaRealPol;
\end{verbatim}
\caption{Polynomial data structure.} \label{fig:biaRealPol}
\end{figure}

\subsection{How to use it}
