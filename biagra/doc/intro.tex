%
% CAPITULO 1 �Que es B.I.A.G.R.A?
%

\chapter{?`Que es \BI?}

\begin{itemize}

\item \BI es una BIbliotecA de proGRamaci\'on cient\'{\i}ficA programada 
enteramente en \textbf{C} y pensada para ser utilizada en programas
escritos en \textbf{C}.
\item \BI ha sido desarrollada y testeada bajo \emph{LiNUX}, por lo cual sus
caracter\'{\i}sticas se adaptan a las de este sistema operativo. 
\item \BI es de libre distribuci\'on y su autor no se hace responsable de su
mal uso y/o de sus posibles fallos.

\end{itemize}

\section{?`Lenguaje C? }
Alguien se puede preguntar el motivo por el cual he elegido el Lenguaje 
\textbf{C} y no \emph{FORTRAN}, lenguaje tradicional para la programaci\'on
cient\'{\i}fica.\newline
Los motivos son varios siendo el fundamental la potencia,
la modularidad de \textbf{C} y el hecho de ser un lenguaje de proposito general.
\newline

Las razones que alegan la inmesa mayor\'{\i}a de programadores de 
\emph{FORTRAN} sobre las ventajas que tiene \emph{FORTRAN}:

\begin{enumerate}
\item Es un lenguaje facil de usar.
\item Casi todo el mundo utiliza \emph{FORTRAN}, en este tipo de
      programaci\'on.
\item Hay mucho c\'odigo escrito en \emph{FORTRAN}.
\end{enumerate}

Aplicando esta forma de pensar bien podr\'{\i}amos pensar que el 
\emph{sol gira sobre la tierra}, ya que hace muchos a\~nos la mayor\'{\i}a de
la gente estaba de acuerdo con esta idea.

\newpage

Las razones por las que fundamento la elecci\'on de \textbf{C} son:

\begin{enumerate}
\item Es un lenguaje potente y que genera un c\'odigo de muy r\'apida 
      ejecuci\'on.
\item Es un lenguaje flexible.
\item Es un lenguaje estructurado.
\item Es un lenguaje de proposito general.
\item El c\'odigo escrito en \textbf{C} es portable y la depuraci\'on y 
      mantenimiento de programas es ``sencilla''\footnote{Si el c\'odigo 
      esta bien estructurado y se ajusta al \emph{ANSI C}.}.
\item Tiene una gran potencia en el manejo del hardware, lo que implica
      una gran potencia en el manejo de modos gr\'aficos\footnote{Deseable
      para seg\'un que aplicaciones.}.
\end{enumerate}

\section{Algunas ideas generales sobre \BI}
Lo primero que debemos destacar es que esta biblioteca esta pensada para
programadores de \textbf{C} y ha sido desarrollada y testeada bajo 
\textbf{LiNUX}, luego en el resto de este documento se supondra que el lector
esta utilizando \textbf{LiNUX} o cualquier otro sistema tipo \textbf{UNIX}.\\

Esta biblioteca esta pensada para resolver problemas generales, no para 
resolver tipos particulares de problemas, es decir, esta pensada para
realizar programas, que, interactuando con el usuario puedan resolver un
problema sin tener que modificar el c\'odigo fuente cuando varien las
condiciones del problema.\newline

Por ejemplo, escribir un programa para invertir matrices, de
cualquier orden, en vez de escribir un programa para invertir matrices de
orden \emph{n} y cuando se quiera invertir una matriz de un orden $m \neq n$
sea necesario cambiar el c\'odigo fuente y recompilar el programa.\newline

Para lograr esto se han utilizado \emph{punteros}, es decir se ha evitado,
en la medida de lo posible, el uso de \emph{arrays}.\newpage

Cuando hablemos de \emph{vectores} nos estaremos refiriendo a un
\emph{puntero}, normalmente a datos del tipo \emph{double}, para el cual se
ha reservado memoria para contener \emph{n} datos. Del mismo modo cuando nos
refiramos a \emph{matrices} estaremos hablando de matrices de \emph{punteros}, 
tambi\'en a datos del tipo \emph{double}(normalmente), para las cuales se ha
reservado memoria para contener $\mbox{\emph{n * m}}$ datos.\newline

En aquellos problemas que se utiliza un conjunto de varios datos se ha
optado por agruparlos dentro de una estructura\footnote{Declarada mediante
typedef.} para mayor comodidad.\newline

En aquellos problemas en los que se pueda cometer alg\'un error como por
ejemplo:

\begin{itemize}
\item Fallo en asignaciones de memoria.
\item Divisiones por cero.
\item \ldots
\end{itemize}

la funci\'on que resuelve ese problema devolver\'a un entero indicando cual
fue el estado en el que se termin\'o la ejecuci\'on de la funci\'on.\newline

Cuando una funci\'on devuelva un dato, que no sea un c\'odigo que indique
el estado en el que se termin\'o la ejecuci\'on de la funci\'on, se indicar\'a 
anteponiendo un prefijo al nombre de la funci\'on, el cual indicar\'a que tipo
de dato devuelve, por ejemplo:

\begin{enumerate}
\item \emph{double dblFuncion(\ldots)} funci\'on que devuelve un dato de
      tipo \emph{double}.
\item \emph{int intFuncion(\ldots)} funci\'on que devuelve un dato de tipo
      \emph{int}.
\item \emph{double *dblPtFuncion(\ldots)} funci\'on que devuelve un dato de
      tipo puntero a \emph{double}.
\item \emph{void Funcion(\ldots)} funci\'on que no devuelve ning\'un dato.
\end{enumerate}

\newpage

\section{C\'odigos devueltos por las funciones}

No es obligatorio que las funciones devuelvan un valor.\newline
Cuando una funci\'on devuelva un valor ser\'a por dos razones:

\begin{itemize}
\item Para devolver el resultado de una operaci\'on.
\item Para informar de como termin\'o una operaci\'on.
\end{itemize} 

Es este \'ultimo caso el que nos incumbe.\newline

Cuando una funci\'on devuelva un dato indicando como termin\'o una
determinada operaci\'on, este dato ser\'a, necesariamente, un entero y los
c\'odigos devueltos los podemos ver en la p\'agina \pageref{sec:tiposdecodigos}.

\section{Como instalar \BI en LiNUX}
Para instalar \BI lo primero que hay que hacer es entrar en el sistema
como \textbf{root} y situarse en el directorio donde esten los fuentes
de la biblioteca.

\subsection{Intalaci\'on de la biblioteca \BI est\'atica}
Para instalar este tipo bibilioteca se puede hacer de dos formas:

\begin{enumerate}
\item \emph{./instalar estatica}
\item \emph{make estatica}
\end{enumerate}

En realidad ambas hacen lo mismo, la opci\'on con \emph{make} en realidad 
ejecuta \emph{./instalar estatica}.\newline

Al realizar cualquiera de estas dos opciones se copiar\'an los ficheros
cabezera a \emph{\textbf{/usr/include/biagra}} y a continuaci\'on se crear\'a la
biblioteca\\ \emph{\textbf{/usr/lib/libbiagra.a}}.\newline

Luego si se utiliza una funci\'on de esta biblioteca, cuyo prototipo est\'a en 
el fichero de cabecera \emph{rngkutta.h} habr\'a que incluir en las directivas 
al prepocesador \textbf{\#include $<$biagra/rngkutta.h$>$}.

\newpage

\subsection{Intalaci\'on de la biblioteca \BI din\'amica ELF}

\subsection{Instalaci\'on de ambas bibliotecas}
Para instalar la biblioteca \BI en su forma est\'atica y din\'amica ELF:

\begin{center}
\emph{make todo}
\end{center}

Esto lo que hace es ejecutar primero

\begin{center}
\emph{./instalar estatica}
\end{center}

lo cual instalar\'a la biblioteca est\'atica, y luego 

\begin{center}
\emph{./instalar elf}
\end{center}

lo cual instalar\'a la biblioteca din\'amica ELF.

\section{Como utilizar \BI en LiNUX}
\BI es una biblioteca para programaci\'on cient\'{\i}fica, 
desarrollada para ser usada en programas escritos en \textbf{C}, se
distribuye en varios formatos:

\begin{description}
\item[Biblioteca est\'atica]
\item[Biblioteca din\'amica ELF]
\end{description}

\subsection{Biblioteca est\'atica}
Para el uso de esta biblioteca hay que indicarle al \emph{montador} que
biblioteca debe \emph{enlazar}.
Por ejemplo, supongamos que hemos escrito un programa para resolver una
ecuaci\'on diferencial por un m\'etodo \emph{Runge-Kutta} y hemos utilizado
funciones cuyos prototipos estan en \textbf{edo.h} y \textbf{rngkutta.h}, si
nuestro programa es \mbox{\emph{programa.c}}, para crear el ejecutable:

\begin{center}
\emph{gcc programa.c -o programa -l\textbf{biagra} -l\textbf{m}\footnote{\BI 
utiliza la biblioteca estandar matem\'atica.}} 
\end{center}

\subsection{Biblioteca din\'amica ELF}

\section{Novedades en la versi\'on $1.0$}
Pues todo ya que es la primera versi\'on.

\section{Condiciones de uso}

\begin{enumerate}
\item El autor no se responsabiliza de su mal uso, de su posibles fallos%
\footnote{Si los hubiera o hubiese.}, ni de las modificaciones que sean
realizadas a dicha biblioteca.
\item Es una biblioteca de libre distribuci\'on.
\item Se puede modificar y a\~nadir c\'odigo, pero nunca distribuirlo bajo
el nombre de \BI, en caso de modificaci\'on de la biblioteca se deben indicar
aquellas funciones que fueron modificadas o a\~nadidas.
\end{enumerate}

Si tienes alguna sugerencia, pregunta, o has encontrado alg\'un error en la 
biblioteca ya sabes donde encontrarme.\newline
