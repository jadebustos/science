%
% What is B.I.A.G.R.A?
%

\chapter{What is \BI?}

\begin{itemize}

\item \BI stands for \textbf{BI}\emph{bliotec}\textbf{A} \emph{de pro}\textbf{GR}\emph{amaci\'on cient\'{\i}fic}\textbf{A} which means Scientific Programming Library.
\item \BI is enterely coded using \textbf{C} language.
\item \BI has been developed and tested under \emph{LiNUX}.
\item \BI is distribuded as open source and its author does not take any responibility.
\item I wrote \BI in the 90s to help me with some subjects in my degree.

\end{itemize}

\section{C language?}

\textbf{C} language instead of \textbf{FORTRAN}?
%
\begin{itemize}
\item \textbf{C} is modular and structured.
\item \textbf{C} is a general purpose language programming.
\item \textbf{C} is a very powerful language and its code is very fast.
\item \textbf{C} allows dynamic memory allocation.
\item \textbf{C} code is portable.
\item \textbf{C} is able to handle graphic modes. 
\end{itemize}

\section{Some general ideas about \BI}

\BI\ has been developed under \textbf{Linux} and some \textbf{Linux} knowledge is needed.\\

\BI\ was developed to solve general problems instead of particular ones. For instance, instead of writing a program to get the inverse of a 4x4 matrix and having to change the source code to get the inverse of a 5x5 matrix \BI\ was developed to allow to write programs to get the inverse of any matrix without having to change de sorce code.\\

To be able to do that \emph{pointers} were used instead of using \emph{arrays}.\\

When we talk about \emph{vectors} we will be talking about a \emph{pointer} using dynamic memory allocation. When we talk about matrices we will be talking about \emph{pointer} to a \emph{pointer} using dynamic memory allocation.\\

\BI\ uses some data structures to store data.\\

For common errors as:

\begin{itemize}
\item Errors in dynamic memory allocation.
\item Division by zero.
\item \ldots
\end{itemize}

\BI\ uses its own constants to notify these errors (Chapter \ref{ch:mathematicalConsts}).\\

Cuando una funci\'on devuelva un dato, que no sea un c\'odigo que indique
el estado en el que se termin\'o la ejecuci\'on de la funci\'on, se indicar\'a 
anteponiendo un prefijo al nombre de la funci\'on, el cual indicar\'a que tipo
de dato devuelve, por ejemplo:

\begin{enumerate}
\item \emph{double dblFuncion(\ldots)} funci\'on que devuelve un dato de
      tipo \emph{double}.
\item \emph{int intFuncion(\ldots)} funci\'on que devuelve un dato de tipo
      \emph{int}.
\item \emph{double *dblPtFuncion(\ldots)} funci\'on que devuelve un dato de
      tipo puntero a \emph{double}.
\item \emph{void Funcion(\ldots)} funci\'on que no devuelve ning\'un dato.
\end{enumerate}

\newpage

\section{C\'odigos devueltos por las funciones}

No es obligatorio que las funciones devuelvan un valor.\newline
Cuando una funci\'on devuelva un valor ser\'a por dos razones:

\begin{itemize}
\item Para devolver el resultado de una operaci\'on.
\item Para informar de como termin\'o una operaci\'on.
\end{itemize} 

Es este \'ultimo caso el que nos incumbe.\newline

Cuando una funci\'on devuelva un dato indicando como termin\'o una
determinada operaci\'on, este dato ser\'a, necesariamente, un entero y los
c\'odigos devueltos los podemos ver en la p\'agina \pageref{sec:tiposdecodigos}.

\section{Como instalar \BI en LiNUX}
Para instalar \BI lo primero que hay que hacer es entrar en el sistema
como \textbf{root} y situarse en el directorio donde esten los fuentes
de la biblioteca.

\subsection{Intalaci\'on de la biblioteca \BI est\'atica}
Para instalar este tipo bibilioteca se puede hacer de dos formas:

\begin{enumerate}
\item \emph{./instalar estatica}
\item \emph{make estatica}
\end{enumerate}

En realidad ambas hacen lo mismo, la opci\'on con \emph{make} en realidad 
ejecuta \emph{./instalar estatica}.\newline

Al realizar cualquiera de estas dos opciones se copiar\'an los ficheros
cabezera a \emph{\textbf{/usr/include/biagra}} y a continuaci\'on se crear\'a la
biblioteca\\ \emph{\textbf{/usr/lib/libbiagra.a}}.\newline

Luego si se utiliza una funci\'on de esta biblioteca, cuyo prototipo est\'a en 
el fichero de cabecera \emph{rngkutta.h} habr\'a que incluir en las directivas 
al prepocesador \textbf{\#include $<$biagra/rngkutta.h$>$}.

\newpage

\subsection{Intalaci\'on de la biblioteca \BI din\'amica ELF}

\subsection{Instalaci\'on de ambas bibliotecas}
Para instalar la biblioteca \BI en su forma est\'atica y din\'amica ELF:

\begin{center}
\emph{make todo}
\end{center}

Esto lo que hace es ejecutar primero

\begin{center}
\emph{./instalar estatica}
\end{center}

lo cual instalar\'a la biblioteca est\'atica, y luego 

\begin{center}
\emph{./instalar elf}
\end{center}

lo cual instalar\'a la biblioteca din\'amica ELF.

\section{Como utilizar \BI en LiNUX}
\BI es una biblioteca para programaci\'on cient\'{\i}fica, 
desarrollada para ser usada en programas escritos en \textbf{C}, se
distribuye en varios formatos:

\begin{description}
\item[Biblioteca est\'atica]
\item[Biblioteca din\'amica ELF]
\end{description}

\subsection{Biblioteca est\'atica}
Para el uso de esta biblioteca hay que indicarle al \emph{montador} que
biblioteca debe \emph{enlazar}.
Por ejemplo, supongamos que hemos escrito un programa para resolver una
ecuaci\'on diferencial por un m\'etodo \emph{Runge-Kutta} y hemos utilizado
funciones cuyos prototipos estan en \textbf{edo.h} y \textbf{rngkutta.h}, si
nuestro programa es \mbox{\emph{programa.c}}, para crear el ejecutable:

\begin{center}
\emph{gcc programa.c -o programa -l\textbf{biagra} -l\textbf{m}\footnote{\BI 
utiliza la biblioteca estandar matem\'atica.}} 
\end{center}

\subsection{Biblioteca din\'amica ELF}
