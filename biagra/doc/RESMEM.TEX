%
% RESMEN.H
%

\chapter{resmem.h}

\section{Introducci\'on}
En este fichero de cabezera estan los prototipos de funciones para la
reserva de memoria.\newline
\par Si no se pudo reservar memoria todas las funciones de asignaci\'on de 
memoria devuelven el puntero nulo, \emph{NULL}, es recomendable comprobar
des-pu\'es de cada asignaci\'on de memoria si se pudo realizar con \'exito la
asignaci\'on de memoria.

\section{Reserva de memoria para vectores}
Estas funciones admiten como par\'ametro la dimensi\'on del vector para el
cual se quiere reservar memoria y devuelven un puntero al primer elemento del
vector.

\newpage

\subsection{dblPtAsigMemVec} \label{sec:AsigMemVec}
Esta funci\'on reserva memoria para un \emph{array} unidimensional y devuelve 
un puntero a \emph{double} conteniendo la direcci\'on del primer elemento.\\

El prototipo de esta funci\'on es el siguiente:

\begin{center}
\emph{double *\textbf{dblPtAsigMemVec}(int intN)}
\end{center}

\begin{description}
\item[intN] dimensi\'on del \emph{array} para el que se quiere reservar
memoria.
\end{description}

Por ejemplo: 

\begin{center}
\emph{dblPtDato = \textbf{dblPtAsigMemVec}(intN);}
\end{center}

Reservar\'{\i}a memoria para \emph{intN} elementos del tipo \emph{double} y se
almacenar\'{\i}a la direcci\'on del primer elemento en \emph{dblPtDato}, el cual
es un puntero a \emph{double}.\newline

Para hacer referencia a los datos de \emph{dblPtDato}:

\begin{itemize}
\item \emph{dblPtDato[i]} donde i $=$ 0, \ldots , $\emph{intN-1}$
\item \emph{*(dblPtDato + i)} donde i $=$ 0, \ldots, $\emph{intN-1}$
\end{itemize}

\newpage

\section{Reserva de memoria para matrices}
Estas funciones admiten como par\'ametros las filas y columnas de las matrices
para las que se quiere reservar memoria y devuelven un puntero a puntero 
\emph{double}.

\subsection{dblPtAsigMemMat} \label{sec:asigmemmat}
Esta funci\'on reserva memoria para un \emph{array} bidimensional y 
devuelve un puntero a puntero \emph{double} conteniendo la direcci\'on
del primer elemento.\newline

El prototipo de esta funci\'on es el siguiente:

\begin{center}
\emph{double **\textbf{dblPtAsigMemMat}(int intFilas, int intCol)}
\end{center}

\begin{description}
\item[intFilas] n\'umero de filas de la matriz para la cual se quiere reservar
memoria.
\item[intCol] n\'umero de columnas de la matriz para la cual se quiere reservar
memoria.
\end{description}

Por ejemplo:

\begin{center}
\emph{dblPtDato = \textbf{dblPtAsigMemMat}(intFilas, intCol);}
\end{center}

Reservar\'{\i}a memoria para una matriz de \emph{intFilas} filas y 
\emph{intCol} columnas, los elementos de la matriz son de tipo
\emph{double} y la direcci\'on base de la matriz estar\'{\i}a almacenada
en dblPtDato, el cual es un puntero a puntero \emph{double}.

Para hacer referencia a los datos de \emph{dblPtDato}:

\begin{itemize}
\item \emph{dblPtDato[i][j]}
\item \emph{*(*(dblPtDato + i) + j)}
\end{itemize}

donde i $=$ 0, \ldots, $\emph{intFilas-1}$\newline

donde j $=$ 0, \ldots, $\emph{intCol-1}$

\newpage

\subsection{dblPtAsigMemMatTrSup}
Esta funci\'on reserva memoria para una matriz triangular superior, cuadrada
de orden \emph{intN}, devuelve un puntero a puntero \emph{double} conteniendo
la direcci\'on del primer elemento.\newline

El prototipo de esta funci\'on es el siguiente:

\begin{center}
\emph{double **\textbf{dblPtAsigMemMatTrSup}(int intN)}
\end{center}

\begin{description}
\item[intN] orden de la matriz para la cual se pretende reservar memoria.
\end{description}

Por ejemplo:

\begin{center}
\emph{dblPtDato = \textbf{dblPtAsigMemMatTrSup}(intN);}
\end{center}

Reservar\'{\i}a memoria para una matriz triangular superior, cuadrada de 
orden \emph{intN}, los elementos de la matriz son de tipo \emph{double} y
la direcci\'on base de la matriz estar\'{\i}a almacenada en \emph{dblPtDato}, 
el cual es un puntero a puntero \emph{double}.\newline

\par Gr\'aficamente estariamos reservando memoria para una matriz del tipo:

\begin{flushleft}
	\begin{tabular}{|c|c|c|c|c|}
	\hline
	$a_{00}$ & $a_{01}$ & $a_{02}$ & $a_{03}$ & $a_{04}$ \\
	\hline
	\end{tabular}
\newline	
	\begin{tabular}{|c|c|c|c|}
	\hline
	$a_{10}$ & $a_{11}$ & $a_{12}$ & $a_{13}$ \\
	\hline
	\end{tabular}
\newline
	\begin{tabular}{|c|c|c|}
	\hline
	$a_{20}$ & $a_{21}$ & $a_{22}$ \\
	\hline
	\end{tabular}
\newline
	\begin{tabular}{|c|c|}
	\hline
	$a_{30}$ & $a_{31}$ \\
	\hline
	\end{tabular}
\newline
	\begin{tabular}{|c|}
	\hline
	$a_{40}$ \\
	\hline
	\end{tabular}
\end{flushleft}

Este es un caso particular tomando \emph{intN} = 5.\newline

Para hacer referencia a los datos de \emph{dblPtDato}:

\begin{itemize}
\item \emph{dblPtDato[i][j]} donde $i + j < \emph{intN}$
\item \emph{*(*(dblPtDato + i) + j)} donde $i + j < \emph{intN}$
\end{itemize}

\subsection{dblPtAsigMemMatTrInf}
Esta funci\'on reserva memoria para una matriz triangular inferior, cuadrada
de orden \emph{intN}, devuelve un puntero a puntero \emph{double} conteniendo
la direcci\'on del primer elemento.\newline

El prototipo de esta funci\'on es el siguiente:

\begin{center}
\emph{double **\textbf{dblPtAsigMemMatTrInf}(int intN)}
\end{center}

\begin{description}
\item[intN] orden de la matriz para la cual se pretende reservar memoria.
\end{description}

Por ejemplo:

\begin{center}
\emph{dblPtDato = \textbf{dblPtAsigMemMatTrInf}(intN);}
\end{center}

Reservar\'{\i}a memoria para una matriz triangular inferior, cuadrada de 
orden \emph{intN}, los elementos de la matriz son de tipo \emph{double} y
la direcci\'on base de la matriz estar\'{\i}a almacenada en \emph{dblPtDato}, 
el cual es un puntero a puntero \emph{double}.\newline

Gr\'aficamente estariamos reservando memoria para una matriz del tipo:

\begin{flushleft}
	\begin{tabular}{|c|}
	\hline
	$a_{00}$ \\
	\hline
	\end{tabular}
\newline
	\begin{tabular}{|c|c|}
	\hline
	$a_{10}$ & $a_{11}$ \\
	\hline
	\end{tabular}
\newline
	\begin{tabular}{|c|c|c|}
	\hline
	$a_{20}$ & $a_{21}$ & $a_{22}$ \\
	\hline
	\end{tabular}
\newline	
	\begin{tabular}{|c|c|c|c|}
	\hline
	$a_{30}$ & $a_{31}$ & $a_{32}$ & $a_{33}$ \\
	\hline
	\end{tabular}
\newline
	\begin{tabular}{|c|c|c|c|c|}
	\hline
	$a_{40}$ & $a_{41}$ & $a_{42}$ & $a_{43}$ & $a_{44}$ \\
	\hline
	\end{tabular}
\end{flushleft}

Este es un caso particular tomando \emph{intN} = 5.\newline

Para hacer referencia a los datos de \emph{dblPtDato}:

\begin{itemize}
\item \emph{dblPtDato[i][j]} donde $j \leq i$
\item \emph{*(*(dblPtDato + i) + j)} donde $j \leq i$
\end{itemize}

\newpage

\section{Liberaci\'on de memoria}
Estas funciones liberan la memoria previamente reservada a un \emph{puntero}.

\subsection{FreeMemDblMat}
Esta funci\'on libera la memoria asignada a una matriz de \emph{punteros}.\\

El prototipo de esta funci\'on es el siguiente:

\begin{center} 
\emph{void \textbf{FreeMemDblMat}(double **dblMatriz, int intFilas)}
\end{center}

\begin{description}
\item[dblMatriz] matriz de \emph{punteros} que se quiere destruir.
\item[intFilas] filas que tiene la matriz de \emph{punteros}.
\end{description}

Por ejemplo:

\begin{center}
\emph{\textbf{FreeMemDblMat}(dblMatriz, intFilas);}
\end{center}

Liberar\'{\i}a la memoria asignada a una matriz de \emph{punteros} de tipo
\emph{double}, \emph{dblMatriz}, la cual tiene \emph{intFilas} filas.

%
% INDICE DE MATERIAS
%

\index{matrices!reserva de memoria}
\index{vectores!reserva de memoria}
\index{memoria!liberaci\'on de}
