%
% datamatrix.h
%

\chapter{Matrices data structures (datamatrix.h)}

\section{Introduction}

Data structures for matrices are defined in \texttt{datamatrix.h} file.\\

\section{\textbf{biaMatrix} data structure} \label{sec:biaMatrix}

This data structure is used to store a matrix. \textbf{biaMatrix} data structure is defined in figure \ref{fig:biaMatrix} where:

\begin{description}
\item[intRows] number of rows.
\item[intCols] number of columns.
\item[**dblCoefs] pointer to store matrix coeficients.
\end{description}

\begin{figure}[!h]
\begin{verbatim}
typedef struct {
  int intRows,
      intCols;

  double **dblCoefs;
  } biaMatrix;   
\end{verbatim}
\caption{biaMatrix data structure.} \label{fig:biaMatrix}
\end{figure}

\FloatBarrier

\subsection{How to use it}

%
\begin{verbatim}
#include <stdlib.h>
#include <time.h>
#include <datamatrix.h>

int main (void) {

  /* Matrix data */
  int  rows = 8,
       cols = 8;

  /* Matrix declaration */
  biaMatrix myMatrix;

  /* Matrix data */
  myMatrix.intCols = cols;
  myMatrix.intRows = rows;

  /* Memory reservation por Matrix coefs */
  myMatrix.dblCoefs = 
      (double **)calloc(myMatrix.intRows, sizeof(double *));

  for(int i=0;i<=myMatrix.intRows;i++)
    myMatrix.dblCoefs[i] = 
        (double *)calloc(myMatrix.intCols, sizeof(double));

  /* Random coefs between 0 and 100 (not cryptographically secure) */
  srand(time(NULL));
  for(int i=0;i<myMatrix.intRows;i++)
    for(int j=0;j<myMatrix.intCols;j++)
      myMatrix.dblCoefs[i][j] = (double) (rand() % 100);

  /* your stuff here */
  ...

  return 0;
}
\end{verbatim}
