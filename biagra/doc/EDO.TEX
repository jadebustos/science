%
% EDO.H
%

\chapter{edo.h}

\section{Introducci\'on}
En este fichero de cabezera estan los prototipos de funciones para la
resoluci\'on num\'erica de ``\emph{\textbf{E.D.O's}}\footnote{Ecuaciones
diferenciales ordinarias.}''.

\section{M\'etodos Runge-Kutta para problemas escalares}
Para comprender mejor este apartado ser\'{\i}a buena idea mirar el 
apartado (\ref{sec:datosEDO}) en la p\'agina \pageref{sec:datosEDO} 
para comprender las \emph{estructuras de datos} aqui utilizadas, asi
como el ap\'endice \ref{sec:Runge} en la p\'agina \pageref{sec:Runge}.

\newpage

\subsection{ExplicitRungeKutta}

Funci\'on que resuelve un \emph{P.V.I.}\footnote{Problema de valores 
iniciales.} mediante un m\'etodo \emph{Runge-Kutta expl\'{\i}cito}.\newline

El prototipo de esta funci\'on es el siguiente:

\begin{center}
\emph{int \textbf{ExplicitRungeKutta}(DatosRK *ptstrDatos,\\
double (*PVI)(double dblX, double dblY))}
\end{center} 

\begin{description}
\item[ptstrDatos] \emph{puntero} a una variable del tipo 
\textbf{DatosRK}\footnote{Apartado (\ref{sec:datosRK}) en la p\'agina 
\pageref{sec:datosRK}.}, la cual contiene los datos necesarios para
la resoluci\'on del problema.
\item[PVI] \emph{puntero} a una funci\'on que devuelve el valor de la
ecuaci\'on diferencial en un punto.
\begin{description}
	\item[dblX] punto en el que queremos evaluar la ecuaci\'on diferencial.
	\item[dblY] valor de la ecuaci\'on diferencial en \emph{dblX}.
\end{description}
\end{description}

Por ejemplo, si tenemos el siguiente \emph{P.V.I.}:

\begin{center}
$
\left \{ \begin{array}{l}
y' = y(x) * \frac{x-y(x)}{x^2} \\
y(1) = 2
\end{array} \right.
$
\end{center}

la funci\'on \emph{PVI} ser\'{\i}a:

\begin{verbatim}
double PVI(double dblX, double dblY)
{
double  dblResultado;

dblResultado = dblY*((dblX-dblY)/(dblX*dblX));

return(dblResultado);
}
\end{verbatim}

La funci\'on, \emph{ExplicitRungeKutta}, devuelve los siguientes c\'odigos:

\begin{center}
\begin{tabular}{|l|l|}
\hline
\textbf{ERR\_AMEM} & Hubo error en la asignaci\'on de memoria. \\
\hline
\textbf{TRUE} & Se calcul\'o la aproximaci\'on. \\
\hline
\end{tabular}
\end{center}

\begin{center}
\emph{intResultado = \textbf{ExplicitRungeKutta}(\&varstrDatRK, PVI);}
\end{center}

Resolver\'{\i}a el \emph{P.V.I.} representado por la funci\'on \emph{PVI} 
utilizando los datos almacenados en la variable, del tipo \emph{DatosRK},
\emph{varstrDatRK} y almacenar\'{\i}a en \emph{intResultado} el c\'odigo
devuelto por la funci\'on.

\section{C\'alculo del n\'umero de nodos}

\subsection{intNumNodos}
Esta funci\'on calcula el n\'umero de nodos que podemos situar en un
intervalo.\newline

El prototipo de esta funci\'on es el siguiente:

\begin{center}
\emph{int \textbf{intNumNodos}(double dblLong, double dblPaso)}
\end{center}

\begin{description}
\item[dblLong] longitud del intervalo.
\item[dblPaso] distancia entre los nodos.
\end{description}

La funci\'on devuelve el n\'umero de nodos que podemos situar en dicho
intervalo.\newline

Por ejemplo:

\begin{center}
\emph{intResultado = \textbf{intNumNodos}(dblLong, dblPaso);}
\end{center}

Almacenar\'{\i}a en \emph{intResultado} el n\'umero de nodos que podemos 
situar en un intervalo de longitud \emph{dblLong} e igualmente espaciados
por \emph{dblPaso}.\\

Observaciones:

\begin{itemize}
\item La funci\'on supone que los argumentos que recibe son distintos de cero.
\item La funci\'on considerar\'a siempre los argumentos que recibe como
positivos, aunque sean negativos.
\end{itemize}


