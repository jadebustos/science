%
% struct.h
%

\chapter{struct.h}

\section{Introducci\'on}

En este fichero de cabezera se encuentran las estructuras de datos que utiliza
\BI para la resoluci\'on de problemas.\newline

\par Como ya se dijo anteriormente todas las estructuras ser\'an declaradas
mediante la palabra reservada \emph{typedef}.

\newpage

\section{Estructuras de datos para E.D.O's} \label{sec:datosEDO}

Estas estructuras de datos se utilizan para la resoluci\'on num\'erica de
E.D.O's\footnote{Ecuaciones Diferenciales Ordinarias.}.

\subsection{ButcherArray}

Esta estructura se utiliza para almacenar la \emph{notaci\'on matricial}
de los m\'etodos \emph{Runge-Kutta}\footnote{Ver ap\'endice sobre 
Runge-Kutta en la p\'agina \pageref{sec:Runge}.}.\newline

La declaraci\'on de esta estructura es la siguiente:

\begin{verbatim}
typedef struct
        {
        double  *dblC,
                *dblB,
                **dblMatriz;

        int intEtapas;
        } ButcherArray;
\end{verbatim}

El significado de cada uno de los miembros de esta estructura es el
siguiente:

\begin{description}
\item[dblC] es un vector de dimensi\'on \emph{strVariable.intEtapas} el cual 
contiene los
elementos $c_i$ del \emph{Runge-Kutta}, donde $0 \leq i < 
\emph{strVariable.intEtapas}$.
\item[dblB] es un vector de dimensi\'on \emph{strVariable.intEtapas} el cual 
contiene los
elementos $b_i$ del \emph{Runge-Kutta}, donde $0 \leq i < 
strVariable.intEtapas$.
\item[dblMatriz] es una matriz la cual contiene la matriz de coeficientes del
\emph{Runge-Kutta}.
\item[intEtapas] es el n\'umero de etapas que tiene el m\'etodo.
\end{description}

Para declarar una variable de este tipo:
\begin{center}
\emph{\textbf{ButcherArray} strVariable;}
\end{center}

\emph{strVariable} ser\'{\i}a una variable del tipo 
\emph{\textbf{ButcherArray}}.

\newpage

\subsection{DatosRK} \label{sec:datosRK}

Esta estructura se utiliza para almacenar todos los datos necesarios
en la ejecuci\'on del \emph{Runge-Kutta}, desde su notaci\'on matricial
hasta su inicializaci\'on, pasando por el paso utilizado.\newline

La declaraci\'on de esta \emph{estructura} es la siguiente:

\begin{verbatim}
typedef struct
        {	
        int     intNumAprox,
                intImplicito;
	        
        double  *dblPuntos,
                dblPaso,
                dblInicio,
                dblFinal;

        ButcherArray strCoefi;
        } DatosRK;
\end{verbatim}

El significado de cada uno de los miembros de esta estructura es el
siguiente:

\begin{description}
\item[intNumAprox] n\'umero de aproximaciones que se realizar\'an con el
m\'etodo, es decir contendr\'a la dimensi\'on de \emph{strVariable.dblPuntos}.
\item[intImplicito] contendr\'a el valor \textbf{TRUE} si se trata de un 
m\'etodo \emph{impl\'{\i}cito}, en caso contrario su valor ser\'a distinto de
\textbf{TRUE}.
\item[dblPuntos] vector de dimensi\'on \emph{strVariable.intNumAprox} el 
cual contendr\'a, en \emph{strVariable.dblPuntos[i],} 
las aproximaciones en los diferentes $x_i$, donde:

\begin{description}
\item[]$x_i = (\emph{strVariable.dblInicio}) + i * (\emph{strVariable.dblPaso})$
\item[]$0 \leq i < \emph{strVariable.intNumAprox}$
\end{description}

\item[dblPaso] tama\~no del paso que utilizara el m\'etodo.\newpage
\item[dblInicio] primer punto, en el que es conocido el valor de la funci\'on,
en el cual nos apoyamos para calcular las dem\'as aproximaciones.
\item[dblFinal] es el \'ultimo punto en el que calcularemos una aproximaci\'on
de la ecuaci\'on diferencial.
\item[strCoefi] es una variable del tipo \emph{\textbf{ButcherArray}}, la cual
contendr\'a la notaci\'on matricial del m\'etodo.
\end{description}

Para declarar una variable de este tipo:

\begin{center}
\emph{\textbf{DatosRK} strVariable;}
\end{center}

\emph{strVariable} ser\'{\i}a una variable del tipo \emph{\textbf{DatosRK}}.

\newpage

\section{Estructuras de datos para la aproximaci\'on de funciones}

\subsection{DatosAprxFunc} \label{sec:DatosAprxFunc}

La declaraci\'on de esta estructura es la siguiente:

\begin{verbatim}
typedef struct

        {
        int     intNMI;

        double  dblx0,
                dblSolucion,
                dblTol,
                dblError; 
        } DatosAprxFunc;
\end{verbatim}

El significado de cada uno de los miembros de esta estructura es el siguiente:

\begin{description}
\item[intNMI] n\'umero m\'aximo de iteraciones.
\item[dblx0] aproximaci\'on inicial.
\item[dblSolucion] aproximaci\'on final de la raiz.
\item[dblTol] tolerancia con la que se va a aproximar la raiz.
\item[dblError] error cometido al aproximar la raiz, valor absoluto de la 
distancia entre las dos \'ultimas aproximaciones.
\end{description}

Para declarar una variable de este tipo:

\begin{center}
\emph{\textbf{DatosAprxFunc} strVariable;}
\end{center}

\emph{strVariable} ser\'{\i}a una variable del tipo \textbf{DatosAprxFunc}.

\section{Estructuras de datos para matrices}

Estas estructuras de datos se utilizan para el uso de matrices.

\subsection{Matriz} \label{sec:strMatriz}

Esta estructura se utiliza para almacenar matrices.\newline

La declaraci\'on de esta estructura es la siguiente:

\begin{verbatim}
typedef struct

        {
        int     intFilas,
                intColumnas;

        double  **dblCoefi;
        } Matriz;
\end{verbatim}

El significado de cada uno de los miembros de esta estructura es el siguiente:

\begin{description}
\item[intFilas] n\'umero de filas de la matriz.
\item[intColumnas] n\'umero de columnas de la matriz.
\item[dblCoefi] matriz que contiene los coeficientes de la matriz que estamos
representando mediante esta estructura de datos.
\end{description}

Para declarar una variable de este tipo:

\begin{center}
\emph{\textbf{Matriz} strVariable;}
\end{center}

\emph{strVariable} ser\'{\i}a una variable del tipo \textbf{Matriz}.

\section{Estructuras de datos para n\'umeros complejos} \label{sec:complejos}

Estas estructuras de datos se utilizan para el uso de n\'umeros complejos.

\subsection{Complejo}
Esta estructura se utiliza para almacenar las coordenadas de un n\'umero
complejo, en coordenadas cartesianas.\newline

La declaraci\'on de esta \emph{estructura} es la siguiente:

\begin{verbatim}
typedef struct

        {
        double  dblReal,
                dblImag;
        } Complejo;
\end{verbatim}

El significado de cada uno de los miembros de esta estructura es el siguiente:

\begin{description}
\item[dblReal] parte real del n\'umero complejo.
\item[dblImag] parte imaginaria del n\'umero complejo.
\end{description}

Para declarar una variable de este tipo:

\begin{center}
\emph{\textbf{Complejo} strVariable;}
\end{center}

\emph{strVariable} ser\'{\i}a una variable del tipo \textbf{Complejo}.

\newpage

\subsection{Polar}
Esta estructura se utiliza para almacenar un n\'umero complejo en 
coordenadas polares.\newline

La declaraci\'on de esta \emph{estructura} es la siguiente:

\begin{verbatim}
typedef struct

        {
        double  dblMod,
                dblArg;	
        } Polar;
\end{verbatim}

El significado de cada uno de los miembros de esta estructura es el siguiente:

\begin{description}
\item[dblMod] m\'odulo del n\'umero complejo.
\item[dblArg] argumento del n\'umero complejo.
\end{description}

Para declarar una variable de este tipo:

\begin{center}
\emph{\textbf{Polar} strVariable;}
\end{center}

\emph{strVariable} ser\'{\i}a una variable del tipo \textbf{Polar}.

