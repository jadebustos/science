%
% MATRIZ.H
%

\chapter{matriz.h}

\section{Introducci\'on}
En este fichero de cabezera estan los prototipos de funciones para el
tratamiento de \emph{matrices}.

\section{Almacenamiento de matrices}
Las matrices se almacenar\'an en la estructura de datos \textbf{Matriz}%
\footnote{Apartado (\ref{sec:strMatriz}) en la p\'agina
\pageref{sec:strMatriz}.}, y almacenaremos los coeficientes de la matriz en el
miembro \emph{dblCoefi}, para ello este miembro ha de estar correctamente
dimensionado\footnote{Ve\'ase el apartado (\ref{sec:asigmemmat}) en la
p\'agina \pageref{sec:asigmemmat}}.

\subsection{MatrizIdentidad}
Funci\'on que almacena la matriz identidad.\newline

El prototipo de esta funci\'on es el siguiente:

\begin{center}
\emph{void \textbf{MatrizIdentidad}(Matriz *ptstrMatriz)};
\end{center}

\begin{description}
\item[ptstrMatriz] puntero a una variable del tipo \textbf{Matriz}.
\end{description}

Por ejemplo:

\begin{center}
\emph{\textbf{MatrizIdentidad}(\&Mat);}
\end{center}

Almacenar\'{\i}a la matriz identidad de orden \emph{Mat.intFilas} en 
\emph{Mat}, donde \emph{Mat} es una variable del tipo \textbf{Matriz}%
\footnote{Apartado (\ref{sec:strMatriz}) en la p\'agina 
\pageref{sec:strMatriz}.}.

\newpage

\subsection{MatrizHomotecia}
Funci\'on que almacena la matriz de la homotecia de raz\'on \emph{dblRazon}.\\

El prototipo de esta funci\'on es el siguiente:

\begin{center}
\emph{void \textbf{MatrizHomotecia}(Matriz *ptstrMatriz, double dblRazon)}
\end{center}

\begin{description}
\item[ptstrMatriz] puntero a una variable del tipo \textbf{Matriz}.
\item[dblRazon] raz\'on de la homotecia.
\end{description}

Por ejemplo:

\begin{center}
\emph{\textbf{MatrizHomotecia}(\&Mat, dblRazon);}
\end{center}

Almacenar\'{\i}a la matriz de la homotecia de orden \emph{Mat.intFilas} y de 
raz\'on \emph{dblRazon} $= \lambda$, en \emph{Mat}, es decir:

\begin{center}
\emph{dblMatriz}
$
= \left( \begin{array}{ccccc}
\lambda & 0 & \cdots & \cdots & 0 \\
0 & \lambda & 0 & & \vdots \\
\vdots & 0 & \ddots & \ddots & \vdots \\
\vdots & & \ddots & \lambda & 0 \\
0 & \cdots & \cdots & 0 & \lambda \\
\end{array} \right)
$
\end{center}

donde \emph{Mat} es una variable del tipo \textbf{Matriz}\footnote{Apartado
(\ref{sec:strMatriz}) en la p\'agina \pageref{sec:strMatriz}.}.

\subsection{MatrizNula}
Funci\'on que almacena la matriz nula.\newline

El prototipo de esta funci\'on es el siguiente:

\begin{center}
\emph{void \textbf{MatrizNula}(Matriz *ptstrMatriz)}
\end{center}

\begin{description}
\item[ptstrMatriz] puntero a una variable del tipo \textbf{Matriz}.
\end{description}

Por ejemplo:

\begin{center}
\emph{\textbf{MatrizNula}(\&Mat);}
\end{center}

Almacenar\'{\i}a la matriz nula, de \emph{Mat.intFilas} y 
\emph{Mat.intColumnas}, en \emph{Mat}, donde \emph{Mat} es una variable del
tipo \textbf{Matriz}\footnote{Apartado (\ref{sec:strMatriz}) en la p\'agina
\pageref{sec:strMatriz}.}.

\newpage

\subsection{MatrizTraspuesta}
Funci\'on que almacena la matriz traspuesta de una dada.\newline

El prototipo de esta funci\'on es el siguiente:

\begin{center}
\emph{void \textbf{MatrizTraspuesta}(Matriz *ptstrMatriz, Matriz *ptstrTras)} 
\end{center}

\begin{description}
\item[ptstrMatriz] puntero a una variable del tipo \textbf{Matriz}.
\item[ptstrTras] puntero a una variable del tipo \textbf{Matriz}.
\end{description}

Por ejemplo:

\begin{center}
\emph{\textbf{MatrizTraspuesta}(\&Mat, \&Tras);}
\end{center}

Almacenar\'{\i}a la matriz traspuesta de \emph{Mat}\footnote{Matriz con
\emph{Mat.intFilas} filas y \emph{Mat.intColumnas} columnas.} en \emph{Tras}%
\footnote{Matriz con \emph{Tras.intFilas} filas y \emph{Tras.intColumnas} 
columnas.}, donde ambas son variables del tipo \textbf{Matriz}\footnote{%
Apartado (\ref{sec:strMatriz}) en la p\'agina \pageref{sec:strMatriz}.}.\newline

Observaciones:

\begin{itemize}
\item El miembro \emph{dblCoefi} de la variable \emph{Tras} ha de estar 
correctamente dimensionado.
\item Los miembros \emph{intFilas} e \emph{intColumnas} de la variable 
\emph{Tras} han de estar correctamente inicializados. No es obligatorio, pero
si recomendable.
\end{itemize}
