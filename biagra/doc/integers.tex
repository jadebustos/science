%
% integers.h
%

\chapter{Integer numbers (integers.h)} \label{ch:integers}

\section{Introduction}

\BI \ includes functions about integer numbers in \texttt{integers.h} file.\\

\section{Sum integers}

\subsection{\texttt{uintSumFirstNIntegers} function} \label{sec:uintSumFirstNIntegers}

This function gets the sum of the first $n$ integers.\\ \\
%
The definition of this function:
%
\begin{verbatim}
unsigned uintSumFirstNIntegers(int n);
\end{verbatim}
%
If the sum is bigger than an unsigned int $0$ is returned.

\section{Prime numbers}

\subsection{\texttt{isPrime} function}

This function checks if a number is a prime number.\\ \\
%
The definition of this function:
%
\begin{verbatim}
int isPrime(int intN);  
\end{verbatim}
%
The following codes are returned:

\begin{center}
\begin{tabular}{|l|l|}
\hline
\textbf{BIA\_FALSE} & \texttt{intN} is not a prime number \\
\hline
\textbf{BIA\_TRUE} & \texttt{intN} is a prime number \\
\hline
\end{tabular}
\end{center}

\subsection{\texttt{getFirstNPrimes} function}

This function checks if a number is a prime number.\\ \\
%
The definition of this function:
%
\begin{verbatim}
void getFirstNPrimes(unsigned int *ptPrimes, int intNumber, int *ptCalc);
\end{verbatim}
%
where:
%
\begin{description}
\item[*ptPrimes] array where primes will be stored. Memory allocation for this array has to be initialized before using this function.
\item[intNumber] number of primes to be computed.
\item[*ptCalc] in this variable the total amount of computed primes will be stored.
\end{description}
