\documentclass[a4paper,12pt,twoside,openright]{report}

%\addtolength{\voffset}{-4cm}
\usepackage[spanish]{babel}
\usepackage{draftwatermark}

\pagestyle{headings}

\SetWatermarkText{Draft}
\SetWatermarkScale{5}
\SetWatermarkColor{blue}

\author{Jos\'e Angel de Bustos P\'erez}

\newcommand{\BI}
	{\emph{B.I.A.G.R.A} }

\newcommand{\yo}
	{\emph{Jos\'e Angel de Bustos P\'erez}}

\frenchspacing

\hyphenation{si-guien-te apro-xi-ma-cio-nes FORTRAN double intGrado 
dblDerivada dblResultado intNMI intOrden dblMatriz Newton intResultado
cabezera coor-de-na-das de-pen-dien-do di-men-sio-na-do re-pre-sen-tan-do}

\begin{document}

\thispagestyle{empty}

\textbf{
\begin{center}
\Huge{B.I.A.G.R.A.} \\[.75cm]
%\end{center}
\LARGE BI\Large bliotec\LARGE A \Large de pro\LARGE GR\Large amaci\'on 
cient\'{\i}fic\LARGE A\\[5cm]
\end{center}
}

\large 
\begin{flushright}
\yo \\
$<$jadebustos@mixmail.com$>$\\ \ \\ 
Versi\'on $1.0$, \today .\\
\textbf{\LaTeXe}
\end{flushright}

\normalsize

\tableofcontents

\listoftables

%
% CAPITULO 1 �Que es B.I.A.G.R.A?
%

\chapter{?`Que es \BI?}

\begin{itemize}

\item \BI es una BIbliotecA de proGRamaci\'on cient\'{\i}ficA programada 
enteramente en \textbf{C} y pensada para ser utilizada en programas
escritos en \textbf{C}.
\item \BI ha sido desarrollada y testeada bajo \emph{LiNUX}, por lo cual sus
caracter\'{\i}sticas se adaptan a las de este sistema operativo. 
\item \BI es de libre distribuci\'on y su autor no se hace responsable de su
mal uso y/o de sus posibles fallos.

\end{itemize}

\section{?`Lenguaje C? }
Alguien se puede preguntar el motivo por el cual he elegido el Lenguaje 
\textbf{C} y no \emph{FORTRAN}, lenguaje tradicional para la programaci\'on
cient\'{\i}fica.\newline
Los motivos son varios siendo el fundamental la potencia,
la modularidad de \textbf{C} y el hecho de ser un lenguaje de proposito general.
\newline

Las razones que alegan la inmesa mayor\'{\i}a de programadores de 
\emph{FORTRAN} sobre las ventajas que tiene \emph{FORTRAN}:

\begin{enumerate}
\item Es un lenguaje facil de usar.
\item Casi todo el mundo utiliza \emph{FORTRAN}, en este tipo de
      programaci\'on.
\item Hay mucho c\'odigo escrito en \emph{FORTRAN}.
\end{enumerate}

Aplicando esta forma de pensar bien podr\'{\i}amos pensar que el 
\emph{sol gira sobre la tierra}, ya que hace muchos a\~nos la mayor\'{\i}a de
la gente estaba de acuerdo con esta idea.

\newpage

Las razones por las que fundamento la elecci\'on de \textbf{C} son:

\begin{enumerate}
\item Es un lenguaje potente y que genera un c\'odigo de muy r\'apida 
      ejecuci\'on.
\item Es un lenguaje flexible.
\item Es un lenguaje estructurado.
\item Es un lenguaje de proposito general.
\item El c\'odigo escrito en \textbf{C} es portable y la depuraci\'on y 
      mantenimiento de programas es ``sencilla''\footnote{Si el c\'odigo 
      esta bien estructurado y se ajusta al \emph{ANSI C}.}.
\item Tiene una gran potencia en el manejo del hardware, lo que implica
      una gran potencia en el manejo de modos gr\'aficos\footnote{Deseable
      para seg\'un que aplicaciones.}.
\end{enumerate}

\section{Algunas ideas generales sobre \BI}
Lo primero que debemos destacar es que esta biblioteca esta pensada para
programadores de \textbf{C} y ha sido desarrollada y testeada bajo 
\textbf{LiNUX}, luego en el resto de este documento se supondra que el lector
esta utilizando \textbf{LiNUX} o cualquier otro sistema tipo \textbf{UNIX}.\\

Esta biblioteca esta pensada para resolver problemas generales, no para 
resolver tipos particulares de problemas, es decir, esta pensada para
realizar programas, que, interactuando con el usuario puedan resolver un
problema sin tener que modificar el c\'odigo fuente cuando varien las
condiciones del problema.\newline

Por ejemplo, escribir un programa para invertir matrices, de
cualquier orden, en vez de escribir un programa para invertir matrices de
orden \emph{n} y cuando se quiera invertir una matriz de un orden $m \neq n$
sea necesario cambiar el c\'odigo fuente y recompilar el programa.\newline

Para lograr esto se han utilizado \emph{punteros}, es decir se ha evitado,
en la medida de lo posible, el uso de \emph{arrays}.\newpage

Cuando hablemos de \emph{vectores} nos estaremos refiriendo a un
\emph{puntero}, normalmente a datos del tipo \emph{double}, para el cual se
ha reservado memoria para contener \emph{n} datos. Del mismo modo cuando nos
refiramos a \emph{matrices} estaremos hablando de matrices de \emph{punteros}, 
tambi\'en a datos del tipo \emph{double}(normalmente), para las cuales se ha
reservado memoria para contener $\mbox{\emph{n * m}}$ datos.\newline

En aquellos problemas que se utiliza un conjunto de varios datos se ha
optado por agruparlos dentro de una estructura\footnote{Declarada mediante
typedef.} para mayor comodidad.\newline

En aquellos problemas en los que se pueda cometer alg\'un error como por
ejemplo:

\begin{itemize}
\item Fallo en asignaciones de memoria.
\item Divisiones por cero.
\item \ldots
\end{itemize}

la funci\'on que resuelve ese problema devolver\'a un entero indicando cual
fue el estado en el que se termin\'o la ejecuci\'on de la funci\'on.\newline

Cuando una funci\'on devuelva un dato, que no sea un c\'odigo que indique
el estado en el que se termin\'o la ejecuci\'on de la funci\'on, se indicar\'a 
anteponiendo un prefijo al nombre de la funci\'on, el cual indicar\'a que tipo
de dato devuelve, por ejemplo:

\begin{enumerate}
\item \emph{double dblFuncion(\ldots)} funci\'on que devuelve un dato de
      tipo \emph{double}.
\item \emph{int intFuncion(\ldots)} funci\'on que devuelve un dato de tipo
      \emph{int}.
\item \emph{double *dblPtFuncion(\ldots)} funci\'on que devuelve un dato de
      tipo puntero a \emph{double}.
\item \emph{void Funcion(\ldots)} funci\'on que no devuelve ning\'un dato.
\end{enumerate}

\newpage

\section{C\'odigos devueltos por las funciones}

No es obligatorio que las funciones devuelvan un valor.\newline
Cuando una funci\'on devuelva un valor ser\'a por dos razones:

\begin{itemize}
\item Para devolver el resultado de una operaci\'on.
\item Para informar de como termin\'o una operaci\'on.
\end{itemize} 

Es este \'ultimo caso el que nos incumbe.\newline

Cuando una funci\'on devuelva un dato indicando como termin\'o una
determinada operaci\'on, este dato ser\'a, necesariamente, un entero y los
c\'odigos devueltos los podemos ver en la p\'agina \pageref{sec:tiposdecodigos}.

\section{Como instalar \BI en LiNUX}
Para instalar \BI lo primero que hay que hacer es entrar en el sistema
como \textbf{root} y situarse en el directorio donde esten los fuentes
de la biblioteca.

\subsection{Intalaci\'on de la biblioteca \BI est\'atica}
Para instalar este tipo bibilioteca se puede hacer de dos formas:

\begin{enumerate}
\item \emph{./instalar estatica}
\item \emph{make estatica}
\end{enumerate}

En realidad ambas hacen lo mismo, la opci\'on con \emph{make} en realidad 
ejecuta \emph{./instalar estatica}.\newline

Al realizar cualquiera de estas dos opciones se copiar\'an los ficheros
cabezera a \emph{\textbf{/usr/include/biagra}} y a continuaci\'on se crear\'a la
biblioteca\\ \emph{\textbf{/usr/lib/libbiagra.a}}.\newline

Luego si se utiliza una funci\'on de esta biblioteca, cuyo prototipo est\'a en 
el fichero de cabecera \emph{rngkutta.h} habr\'a que incluir en las directivas 
al prepocesador \textbf{\#include $<$biagra/rngkutta.h$>$}.

\newpage

\subsection{Intalaci\'on de la biblioteca \BI din\'amica ELF}

\subsection{Instalaci\'on de ambas bibliotecas}
Para instalar la biblioteca \BI en su forma est\'atica y din\'amica ELF:

\begin{center}
\emph{make todo}
\end{center}

Esto lo que hace es ejecutar primero

\begin{center}
\emph{./instalar estatica}
\end{center}

lo cual instalar\'a la biblioteca est\'atica, y luego 

\begin{center}
\emph{./instalar elf}
\end{center}

lo cual instalar\'a la biblioteca din\'amica ELF.

\section{Como utilizar \BI en LiNUX}
\BI es una biblioteca para programaci\'on cient\'{\i}fica, 
desarrollada para ser usada en programas escritos en \textbf{C}, se
distribuye en varios formatos:

\begin{description}
\item[Biblioteca est\'atica]
\item[Biblioteca din\'amica ELF]
\end{description}

\subsection{Biblioteca est\'atica}
Para el uso de esta biblioteca hay que indicarle al \emph{montador} que
biblioteca debe \emph{enlazar}.
Por ejemplo, supongamos que hemos escrito un programa para resolver una
ecuaci\'on diferencial por un m\'etodo \emph{Runge-Kutta} y hemos utilizado
funciones cuyos prototipos estan en \textbf{edo.h} y \textbf{rngkutta.h}, si
nuestro programa es \mbox{\emph{programa.c}}, para crear el ejecutable:

\begin{center}
\emph{gcc programa.c -o programa -l\textbf{biagra} -l\textbf{m}\footnote{\BI 
utiliza la biblioteca estandar matem\'atica.}} 
\end{center}

\subsection{Biblioteca din\'amica ELF}

\section{Novedades en la versi\'on $1.0$}
Pues todo ya que es la primera versi\'on.

\section{Condiciones de uso}

\begin{enumerate}
\item El autor no se responsabiliza de su mal uso, de su posibles fallos%
\footnote{Si los hubiera o hubiese.}, ni de las modificaciones que sean
realizadas a dicha biblioteca.
\item Es una biblioteca de libre distribuci\'on.
\item Se puede modificar y a\~nadir c\'odigo, pero nunca distribuirlo bajo
el nombre de \BI, en caso de modificaci\'on de la biblioteca se deben indicar
aquellas funciones que fueron modificadas o a\~nadidas.
\end{enumerate}

\section{Sobre el autor}

Si alguien desea ponerse en contacto con el autor de esta biblioteca:

\begin{center}
\textbf{jadebustos@mixmail.com}
\end{center}

Si tienes alguna sugerencia, pregunta, o has encontrado alg\'un error en la 
biblioteca ya sabes donde encontrarme.\newline

%
% COMO UTILIZAR BIAGRA
%

\include{como}

%
% COMPLEJO.H
%

\include{complejo}

%
% CONST.H
%

%
% const.h
%

\chapter{B.I.A.G.R.A constants (const.h)} \label{ch:mathematicalConsts}

\section{Introduction}

\BI \ includes its own constants to be used if needed.\\

These constants are defined in \texttt{const.h}.

\section{Mathematical constants} \label{sec:mathematicalConsts}

Table \ref{tab:mathematicalConsts} shows the \BI's mathematical constanst.

\begin{table}[!h]
  \begin{center}
  \begin{tabular}{|c|c|c|}
    \hline
    \textbf{Constant} & \textbf{Name} & \textbf{Value} \\
    \hline
    $\texttt{e}$ & \texttt{BIA\_E} & $2.71828182845904523536029$ \\
    \hline
    $\pi$ & \texttt{BIA\_PI} & 3.14159265358979323846264 \\
    \hline
  \end{tabular}
  \end{center}
\caption{\BI\ mathematical constants.} \label{tab:mathematicalConsts}
\end{table}

\FloatBarrier

\section{Logical constants}

The following logical constants are defined:

\begin{description}
\item[BIA\_FALSE] when a condition is not met.
\item[BIA\_TRUE] when a condiction is met.
\end{description}

\section{Error constants}

The following error constants are defined:

\begin{description}
\item[BIA\_ZERO\_DIV] division by zero.
\item[BIA\_MEM\_ALLOC] error in memory allocation.
\end{description}


%
% EDO.H
%

\include{edo}

%
% ENTEROS.H
%

\include{enteros}

%
% MATRIZ.H
%

\include{matriz}

%
% POLINOMIOS.H
%

\include{polinomios}

%
% PRIMOS.H
%

\include{primos}

%
% RAIZFUNC.H
%

\include{raizfunc}

%
% RESMEN.H
%

%
% resmem.h
%

\chapter{Memory allocation (resmem.h)}

\section{Introduction}

\BI\ includes its own memory allocation functions which are defined in \texttt{resmem.h} file.

\section{Vector's memory allocation}

Some functions are provided to handle memory allocations for vectors.

\subsection{\texttt{dblPtMemAllocVec} function} \label{sec:dblPtMemAllocVec}

This functions allocates memory for a vector of doubles.\\

The definition of this function:
%
\begin{verbatim}
double *dblPtMemAllocVec(int intElements);  
\end{verbatim}
%
This function has only one argument, \texttt{intElements}, which is the dimension of the vector and a \texttt{double} pointer is returned.

\section{Matrix memory allocation}

Some functions are provided to handle memory allocations for vectors.

\subsection{\texttt{dblPtMemAllocMat} function} \label{sec:dblPtMemAllocMat}

This function allocates memory for a matrix of doubles.\\

The definition of this function:
%
\begin{verbatim}
double **dblPtMemAllocMat(int intRows, int intCols);
\end{verbatim}
%
where:
%
\begin{description}
\item[intRows] number of rows.
\item[intCols] number of columns.
\end{description}

\subsection{\texttt{dblPtMemAllocUpperTrMat} function} \label{sec:dblPtMemAllocUpperTrMat}

This function allowcates memory for a upper triangular square matrix.\\

The definition of this function:
%
\begin{verbatim}
double **dblPtMemAllocUpperTrMat(int intOrder);
\end{verbatim}

This function has only one argument, \texttt{intOrder}, which is the order of the matrix and a \texttt{double} pointer to pointer is returned.\\

In a upper triangular square matrix all elements below the diagonal are zero: 
%
\begin{displaymath}
\left( \begin{array}{ccccc}
  a_{0,0} & a_{0,1} & a_{0,2} & a_{0,3} & a_{0,4} \\
  0      & a_{1,1} & a_{1,2} & a_{1,3} & a_{1,4} \\ 
  0      & 0      & a_{2,2} & a_{2,3} & a_{2,4} \\
  0      & 0      & 0      & a_{3,3} & a_{3,4} \\
  0      & 0      & 0      & 0      & a_{4,4} \\
\end{array} \right)
\end{displaymath}
%
For \texttt{intOrder = 5}:
%
\begin{verbatim}
myMatrix = dbpPtMemAllocUpperTrMat(5);  
\end{verbatim}
%
and:
\begin{center}
  \begin{tabular}{|c|c|c|c|}
    \hline
    \textbf{Pointer} & \textbf{\# elements} & \textbf{First element} & \textbf{Last element}\\
    \hline
    \texttt{myMatrix[0]} & 5 & 0 & 4\\
    \hline
    \texttt{myMatrix[1]} & 4 & 0 & 3\\
    \hline
    \texttt{myMatrix[2]} & 3 & 0 & 2\\
    \hline
    \texttt{myMatrix[3]} & 2 & 0 & 1\\
    \hline
    \texttt{myMatrix[4]} & 1 & 0 & 0\\
    \hline
  \end{tabular}
\end{center}
%
so:
%
\begin{displaymath}
  myMatrix[i][j] = *(*(myMatrix + i) + j) = \left\{ \begin{array}{ll}
    a_{i,j+i} & \forall \ i \le j \\
     & \\
    0 & \forall \ i > j
    \end{array} \right.    
\end{displaymath}

\subsection{\texttt{dblPtMemAllocLowerMat} function} \label{sec:dblPtMemAllocLowerMat}

This function allowcates memory for a lower triangular square matrix.\\

The definition of this function:
%
\begin{verbatim}
double **dblPtMemAllocLowerTrMat(int intOrder);
\end{verbatim}

This function has only one argument, \texttt{intOrder}, which is the order of the matrix and a \texttt{double} pointer to pointer is returned.\\

In a lower triangular square matrix all elements above the diagonal are zero: 
%
\begin{displaymath}
\left( \begin{array}{ccccc}
  a_{0,0} & 0      & 0      & 0      & 0 \\
  a_{1,0} & a_{1,1} & 0      & 0      & 0 \\ 
  a_{2,0} & a_{2,1} & a_{2,2} & 0      & 0 \\
  a_{3,0} & a_{3,1} & a_{3,2} & a_{3,3} & 0 \\
  a_{4,0} & a_{4,1} & a_{4,2} & a_{4,3} & a_{4,4} \\
\end{array} \right)
\end{displaymath}
%
For \texttt{intOrder = 5}:
%
\begin{verbatim}
myMatrix = dbpPtMemAllocLowerTrMat(5);  
\end{verbatim}
%
and:
\begin{center}
  \begin{tabular}{|c|c|c|c|}
    \hline
    \textbf{Pointer} & \textbf{\# elements} & \textbf{First element} & \textbf{Last element}\\
    \hline
    \texttt{myMatrix[0]} & 1 & 0 & 0\\
    \hline
    \texttt{myMatrix[1]} & 2 & 0 & 1\\
    \hline
    \texttt{myMatrix[2]} & 3 & 0 & 2\\
    \hline
    \texttt{myMatrix[3]} & 4 & 0 & 3\\
    \hline
    \texttt{myMatrix[4]} & 5 & 0 & 4\\
    \hline
  \end{tabular}
\end{center}
%
so:
%
\begin{displaymath}
  myMatrix[i][j] = *(*(myMatrix + i) + j) = \left\{ \begin{array}{ll}
    a_{i,j} & \forall \ i \le j \\
     & \\
    0 & \forall \ i < j
    \end{array} \right.    
\end{displaymath}

\section{Freeing memory} \label{sec:freeingMemory}

\BI\ includes its own functions to free memory.

\subsection{\texttt{freeMemDblMat} function} \label{sec:freeMemDblMat}




%
% RNGKUTTA.H
%

%
% RNGKUTTA.H
%

\chapter{Runge-Kutta methods (rngkutta.h)}

\section{Introduction}

\textbf{Runge-Kutta} are a family of implicit and explicit iterative methods used to approximate solutions of ordinary differential equations or \textbf{ODE}.\\

Butcher matricial notation is used in this implementation.\\

\section{Data structures}

\subsection{\texttt{biaButcherArray} data structure}

This structure is used to store the Butcher matricial notation.\\

Data structure is defined in figure \ref{fig:biaButcherArray} where:
%
\begin{description}
%
\item[intStages] method stages.
%
\item[*dblC] $c_i$ coefficients stored in an array with size \texttt{intStages}.
%
\item[*dblB] $b_i$ coefficients stored in an array with size \texttt{intStages}.
%
\item[**dblMatrix] matrix to store $a_{i,j}$ method's coeficients.  
%
\end{description}

\begin{figure}[!h]
\begin{verbatim}
typedef struct {
  double  *dblC,
          *dblB,
          **dblMatrix;

  int     intStages;
} biaButcherArray;
\end{verbatim}
\caption{biaButcherArray data structure.} \label{fig:biaButcherArray}
\end{figure}
%
\FloatBarrier

\subsection{\texttt{DataRK} data structure}

This structure is used to store all the data needed to apply a Runge-Kutta method.\\

Data structure is defined in figure \ref{fig:biaDataRK} where:
%
\begin{description}
%
\item[intNumApprox] number of approximations to be done (size of the array \texttt{dblPoints}).
%
\item[intImplicit] when the Runge-Kutta method is implicit or not. The following constants are defined in the header file:
%
  \begin{center}
  \begin{tabular}{|c|c|}
    \hline
    \textbf{Name} & \textbf{Value} \\
    \hline
    \textbf{BIA\_IMPLICIT\_RK\_TRUE} & $0$ \\
    \hline
    \textbf{BIA\_IMPLICIT\_RK\_FALSE} & $1$ \\
    \hline
  \end{tabular}
  \end{center}  
%
\item[*dblPoints]
%
\item[dblStepSize]
%
\end{description}

\begin{figure}[!h]
\begin{verbatim}
typedef struct {
  int intNumApprox,
      intImplicit;

  double  *dblPoints,
          dblStepSize,
          dblFirst,
          dblLast;

  biaButcherArray strCoefs;
} biaDataRK;
\end{verbatim}
\caption{biaDataRK data structure.} \label{fig:biaDataRK}
\end{figure}
%
\FloatBarrier

\section{Explicit Runge-Kutta methods}

Let's assume that the initial value problem (I.V.P.) or Cauchy problem we want to solve is:
%
\begin{eqnarray*}
  y'(x) & = & f(x, y(x)) \\
  y(x_0) & = & y_0
\end{eqnarray*}
%
The family of explicit \textbf{Runge-Kutta} methods is given by:
%
\begin{displaymath}
  y_{n+1} = y_n  + h \cdot \sum_{i=1}^s b_i \cdot k_i
\end{displaymath}
%
where:
%
\begin{eqnarray*}
  k_i & = & f(x_n + (h\cdot c_i), y_n + h\cdot \left(\sum_{j=1}^{i-1} a_{i,j} \cdot k_j \right) )\\
  c_i & = & \sum_{j=1}^{i-1} a_{i,j} \qquad \textrm{where} \qquad i \in \{2,\dots,s\}
\end{eqnarray*}

\subsection{\texttt{ExplicitRungeKutta} function}

\subsection{\texttt{RungeKuttaClasico} function}

Funci\'on que inicializa los coeficientes para el m\'etodo \emph{Runge-Kutta 
Cl\'asico}, el cual es un m\'etodo de $4$ etapas y orden $4$.\newline

La notaci\'on matricial del m\'etodo es la siguiente:

\begin{center}
$
\begin{array}{c|cccc}
0 & 0 \\
\frac{1}{2} & \frac{1}{2} & 0 \\
\frac{1}{2} & 0 & \frac{1}{2} & 0 \\
1 & 0 & 0 & 1 & 0 \\
\hline
 & \frac{1}{6} & \frac{1}{3} & \frac{1}{3} & \frac{1}{6} \\
\end{array}
$
\end{center}

El prototipo de esta funci\'on es el siguiente:

\begin{center}
\emph{int \textbf{RungeKuttaClasico}(DatosRK *ptstrDatos)}
\end{center}

\begin{description}
\item[ptstrDatos] puntero a una variable de \emph{estructura} del tipo
\emph{DatosRK}.
\end{description}

La funci\'on devuelve los siguientes c\'odigos:

\begin{center}
\begin{tabular}{|l|l|}
\hline
\textbf{ERR\_AMEM} & Hubo un error en la asignaci\'on de memoria. \\
\hline
\textbf{TRUE} & Se inicializaron con \'exito los coeficientes. \\
\hline
\end{tabular}
\end{center}

Por ejemplo:

\begin{center}
\emph{intResultado = \textbf{RungeKuttaClasico}(\&varstrDatRK);}
\end{center}

Inicializar\'{\i}a los coeficientes del m\'etodo en la variable 
\emph{varstrDatRK}, en \emph{intResultado} el valor \textbf{TRUE} si se pudieron
inicializar los coeficientes y en caso contrario \textbf{ERR\_AMEM}.

\subsection{MetodoHeun}
Funci\'on que inicializa los coeficientes para el m\'etodo de \emph{Heun}, el
cual es un m\'etodo \emph{Runge-Kutta} de $3$ etapas y orden $3$.\newline

La notaci\'on matricial del m\'etodo es la siguiente:

\begin{center}
$
\begin{array}{c|ccc}
0 & 0 \\
\frac{1}{3} & \frac{1}{3} & 0 \\
\frac{2}{3} & 0 & \frac{2}{3} & 0 \\
\hline
 & \frac{1}{4} & 0 & \frac{3}{4}
\end{array}
$
\end{center}

El prototipo de esta funci\'on es el siguiente:

\begin{center}
\emph{int \textbf{MetodoHeun}(DatosRK *ptstrDatos)}
\end{center}

\begin{description}
\item[ptstrDatos] puntero a una variable de \emph{estructura} del tipo
\emph{DatosRK}.
\end{description}

La funci\'on devuelve los siguientes c\'odigos:

\begin{center}
\begin{tabular}{|l|l|}
\hline
\textbf{ERR\_AMEM} & Hubo un error en la asignaci\'on de memoria. \\
\hline
\textbf{TRUE} & Se inicializaron con \'exito los coeficientes. \\
\hline
\end{tabular}
\end{center}

Por ejemplo:

\begin{center}
\emph{intResultados = \textbf{MetodoHeun}(\&varstrDatRK);}
\end{center}

Inicializar\'{\i}a los coeficientes del m\'etodo en la variable
\emph{varstrDatRK}, en \emph{intResultado} el valor \textbf{TRUE} si se pudieron
inicializar los coeficientes y en caso contrario \textbf{ERR\_AMEM}.

\subsection{MetodoKutta}

Funci\'on que inicializa los coeficientes para el m\'etodo de \emph{Kutta}, el
cual es un m\'etodo \emph{Runge-Kutta} de $3$ etapas y orden $3$.\newline

La notaci\'on matricial del m\'etodo es la siguiente:

\begin{center}
$
\begin{array}{c|ccc}
0 & 0 \\
\frac{1}{2} & \frac{1}{2} & 0 \\
1 & -1 & 2 & 0 \\
\hline
 & \frac{1}{6} & \frac{2}{3} & \frac{1}{6}
\end{array}
$
\end{center}

El prototipo de esta funci\'on es el siguiente:

\begin{center}
\emph{int \textbf{MetodoKutta}(DatosRK *ptstrDatos)}
\end{center}

\begin{description}
\item[ptstrDatos] puntero a una variable de \emph{estructura} del tipo
\emph{DatosRK}.
\end{description}

La funci\'on devuelve los siguientes c\'odigos:

\begin{center}
\begin{tabular}{|l|l|}
\hline
\textbf{ERR\_AMEM} & Hubo un error en la asignaci\'on de memoria. \\
\hline
\textbf{TRUE} & Se inicializaron con \'exito los coeficientes. \\
\hline
\end{tabular}
\end{center}

Por ejemplo:

\begin{center}
\emph{intResultado = \textbf{MetodoKutta}(\&varstrDatRK);}
\end{center}

Inicializar\'{\i}a los coeficientes del m\'etodo en la variable
\emph{varstrDatRK}, en \emph{intResultado} el valor \textbf{TRUE} si se pudieron
inicializar los coeficientes y en caso contrario \textbf{ERR\_AMEM}.

\subsection{\texttt{EulerModificado} function}

Funci\'on que inicializa los coeficientes para el m\'etodo de \emph{Euler 
modificado}, el cual es un m\'etodo \emph{Runge-Kutta} de $2$ etapas y 
orden $2$.\newline

La notaci\'on matricial del m\'etodo es la siguiente:

\begin{center}
$
\begin{array}{c|cc}
0 & 0 \\
\frac{1}{2} & \frac{1}{2} & 0 \\
\hline
 & 0 & 1
\end{array}
$
\end{center}

El prototipo de esta funci\'on es el siguiente:

\begin{center}
\emph{int \textbf{EulerModificado}(DatosRK *ptstrDatos)}
\end{center}

\begin{description}
\item[ptstrDatos] puntero a una variable de \emph{estructura} del tipo
\emph{DatosRK}.
\end{description}

La funci\'on devuelve los siguientes c\'odigos:

\begin{center}
\begin{tabular}{|l|l|}
\hline
\textbf{ERR\_AMEM} & Hubo un error en la asignaci\'on de memoria. \\
\hline
\textbf{TRUE} & Se inicializaron con \'exito los coeficientes. \\
\hline
\end{tabular}
\end{center}

Por ejemplo:

\begin{center}
\emph{intResultado = \textbf{EulerModificado}(\&varstrDatRK);}
\end{center}


Inicializar\'{\i}a los coeficientes del m\'etodo en la variable
\emph{varstrDatRK}, en \emph{intResultado} el valor \textbf{TRUE} si se pudieron
inicializar los coeficientes y en caso contrario \textbf{ERR\_AMEM}.

\subsection{\texttt{EulerMejorado} function}

Funci\'on que inicializa los coeficientes para el m\'etodo de \emph{Euler mejorado},
el cual es un m\'etodo \emph{Runge-Kutta} de $2$ etapas y orden $2$.\newline

La notaci\'on matricial del m\'etodo es la siguiente:

\begin{center}
$
\begin{array}{c|cc}
0 & 0 \\
1 & 1 & 0 \\
\hline
 & \frac{1}{2} & \frac{1}{2}
\end{array}
$
\end{center}

El prototipo de esta funci\'on es el siguiente:

\begin{center}
\emph{int \textbf{EulerMejorado}(DatosRK *ptstrDatos)}
\end{center}

\begin{description}
\item[ptstrDatos] puntero a una variable de \emph{estructura} del tipo
\emph{DatosRK}.
\end{description}

La funci\'on devuelve los siguientes c\'odigos:

\begin{center}
\begin{tabular}{|l|l|}
\hline
\textbf{ERR\_AMEM} & Hubo un error en la asignaci\'on de memoria. \\
\hline
\textbf{TRUE} & Se inicializaron con \'exito los coeficientes. \\
\hline
\end{tabular}
\end{center}

Por ejemplo:

\begin{center}
\emph{intResultado = \textbf{EulerMejorado}(\&varstrDatRK);}
\end{center}


Inicializar\'{\i}a los coeficientes del m\'etodo en la variable
\emph{varstrDatRK}, en \emph{intResultado} el valor \textbf{TRUE} si se pudieron
inicializar los coeficientes y en caso contrario \textbf{ERR\_AMEM}.





Todas estas funciones suponen que la variable de \emph{estructura}, del tipo
\emph{DatosRK}\footnote{Apartado (\ref{sec:datosRK}) en la p\'agina 
\pageref{sec:datosRK}}, no tienen dimensionados los punteros en ella 
contenidos, raz\'on por la cual ser\'a necesario liberar la memoria asignada
a estos antes de pasarle como parametro una variable de este tipo a una de
las siguientes funciones(siempre y cuando se hayan dimensionado dichos
punteros).\newline

Hay que destacar que \textbf{NO} se inicializan todos los miembros de esta
estructura, s\'olo aquellos miembros que contienen los coeficientes del 
m\'etodo.\newline

Los siguientes miembros \textbf{NO} se inicializan:
%
\begin{description}
\item[intNumAprox]
\item[dblPuntos]
\item[dblPaso]
\item[dblInicio]
\item[dblFinal]
\end{description}

Estos miembros son independientes del m\'etodo, dependen del problema que
se quiera resolver y tendr\'an que ser inicializados por el usuario.


%
% struct.h
%

%
% struct.h
%

\chapter{struct.h}

\section{Introducci\'on}

En este fichero de cabezera se encuentran las estructuras de datos que utiliza
\BI para la resoluci\'on de problemas.\newline

\par Como ya se dijo anteriormente todas las estructuras ser\'an declaradas
mediante la palabra reservada \emph{typedef}.

\newpage

\section{Estructuras de datos para E.D.O's} \label{sec:datosEDO}

Estas estructuras de datos se utilizan para la resoluci\'on num\'erica de
E.D.O's\footnote{Ecuaciones Diferenciales Ordinarias.}.

\subsection{ButcherArray}

Esta estructura se utiliza para almacenar la \emph{notaci\'on matricial}
de los m\'etodos \emph{Runge-Kutta}\footnote{Ver ap\'endice sobre 
Runge-Kutta en la p\'agina \pageref{sec:Runge}.}.\newline

La declaraci\'on de esta estructura es la siguiente:

\begin{verbatim}
typedef struct
        {
        double  *dblC,
                *dblB,
                **dblMatriz;

        int intEtapas;
        } ButcherArray;
\end{verbatim}

El significado de cada uno de los miembros de esta estructura es el
siguiente:

\begin{description}
\item[dblC] es un vector de dimensi\'on \emph{strVariable.intEtapas} el cual 
contiene los
elementos $c_i$ del \emph{Runge-Kutta}, donde $0 \leq i < 
\emph{strVariable.intEtapas}$.
\item[dblB] es un vector de dimensi\'on \emph{strVariable.intEtapas} el cual 
contiene los
elementos $b_i$ del \emph{Runge-Kutta}, donde $0 \leq i < 
strVariable.intEtapas$.
\item[dblMatriz] es una matriz la cual contiene la matriz de coeficientes del
\emph{Runge-Kutta}.
\item[intEtapas] es el n\'umero de etapas que tiene el m\'etodo.
\end{description}

Para declarar una variable de este tipo:
\begin{center}
\emph{\textbf{ButcherArray} strVariable;}
\end{center}

\emph{strVariable} ser\'{\i}a una variable del tipo 
\emph{\textbf{ButcherArray}}.

\newpage

\subsection{DatosRK} \label{sec:datosRK}

Esta estructura se utiliza para almacenar todos los datos necesarios
en la ejecuci\'on del \emph{Runge-Kutta}, desde su notaci\'on matricial
hasta su inicializaci\'on, pasando por el paso utilizado.\newline

La declaraci\'on de esta \emph{estructura} es la siguiente:

\begin{verbatim}
typedef struct
        {	
        int     intNumAprox,
                intImplicito;
	        
        double  *dblPuntos,
                dblPaso,
                dblInicio,
                dblFinal;

        ButcherArray strCoefi;
        } DatosRK;
\end{verbatim}

El significado de cada uno de los miembros de esta estructura es el
siguiente:

\begin{description}
\item[intNumAprox] n\'umero de aproximaciones que se realizar\'an con el
m\'etodo, es decir contendr\'a la dimensi\'on de \emph{strVariable.dblPuntos}.
\item[intImplicito] contendr\'a el valor \textbf{TRUE} si se trata de un 
m\'etodo \emph{impl\'{\i}cito}, en caso contrario su valor ser\'a distinto de
\textbf{TRUE}.
\item[dblPuntos] vector de dimensi\'on \emph{strVariable.intNumAprox} el 
cual contendr\'a, en \emph{strVariable.dblPuntos[i],} 
las aproximaciones en los diferentes $x_i$, donde:

\begin{description}
\item[]$x_i = (\emph{strVariable.dblInicio}) + i * (\emph{strVariable.dblPaso})$
\item[]$0 \leq i < \emph{strVariable.intNumAprox}$
\end{description}

\item[dblPaso] tama\~no del paso que utilizara el m\'etodo.\newpage
\item[dblInicio] primer punto, en el que es conocido el valor de la funci\'on,
en el cual nos apoyamos para calcular las dem\'as aproximaciones.
\item[dblFinal] es el \'ultimo punto en el que calcularemos una aproximaci\'on
de la ecuaci\'on diferencial.
\item[strCoefi] es una variable del tipo \emph{\textbf{ButcherArray}}, la cual
contendr\'a la notaci\'on matricial del m\'etodo.
\end{description}

Para declarar una variable de este tipo:

\begin{center}
\emph{\textbf{DatosRK} strVariable;}
\end{center}

\emph{strVariable} ser\'{\i}a una variable del tipo \emph{\textbf{DatosRK}}.

\newpage

\section{Polynomial data structures}

\subsection{Polynomials}

\begin{equation}
p(x) = a_0 + a_1 \cdot x + \cdots + a_n \cdot x^n = \sum_{i=0}^n a_i \cdot x^i
\end{equation}

\subsection{biaPol data structure} \label{sec:biaPol}

\textbf{biaPol} data structure is defined in figure \ref{fig:biaPol} where:

\begin{description}
\item[intDegree] polynomial degree.
\item[intRealRoots] number of real roots (if any).
\item[intCompRoots] number of complex roots (if any).
\item[*dblCoef] pointer to store polynomial coeficients.
\end{description}

\begin{figure}[!h]
\begin{verbatim}
typedef struct {
  int  intDegree    = 0,
       intRealRoots = 0,
       intCompRoots = 0;

  double  *dblCoefs;
  } biaPol;
\end{verbatim}
\caption{Polynomial data structure.} \label{fig:biaPol}
\end{figure}

\subsection{How to use it}


\section{Estructuras de datos para la aproximaci\'on de funciones}

\subsection{DatosAprxFunc} \label{sec:DatosAprxFunc}

La declaraci\'on de esta estructura es la siguiente:

\begin{verbatim}
typedef struct

        {
        int     intNMI;

        double  dblx0,
                dblSolucion,
                dblTol,
                dblError; 
        } DatosAprxFunc;
\end{verbatim}

El significado de cada uno de los miembros de esta estructura es el siguiente:

\begin{description}
\item[intNMI] n\'umero m\'aximo de iteraciones.
\item[dblx0] aproximaci\'on inicial.
\item[dblSolucion] aproximaci\'on final de la raiz.
\item[dblTol] tolerancia con la que se va a aproximar la raiz.
\item[dblError] error cometido al aproximar la raiz, valor absoluto de la 
distancia entre las dos \'ultimas aproximaciones.
\end{description}

Para declarar una variable de este tipo:

\begin{center}
\emph{\textbf{DatosAprxFunc} strVariable;}
\end{center}

\emph{strVariable} ser\'{\i}a una variable del tipo \textbf{DatosAprxFunc}.

\section{Estructuras de datos para matrices}

Estas estructuras de datos se utilizan para el uso de matrices.

\subsection{Matriz} \label{sec:strMatriz}

Esta estructura se utiliza para almacenar matrices.\newline

La declaraci\'on de esta estructura es la siguiente:

\begin{verbatim}
typedef struct

        {
        int     intFilas,
                intColumnas;

        double  **dblCoefi;
        } Matriz;
\end{verbatim}

El significado de cada uno de los miembros de esta estructura es el siguiente:

\begin{description}
\item[intFilas] n\'umero de filas de la matriz.
\item[intColumnas] n\'umero de columnas de la matriz.
\item[dblCoefi] matriz que contiene los coeficientes de la matriz que estamos
representando mediante esta estructura de datos.
\end{description}

Para declarar una variable de este tipo:

\begin{center}
\emph{\textbf{Matriz} strVariable;}
\end{center}

\emph{strVariable} ser\'{\i}a una variable del tipo \textbf{Matriz}.

\section{Estructuras de datos para n\'umeros complejos} \label{sec:complejos}

Estas estructuras de datos se utilizan para el uso de n\'umeros complejos.

\subsection{Complejo}
Esta estructura se utiliza para almacenar las coordenadas de un n\'umero
complejo, en coordenadas cartesianas.\newline

La declaraci\'on de esta \emph{estructura} es la siguiente:

\begin{verbatim}
typedef struct

        {
        double  dblReal,
                dblImag;
        } Complejo;
\end{verbatim}

El significado de cada uno de los miembros de esta estructura es el siguiente:

\begin{description}
\item[dblReal] parte real del n\'umero complejo.
\item[dblImag] parte imaginaria del n\'umero complejo.
\end{description}

Para declarar una variable de este tipo:

\begin{center}
\emph{\textbf{Complejo} strVariable;}
\end{center}

\emph{strVariable} ser\'{\i}a una variable del tipo \textbf{Complejo}.

\newpage

\subsection{Polar}
Esta estructura se utiliza para almacenar un n\'umero complejo en 
coordenadas polares.\newline

La declaraci\'on de esta \emph{estructura} es la siguiente:

\begin{verbatim}
typedef struct

        {
        double  dblMod,
                dblArg;	
        } Polar;
\end{verbatim}

El significado de cada uno de los miembros de esta estructura es el siguiente:

\begin{description}
\item[dblMod] m\'odulo del n\'umero complejo.
\item[dblArg] argumento del n\'umero complejo.
\end{description}

Para declarar una variable de este tipo:

\begin{center}
\emph{\textbf{Polar} strVariable;}
\end{center}

\emph{strVariable} ser\'{\i}a una variable del tipo \textbf{Polar}.



%
% TIPOMATRIZ.H
%

\include{tipomatriz}

%%%%%%%%%%%%%
% Apendices %
%%%%%%%%%%%%%

\appendix

%
% Apendice sobre metodos Runge - Kutta 
%

%
% Apendice sobre metodos Runge - Kutta 
%

\chapter{Runge-Kutta methods} \label{sec:Runge}

This appendix is intended to help to know how Runge-Kutta methods are implemented and used in this library.

\section{What is a Runge-Kutta method?}

\textbf{Runge-Kutta} methods are a family of numerical methods to approach solutions of ordinary differential equations (O.D.E). These methods are iterative methods used to solve ``\emph{initial problem value}'' (\textbf{I.P.V}) or ``\emph{Cauchy problem}''.\\

These methods are only-one-step methods with a fixed size for the method step\footnote{It is also possible to implement methods with a variable step known as \emph{embedding}.}.\\

\subsection{What is a I.V.P.?}

An \emph{I.V.P.} is:

\begin{equation} \label{eq:IVP}
\left\{ \begin{array}{l}
y' = f(x, y(x))\\
y(x_0) = y_0\\
\end{array} \right.
\end{equation}
%
So $y'$ is a function depending on the variable $x$, and the function $y(x)$. $y(x)$ is the solution of the equation \ref{eq:IVP} and the point $(x_0,y_0)$ belongs to the curve $y(x)$.\\

Solving the \emph{I.V.P.} \ref{eq:IVP} is finding a function $y(x)$ such as the equation \ref{eq:IVP} is met.\\

An example of a \emph{I.V.P.}:
%
\begin{equation} \label{eq:IVPej}
\left\{ \begin{array}{l}
y' = \frac{x * y(x) - y(x)^2}{x^2} \\
y(1) = 2 \\
\end{array} \right.
\end{equation}
%
The solution of the \ref{eq:IVPej} will be:
%
\begin{equation}
y(x) = \frac{x}{\frac{1}{2}+\ln x}
\end{equation}

\section{Runge-Kutta's method notation}

$y(x_i)$ will be the exact value of the function $y(x)$ evaluated in $x_i$.\\ \\
$y_i$ will be the approximation of the function $y(x)$ in the point $x_i$.\\ \\
$h$ is the step used by the method in each iteration.

\subsection{General formulation}

A $s$-stages \textbf{Runge-Kutta}'s method formulation is:
%
\begin{equation}
y_{n+1} = y_{n} + h \cdot \sum_{i=0}^{s-1} b_i \cdot k_i
\end{equation}
%
where:
%
\begin{equation}
k_i = f(x_n + c_i \cdot h, y_n + h \cdot \sum_{j=0}^{s-1} a_{i,j} \cdot k_j)
\end{equation}
%
satisfying:
%
\begin{equation}
\sum_{j=0}^{s-1} a_{i,j} = c_i
\end{equation}

\subsection{Matricial notation (Butcher's)}

Matricial notation is used to represent method's coeficients using a matrix.\\

For a $s$-stages \textbf{Runge-Kutta} method the matricial notation will be:
%
\begin{center}
\begin{displaymath}
\begin{array}{c|ccc}
c_0 & a_{0,0} & \cdots \cdots & a_{0,s-1} \\
\vdots & \vdots & & \vdots \\
\vdots & \vdots & & \vdots \\
c_{s-1} & a_{s-1,0} & \cdots \cdots & a_{s-1,s-1} \\
\hline
 & b_0 & \cdots \cdots & b_{s-1} \\
\end{array}
\end{displaymath}
\end{center}

\note{In section \ref{sec:biaButcherArray} is shown a data structure used to store the Butcher array.}

\section{Runge-Kutta types}

There are several types of \textbf{Runge-Kutta} methods.

\subsection{Implicit Runge-Kutta}

A \textbf{Runge-Kutta} method is said to be implicit when the $a_{i,j} \neq 0$ for some $j > i$.\\

The $2$-stages Gauss method is an implicit \textbf{Runge-Kutta} method of $2$-stages:
%
\begin{center}
\begin{displaymath}
\begin{array}{c|cc}
\frac{3-\sqrt 3}{6} & \frac{1}{4} & \frac{3-2*\sqrt 3}{12} \\
\frac{3+\sqrt 3}{6} & \frac{3+2*\sqrt 3}{12} &\frac{1}{4} \\
\hline
 & \frac{1}{2} & \frac{1}{2}
\end{array}
\end{displaymath}
\end{center}

\subsection{Semi-implicit Runge-Kutta}

A \textbf{Runge-Kutta} method is said to be semi-implicit when the $a_{i,j} = 0$ when $j > i$.\\

A $2$-stages semi-implicit \textbf{Runge-Kutta} method:
%
\begin{center}
\begin{displaymath}
\begin{array}{c|cc}
\frac{3+\sqrt 3}{6} & \frac{3+\sqrt 3}{6} & 0 \\
\frac{3-\sqrt 3}{6} & \frac{-\sqrt 3}{3} & \frac {3+\sqrt 3}{6} \\
\hline
 & \frac{1}{2} & \frac{1}{2}
\end{array}
\end{displaymath}
\end{center}

\subsection{Explicit Runge-Kutta}

A \textbf{Runge-Kutta} method is said to be explicit when the $a_{i,j} = 0$ when $j \geq i$.\\

A $4$-stages explicit \textbf{Runge-Kutta} method also known as ``\textbf{classic Runge-Kutta}'':
%
\begin{center}
\begin{displaymath}
\begin{array}{c|cccc}
0 & 0 \\
\frac{1}{2} & \frac{1}{2} & 0 \\
\frac{1}{2} & 0 & \frac{1}{2} & 0 \\
1 & 0 & 0 & 1 & 0 \\
\hline
 & \frac{1}{6} & \frac{1}{3} & \frac{1}{3} & \frac{1}{6}
\end{array}
\end{displaymath}
\end{center}


\end{document}
