%
% COMPLEJO.H
%

\chapter{complejo.h}

\section{Introducci\'on}
En este fichero de cabezera estan los prototipos de funciones para el manejo
de n\'umeros complejos as\'{\i} como los prototipos de funciones para el
c\'alculo con n\'umeros complejos.\newline

Ser\'{\i}a buena idea consultar el apartado (\ref{sec:complejos}) en la
p\'agina \pageref{sec:complejos} para comprender las estructuras de datos
que utilizan estas funciones.

\begin{center}
\begin{tabular}{|c|}
\hline
\textbf{Se entender\'a como argumento de un n\'umero complejo,} \\
\textbf{al \'angulo formado con el eje REAL POSITIVO y en} \\
\textbf{sentido contrario al de las agujas del reloj.} \\
\hline
\end{tabular}
\end{center}

\newpage

\section{Operaciones entre n\'umeros complejos}

\subsection{strSumaComplejos}
Esta funci\'on suma n\'umeros complejos.\newline

El prototipo de esta funci\'on es el siguiente:

\begin{center}
\emph{Complejo \textbf{strSumaComplejos}(Complejo strComp1, Complejo strComp2)}
\end{center}

\begin{description}
\item[strComp1] variable del tipo \emph{Complejo} que contiene un n\'umero 
complejo.
\item[strComp2] variable del tipo \emph{Complejo} que contiene un n\'umero
complejo.
\end{description}

La funci\'on devuelve una variable del tipo \emph{Complejo}, la cual contiene
la suma de los argumentos que se le han pasado a la funci\'on.

Por ejemplo:

\begin{center}
\emph{strResultado = \textbf{strSumaComplejos}(strComp1, strComp2);}
\end{center}

Almacenar\'{\i}a en \emph{strResultado}, variable del tipo \emph{Complejo}, el
resultado de sumar los n\'umeros complejos \emph{strComp1} y \emph{strComp2}.\\

Observaciones:

\begin{itemize}
\item La funci\'on trabaja en coordenadas cartesianas.
\end{itemize}

\subsection{strRestaComplejos}
Esta funci\'on resta n\'umeros complejos.\newline

El prototipo de esta funci\'on es el siguiente:

\begin{center}
\emph{Complejo \textbf{strRestaComplejos}(Complejo strComp1, Complejo strComp2)}
\end{center}

\begin{description}
\item[strComp1] variable del tipo \emph{Complejo} que contiene un n\'umero 
complejo.
\item[strComp2] variable del tipo \emph{Complejo} que contiene un n\'umero
complejo.
\end{description}

La funci\'on devuelve una variable del tipo \emph{Complejo}, la cual contiene
la resta de los argumentos que se le han pasado a la funci\'on.\newpage

Por ejemplo:

\begin{center}
\emph{strResultado = \textbf{strRestaComplejos}(strComp1, strComp2);}
\end{center}

Almacenar\'{\i}a en \emph{strResultado}, variable del tipo \emph{Complejo}, el
resultado de restar al n\'umero complejo \emph{strComp1} el n\'umero complejo
\emph{strComp2}. \\

Observaciones:

\begin{itemize}
\item La funci\'on trabaja en coordenadas cartesianas.
\end{itemize}

\subsection{strProdComplejos}
Esta funci\'on multiplica n\'umeros complejos.\newline

El prototipo de esta funci\'on es el siguiente:

\begin{center}
\emph{Complejo \textbf{strProdComplejo}(Complejo strComp1, Complejo strComp2)}
\end{center}

\begin{description}
\item[strComp1] variable del tipo \emph{Complejo} que contiene un n\'umero 
complejo.
\item[strComp2] variable del tipo \emph{Complejo} que contiene un n\'umero
complejo.
\end{description}

La funci\'on devuelve una variable del tipo \emph{Complejo}, la cual contiene
el producto de los argumentos que se le han pasado a la funci\'on.\newline

Por ejemplo:

\begin{center}
\emph{strResultado = \textbf{strProdComplejos}(strComp1, strComp2);}
\end{center}

Almacenar\'{\i}a en \emph{strResultado}, variable del tipo \emph{Complejo}, el
resultado de multiplicar los n\'umeros complejos \emph{strComp1} y \emph{strComp2}.\\

Observaciones:

\begin{itemize}
\item La funci\'on trabaja en coordenadas cartesianas.
\end{itemize}

\subsection{DivComplejos}
Esta funci\'on divide n\'umeros complejos.\newline

El prototipo de esta funci\'on es el siguiente:

\begin{center}
\emph{int \textbf{DivComplejos}(Complejo strComp1, Complejo strComp2,\\
Complejo *ptstrRes)}
\end{center}

\begin{description}
\item[strComp1] variable del tipo \emph{Complejo} que contiene un n\'umero 
complejo.
\item[strComp2] variable del tipo \emph{Complejo} que contiene un n\'umero
complejo.
\item[ptstrRes] puntero a una variable del tipo \emph{Complejo} en la cual
se almacenar\'a el resultado.
\end{description}

La funci\'on devuelve los siguientes c\'odigos:

\begin{center}
\begin{tabular}{|l|l|}
\hline
\textbf{DIV\_CERO} & Divisi\'on por cero. \\
\hline
\textbf{TRUE} & Se efectu\'o con \'exito la divisi\'on. \\
\hline
\end{tabular}
\end{center} 

Por ejemplo:

\begin{center}
\emph{intResultado = \textbf{DivComplejos}(strComp1, strComp2, \&strRes);}
\end{center}

Almacenar\'{\i}a en \emph{strRes}, variable del tipo \emph{Complejo}, el
resultado de dividir el n\'umero complejo \emph{strComp1} por el n\'umero 
complejo \emph{strComp2} y en \emph{intResultado} el valor $0$, siempre y cuando
\emph{strComp2} no sea el n\'umero complejo $0+0*i$, en cuyo caso almacenar\'a
en \emph{intResultado} el valor \textbf{DIV\_CERO}.\\

Observaciones:

\begin{itemize}
\item La funci\'on trabaja en coordenadas cartesianas.
\end{itemize}

\newpage

\section{Operaciones con n\'umeros complejos}

\subsection{dblModulo}
Esta funci\'on calcula el m\'odulo de un n\'umero complejo.\newline

El prototipo de esta funci\'on es el siguiente:

\begin{center}
\emph{double \textbf{dblModulo}(Complejo strComp)}
\end{center}

\begin{description}
\item[strComp] variable del tipo \emph{Complejo} que contiene un n\'umero
complejo.
\end{description}

La funci\'on devuelve el m\'odulo del n\'umero complejo que se le pasa como
argumento.\newline

Por ejemplo:

\begin{center}
\emph{dblResultado = \textbf{dblModulo}(strComp);}
\end{center}

Almacenar\'{\i}a en \emph{dblResultado} el m\'odulo del n\'umero complejo
\emph{strComp}.\\

Observaciones:

\begin{itemize}
\item La funci\'on trabaja en coordenadas cartesianas.
\end{itemize}

\subsection{dblArgumento}
Esta funci\'on calcula el argumento de un n\'umero complejo.\newline

El prototipo de esta funci\'on es el siguiente:

\begin{center}
\emph{double \textbf{dblArgumento}(Complejo strComp)}
\end{center}

\begin{description}
\item[strComp] variable del tipo \emph{Complejo} que contiene un n\'umero
complejo.
\end{description}

La funci\'on devuelve el argumento, en radianes, del n\'umero complejo que se
le pasa como argumento.\newline

Por ejemplo:

\begin{center}
\emph{dblResultado = \textbf{dblArgumento}(strComp);}
\end{center}

Almacenar\'{\i}a en \emph{dblResultado} el argumento, en radianes, del
n\'umero complejo \emph{strComp}.\\

Observaciones:

\begin{itemize}
\item La funci\'on trabaja en coordenadas cartesianas.
\end{itemize}

\subsection{strOpuComplejo}
Esta funci\'on calcula el opuesto, respecto de la suma, de un 
n\'umero complejo.\newline

El prototipo de esta funci\'on es el siguiente:

\begin{center}
\emph{Complejo \textbf{strOpuComplejo}(Complejo strComp)}
\end{center}

\begin{description}
\item[strComp] variable del tipo \emph{Complejo} que contiene un n\'umero 
complejo.
\end{description}

La funci\'on devuelve una variable del tipo \emph{Complejo}, la cual contiene
el opuesto del argumento que se le ha pasado a la funci\'on.\newline

Por ejemplo:

\begin{center}
\emph{strResultado = \textbf{strOpuComplejo}(strComp);}
\end{center}

Almacenar\'{\i}a en \emph{strResultado}, variable del tipo \emph{Complejo}, el
opuesto, respecto de la suma, del n\'umero complejo \emph{strComp}.\\

Observaciones:

\begin{itemize}
\item La funci\'on trabaja en coordenadas cartesianas.
\end{itemize}

\subsection{InvComplejo}
Esta funci\'on calcula el inverso, respecto del producto, de un 
n\'umero complejo.\newline

El prototipo de esta funci\'on es el siguiente:

\begin{center}
\emph{int \textbf{InvComplejo}(Complejo strComp, Complejo *ptstrRes)}
\end{center}

\begin{description}
\item[strComp] variable del tipo \emph{Complejo} que contiene un n\'umero 
complejo.
\item[ptstrRes] puntero a una variable del tipo \emph{Complejo} en la cual se
almacenar\'a el resultado.
\end{description}

La funci\'on devuelve los siguientes c\'odigos:

\begin{center}
\begin{tabular}{|l|l|}
\hline
\textbf{DIV\_CERO} & Divisi\'on por cero. \\
\hline
\textbf{TRUE} & Se calcul\'o con \'exito el inverso. \\
\hline
\end{tabular}
\end{center}

\newpage

Por ejemplo:

\begin{center}
\emph{intResultado = \textbf{InvComplejo}(strComp, \&strRes);}
\end{center}

Almacenar\'{\i}a en \emph{strRes}, variable del tipo \emph{Complejo}, el
inverso, respecto de la multiplicaci\'on, del n\'umero complejo \emph{strComp} 
y en \emph{intResultado} el valor $0$, siempre y cuando \emph{strComp} no sea
el n\'umero complejo $0+0*i$, en cuyo caso almacenar\'a en \emph{intResultado} 
el valor \textbf{DIV\_CERO}.\\

Observaciones:

\begin{itemize}
\item La funci\'on trabaja en coordenadas cartesianas.
\end{itemize}

\subsection{strComplejoConj}
Esta funci\'on calcula el complejo conjugado de un n\'umero complejo.\newline

El prototipo de esta funci\'on es el siguiente:

\begin{center}
\emph{Complejo \textbf{strComplejoConj}(Complejo strComp)}
\end{center}

\begin{description}
\item[strComp] variable del tipo \emph{Complejo} que contiene un n\'umero 
complejo.
\end{description}

La funci\'on devuelve una variable del tipo \emph{Complejo}, la cual contiene
el complejo conjugado del argumento que se le ha pasado a la funci\'on.\newline

Por ejemplo:

\begin{center}
\emph{strResultado = \textbf{strComplejoConj}(strComp);}
\end{center}

Almacenar\'{\i}a en \emph{strResultado}, variable del tipo \emph{Complejo}, el
complejo conjugado del n\'umero complejo \emph{strComp}.\\

Observaciones:

\begin{itemize}
\item La funci\'on trabaja en coordenadas cartesianas.
\end{itemize}

\subsection{strCartePolar}
Esta funci\'on calcula las coordenadas \emph{polares} de un n\'umero complejo,
dado en coordenadas \emph{cartesianas}.\newline

El prototipo de esta funci\'on es el siguiente:

\begin{center}
\emph{Polar \textbf{strCartePolar}(Complejo strComp)}
\end{center}

\begin{description}
\item[strComp] variable del tipo \emph{Complejo} que contiene un n\'umero 
complejo.
\end{description}

La funci\'on devuelve una variable del tipo \emph{Polar}, la cual contiene
las coordenadas polares del argumento que se le ha pasado a la funci\'on.\\

Por ejemplo:

\begin{center}
\emph{strResultado = \textbf{strCartePolar}(strComp);}
\end{center}

Almacenar\'{\i}a en \emph{strResultado}, variable del tipo \emph{Polar}, las
coordenadas polares del n\'umero complejo \emph{strComp}\footnote{En coordenadas
cartesianas.}.\\

Observaciones:

\begin{itemize}
\item La funci\'on calcula el argumento en radianes.
\end{itemize}

\subsection{strPolarCarte}
Esta funci\'on calcula las coordenadas \emph{cartesianas} de un n\'umero
complejo, dado en coordenadas \emph{polares}.\newline

El prototipo de esta funci\'on es el siguiente:

\begin{center}
\emph{Complejo \textbf{strPolarCarte}(Polar strComp)}
\end{center}

\begin{description}
\item[strComp] variable del tipo \emph{Complejo} que contiene un n\'umero 
complejo.
\end{description}

La funci\'on devuelve una variable del tipo \emph{Complejo}, la cual contiene
las coordenadas cartesianas del argumento que se le ha pasado a la funci\'on.

\newpage

Por ejemplo:

\begin{center}
\emph{strResultado = \textbf{strPolarCarte}(strComp);}
\end{center}

Almacenar\'{\i}a en \emph{strResultado}, variable del tipo \emph{Complejo}, las
coordenadas cartesianas del n\'umero complejo \emph{strComp}\footnote{En 
coordenadas polares.}.\\

Observaciones:

\begin{itemize}
\item La funci\'on supone el argumento en radianes.
\end{itemize}
