%
% PRIMOS.H
%

\chapter{primos.h}

\section{Introducci\'on}
En este fichero de cabezera estan los prototipos de funciones relacionadas
con n\'umeros primos. 

\section{C\'alculo de n\'umeros primos}

\subsection{Primo}
Comprueba si un n\'umero es primo o no.\newline

El prototipo de esta funci\'on es el siguiente:

\begin{center}
\emph{int \textbf{Primo}(int intNumero)};
\end{center}

\begin{description}
\item[intNumero] n\'umero que se quiere comprobar si es o no primo.
\end{description}

La funci\'on devuelve los siguientes c\'odigos:

\begin{center}
\begin{tabular}{|l|l|}
\hline
\textbf{FALSE} & Si el n\'umero no es primo. \\
\hline
\textbf{TRUE} & Si el n\'umero es primo. \\
\hline
\end{tabular}
\end{center}

\par Por ejemplo:

\begin{center}
\emph{intResultado = \textbf{Primo}(intNumero)};
\end{center}

Almacenar\'{\i}a en \emph{intResultado} el valor \textbf{TRUE} si
\emph{intNumero} es un n\'umero primo y \textbf{FALSE} en caso contrario.

\newpage

Observaciones:

\begin{itemize}
\item La funci\'on supone que \emph{intNumero} es no nulo.
\item La funci\'on considerar\'a \emph{intNumero} como positivo aunque no lo
sea.
\end{itemize}

\subsection{PrimerosPrimos}

Esta funci\'on calcula los primeros n\'umeros primos.\newline

El prototipo de esta funci\'on es el siguiente:

\begin{center}
\emph{\textbf{PrimerosPrimos}(unsigned int *ptunsPrimos, int intNumero,\\
int *ptintCalc)};
\end{center}

\begin{description}
\item[ptunsPrimos] puntero que tiene que estar dimensionado adecuadamente
para almancenar \emph{intNumeros} n\'umeros primos, del tipo
\emph{unsigned int}.
\item[intNumero] cantidad de n\'umeros primos que se quieren calcular.
\item[ptintCalc] puntero a la variable donde se va a almacenar la cantidad
de n\'umeros primos calculados.
\end{description}

Por ejemplo:

\begin{center}
\emph{\textbf{PrimerosPrimos}(unsPrimos, intNumero, \&intCalc)};
\end{center}

Almacenar\'{\i}a en \emph{intCalc} la cantidad de n\'umeros primos calculados y
los n\'umeros primos calculados estar\'{\i}an almacenados en:

\begin{displaymath} 
unsPrimos[i] \qquad donde \quad 0 \le i \le intCalc-1
\end{displaymath}

Observaciones:

\begin{itemize}
\item \emph{ptunsPrimos} ha de estar dimensionado adecuadamente para que la
funci\'on pueda almacenar en \'el \emph{intNumero} datos \emph{unsigned}.
\item La funci\'on utiliza \emph{intNumero} como un n\'umero positivo.
\end{itemize}
