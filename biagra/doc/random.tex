%
% random.h
%

\chapter{Pseudo random numbers (random.h)} \label{ch:random}

\section{Introduction}

\BI \ includes its own function to pseudo random number generation and they are defined in \texttt{random.h} file.\\

\warning{This functions are not tested to produce unpredictable sequences, so be careful when use them.}

\section{Pseudo random integer numbers}

\subsection{\textbf{intRandom} function} \label{sec:intRandom}

This function generates random integers.\\

The definition of this function:
%
\begin{verbatim}
int intRandom(int limit);  
\end{verbatim}
%
The pseudo random integer is placed in the interval $(-limit,limit)$.\\ \\
%
\note{Before using this function \texttt{srand} must be used to initialize \texttt{rand}. You can use \texttt{srand((unsigned)time(NULL))}.}
%
\ \\
%
The pseudo random number is generated with the following formula:
%
\begin{displaymath}
\left[ \frac{limit \cdot rand()}{RAND\_MAX + 1} \right] \in (-limit,limit)
\end{displaymath}
%
Then randomly is choosed if the number is positive or negative using the above formula with $limit=2$ and then taking modulus $2$. If modulus is $1$ then the number will be a negative one.

\subsection{\textbf{uintRandom} function} \label{sec:uintRandom}

This function generates random integers.\\

The definition of this function:
%
\begin{verbatim}
int uintRandom(int limit);  
\end{verbatim}
%
The pseudo random integer is placed in the interval $[0,limit)$.\\ \\
%
\note{Before using this function \texttt{srand} must be used to initialize \texttt{rand}. You can use \texttt{srand((unsigned)time(NULL))}.}
%
\ \\
%
The pseudo random number is generated with the following formula:
%
\begin{displaymath}
\left[ \frac{limit \cdot rand()}{RAND\_MAX + 1} \right] \in [0,limit)
\end{displaymath}
%

\section{Pseudo random floating point numbers}

\subsection{\textbf{dblRandom} function} \label{sec:dblRandom}

This function generates random floating point numbers.\\

The definition of this function:
%
\begin{verbatim}
int dblRandom(int limit);  
\end{verbatim}
%
The pseudo random floating point number is placed in the interval $(-limit,limit)$.\\ \\
%
\note{Before using this function \texttt{srand} must be used to initialize \texttt{rand}. You can use \texttt{srand((unsigned)time(NULL))}.}
%
\ \\
%
The pseudo random number is generated with the following formula:
%
\begin{displaymath}
\frac{limit \cdot rand()}{RAND\_MAX + 1} \in (-limit,limit)
\end{displaymath}
%
Then randomly is choosed if the number is positive or negative using the above formula with $limit=2$ and then taking modulus $2$. If modulus is $1$ then the number will be a negative one.

\subsection{\textbf{udblRandom} function} \label{sec:udblRandom}

This function generates random floating point numbers.\\

The definition of this function:
%
\begin{verbatim}
int udblRandom(int limit);  
\end{verbatim}
%
The pseudo random floating point number is placed in the interval $[0,limit)$.\\ \\
%
\note{Before using this function \texttt{srand} must be used to initialize \texttt{rand}. You can use \texttt{srand((unsigned)time(NULL))}.}
%
\ \\
%
The pseudo random number is generated with the following formula:
%
\begin{displaymath}
\frac{limit \cdot rand()}{RAND\_MAX + 1} \in [0,limit)
\end{displaymath}
%
