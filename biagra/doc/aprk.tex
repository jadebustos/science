%
% Apendice sobre metodos Runge - Kutta 
%

\chapter{Runge-Kutta methods} \label{sec:Runge}

This appendix is intended to help to know how Runge-Kutta methods are implemented and used in this library.

\section{What is a Runge-Kutta method?}

\textbf{Runge-Kutta} methods are a family of numerical methods to approach solutions of ordinary differential equations (O.D.E). These methods are iterative methods used to solve ``\emph{initial problem value}'' (\textbf{I.P.V}) or ``\emph{Cauchy problem}''.\\

These methods are only-one-step methods with a fixed size for the method step\footnote{It is also possible to implement methods with a variable step known as \emph{embedding}.}.\\

\subsection{What is a I.V.P.?}

An \emph{I.V.P.} is:

\begin{equation} \label{eq:IVP}
\left\{ \begin{array}{l}
y' = f(x, y(x))\\
y(x_0) = y_0\\
\end{array} \right.
\end{equation}
%
So $y'$ is a function depending on the variable $x$, and the function $y(x)$. $y(x)$ is the solution of the equation \ref{eq:IVP} and the point $(x_0,y_0)$ belongs to the curve $y(x)$.\\

Solving the \emph{I.V.P.} \ref{eq:IVP} is finding a function $y(x)$ such as the equation \ref{eq:IVP} is met.\\

An example of a \emph{I.V.P.}:
%
\begin{equation} \label{eq:IVPej}
\left\{ \begin{array}{l}
y' = \frac{x * y(x) - y(x)^2}{x^2} \\
y(1) = 2 \\
\end{array} \right.
\end{equation}
%
The solution of the \ref{eq:IVPej} will be:
%
\begin{equation}
y(x) = \frac{x}{\frac{1}{2}+\ln x}
\end{equation}

\section{Runge-Kutta's method notation}

$y(x_i)$ will be the exact value of the function $y(x)$ evaluated in $x_i$.\\ \\
$y_i$ will be the approximation of the function $y(x)$ in the point $x_i$.\\ \\
$h$ is the step used by the method in each iteration.

\subsection{General formulation}

A $s$-stages \textbf{Runge-Kutta}'s method formulation is:
%
\begin{equation}
y_{n+1} = y_{n} + h \cdot \sum_{i=0}^{s-1} b_i \cdot k_i
\end{equation}
%
where:
%
\begin{equation}
k_i = f(x_n + c_i \cdot h, y_n + h \cdot \sum_{j=0}^{s-1} a_{i,j} \cdot k_j)
\end{equation}
%
satisfying:
%
\begin{equation}
\sum_{j=0}^{s-1} a_{i,j} = c_i
\end{equation}

\subsection{Matricial notation (Butcher's)}

Matricial notation is used to represent method's coeficients using a matrix.\\

For a $s$-stages \textbf{Runge-Kutta} method the matricial notation will be:
%
\begin{center}
\begin{displaymath}
\begin{array}{c|ccc}
c_0 & a_{0,0} & \cdots \cdots & a_{0,s-1} \\
\vdots & \vdots & & \vdots \\
\vdots & \vdots & & \vdots \\
c_{s-1} & a_{s-1,0} & \cdots \cdots & a_{s-1,s-1} \\
\hline
 & b_0 & \cdots \cdots & b_{s-1} \\
\end{array}
\end{displaymath}
\end{center}

\note{In section \ref{sec:biaButcherArray} is shown a data structure used to store the Butcher array.}

\section{Runge-Kutta types}

There are several types of \textbf{Runge-Kutta} methods.

\subsection{Implicit Runge-Kutta}

A \textbf{Runge-Kutta} method is said to be implicit when the $a_{i,j} \neq 0$ for some $j > i$.\\

The $2$-stages Gauss method is an implicit \textbf{Runge-Kutta} method of $2$-stages:
%
\begin{center}
\begin{displaymath}
\begin{array}{c|cc}
\frac{3-\sqrt 3}{6} & \frac{1}{4} & \frac{3-2*\sqrt 3}{12} \\
\frac{3+\sqrt 3}{6} & \frac{3+2*\sqrt 3}{12} &\frac{1}{4} \\
\hline
 & \frac{1}{2} & \frac{1}{2}
\end{array}
\end{displaymath}
\end{center}

\subsection{Semi-implicit Runge-Kutta}

A \textbf{Runge-Kutta} method is said to be semi-implicit when the $a_{i,j} = 0$ when $j > i$.\\

A $2$-stages semi-implicit \textbf{Runge-Kutta} method:
%
\begin{center}
\begin{displaymath}
\begin{array}{c|cc}
\frac{3+\sqrt 3}{6} & \frac{3+\sqrt 3}{6} & 0 \\
\frac{3-\sqrt 3}{6} & \frac{-\sqrt 3}{3} & \frac {3+\sqrt 3}{6} \\
\hline
 & \frac{1}{2} & \frac{1}{2}
\end{array}
\end{displaymath}
\end{center}

\subsection{Explicit Runge-Kutta}

A \textbf{Runge-Kutta} method is said to be explicit when the $a_{i,j} = 0$ when $j \geq i$.\\

A $4$-stages explicit \textbf{Runge-Kutta} method also known as ``\textbf{classic Runge-Kutta}'':
%
\begin{center}
\begin{displaymath}
\begin{array}{c|cccc}
0 & 0 \\
\frac{1}{2} & \frac{1}{2} & 0 \\
\frac{1}{2} & 0 & \frac{1}{2} & 0 \\
1 & 0 & 0 & 1 & 0 \\
\hline
 & \frac{1}{6} & \frac{1}{3} & \frac{1}{3} & \frac{1}{6}
\end{array}
\end{displaymath}
\end{center}
