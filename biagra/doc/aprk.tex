%
% Apendice sobre metodos Runge - Kutta 
%

\chapter{M\'etodos Runge-Kutta} \label{sec:Runge}

\section{?`Que es un m\'etodo Runge-Kutta?}

Los m\'etodos \emph{Runge-Kutta} son m\'etodos num\'ericos para la resoluci\'on
de \emph{P.V.I.}\footnote{Problema de Valores Iniciales.}.\newline

Estos m\'etodos son de un \emph{s\'olo paso} y la longitud del paso es 
fija\footnote{Tambi\'en se pueden implementar con longitud de paso variable,
son conocidos como \emph{embedding} o \emph{adaptativos}.}.\newline

No trataremos de estudiar con detalle este tipo de m\'etodos, sino de
dar su formulaci\'on para entender como fueron programados los m\'etodos
num\'ericos que tratan sobre estos m\'etodos.
\newpage

\subsection{?`Que es un P.V.I.?}
Un \emph{P.V.I.} es:

\begin{equation} \label{eq:PVI}
\left\{ \begin{array}{l}
y' = f(x, y(x))\\
y(x_0) = y_0\\
\end{array} \right.
\end{equation}

Esta notaci\'on nos indica que $y'$ es una funci\'on en la variable $x$, que
depende de la funci\'on $y(x)$ y que la curva definida por la funci\'on $y(x)$,
soluci\'on de la ecuaci\'on diferencial (\ref{eq:PVI}) pasa por el punto 
$(x_0,y_0)$.\newline

Resolver el \emph{P.V.I.} (\ref{eq:PVI}) es encontrar una funci\'on $y(x)$
que satisfaga (\ref{eq:PVI}).\newline

Dado el siguiente \emph{P.V.I.}

\begin{equation} \label{eq:PVIej}
\left\{ \begin{array}{l}
y' = \frac{x * y(x) - y(x)^2}{x^2} \\
y(1) = 2 \\
\end{array} \right.
\end{equation}

podemos ver como $y'$ depende, efectivamente, de la variable $x$ y de la
funci\'on $y(x)$, la cual, a su vez, tambi\'en depende de la variable $x$.
\newline

La soluci\'on de (\ref{eq:PVIej}) ser\'a la funci\'on:

\begin{equation}
y(x) = \frac{x}{\frac{1}{2}+\ln x}
\end{equation}

\section{Formulaci\'on de los m\'etodos Runge-Kutta}

\subsection{Notaci\'on}

$y(x_i)$ ser\'a el valor exacto de la funci\'on $y(x)$ en el punto $x_i$.\\ \\
$y_i$ ser\'a una aproximaci\'on del valor de la funci\'on $y(x)$ en el punto
$x_i$.\newline \newline
$h$ es el paso utilizado por el m\'etodo en cada iteraci\'on.

\newpage

\subsection{Formulaci\'on general}

La formulaci\'on de un m\'etodo \emph{Runge-Kutta} de \emph{\textbf{s}} 
etapas es:

\begin{equation}
y_{n+1} = y_{n} + h * \sum_{i=0}^{s-1} b_i k_i
\end{equation}

donde

\begin{equation}
k_i = f(x_n + c_i * h, y_n + h * \sum_{j=0}^{s-1} a_{ij} k_j)
\end{equation}

verificando

\begin{equation}
\sum_{j=0}^{s-1} a_{ij} = c_i
\end{equation}

\subsection{Notaci\'on matricial(Butcher)}

La notaci\'on matricial se utiliza para dar los coeficientes del m\'etodo, y
como su propio nombre indica, en forma matricial.\newline

La notaci\'on matricial para un \emph{Runge-Kutta} de \emph{s} etapas 
ser\'{\i}a la siguiente: 

\begin{center}
$
\begin{array}{c|ccc}
c_0 & a_{0\,0} & \cdots \cdots & a_{0\,s-1} \\
\vdots & \vdots & & \vdots \\
\vdots & \vdots & & \vdots \\
c_{s-1} & a_{s-1\,0} & \cdots \cdots & a_{s-1\,s-1} \\
\hline
 & b_0 & \cdots \cdots & b_{s-1} \\
\end{array}
$
\end{center}

\newpage

\section{Tipos de Runge-Kutta}

\subsection{Runge-Kutta impl\'{\i}citos}

Un m\'etodo \emph{Runge-Kutta} se dice \emph{impl\'{\i}cito} cuando los $a_{ij}
\neq 0$ para alg\'un $j > i$.\newline

El siguiente m\'etodo es un \emph{Runge-Kutta} de \emph{$2$ etapas},
\emph{impl\'{\i}cito} y recibe el nombre de \emph{\textbf{``M\'etodo de Gauss 
de dos etapas''}}:

\begin{center}
$
\begin{array}{c|cc}
\frac{3-\sqrt 3}{6} & \frac{1}{4} & \frac{3-2*\sqrt 3}{12} \\
\frac{3+\sqrt 3}{6} & \frac{3+2*\sqrt 3}{12} &\frac{1}{4} \\
\hline
 & \frac{1}{2} & \frac{1}{2}
\end{array}
$
\end{center}

\subsection{Runge-Kutta semiimpl\'{\i}citos}

Un m\'etodo \emph{Runge-Kutta} se dice \emph{semiimpl\'{\i}cito} cuando los
$a_{ij} = 0$ para $j > i$.\newline

El siguiente m\'etodo es un \emph{Runge-Kutta} de \emph{$2$ etapas},
\emph{semiimpl\'{\i}cito:}

\begin{center}
$
\begin{array}{c|cc}
\frac{3+\sqrt 3}{6} & \frac{3+\sqrt 3}{6} & 0 \\
\frac{3-\sqrt 3}{6} & \frac{-\sqrt 3}{3} & \frac {3+\sqrt 3}{6} \\
\hline
 & \frac{1}{2} & \frac{1}{2}
\end{array}
$
\end{center}

\subsection{Runge-Kutta expl\'{\i}citos}

Un m\'etodo \emph{Runge-Kutta} se dice \emph{expl\'{\i}cito} cuando los 
$a_{ij} = 0$ para $j \geq i$.\newline

El siguiente m\'etodo es un \emph{Runge-Kutta} de \emph{$4$ etapas}, 
\emph{expl\'{\i}cito} y recibe el nombre de 
\emph{\textbf{``Runge-Kutta cl\'asico''}}:\newline

\begin{center}
$
\begin{array}{c|cccc}
0 & 0 \\
\frac{1}{2} & \frac{1}{2} & 0 \\
\frac{1}{2} & 0 & \frac{1}{2} & 0 \\
1 & 0 & 0 & 1 & 0 \\
\hline
 & \frac{1}{6} & \frac{1}{3} & \frac{1}{3} & \frac{1}{6}
\end{array}
$
\end{center}
