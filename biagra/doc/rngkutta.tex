%
% RNGKUTTA.H
%

\chapter{Runge-Kutta methods (rngkutta.h)}

\section{Introduction}

\textbf{Runge-Kutta} are a family of implicit and explicit iterative methods used to approximate solutions of ordinary differential equations or \textbf{ODE}.\\

Butcher matricial notation is used in this implementation.\\

\section{Data structures}

\subsection{\texttt{biaButcherArray} data structure} \label{sec:biaButcherArray}

This structure is used to store the Butcher matricial notation.\\

Data structure is defined in figure \ref{fig:biaButcherArray} where:
%
\begin{description}
%
\item[intStages] method stages.
%
\item[*dblC] $c_i$ coefficients stored in an array with size \texttt{intStages}.
%
\item[*dblB] $b_i$ coefficients stored in an array with size \texttt{intStages}.
%
\item[**dblMatrix] matrix to store $a_{i,j}$ method's coeficients.  
%
\end{description}

\begin{figure}[!h]
\begin{verbatim}
typedef struct {
  double  *dblC,
          *dblB,
          **dblMatrix;

  int     intStages;
} biaButcherArray;
\end{verbatim}
\caption{biaButcherArray data structure.} \label{fig:biaButcherArray}
\end{figure}
%
\FloatBarrier

\subsection{\texttt{biaDataRK} data structure}

This structure is used to store all the data needed to apply a Runge-Kutta method.\\

Data structure is defined in figure \ref{fig:biaDataRK} where:
%
\begin{description}
%
\item[intNumApprox] number of approximations to be done (size of the array \texttt{dblPoints}).
%
\item[intImplicit] when the Runge-Kutta method is implicit or not. The following constants are defined in the header file:
%
\begin{center}
\begin{tabular}{|c|c|}
\hline
\textbf{Name} & \textbf{Value} \\
\hline
\textbf{BIA\_IMPLICIT\_RK\_TRUE} & $0$ \\
\hline
\textbf{BIA\_IMPLICIT\_RK\_FALSE} & $1$ \\
\hline
\end{tabular}
\end{center}  
%
\item[*dblPoints] array with dimension \texttt{intNumApprox} and its elements will be the approximations in $x_i$ where:
%
\begin{displaymath}
x_i = dblFirst + i \cdot dblStepSize \qquad \textrm{where} \qquad 0 \le i < intNumApprox     
\end{displaymath}
%
\item[dblStepSize] method's step-size.
%
\item[dblFirst] first point used to compute all the approximations. The value of the function in this point is known (initial condition).
%
\item[dblLast] last point in which approximations will be computed.
%
\item[strCoefs] variable of type biaButcherArray (section \ref{sec:biaButcherArray}) storing Butcher matricial notation.
%
\end{description}

\begin{figure}[!h]
\begin{verbatim}
typedef struct {
  int intNumApprox,
      intImplicit;

  double  *dblPoints,
          dblStepSize,
          dblFirst,
          dblLast;

  biaButcherArray strCoefs;
} biaDataRK;
\end{verbatim}
\caption{biaDataRK data structure.} \label{fig:biaDataRK}
\end{figure}
%
\FloatBarrier

\section{C\'alculo del n\'umero de nodos}

\subsection{intNumNodos}
Esta funci\'on calcula el n\'umero de nodos que podemos situar en un
intervalo.\newline

El prototipo de esta funci\'on es el siguiente:

\begin{center}
\emph{int \textbf{intNumNodos}(double dblLong, double dblPaso)}
\end{center}

\begin{description}
\item[dblLong] longitud del intervalo.
\item[dblPaso] distancia entre los nodos.
\end{description}

La funci\'on devuelve el n\'umero de nodos que podemos situar en dicho
intervalo.\newline

Por ejemplo:

\begin{center}
\emph{intResultado = \textbf{intNumNodos}(dblLong, dblPaso);}
\end{center}

Almacenar\'{\i}a en \emph{intResultado} el n\'umero de nodos que podemos 
situar en un intervalo de longitud \emph{dblLong} e igualmente espaciados
por \emph{dblPaso}.\\

Observaciones:

\begin{itemize}
\item La funci\'on supone que los argumentos que recibe son distintos de cero.
\item La funci\'on considerar\'a siempre los argumentos que recibe como
positivos, aunque sean negativos.
\end{itemize}

\section{Explicit Runge-Kutta methods}

\subsection{\texttt{ExplicitRungeKutta} function}

Funci\'on que resuelve un \emph{P.V.I.}\footnote{Problema de valores 
iniciales.} mediante un m\'etodo \emph{Runge-Kutta expl\'{\i}cito}.\newline

El prototipo de esta funci\'on es el siguiente:

\begin{center}
\emph{int \textbf{ExplicitRungeKutta}(DatosRK *ptstrDatos,\\
double (*PVI)(double dblX, double dblY))}
\end{center} 

\begin{description}
\item[ptstrDatos] \emph{puntero} a una variable del tipo 
\textbf{DatosRK}\footnote{Apartado (\ref{sec:datosRK}) en la p\'agina 
\pageref{sec:datosRK}.}, la cual contiene los datos necesarios para
la resoluci\'on del problema.
\item[PVI] \emph{puntero} a una funci\'on que devuelve el valor de la
ecuaci\'on diferencial en un punto.
\begin{description}
	\item[dblX] punto en el que queremos evaluar la ecuaci\'on diferencial.
	\item[dblY] valor de la ecuaci\'on diferencial en \emph{dblX}.
\end{description}
\end{description}

Por ejemplo, si tenemos el siguiente \emph{P.V.I.}:

\begin{center}
$
\left \{ \begin{array}{l}
y' = y(x) * \frac{x-y(x)}{x^2} \\
y(1) = 2
\end{array} \right.
$
\end{center}

la funci\'on \emph{PVI} ser\'{\i}a:

\begin{verbatim}
double PVI(double dblX, double dblY)
{
double  dblResultado;

dblResultado = dblY*((dblX-dblY)/(dblX*dblX));

return(dblResultado);
}
\end{verbatim}

La funci\'on, \emph{ExplicitRungeKutta}, devuelve los siguientes c\'odigos:

\begin{center}
\begin{tabular}{|l|l|}
\hline
\textbf{ERR\_AMEM} & Hubo error en la asignaci\'on de memoria. \\
\hline
\textbf{BIA\_TRUE} & Success. \\
\hline
\end{tabular}
\end{center}

\begin{center}
\emph{intResultado = \textbf{ExplicitRungeKutta}(\&varstrDatRK, PVI);}
\end{center}

Resolver\'{\i}a el \emph{P.V.I.} representado por la funci\'on \emph{PVI} 
utilizando los datos almacenados en la variable, del tipo \emph{DatosRK},
\emph{varstrDatRK} y almacenar\'{\i}a en \emph{intResultado} el c\'odigo
devuelto por la funci\'on.

\subsection{\texttt{RungeKuttaClasico} function}

Funci\'on que inicializa los coeficientes para el m\'etodo \emph{Runge-Kutta 
Cl\'asico}, el cual es un m\'etodo de $4$ etapas y orden $4$.\newline

La notaci\'on matricial del m\'etodo es la siguiente:

\begin{center}
$
\begin{array}{c|cccc}
0 & 0 \\
\frac{1}{2} & \frac{1}{2} & 0 \\
\frac{1}{2} & 0 & \frac{1}{2} & 0 \\
1 & 0 & 0 & 1 & 0 \\
\hline
 & \frac{1}{6} & \frac{1}{3} & \frac{1}{3} & \frac{1}{6} \\
\end{array}
$
\end{center}

El prototipo de esta funci\'on es el siguiente:

\begin{center}
\emph{int \textbf{RungeKuttaClasico}(DatosRK *ptstrDatos)}
\end{center}

\begin{description}
\item[ptstrDatos] puntero a una variable de \emph{estructura} del tipo
\emph{DatosRK}.
\end{description}

La funci\'on devuelve los siguientes c\'odigos:

\begin{center}
\begin{tabular}{|l|l|}
\hline
\textbf{ERR\_AMEM} & Hubo un error en la asignaci\'on de memoria. \\
\hline
\textbf{TRUE} & Se inicializaron con \'exito los coeficientes. \\
\hline
\end{tabular}
\end{center}

Por ejemplo:

\begin{center}
\emph{intResultado = \textbf{RungeKuttaClasico}(\&varstrDatRK);}
\end{center}

Inicializar\'{\i}a los coeficientes del m\'etodo en la variable 
\emph{varstrDatRK}, en \emph{intResultado} el valor \textbf{TRUE} si se pudieron
inicializar los coeficientes y en caso contrario \textbf{ERR\_AMEM}.

\subsection{MetodoHeun}
Funci\'on que inicializa los coeficientes para el m\'etodo de \emph{Heun}, el
cual es un m\'etodo \emph{Runge-Kutta} de $3$ etapas y orden $3$.\newline

La notaci\'on matricial del m\'etodo es la siguiente:

\begin{center}
$
\begin{array}{c|ccc}
0 & 0 \\
\frac{1}{3} & \frac{1}{3} & 0 \\
\frac{2}{3} & 0 & \frac{2}{3} & 0 \\
\hline
 & \frac{1}{4} & 0 & \frac{3}{4}
\end{array}
$
\end{center}

El prototipo de esta funci\'on es el siguiente:

\begin{center}
\emph{int \textbf{MetodoHeun}(DatosRK *ptstrDatos)}
\end{center}

\begin{description}
\item[ptstrDatos] puntero a una variable de \emph{estructura} del tipo
\emph{DatosRK}.
\end{description}

La funci\'on devuelve los siguientes c\'odigos:

\begin{center}
\begin{tabular}{|l|l|}
\hline
\textbf{ERR\_AMEM} & Hubo un error en la asignaci\'on de memoria. \\
\hline
\textbf{TRUE} & Se inicializaron con \'exito los coeficientes. \\
\hline
\end{tabular}
\end{center}

Por ejemplo:

\begin{center}
\emph{intResultados = \textbf{MetodoHeun}(\&varstrDatRK);}
\end{center}

Inicializar\'{\i}a los coeficientes del m\'etodo en la variable
\emph{varstrDatRK}, en \emph{intResultado} el valor \textbf{TRUE} si se pudieron
inicializar los coeficientes y en caso contrario \textbf{ERR\_AMEM}.

\subsection{MetodoKutta}

Funci\'on que inicializa los coeficientes para el m\'etodo de \emph{Kutta}, el
cual es un m\'etodo \emph{Runge-Kutta} de $3$ etapas y orden $3$.\newline

La notaci\'on matricial del m\'etodo es la siguiente:

\begin{center}
$
\begin{array}{c|ccc}
0 & 0 \\
\frac{1}{2} & \frac{1}{2} & 0 \\
1 & -1 & 2 & 0 \\
\hline
 & \frac{1}{6} & \frac{2}{3} & \frac{1}{6}
\end{array}
$
\end{center}

El prototipo de esta funci\'on es el siguiente:

\begin{center}
\emph{int \textbf{MetodoKutta}(DatosRK *ptstrDatos)}
\end{center}

\begin{description}
\item[ptstrDatos] puntero a una variable de \emph{estructura} del tipo
\emph{DatosRK}.
\end{description}

La funci\'on devuelve los siguientes c\'odigos:

\begin{center}
\begin{tabular}{|l|l|}
\hline
\textbf{ERR\_AMEM} & Hubo un error en la asignaci\'on de memoria. \\
\hline
\textbf{TRUE} & Se inicializaron con \'exito los coeficientes. \\
\hline
\end{tabular}
\end{center}

Por ejemplo:

\begin{center}
\emph{intResultado = \textbf{MetodoKutta}(\&varstrDatRK);}
\end{center}

Inicializar\'{\i}a los coeficientes del m\'etodo en la variable
\emph{varstrDatRK}, en \emph{intResultado} el valor \textbf{TRUE} si se pudieron
inicializar los coeficientes y en caso contrario \textbf{ERR\_AMEM}.

\subsection{\texttt{EulerModificado} function}

Funci\'on que inicializa los coeficientes para el m\'etodo de \emph{Euler 
modificado}, el cual es un m\'etodo \emph{Runge-Kutta} de $2$ etapas y 
orden $2$.\newline

La notaci\'on matricial del m\'etodo es la siguiente:

\begin{center}
$
\begin{array}{c|cc}
0 & 0 \\
\frac{1}{2} & \frac{1}{2} & 0 \\
\hline
 & 0 & 1
\end{array}
$
\end{center}

El prototipo de esta funci\'on es el siguiente:

\begin{center}
\emph{int \textbf{EulerModificado}(DatosRK *ptstrDatos)}
\end{center}

\begin{description}
\item[ptstrDatos] puntero a una variable de \emph{estructura} del tipo
\emph{DatosRK}.
\end{description}

La funci\'on devuelve los siguientes c\'odigos:

\begin{center}
\begin{tabular}{|l|l|}
\hline
\textbf{ERR\_AMEM} & Hubo un error en la asignaci\'on de memoria. \\
\hline
\textbf{TRUE} & Se inicializaron con \'exito los coeficientes. \\
\hline
\end{tabular}
\end{center}

Por ejemplo:

\begin{center}
\emph{intResultado = \textbf{EulerModificado}(\&varstrDatRK);}
\end{center}


Inicializar\'{\i}a los coeficientes del m\'etodo en la variable
\emph{varstrDatRK}, en \emph{intResultado} el valor \textbf{TRUE} si se pudieron
inicializar los coeficientes y en caso contrario \textbf{ERR\_AMEM}.

\subsection{\texttt{EulerMejorado} function}

Funci\'on que inicializa los coeficientes para el m\'etodo de \emph{Euler mejorado},
el cual es un m\'etodo \emph{Runge-Kutta} de $2$ etapas y orden $2$.\newline

La notaci\'on matricial del m\'etodo es la siguiente:

\begin{center}
$
\begin{array}{c|cc}
0 & 0 \\
1 & 1 & 0 \\
\hline
 & \frac{1}{2} & \frac{1}{2}
\end{array}
$
\end{center}

El prototipo de esta funci\'on es el siguiente:

\begin{center}
\emph{int \textbf{EulerMejorado}(DatosRK *ptstrDatos)}
\end{center}

\begin{description}
\item[ptstrDatos] puntero a una variable de \emph{estructura} del tipo
\emph{DatosRK}.
\end{description}

La funci\'on devuelve los siguientes c\'odigos:

\begin{center}
\begin{tabular}{|l|l|}
\hline
\textbf{ERR\_AMEM} & Hubo un error en la asignaci\'on de memoria. \\
\hline
\textbf{TRUE} & Se inicializaron con \'exito los coeficientes. \\
\hline
\end{tabular}
\end{center}

Por ejemplo:

\begin{center}
\emph{intResultado = \textbf{EulerMejorado}(\&varstrDatRK);}
\end{center}


Inicializar\'{\i}a los coeficientes del m\'etodo en la variable
\emph{varstrDatRK}, en \emph{intResultado} el valor \textbf{TRUE} si se pudieron
inicializar los coeficientes y en caso contrario \textbf{ERR\_AMEM}.





Todas estas funciones suponen que la variable de \emph{estructura}, del tipo
\emph{DatosRK}\footnote{Apartado (\ref{sec:datosRK}) en la p\'agina 
\pageref{sec:datosRK}}, no tienen dimensionados los punteros en ella 
contenidos, raz\'on por la cual ser\'a necesario liberar la memoria asignada
a estos antes de pasarle como parametro una variable de este tipo a una de
las siguientes funciones(siempre y cuando se hayan dimensionado dichos
punteros).\newline

Hay que destacar que \textbf{NO} se inicializan todos los miembros de esta
estructura, s\'olo aquellos miembros que contienen los coeficientes del 
m\'etodo.\newline

Los siguientes miembros \textbf{NO} se inicializan:
%
\begin{description}
\item[intNumAprox]
\item[dblPuntos]
\item[dblPaso]
\item[dblInicio]
\item[dblFinal]
\end{description}

Estos miembros son independientes del m\'etodo, dependen del problema que
se quiera resolver y tendr\'an que ser inicializados por el usuario.
