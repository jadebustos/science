%
% roots.h
%

\chapter{Roots approximation (roots.h)} \label{sec:roots}

\section{Introduction}

Functions to compute function's roots approximation are defined in \texttt{roots.h} file.

\section{Data structures}

Some data structures are defined in \BI to manage roots.

\subsection{\texttt{biaRealRoot} data structure} \label{sec:biaRealRoot}

This data structure is used to store data for root approximation.\\

Data structure is defined in figure \ref{fig:biaRealRoot} where:
%
\begin{description}
\item[intNMI] maximum number of iterations to get the root with a maximum error of \emph{dblTol}.
\item[intIte] iterations used to get the root.
\item[dblx0] initial approximation to get the root.
\item[dblRoot] root approximation.
\item[dblTol] maximum tolerance when calculating the root.
\item[dblError] error in root approximation. Difference between the las two root approximations.
\end{description}
%
\begin{figure}[!h]
\begin{verbatim}
typedef struct {
  int intNMI,
      intIte;

  double dblx0,
         dblRoot,
         dblTol,
         dblError;
  } biaRealRoot;
\end{verbatim}
\caption{biaRealRoot data structure.} \label{fig:biaRealRoot}
\end{figure}

\FloatBarrier

\section{Function roots approximation}

\subsection{\texttt{newtonMethod} function}

This function approaches a function's root using the Newton method.\\ \\
%
The definition of this function:
%
\begin{verbatim}
int newtonMethod(biaRealRoot *ptRoot, 
       int (*func)(double dblx0, double *ptRes),
       int (*der)(double dblx0, double *ptRes));  
\end{verbatim}


