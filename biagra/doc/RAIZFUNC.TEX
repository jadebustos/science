%
% RAIZFUNC.H
%

\chapter{raizfunc.h}

\section{Introducci\'on}

En este fichero de cabezera estan los prototipos de funciones que se
utilizar\'an para aproximar raizes de funciones.\newline

Los datos que utilizaremos para aproximar la raiz vendr\'an dados por la
estructura de datos \textbf{DatosAprxFunc}\footnote{Apartado
(\ref{sec:DatosAprxFunc}) en la p\'agina \pageref{sec:DatosAprxFunc}.}.

\section{M\'etodo de Newton}

\subsection{NewtonFunc}

Esta funci\'on calcula una aproximaci\'on a una raiz de una funci\'on
utilizando el m\'etodo de Newton.\newline

El prototipo de esta funci\'on es el siguiente:

\begin{center}
\emph{int \textbf{NewtonFunc}(DatosAprxFunc *ptstrDatos,\\
	int (*Funcion)(double dblPunto, double *ptdblRes),\\
	int (*Derivada)(double dblPunto, double *ptdblRes))};
\end{center}

\begin{description}
\item[ptstrDatos] puntero a una variable del tipo \textbf{DatosAprxFunc}.
\item[Funcion] puntero a una funci\'on que calcula el valor de la funci\'on,
de la cual queremos aproximar una raiz, en \emph{dblPunto} y lo almacena en
\emph{ptdblRes}.\newline

La funci\'on devuelve los siguientes c\'odigos:

\begin{center}
\begin{tabular}{|l|l|}
\hline
\textbf{DIV\_CERO} & Hubo divisi\'on por cero. \\
\hline
\textbf{TRUE} & No hubo divisi\'on por cero. \\
\hline
\end{tabular}
\end{center}
\item[Derivada] igual que la anterior pero con la derivada de la funci\'on de
la que estamos aproximando una raiz. 
\end{description}

La funci\'on devuelve los siguientes c\'odigos:

\begin{center}
\begin{tabular}{|l|l|}
\hline
\textbf{DIV\_CERO} & Hubo divisi\'on por cero. \\
\hline
\textbf{FALSE} & No se calcul\'o la soluci\'on en las condiciones 
del problema \\
& (intNMI y dblTol). \\
\hline
\textbf{TRUE} & Se calcul\'o la soluci\'on en las condiciones del problema.\\
\hline
\end{tabular}
\end{center}

Por ejemplo supongamos que queremos aproximar la raiz de $f(x)$:

\begin{displaymath}
f(x) = \cos{(x)} - \sqrt{x} + \frac{1}{e^{x}}
\end{displaymath}

y la derivada de $f(x)$ viene dada por:

\begin{displaymath}
f'(x) = -\{ \sin{(x)}+\frac{1}{2 \sqrt{x}}+\frac{1}{e^x} \}
\end{displaymath}

Podemos ver que $f'(x)$ no esta definida para $x=0$.

\begin{center}
\emph{intResultado = \textbf{NewtonFunc}(\&strDatos, Funcion, Derivada)};
\end{center}

Aproximar\'{\i}a una ra\'{\i}z de la funci\'on utilizando los datos de la
variable \emph{strDatos}, la cual es una variable del tipo
\textbf{DatosAprxFunc}\footnote{Apartado (\ref{sec:DatosAprxFunc}) en la
p\'agina \pageref{sec:DatosAprxFunc}.}.\newline

La funci\'on almacenar\'a en el miembro \emph{dblSolucion} la aproximaci\'on
a la ra\'{\i}z y en el miembro \emph{dblError} el error cometido al aproximar
dichar ra\'{\i}z, entendiendo como error la distancia, en valor absoluto, de dos
aproximaciones sucesivas.\newline

En \emph{intResultado} estar\'a el c\'odigo devuelto por la funci\'on, el cual
indicar\'a como termin\'o la ejecuci\'on de la funci\'on.\newline

Para el caso anterior la funci\'on de la que queremos aproximar una ra\'{\i}z:

\begin{verbatim}
int Funcion (double dblPunto, double *ptdblRes)

{
double dblRes = .0;

dblRes = cos(dblPunto)-sqrt(dblPunto)+(1./exp(dblPunto));

*ptdblRes = dblRes;

return (TRUE);
}
\end{verbatim}

y su derivada:

\begin{verbatim}
int Derivada (double dblPunto, double *ptdblRes)

{
double 	dblRes = .0,
	dblTest = .0;

dblTest = sqrt(dblPunto);

if ( dblTest == .0 )            /* DIVISION POR CERO */
        return (DIV_CERO);      /* FIN */

dblRes = -sin(dblPunto)-(1./exp(dblPunto))-(1./(2.*dblTest));

*ptdblRes = dblRes;

return (TRUE);
}
\end{verbatim}

Los ficheros que contengan a estas funciones deben tener los siguientes
\emph{includes}:

\begin{verbatim}
#include <math.h>

#include <biagra/const.h>
\end{verbatim}
