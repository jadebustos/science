%
% TIPOMATRIZ.H
%

\chapter{tipomatriz.h}

\section{Introducci\'on}
En este fichero de cabezera se encuentran los prototipos de funciones que
determinan los diferentes tipos de matrices.

\section{Matrices sim\'etricas}

\subsection{EsMatrizSimetrica}
Determina si una matriz cuadrada es sim\'etrica.\newline

El prototipo de esta funci\'on es el siguiente:

\begin{center}
\emph{int \textbf{EsMatrizSimetrica}(Matriz *ptstrMatriz)}
\end{center}

\begin{description}
\item[ptstrMatriz] puntero a una variable del tipo \textbf{Matriz}. 
\end{description}

La funci\'on devuelve los siguientes c\'odigos:

\begin{center}
\begin{tabular}{|l|l|}
\hline
\textbf{FALSE} & La matriz no es sim\'etrica. \\
\hline
\textbf{TRUE} & La matriz es sim\'etrica. \\
\hline
\end{tabular}
\end{center}

Por ejemplo:

\begin{center}
\emph{intResultado = \textbf{EsMatrizSimetrica}(\&Mat);}
\end{center}

Almacenar\'{\i}a en \emph{intResultado} el valor \textbf{TRUE} o el valor
\textbf{FALSE} dependiendo de si la matriz \emph{Mat}, cuadrada de orden 
\emph{Mat.intFilas}, es o no sim\'etrica, donde \emph{Mat} es una variable del
tipo \textbf{Matriz}.

\newpage

\section{Matriz identidad}

\subsection{EsMatrizIdentidad}
Determina si una matriz cuadrada es o no la matriz identidad.\newline

El prototipo de esta funci\'on es el siguiente:

\begin{center}
\emph{int \textbf{EsMatrizIdentidad}(Matriz *ptstrMatriz)}
\end{center}

\begin{description}
\item[ptstrMatriz] puntero a una variable del tipo \textbf{Matriz}.
\end{description}

La funci\'on devuelve los siguientes c\'odigos:

\begin{center}
\begin{tabular}{|l|l|}
\hline
\textbf{FALSE} & La matriz no es la matriz identidad. \\
\hline
\textbf{TRUE} & La matriz es la matriz identidad. \\
\hline
\end{tabular}
\end{center}

Por ejemplo:

\begin{center}
\emph{intResultado = \textbf{EsMatrizIdentidad}(\&Mat);}
\end{center}

Almacenar\'{\i}a en \emph{intResultado} el valor \textbf{TRUE} o el valor
\textbf{FALSE} dependiendo de si la matriz \emph{Mat}, cuadrada de orden 
\emph{Mat.intFilas}, es o no la matriz identidad, donde \emph{Mat} es una
variable del tipo \textbf{Matriz}\footnote{Apartado (\ref{sec:strMatriz})
en la p\'agina \pageref{sec:strMatriz}.}.

\newpage

\section{Matrices nulas}

\subsection{EsMatrizNula}
Determina si una matriz es o no nula, permitiendo definir lo que sus 
coeficientes pueden distar de cero.\newline

El prototipo de esta funci\'on es el siguiente:

\begin{center}
\emph{int \textbf{EsMatrizNula}(Matriz *ptstrMatriz, double dblTol)}
\end{center}

\begin{description}
\item[ptstrMatriz] puntero a una variable del tipo \textbf{Matriz}.
\item[dblTol] distancia de los coeficientes a cero.
\end{description}

La funci\'on devuelve los siguientes c\'odigos:

\begin{center}
\begin{tabular}{|l|l|}
\hline
\textbf{FALSE} & La matriz no es la matriz nula. \\
\hline
\textbf{TRUE} & La matriz es la matriz nula. \\
\hline
\end{tabular}
\end{center}

Por ejemplo:

\begin{center}
\emph{intResultado = \textbf{EsMatrizNula}(\&Mat, dblTol);}
\end{center}

Almacenar\'{\i}a en \emph{intResultado} el valor \textbf{TRUE} si los
coeficientes de \emph{Mat} son menores o iguales, en valor absoluto, que 
\emph{dblTol} y en caso contrario \emph{intResultado} tendr\'{\i}a el valor
\textbf{FALSE}, donde \emph{Mat} es una variable del tipo \textbf{Matriz}%
\footnote{Apartado (\ref{sec:strMatriz}) en la p\'agina 
\pageref{sec:strMatriz}.}.
