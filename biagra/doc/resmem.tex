%
% resmem.h
%

\chapter{Memory allocation (resmem.h)}

\section{Introduction}

\BI\ includes its own memory allocation functions which are defined in \texttt{resmem.h} file.

\section{Vector's memory allocation}

Some functions are provided to handle memory allocations for vectors.

\subsection{\texttt{dblPtMemAllocVec} function} \label{sec:dblPtMemAllocVec}

This functions allocates memory for a vector of doubles.\\

The definition of this function:
%
\begin{verbatim}
double *dblPtMemAllocVec(int intElements);  
\end{verbatim}
%
This function has only one argument, \texttt{intElements}, which is the dimension of the vector and a \texttt{double} pointer is returned.

\section{Matrix memory allocation}

Some functions are provided to handle memory allocations for vectors.

\subsection{\texttt{dblPtMemAllocMat} function} \label{sec:dblPtMemAllocMat}

This function allocates memory for a matrix of doubles.\\

The definition of this function:
%
\begin{verbatim}
double **dblPtMemAllocMat(int intRows, int intCols);
\end{verbatim}
%
where:
%
\begin{description}
\item[intRows] number of rows.
\item[intCols] number of columns.
\end{description}

\subsection{\texttt{dblPtMemAllocUpperTrMat} function} \label{sec:dblPtMemAllocUpperTrMat}

This function allowcates memory for a upper triangular square matrix.\\

The definition of this function:
%
\begin{verbatim}
double **dblPtMemAllocUpperTrMat(int intOrder);
\end{verbatim}

This function has only one argument, \texttt{intOrder}, which is the order of the matrix and a \texttt{double} pointer to pointer is returned.\\

In a upper triangular square matrix all elements below the diagonal are zero: 
%
\begin{displaymath}
\left( \begin{array}{ccccc}
  a_{0,0} & a_{0,1} & a_{0,2} & a_{0,3} & a_{0,4} \\
  0      & a_{1,1} & a_{1,2} & a_{1,3} & a_{1,4} \\ 
  0      & 0      & a_{2,2} & a_{2,3} & a_{2,4} \\
  0      & 0      & 0      & a_{3,3} & a_{3,4} \\
  0      & 0      & 0      & 0      & a_{4,4} \\
\end{array} \right)
\end{displaymath}
%
For \texttt{intOrder = 5}:
%
\begin{verbatim}
myMatrix = dbpPtMemAllocUpperTrMat(5);  
\end{verbatim}
%
and:
\begin{center}
  \begin{tabular}{|c|c|c|c|}
    \hline
    \textbf{Pointer} & \textbf{\# elements} & \textbf{First element} & \textbf{Last element}\\
    \hline
    \texttt{myMatrix[0]} & 5 & 0 & 4\\
    \hline
    \texttt{myMatrix[1]} & 4 & 0 & 3\\
    \hline
    \texttt{myMatrix[2]} & 3 & 0 & 2\\
    \hline
    \texttt{myMatrix[3]} & 2 & 0 & 1\\
    \hline
    \texttt{myMatrix[4]} & 1 & 0 & 0\\
    \hline
  \end{tabular}
\end{center}
%
so:
%
\begin{displaymath}
  myMatrix[i][j] = *(*(myMatrix + i) + j) = \left\{ \begin{array}{ll}
    a_{i,j+i} & \forall \ i \le j \\
     & \\
    0 & \forall \ i > j
    \end{array} \right.    
\end{displaymath}

\subsection{\texttt{dblPtMemAllocLowerMat} function} \label{sec:dblPtMemAllocLowerMat}

This function allowcates memory for a lower triangular square matrix.\\

The definition of this function:
%
\begin{verbatim}
double **dblPtMemAllocLowerTrMat(int intOrder);
\end{verbatim}

This function has only one argument, \texttt{intOrder}, which is the order of the matrix and a \texttt{double} pointer to pointer is returned.\\

In a lower triangular square matrix all elements above the diagonal are zero: 
%
\begin{displaymath}
\left( \begin{array}{ccccc}
  a_{0,0} & 0      & 0      & 0      & 0 \\
  a_{1,0} & a_{1,1} & 0      & 0      & 0 \\ 
  a_{2,0} & a_{2,1} & a_{2,2} & 0      & 0 \\
  a_{3,0} & a_{3,1} & a_{3,2} & a_{3,3} & 0 \\
  a_{4,0} & a_{4,1} & a_{4,2} & a_{4,3} & a_{4,4} \\
\end{array} \right)
\end{displaymath}
%
For \texttt{intOrder = 5}:
%
\begin{verbatim}
myMatrix = dbpPtMemAllocLowerTrMat(5);  
\end{verbatim}
%
and:
\begin{center}
  \begin{tabular}{|c|c|c|c|}
    \hline
    \textbf{Pointer} & \textbf{\# elements} & \textbf{First element} & \textbf{Last element}\\
    \hline
    \texttt{myMatrix[0]} & 1 & 0 & 0\\
    \hline
    \texttt{myMatrix[1]} & 2 & 0 & 1\\
    \hline
    \texttt{myMatrix[2]} & 3 & 0 & 2\\
    \hline
    \texttt{myMatrix[3]} & 4 & 0 & 3\\
    \hline
    \texttt{myMatrix[4]} & 5 & 0 & 4\\
    \hline
  \end{tabular}
\end{center}
%
so:
%
\begin{displaymath}
  myMatrix[i][j] = *(*(myMatrix + i) + j) = \left\{ \begin{array}{ll}
    a_{i,j} & \forall \ i \le j \\
     & \\
    0 & \forall \ i < j
    \end{array} \right.    
\end{displaymath}

\section{Freeing memory} \label{sec:freeingMemory}

\BI\ includes its own functions to free memory.

\subsection{\texttt{freeMemDblMat} function} \label{sec:freeMemDblMat}


