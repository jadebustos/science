%
% POLINOMIOS.H
%

\chapter{polinomios.h} \label{sec:polinomios}

\section{Introducci\'on}
En este fichero de cabezera estan los prototipos de funciones para el
tratamiento de polinomios.\newline

\par Los polinomios los guardaremos en una estructura llamada
\textbf{Polinomio}\footnote{Apartado (\ref{sec:strPolinomio}) en la p\'agina
\pageref{sec:strPolinomio}.}. Para declarar una variable de este tipo:

\begin{center}
\textbf{Polinomio} \emph{Poli};
\end{center}

El grado del polinomio lo almacenaremos en el miembro \emph{intGrado}.\\

Los coeficientes del polinomio los guardaremos en el miembro \emph{dblCoefi}.\\

Supongamos que tenemos el siguiente polinomio:

\begin{center}
$
P(x) = a_0 + a_1 x + a_2 x^2 + a_3 x^3 + \cdots + a_n x^n
$
\end{center}

y queremos almacenar sus coeficientes en \emph{Poli}, el cual es una variable
del tipo \textbf{Polinomio}. Como el polinomio tiene $n+1$ coeficientes
deberemos dimensionar\footnote{Vease el apartado (\ref{sec:AsigMemVec}) de la
p\'agina \pageref{sec:AsigMemVec}.} \emph{dblCoefi} como un vector $n+1$
dimensional, es decir:

\begin{center}
\emph{Poli.dblCoefi} = \textbf{dblPtAsigMemVec} (\emph{(Poli.intGrado)$+1$});
\end{center}

Una vez dimensionado \emph{dblCoefi}, correctamente, almacenaremos los
coeficientes del polinomio de la siguiente manera:

\begin{center}
$Poli.dblCoefi[i] = a_i$ para $0 \leq i < n$
\end{center}

Observaciones:

\begin{itemize}
\item Las funciones de esta biblioteca que utilizan la estructura de datos
\textbf{Polinomio} utilizar\'an los datos que necesiten de dicha estructura,
por lo que si alg\'un dato es incorrecto, el resultado tambi\'en lo ser\'a.
\item Algunas funciones calculan polinomios, estas funciones 
comprueban si fue dimensionado el miembro \emph{dblCoefi}, sino fue dimensionado
lo dimensionan ellas y si fue dimensionado liberan la memoria que tenga asignada
y asignan la memoria que necesiten. Razon por la cual cuando no se haya
dimensionado dicho miembro debe se inicializado a $0$:

\begin{center}
\emph{Poli.dblCoefi = 0};
\end{center}
\end{itemize}

\section{Evaluaci\'on de polinomios}

\subsection{dblEvaluarPolinomio}

Esta funci\'on devuelve el valor de un polinomio en un punto.\newline

El prototipo de esta funci\'on es el siguiente:

\begin{center}
\emph{double \textbf{dblEvaluarPolinomio}(Polinomio *ptstrPoli,\\
double dblPunto)}
\end{center}

\begin{description} 
\item[ptstrPoli] puntero a una variable del tipo \textbf{Polinomio}. 
\item[dblPunto] punto en el que se quiere evaluar el polinomio.
\end{description}

Por ejemplo:

\begin{center}
\emph{dblResultado = \textbf{dblEvaluarPolinomio}(\&Poli, dblPunto);}
\end{center}

Almacenar\'{\i}a en \emph{dblResultado} el valor del polinomio que esta
almacenado en \emph{Poli} en el punto \emph{dblPunto}, donde \emph{Poli} es
una variable del tipo \textbf{Polinomio}\footnote{Apartado
(\ref{sec:strPolinomio}) en la p\'agina \pageref{sec:strPolinomio}.}.

\newpage

\section{C\'alculo de derivadas}

\subsection{DerivadaPolinomio}

Esta funci\'on calcula la derivada \emph{intN$-$\'esima} de un polinomio.\\

El prototipo de esta funci\'on es el siguiente:

\begin{center}
\emph{int \textbf{DerivadaPolinomio}(Polinomio *ptstrPoli,
Polinomio *ptstrDerivada,\\ int intN)}
\end{center}

\begin{description}
\item[ptstrPoli] puntero a una variable del tipo \textbf{Polinomio}.
\item[ptstrDerivada] puntero a una variable del tipo \textbf{Polinomio}.
\item[intN] orden de la derivada que se va a calcular. La funci\'on lo 
considerar\'a siempre como un n\'umero positivo, aunque sea negativo.
\end{description}

La funci\'on devuelve los siguientes c\'odigos:

\begin{center}
\begin{tabular}{|l|l|}
\hline
\textbf{ERR\_AMEM} & Hubo un error en la asignaci\'on de memoria. \\
\hline
\textbf{TRUE} & Se calcul\'o con \'exito la derivada. \\
\hline
\end{tabular}
\end{center}

Por ejemplo:

\begin{center}
\emph{intResultado = \textbf{DerivadaPolinomio}(\&Poli, \&Derivada, intN);}
\end{center}

Calcular\'{\i}a la derivada \emph{intN$-$\'esima} de \emph{Poli}, 
la almacenar\'{\i}a en \emph{Derivada}, donde ambos son datos del
tipo \textbf{Polinomio}\footnote{Apartado (\ref{sec:strPolinomio}) en la
p\'agina \pageref{sec:strPolinomio}.} y tambi\'en almacena en el miembro
\emph{intGrado} de la variable \emph{Derivada} el grado de la derivada que
se ha calculado.\\

En \emph{intResultado} estar\'a el c\'odigo devuelto por la funci\'on, el cual
indica el estado en el que termin\'o la ejecuci\'on de la funci\'on.\newline

Observaciones:

\begin{itemize}
\item La funci\'on comprueba si se le ha asignado memoria al miembro 
\emph{dblCoefi} de \emph{ptstrDerivada}. En caso afirmativo libera esa
memoria y asigna la memoria necesaria, en caso contrario simplemente asigna
la memoria necesaria a este miembro para contener los coeficientes de la
derivada.\\

\textbf{Esta comprobaci\'on la realiza comparando con el puntero nulo, 
\emph{NULL} = $0$, si no se ha dimensionado este miembro es necesario almacenar
en \'el el valor \emph{NULL}, ya que de lo contrario tendr\'{\i}amos un error
en tiempo de ejecuci\'on del tipo \emph{Segment Fault}.}.\\

Luego antes de utilizar esta funci\'on, si previamente no se ha dimensionado
el miembro \emph{dblCoefi} de la variable \emph{Derivada}, se ha de inicializar
a cero, es decir:

\begin{center}
Derivada.dblCoefi = 0;
\end{center}
\item En el caso en el que la funci\'on devuelva el valor \textbf{ERR\_AMEM} el
miembro \emph{dblCoefi} de \emph{ptstrDerivada} no estar\'a dimensionado, con lo
cual si se quiere volver a utilizar habr\'a que dimensionarlo de nuevo.
\end{itemize}

\section{Operaciones entre polinomios}

\subsection{SumarPolinomios}

Esta funci\'on suma dos polinomios.\newline

El prototipo de esta funci\'on es el siguiente:

\begin{center}
\emph{int \textbf{SumarPolinomios}(Polinomio *ptstrPoli$1$,
Polinomio *ptstrPoli$2$, \\Polinomio *ptstrRes);}
\end{center}

\begin{description}
\item[ptstrPoli$1$] puntero a una variable del tipo \textbf{Polinomio}. 
\item[ptstrPoli$2$] puntero a una variable del tipo \textbf{Polinomio}. 
\item[ptstrRes] puntero a una variable del tipo \textbf{Polinomio}.
\end{description}

La funci\'on devuelve los siguientes c\'odigos:

\begin{center}
\begin{tabular}{|l|l|}
\hline
\textbf{ERR\_AMEM} & Hubo un error en la asignaci\'on de memoria. \\
\hline
\textbf{TRUE} & Se calcul\'o con \'exito la suma. \\
\hline
\end{tabular}
\end{center}

Por ejemplo:

\begin{center}
\emph{intResultado = \textbf{SumarPolinomios}(\&Poli1, \&Poli2, \&Res);}
\end{center}

Sumar\'{\i}a el polinomio almacenado en \emph{Poli$1$} con el almacenado
en \emph{Poli$2$} y guardar\'{\i}a el resultado en \emph{Res}, donde todas las
variables son del tipo \textbf{Polinomio}\footnote{Apartado 
(\ref{sec:strPolinomio}) en la p\'agina \pageref{sec:strPolinomio}.}.\\

En \emph{intResultado} estar\'a el c\'odigo devuelto por la funci\'on, el cual
indica el estado en el que termin\'o la ejecuci\'on de la funci\'on.\newline

Observaciones:

\begin{itemize}
\item La funci\'on comprueba si se le ha asignado memoria al miembro 
\emph{dblCoefi} de \emph{ptstrRes}. En caso afirmativo libera esa
memoria y asigna la memoria necesaria, en caso contrario simplemente asigna
la memoria necesaria a este miembro para contener los coeficientes del
producto de ambos polinomios.\\

\textbf{Esta comprobaci\'on la realiza comparando con el puntero nulo, 
\emph{NULL} = $0$, si no se ha dimensionado este miembro es necesario almacenar
en \'el el valor \emph{NULL}, ya que de lo contrario tendr\'{\i}amos un error
en tiempo de ejecuci\'on del tipo \emph{Segment Fault}.}.\\

Luego antes de utilizar esta funci\'on, si previamente no se ha dimensionado
el miembro \emph{dblCoefi} de la variable \emph{Res}, se ha de inicializar
a cero, es decir:

\begin{center}
Res.dblCoefi = 0;
\end{center}
\item El miembro \emph{intGrado} de la variable \emph{Res} tendr\'a como
valor el mayor de los grados de \emph{Poli1} y \emph{Poli2}, aunque la suma
sea de menor grado.
\item En el caso en el que la funci\'on devuelva el valor \textbf{ERR\_AMEM} el
miembro \emph{dblCoefi} de \emph{ptstrRes} no estar\'a dimensionado, con lo
cual si se quiere volver a utilizar habr\'a que dimensionarlo de nuevo.
\end{itemize}

\subsection{RestarPolinomios}

Esta funci\'on resta dos polinomios.\newline

El prototipo de esta funci\'on es el siguiente:

\begin{center}
\emph{int \textbf{RestarPolinomios}(Polinomio *ptstrPoli$1$,
Polinomio *ptstrPoli$2$, \\Polinomio *ptstrRes);}
\end{center}

\begin{description}
\item[ptstrPoli$1$] puntero a una variable del tipo \textbf{Polinomio}. 
\item[ptstrPoli$2$] puntero a una variable del tipo \textbf{Polinomio}. 
\item[ptstrRes] puntero a una variable del tipo \textbf{Polinomio}.
\end{description}

La funci\'on devuelve los siguientes c\'odigos:

\begin{center}
\begin{tabular}{|l|l|}
\hline
\textbf{ERR\_AMEM} & Hubo un error en la asignaci\'on de memoria. \\
\hline
\textbf{TRUE} & Se calcul\'o con \'exito la multiplicaci\'on. \\
\hline
\end{tabular}
\end{center}

Por ejemplo:

\begin{center}
\emph{intResultado = \textbf{RestarPolinomios}(\&Poli1, \&Poli2, \&Res);}
\end{center}

Restar\'{\i}a el polinomio almacenado en \emph{Poli$1$} con el almacenado
en \emph{Poli$2$} y guardar\'{\i}a el resultado en \emph{Res}, donde todas las
variables son del tipo \textbf{Polinomio}\footnote{Apartado 
(\ref{sec:strPolinomio}) en la p\'agina \pageref{sec:strPolinomio}.}.\\

En \emph{intResultado} estar\'a el c\'odigo devuelto por la funci\'on, el cual
indica el estado en el que termin\'o la ejecuci\'on de la funci\'on.\newline

Observaciones:

\begin{itemize}
\item La funci\'on comprueba si se le ha asignado memoria al miembro 
\emph{dblCoefi} de \emph{ptstrRes}. En caso afirmativo libera esa
memoria y asigna la memoria necesaria, en caso contrario simplemente asigna
la memoria necesaria a este miembro para contener los coeficientes del
producto de ambos polinomios.\\

\textbf{Esta comprobaci\'on la realiza comparando con el puntero nulo, 
\emph{NULL} = $0$, si no se ha dimensionado este miembro es necesario almacenar
en \'el el valor \emph{NULL}, ya que de lo contrario tendr\'{\i}amos un error
en tiempo de ejecuci\'on del tipo \emph{Segment Fault}.}.\\

Luego antes de utilizar esta funci\'on, si previamente no se ha dimensionado
el miembro \emph{dblCoefi} de la variable \emph{Res}, se ha de inicializar
a cero, es decir:

\begin{center}
Res.dblCoefi = 0;
\end{center}
\item El miembro \emph{intGrado} de la variable \emph{Res} tendr\'a como
valor el mayor de los grados de \emph{Poli1} y \emph{Poli2}, aunque la resta
sea de menor grado.
\item En el caso en el que la funci\'on devuelva el valor \textbf{ERR\_AMEM} el
miembro \emph{dblCoefi} de \emph{ptstrRes} no estar\'a dimensionado, con lo
cual si se quiere volver a utilizar habr\'a que dimensionarlo de nuevo.
\end{itemize}

\subsection{MultiplicarPolinomios}

Esta funci\'on multiplica dos polinomios.\newline

El prototipo de esta funci\'on es el siguiente:

\begin{center}
\emph{int \textbf{MultiplicarPolinomios}(Polinomio *ptstrPoli$1$,\\
Polinomio *ptstrPoli$2$, Polinomio *ptstrRes);}
\end{center}

\begin{description}
\item[ptstrPoli$1$] puntero a una variable del tipo \textbf{Polinomio}. 
\item[ptstrPoli$2$] puntero a una variable del tipo \textbf{Polinomio}. 
\item[ptstrRes] puntero a una variable del tipo \textbf{Polinomio}.
\end{description}

La funci\'on devuelve los siguientes c\'odigos:

\begin{center}
\begin{tabular}{|l|l|}
\hline
\textbf{ERR\_AMEM} & Hubo un error en la asignaci\'on de memoria. \\
\hline
\textbf{TRUE} & Se calcul\'o con \'exito la multiplicaci\'on. \\
\hline
\end{tabular}
\end{center}

Por ejemplo:

\begin{center}
\emph{intResultado = \textbf{MultiplicarPolinomios}(\&Poli1, \&Poli2, \&Res);}
\end{center}

Multiplicar\'{\i}a el polinomio almacenado en \emph{Poli$1$} por el almacenado
en \emph{Poli$2$} y guardar\'{\i}a el resultado en \emph{Res}, donde todas las
variables son del tipo \textbf{Polinomio}\footnote{Apartado 
(\ref{sec:strPolinomio}) en la p\'agina \pageref{sec:strPolinomio}.}.\\

En \emph{intResultado} estar\'a el c\'odigo devuelto por la funci\'on, el cual
indica el estado en el que termin\'o la ejecuci\'on de la funci\'on.\newline

Observaciones:

\begin{itemize}
\item La funci\'on comprueba si se le ha asignado memoria al miembro 
\emph{dblCoefi} de \emph{ptstrRes}. En caso afirmativo libera esa
memoria y asigna la memoria necesaria, en caso contrario simplemente asigna
la memoria necesaria a este miembro para contener los coeficientes del
producto de ambos polinomios.\\

\textbf{Esta comprobaci\'on la realiza comparando con el puntero nulo, 
\emph{NULL} = $0$, si no se ha dimensionado este miembro es necesario almacenar
en \'el el valor \emph{NULL}, ya que de lo contrario tendr\'{\i}amos un error
en tiempo de ejecuci\'on del tipo \emph{Segment Fault}.}.\\

Luego antes de utilizar esta funci\'on, si previamente no se ha dimensionado
el miembro \emph{dblCoefi} de la variable \emph{Res}, se ha de inicializar
a cero, es decir:

\begin{center}
Res.dblCoefi = 0;
\end{center}
\item En el caso en el que la funci\'on devuelva el valor \textbf{ERR\_AMEM} el
miembro \emph{dblCoefi} de \emph{ptstrRes} no estar\'a dimensionado, con lo
cual si se quiere volver a utilizar habr\'a que dimensionarlo de nuevo.
\end{itemize}

\section{Raices de polinomios}

Estas funciones aproximan las raices de polinomios.

\subsection{NewtonPoli}

Esta funci\'on aproxima una raiz de un polinomio mediante el m\'etodo
de \emph{Newton}.\newline

El prototipo de esta funci\'on es el siguiente:

\begin{center}
\emph{int \textbf{NewtonPoli}(Polinomio *ptstrPoli, DatosAprxFunc *ptstrDatos)}
\end{center}

\begin{description}
\item[ptstrPoli] puntero a una variable del tipo \textbf{Polinomio}.
\item[ptstrDatos] puntero a una variable del tipo \emph{DatosAprxFunc}.
\end{description}

La funci\'on devuelve los siguientes c\'odigos:

\begin{center}
\begin{tabular}{|l|l|}
\hline
\textbf{DIV\_CERO}& Divisi\'on por cero. \\
\hline
\textbf{ERR\_AMEM} & Hubo error en la asignaci\'on de memoria. \\
\hline
\textbf{FALSE} & No se calcul\'o la soluci\'on en las condiciones del problema\\

        & (\emph{intNMI y \emph{dblTol}}). \\ 
\hline
\textbf{TRUE} & Se calcul\'o la soluci\'on en las condiciones del problema. \\
\hline
\end{tabular}
\end{center}

Por ejemplo:

\begin{center}
\emph{intResultado = \textbf{NewtonPoli}(\&strPoli, \&strDatos)};
\end{center}

Aproximar\'{\i}a una raiz del polinomio almacenado en \emph{strPoli} utilizando
los datos almacenados en la variable \emph{strDatos}, donde \emph{strPoli} es
una variable del tipo \textbf{Polinomio}\footnote{Apartado
(\ref{sec:strPolinomio}) en la p\'agina \pageref{sec:strPolinomio}.} y
\emph{strDatos} es una variable del tipo \textbf{DatosAprxFunc}%
\footnote{Apartado (\ref{sec:DatosAprxFunc}) en la p\'agina
\pageref{sec:DatosAprxFunc}.}.\newline

La funci\'on almacenar\'a en el miembro \emph{dblSolucion} la aproximaci\'on a
la raiz y en el miembro \emph{dblError} el error cometido al aproximar dicha
ra\'{\i}z, entendiendo como error la distancia, en valor absoluto, de dos
aproximaciones sucesivas.\newline

En \emph{intResultado} estar\'a el c\'odigo devuelto por la funci\'on, el cual
indicar\'a como termin\'o la ejecuci\'on de la funci\'on.


