\documentclass[a4paper,12pt,twoside,openright]{report}

%\addtolength{\voffset}{-4cm}
\usepackage[english]{babel}
\usepackage[colorlinks = true,
            linkcolor = blue,
            urlcolor  = blue,
            citecolor = blue,
            anchorcolor = blue]{hyperref}
\usepackage{amssymb}
\usepackage{draftwatermark}
\usepackage{placeins}
\usepackage{balloons}

\pagestyle{headings}

\SetWatermarkText{Draft}
\SetWatermarkScale{5}
\SetWatermarkColor{blue}

\author{Jos\'e Angel de Bustos P\'erez}

\newcommand{\BI}
	{\emph{B.I.A.G.R.A} }

\newcommand{\yo}
	{\emph{Jos\'e Angel de Bustos P\'erez}}

\frenchspacing

\hyphenation{si-guien-te apro-xi-ma-cio-nes FORTRAN double intGrado 
dblDerivada dblResultado intNMI intOrden dblMatriz Newton intResultado
cabezera coor-de-na-das de-pen-dien-do di-men-sio-na-do re-pre-sen-tan-do par-ti-cu-lar a-ppro-xi-ma-tion}

\begin{document}

\thispagestyle{empty}

\textbf{
\begin{center}
\Huge{B.I.A.G.R.A.} \\[.75cm]
%\end{center}
\LARGE BI\Large bliotec\LARGE A \Large de pro\LARGE GR\Large amaci\'on 
cient\'{\i}fic\LARGE A\\[5cm]
\end{center}
}

\large 
\begin{flushright}
\yo \\
$<$jadebustos@gmail.com$>$\\ \ \\ 
Versi\'on $1.0$, \today .\\
\textbf{\LaTeXe}
\end{flushright}

\normalsize

\tableofcontents

\listoftables

\part{Introduction}

%
% INTRODUCTION
%

%
% LENGUJES Y SIMBOLOS
%

\section{Lenguajes y simbolos}

Para emitir un mensaje hay que hacerlo utilizando un determinado lenguaje,
com\'un al transmisor y al receptor. Todo lenguaje estar\'a formado por un
conjunto de simbolos, los cuales formar\'an las palabras del lenguaje.\\

Los simbolos tambi\'en recibiran el nombre de ``bits''.

\begin{ejemplo}
\ \\ \\
Cuando queremos comunicarnos con una persona, transmitir un mensaje, hablamos
con esa persona en un idioma conocido por ambas partes, lenguaje, y los
simbolos que utilizamos son las letras del abecedario, n\'umeros, \dots y el
medio o canal utilizado puede ser el habla o la escritura, por ejemplo.
\end{ejemplo} 

\begin{ejemplo}[I.S.B.N.]
\ \\ \\
Sistema internacional de libros. \\

Este sistema se utiliza para catalogar y numerar los libros. Est\'a formado por
palabras de $10$ cifras y los simbolos que utiliza este lenguaje son:

\begin{displaymath}
Smb_{I.S.B.N.} = \{0, 1, 2, 3, 4, 5, 6, 7, 8, 9 \}
\end{displaymath}

Una palabra de este lenguaje ser\'a:

\begin{displaymath}
a=(a_1,a_2,a_3,a_4,a_5,a_6,a_7,a_8,a_9,a_{10})\quad a_i \in Smb_{I.S.B.N.}
\quad \forall \quad i=1,\ldots,10
\end{displaymath}
\end{ejemplo}
%
\newpage
%
\section{Objetivos de la teor\'{\i}a de c\'odigos}

Los objetivos de la teor\'{\i}a de c\'odigos son los siguientes:
\begin{itemize}
\item Construir c\'odigos para la transmisi\'on de informaci\'on.
\item Dichos c\'odigos han de detectar cuando ha ocurrido un error en la
transmisi\'on de la informaci\'on.
\item Dichos c\'odigos deben corregir el mayor n\'umero posible de errores.
\end{itemize}

Con el fin de detectar y corregir los posibles errores ocurridos durante la
transmisi\'on del mensaje original introduciremos informaci\'on redundante
acerca del mensaje original, con el fin de cotejar esta informaci\'on con el 
mensaje original y poder comprobar de esta forma si se produjo alg\'un error en
la transmisi\'on del mensaje.\\

No podemos abusar de la informaci\'on redundante que introducimos. Al
introducir informaci\'on redundante en el mensaje a transmitir obtenemos un
mensaje m\'as largo de transmitir, por lo tanto ser\'a m\'as costoso y lento
el poder transmitirlo. Por lo tanto si introducimos demasiada informaci\'on
redundante tenemos la ventaja de poder detectar y corregir bastantes errores,
pero tambi\'en tenemos el inconveniente de un mayor costo al transmitir la
informaci\'on.\\

Otro objetivo que buscaremos ser\'a que cuando se origine un error en la
transmisi\'on de una palabra se produzca una palabra NO perteneciente al
c\'odigo. Por ejemplo si transmitimos una palabra ``A'', ocurre un error y,
fruto de ese error la palabra que llega al receptor es ``B'', la situaci\'on
ideal ser\'{\i}a que ``B'' NO perteneciera a nuestro c\'odigo. La explicaci\'on
de esto es que si nos llega una palabra que NO es del c\'odigo que estamos
utilizando entonces se ha producido un error, mientras que s\'{\i} la palabra
es del c\'odigo la daremos como buena y no nos percataremos de que se ha 
cometido un error en la transmisi\'on.
%
\newpage
%
\subsection{Ejemplos de la utilizaci\'on de c\'odigos}

Los c\'odigos se usan de una manera continua en muchos campos, como por ejemplo:

\begin{itemize}
\item En las comunicaciones:
\begin{itemize}
\item Radio.
\item Televisi\'on.
\item Sat\'elites.
\item Ordenadores.
\end{itemize}
\item Sistemas de grabaci\'on de datos:
\begin{itemize}
\item Voz.
\item Video.
\item CD-Rom.
\end{itemize}
\item Comunicaci\'on escrita.
\item Comunicaci\'on oral(sonora).
\end{itemize}

\section{Notaciones}

Trabajaremos siempre, salvo que no se diga lo contrario, sobre el conjunto
$\mathbb{F}_q$. Este conjunto es un conjunto finito de $q$ elementos y
supondremos siempre que $q=p^n$, donde $p$ es un n\'umero primo y $n\in
\mathbb{N}$.\\

Nuestros c\'odigos estar\'an formados por palabras de $n$ ``bits'' o simbolos,
donde cada palabra pertenece a $\mathbb{F}_q$.\\

El conjunto de todas las palabras de $n$ simbolos donde cada simbolo es un
elemento de $\mathbb{F}_q$ lo denotaremos como $\mathbb{F}^{^n}_q$.\\

Luego tendremos que:
\begin{displaymath}
\mathbb{F}^{^n}_q = \{(a_1,a_2,\dots,a_n)\ |\ a_i\in \mathbb{F}_q\quad \forall \
i=1,\dots,n\}
\end{displaymath}

Como $\mathbb{F}_q$ tiene un n\'umero finito de elementos, $q$, dado un n\'umero
entero $n<\infty$ podemos calcular la cantidad de elementos que tiene
$\mathbb{F}^{^n}_q$ al cual denotaremos como $|\mathbb{F}^{^n}_q|$.\\

El n\'umero de elementos de $\mathbb{F}^{^n}_q$, al que llamaremos orden de
$\mathbb{F}^{^n}_q$, ser\'a el n\'umero de combinaciones, distintas, que podemos
hacer de $n$ elementos de $\mathbb{F}^{^n}_q$, es decir:
\begin{displaymath}
|\mathbb{F}^{^n}_q| = q^n\ donde\ n\in \mathbb{N}
\end{displaymath}
%
Los elementos de $\mathbb{F}^{^n}_q$ los denotaremos de dos formas:
\begin{itemize}
\item $(a_1,\dots,a_n)$ fundamentalmente cuando consideremos dicho elemento como
elemento de una estructura algebraica.
\item $a_1\dots a_n$ fundamentalmente cuando consideremos dicho elemento como
elemento de un c\'odigo.
\end{itemize}
pero ambas notaciones sirven para hacer referencia al mismo elemento.

\section{Resumen}

Para transmitir informaci\'on lo que haremos ser\'a:
\begin{itemize}
\item Dividir la informaci\'on en bloques o palabras.
\item Codificar cada palabra del mensaje utilizando un c\'odigo. Este paso
supone a\~nadir a cada palabra informaci\'on redundante para poder controlar
cuando se ha producido un error en la transmisi\'on y, en algunos casos,
corregir dicho error.
\item Enviar el mensaje.
\end{itemize}
%
Una vez recibido el mensaje para poder conocer el mensaje original:
\begin{itemize}
\item Comprobar cada palabra del mensaje recibido, utilizando la informaci\'on
redundante\footnote{``Bits'' de control.}, para comprobar que no hubo ning\'un
error en la transmisi\'on del mensaje. En caso de haberlo se tienen dos
opciones:
\begin{itemize}
\item Intentar corregir el error, en el caso que el c\'odigo lo permita.
Corregir el error debe enterderse como calcular la palabra del c\'odigo que,
con mayor probabilidad, fue transmitida originalmente.
\item Solicitar, de nuevo, la informaci\'on erronea.
\end{itemize}
\item Eliminar la informaci\'on redundante introducida\footnote{Decodificar el
mensaje.}.
\end{itemize}

\subsection{Alfabeto}

\begin{definicion}[Alfabeto]
\ \\
Un \textbf{``alfabeto''} ser\'a el conjunto de si\'{\i}mbolos utilizaremos
para codificar el mensaje.
\end{definicion}
En todo alfabeto existe un s\'{\i}mbolo distinguido, que es el ``\emph{espacio
en blanco}''. Este s\'{\i}mbolo nos permite distiguir cuando termina una palabra
y comienza la siguiente.\\

Por ejemplo si utilizamos como alfabeto $\mathbb{F}_2$ los s\'{\i}mbolos 
serian $\{0,1\}$.

\subsection{Palabras}

\begin{definicion}[Palabra]
\ \\
Entenderemos por \textbf{``palabra''} a una sucesi\'on ordenada de simbolos de
un alfabeto.
\end{definicion}
En los casos que vamos a considerar las palabras ser\'an elementos de
$\mathbb{F}^{^n}_q$, puesto que vamos a utilizar palabras de longitud fija.\\ \\
%
El n\'umero de elementos que posee $\mathbb{F}^{^n}_q$ es $|\mathbb{F}_q|^{^n}$.


\subsection{C\'odigos}

\begin{definicion}[C\'odigos]
\ \\
Llamaremos \textbf{``c\'odigo''} al conjunto de palabras que utilizaremos para
codificar informaci\'on.
\end{definicion}
En los casos que vamos a considerar los c\'odigos ser\'an subconjuntos de
$\mathbb{F}^{^n}_q$, es decir $\mathcal{C}\subset \mathbb{F}^{^n}_q$.


%
% COMO UTILIZAR BIAGRA
%

\include{como}

\part{B.I.A.G.R.A Data structures and constants}

%
% const.h
%

%
% const.h
%

\chapter{B.I.A.G.R.A constants (const.h)} \label{ch:mathematicalConsts}

\section{Introduction}

\BI \ includes its own constants to be used if needed.\\

These constants are defined in \texttt{const.h}.

\section{Mathematical constants} \label{sec:mathematicalConsts}

Table \ref{tab:mathematicalConsts} shows the \BI's mathematical constanst.

\begin{table}[!h]
  \begin{center}
  \begin{tabular}{|c|c|c|}
    \hline
    \textbf{Constant} & \textbf{Name} & \textbf{Value} \\
    \hline
    $\texttt{e}$ & \texttt{BIA\_E} & $2.71828182845904523536029$ \\
    \hline
    $\pi$ & \texttt{BIA\_PI} & 3.14159265358979323846264 \\
    \hline
  \end{tabular}
  \end{center}
\caption{\BI\ mathematical constants.} \label{tab:mathematicalConsts}
\end{table}

\FloatBarrier

\section{Logical constants}

The following logical constants are defined:

\begin{description}
\item[BIA\_FALSE] when a condition is not met.
\item[BIA\_TRUE] when a condiction is met.
\end{description}

\section{Error constants}

The following error constants are defined:

\begin{description}
\item[BIA\_ZERO\_DIV] division by zero.
\item[BIA\_MEM\_ALLOC] error in memory allocation.
\end{description}


%
% datapol.h
%

%
% datapol.h
%

\chapter{Polynomial data structures (datapol.h)}

\section{Introduction}

Data structures for polynomials are defined in \texttt{datapol.h} file.\\

A polynomial used to be represented as shown in equation \ref{equ:pol}.

\begin{equation} \label{equ:pol}
p(x) = a_0 + a_1 \cdot x + \cdots + a_n \cdot x^n = \sum_{i=0}^n a_i \cdot x^i \qquad \textrm{where } a_i \in \mathbb{R}
\end{equation}

\section{\textbf{biaRealPol} data structure} \label{sec:biaRealPol}

This data structure is used to handle polinomials $p(x) \in \mathbb{R}[x]$. \textbf{biaPol} data structure is defined in figure \ref{fig:biaRealPol} where:

\begin{description}
\item[intDegree] polynomial degree.
\item[intRealRoots] number of real roots (if any).
\item[intCompRoots] number of complex roots (if any).
\item[*dblCoef] pointer to store polynomial coeficients.
\end{description}

\begin{figure}[!h]
\begin{verbatim}
typedef struct {
  int  intDegree    = 0,
       intRealRoots = 0,
       intCompRoots = 0;

  double  *dblCoefs;
  } biaRealPol;
\end{verbatim}
\caption{biaRealPol data structure.} \label{fig:biaRealPol}
\end{figure}

\FloatBarrier

\subsection{How to use it}

%
\begin{verbatim}
#include <stdlib.h>
#include <datapol.h>

int main(void) {

  /* var declaration */
  biaRealPol myPol;
  ...
  /* polynomial degree */
  myPol.intDegree = 3;  
  ...
  /* memory allocation */
  myPol.dblCoefs = (double *)calloc(myPol.intDegree + 1, sizeof(double));  
  ...
  /* store coeficients */
  myPol.dblCoefs[0] = a0;
  myPol.dblCoefs[1] = a1;
  myPol.dblCoefs[2] = a2;
  ...
  myPol.dblCoefs[n] = an;
  /* your stuff here */
  ...
  return 0;
}
\end{verbatim}

\note{Bare in mind that a $n$ degree polynomial has $n+1$ coeficients.}

\tip{In examples directory \texttt{datapol-biaRealPol.c} shows a simple way to use it.}

\section{\textbf{biaRealRoot} data structure} \label{sec:biaRealRoot}

This data structure is used to store data for root approximation. Data structure is defined in figure \ref{fig:biaRealRoot} where:

\begin{description}
\item[intNMI] maximum number of iterations to get the root with a maximum error of \emph{dblTol}.
\item[intIte] iterations used to get the root.
\item[dblx0] initial approximation to get the root.
\item[dblRoot] root approximation.
\item[dblTol] maximum tolerance when calculating the root.
\item[dblError] error in root approximation. Difference between the las two root approximations.
\end{description}

\begin{figure}[!h]
\begin{verbatim}
typedef struct {
  int intNMI,
      intIte;

  double dblx0,
         dblRoot,
         dblTol,
         dblError;
  } biaRealRoot;
\end{verbatim}
\caption{biaRealRoot data structure.} \label{fig:biaRealRoot}
\end{figure}

\FloatBarrier

\subsection{How to use it}


%
% datacomplex.h
%

%
% datacomplex.h
%

\chapter{Complex number data structures (datacomplex.h)} \label{ch:datacomplex}

\section{Introduction}

Data structures for complex numbers are defined in \texttt{datacomplex.h} file.

\section{\textbf{biaComplex} data structure} \label{sec:biaComplex}

This data structure is used to handle polinomials $p(x) \in \mathbb{R}[x]$. \textbf{biaComplex} data structure is defined in figure \ref{fig:biaRealPol} where:

\begin{description}
\item[intDegree] polynomial degree.
\item[intRealRoots] number of real roots (if any).
\item[intCompRoots] number of complex roots (if any).
\item[*dblCoef] pointer to store polynomial coeficients.
\end{description}

\begin{figure}[!h]
\begin{verbatim}
typedef struct {
  double dblReal,
         dblImag;
  } biaComplex;
\end{verbatim}
\caption{biaComplex data structure.} \label{fig:biaComplex}
\end{figure}

\section{\textbf{biaPolar} data structure} \label{sec:biaPolar}

This data structure is used to store data for root approximation. Data structure is defined in figure \ref{fig:biaRealRoot} where:

\begin{description}
\item[intNMI] maximum number of iterations to get the root with a maximum error of \emph{dblTol}.
\item[intIte] iterations used to get the root.
\item[dblx0] initial approximation to get the root.
\item[dblRoot] root approximation.
\item[dblTol] maximum tolerance when calculating the root.
\item[dblError] error in root approximation. Difference between the las two root approximations.
\end{description}

\begin{figure}[!h]
\begin{verbatim}
typedef struct {
  double dblMod,
         dblArg;
  } biaPolar;
\end{verbatim}
\caption{biaPolar data structure.} \label{fig:biaPolar}
\end{figure}


%
% datamatrix.h
%

%
% datamatrix.h
%

\chapter{Matrices data structures (datamatrix.h)}

\section{Introduction}

Data structures for matrices are defined in \texttt{datamatrix.h} file.\\

\section{\textbf{biaMatrix} data structure} \label{sec:biaMatrix}

This data structure is used to store a matrix. \textbf{biaMatrix} data structure is defined in figure \ref{fig:biaMatrix} where:

\begin{description}
\item[intRows] number of rows.
\item[intCols] number of columns.
\item[**dblCoefs] pointer to store matrix coeficients.
\end{description}

\begin{figure}[!h]
\begin{verbatim}
typedef struct {
  int intRows,
      intCols;

  double **dblCoefs;
  } biaMatrix;   
\end{verbatim}
\caption{biaMatrix data structure.} \label{fig:biaMatrix}
\end{figure}

\FloatBarrier


%
% dataaprox.h
%

\include{dataaprox}

\part{Memory allocation}

%
% resmem.h
%

%
% resmem.h
%

\chapter{Memory allocation (resmem.h)}

\section{Introduction}

\BI\ includes its own memory allocation functions which are defined in \texttt{resmem.h} file.

\section{Vector's memory allocation}

Some functions are provided to handle memory allocations for vectors.

\subsection{\texttt{dblPtMemAllocVec} function} \label{sec:dblPtMemAllocVec}

This functions allocates memory for a vector of doubles.\\

The definition of this function:
%
\begin{verbatim}
double *dblPtMemAllocVec(int intElements);  
\end{verbatim}
%
This function has only one argument, \texttt{intElements}, which is the dimension of the vector and a \texttt{double} pointer is returned.

\section{Matrix memory allocation}

Some functions are provided to handle memory allocations for vectors.

\subsection{\texttt{dblPtMemAllocMat} function} \label{sec:dblPtMemAllocMat}

This function allocates memory for a matrix of doubles.\\

The definition of this function:
%
\begin{verbatim}
double **dblPtMemAllocMat(int intRows, int intCols);
\end{verbatim}
%
where:
%
\begin{description}
\item[intRows] number of rows.
\item[intCols] number of columns.
\end{description}

\subsection{\texttt{dblPtMemAllocUpperTrMat} function} \label{sec:dblPtMemAllocUpperTrMat}

This function allowcates memory for a upper triangular square matrix.\\

The definition of this function:
%
\begin{verbatim}
double **dblPtMemAllocUpperTrMat(int intOrder);
\end{verbatim}

This function has only one argument, \texttt{intOrder}, which is the order of the matrix and a \texttt{double} pointer to pointer is returned.\\

In a upper triangular square matrix all elements below the diagonal are zero: 
%
\begin{displaymath}
\left( \begin{array}{ccccc}
  a_{0,0} & a_{0,1} & a_{0,2} & a_{0,3} & a_{0,4} \\
  0      & a_{1,1} & a_{1,2} & a_{1,3} & a_{1,4} \\ 
  0      & 0      & a_{2,2} & a_{2,3} & a_{2,4} \\
  0      & 0      & 0      & a_{3,3} & a_{3,4} \\
  0      & 0      & 0      & 0      & a_{4,4} \\
\end{array} \right)
\end{displaymath}
%
For \texttt{intOrder = 5}:
%
\begin{verbatim}
myMatrix = dbpPtMemAllocUpperTrMat(5);  
\end{verbatim}
%
and:
\begin{center}
  \begin{tabular}{|c|c|c|c|}
    \hline
    \textbf{Pointer} & \textbf{\# elements} & \textbf{First element} & \textbf{Last element}\\
    \hline
    \texttt{myMatrix[0]} & 5 & 0 & 4\\
    \hline
    \texttt{myMatrix[1]} & 4 & 0 & 3\\
    \hline
    \texttt{myMatrix[2]} & 3 & 0 & 2\\
    \hline
    \texttt{myMatrix[3]} & 2 & 0 & 1\\
    \hline
    \texttt{myMatrix[4]} & 1 & 0 & 0\\
    \hline
  \end{tabular}
\end{center}
%
so:
%
\begin{displaymath}
  myMatrix[i][j] = *(*(myMatrix + i) + j) = \left\{ \begin{array}{ll}
    a_{i,j+i} & \forall \ i \le j \\
     & \\
    0 & \forall \ i > j
    \end{array} \right.    
\end{displaymath}

\subsection{\texttt{dblPtMemAllocLowerMat} function} \label{sec:dblPtMemAllocLowerMat}

This function allowcates memory for a lower triangular square matrix.\\

The definition of this function:
%
\begin{verbatim}
double **dblPtMemAllocLowerTrMat(int intOrder);
\end{verbatim}

This function has only one argument, \texttt{intOrder}, which is the order of the matrix and a \texttt{double} pointer to pointer is returned.\\

In a lower triangular square matrix all elements above the diagonal are zero: 
%
\begin{displaymath}
\left( \begin{array}{ccccc}
  a_{0,0} & 0      & 0      & 0      & 0 \\
  a_{1,0} & a_{1,1} & 0      & 0      & 0 \\ 
  a_{2,0} & a_{2,1} & a_{2,2} & 0      & 0 \\
  a_{3,0} & a_{3,1} & a_{3,2} & a_{3,3} & 0 \\
  a_{4,0} & a_{4,1} & a_{4,2} & a_{4,3} & a_{4,4} \\
\end{array} \right)
\end{displaymath}
%
For \texttt{intOrder = 5}:
%
\begin{verbatim}
myMatrix = dbpPtMemAllocLowerTrMat(5);  
\end{verbatim}
%
and:
\begin{center}
  \begin{tabular}{|c|c|c|c|}
    \hline
    \textbf{Pointer} & \textbf{\# elements} & \textbf{First element} & \textbf{Last element}\\
    \hline
    \texttt{myMatrix[0]} & 1 & 0 & 0\\
    \hline
    \texttt{myMatrix[1]} & 2 & 0 & 1\\
    \hline
    \texttt{myMatrix[2]} & 3 & 0 & 2\\
    \hline
    \texttt{myMatrix[3]} & 4 & 0 & 3\\
    \hline
    \texttt{myMatrix[4]} & 5 & 0 & 4\\
    \hline
  \end{tabular}
\end{center}
%
so:
%
\begin{displaymath}
  myMatrix[i][j] = *(*(myMatrix + i) + j) = \left\{ \begin{array}{ll}
    a_{i,j} & \forall \ i \le j \\
     & \\
    0 & \forall \ i < j
    \end{array} \right.    
\end{displaymath}

\section{Freeing memory} \label{sec:freeingMemory}

\BI\ includes its own functions to free memory.

\subsection{\texttt{freeMemDblMat} function} \label{sec:freeMemDblMat}




%
% MATHEMATICAL FUNCTIONS
%

\part{Mathematical functions}

%
% poliynomial.h
%

%
% polynomial.h
%

\chapter{Polynomial (polynomials.h)} \label{sec:polynomial}

\section{Introduction}

Functions to manage polynomials are defined in \texttt{polynomial.h} file.\\

A polynomial used to be represented as shown in equation \ref{equ:pol}.

\begin{equation} \label{equ:pol}
p(x) = a_0 + a_1 \cdot x + \cdots + a_n \cdot x^n = \sum_{i=0}^n a_i \cdot x^i \qquad \textrm{where } a_i \in \mathbb{R}
\end{equation}

\section{Data structures}

Some data structures are defined in \BI to manage polynomials.

\subsection{\texttt{biaRealPol} data structure} \label{sec:biaRealPol}

This data structure is used to handle polinomials $p(x) \in \mathbb{R}[x]$. \textbf{biaPol} data structure is defined in figure \ref{fig:biaRealPol} where:
%
\begin{description}
\item[intDegree] polynomial degree.
\item[intRealRoots] number of real roots (if any).
\item[intCompRoots] number of complex roots (if any).
\item[*dblCoef] pointer to store polynomial coeficients.
\end{description}
%
\begin{figure}[!h]
\begin{verbatim}
typedef struct {
  int  intDegree    = 0,
       intRealRoots = 0,
       intCompRoots = 0;

  double  *dblCoefs;
  } biaRealPol;
\end{verbatim}
\caption{biaRealPol data structure.} \label{fig:biaRealPol}
\end{figure}
%
\FloatBarrier
%
Polynomial coeficients are stored in \texttt{dblCoefs} pointer which has to be previously initialized:
%
\begin{eqnarray*}
  \mathrm{dblCoefs[0]} & = & a_0 \\
  \mathrm{dblCoefs[1]} & = & a_1 \\
  \dots & \dots & \dots \\
  \mathrm{dblCoefs[n]} & = & a_n \\
\end{eqnarray*}

\section{Polynomial derivatives}

\subsection{\texttt{derivativePol} function}

This function gets the $n$-th derivative of a polynomial.\\ \\
%
The definition of this function:
%
\begin{verbatim}
int derivativePol(biaPol *ptPol, biaPol *ptDer, int intN);
\end{verbatim}
%
where:
\begin{description} 
\item[*ptPol] pointer to a \texttt{biaRealPol} struct with the polynomial to get its derivative is stored.
\item[*ptDer] pointer to a \texttt{biaRealPol} struct where the derivative will be stored.
\item[intN] order of the derivative to get.
\end{description}
%
The following codes are returned:
%
\begin{center}
\begin{tabular}{|l|l|}
\hline
\textbf{BIA\_MEM\_ALLOC} & Memory allocation error \\
\hline
\textbf{BIA\_TRUE} & Success \\
\hline
\end{tabular}
\end{center}
%
\note{\texttt{ptDer} will be released and memory allocation will be carried out to store the derivative.}
%
\warning{\texttt{ptDer} member \texttt{dblCoefs} has to be initialized to a \texttt{NULL} pointer to avoid a \textbf{Segment Fault} error if it was not previously initialized.}

\section{Arithmetical operations using polynomials}

\subsection{\texttt{addPol} function}

This function adds two polynomials.\\ \\
%
The definition of this function:
%
\begin{verbatim}
int addPol(biaPol *ptPol1, biaPol *ptPol2, biaPol *ptRes);  
\end{verbatim}
%
where:
%
\begin{description}
\item[*ptPol1] pointer to a \texttt{biaPol} struct with the first polynomial to be added.
\item[*ptPol2] pointer to a \texttt{biaPol} struct with the second polynomial to be added.
\item[*ptRes] pointer to a \texttt{biaPol} struct where the add operation will be stored.
\end{description}
%
The following codes are returned:
%
\begin{center}
\begin{tabular}{|l|l|}
\hline
\textbf{BIA\_MEM\_ALLOC} & Memory allocation error \\
\hline
\textbf{BIA\_TRUE} & Success \\
\hline
\end{tabular}
\end{center}
%
\note{\texttt{ptRes} will be released and memory allocation will be carried out to store the derivative.}
%
\warning{\texttt{ptRes} member \texttt{dblCoefs} has to be initialized to a \texttt{NULL} pointer to avoid a \textbf{Segment Fault} error if it was not previously initialized.}

\subsection{\texttt{subtractPol} function}

This function subtracts two polynomials.\\ \\
%
The definition of this function:
%
\begin{verbatim}
int subtractPol(biaPol *ptPol1, biaPol *ptPol2, biaPol *ptRes);
\end{verbatim}
%
where:
%
\begin{description}
\item[*ptPol1] pointer to a \texttt{biaPol} struct with the first polynomial.
\item[*ptPol2] pointer to a \texttt{biaPol} struct with the polynomial to be subtracted from the above.
\item[*ptRes] pointer to a \texttt{biaPol} struct where the subtract operation will be stored.
\end{description}
%
The following codes are returned:
%
\begin{center}
\begin{tabular}{|l|l|}
\hline
\textbf{BIA\_MEM\_ALLOC} & Memory allocation error \\
\hline
\textbf{BIA\_TRUE} & Success \\
\hline
\end{tabular}
\end{center}
%
\note{\texttt{ptRes} will be released and memory allocation will be carried out to store the derivative.}
%
\warning{\texttt{ptRes} member \texttt{dblCoefs} has to be initialized to a \texttt{NULL} pointer to avoid a \textbf{Segment Fault} error if it was not previously initialized.}

\subsection{\texttt{multiplyPol} function}

This functions multiplies two polynomials.\\ \\
%
The definition of this function:
%
\begin{verbatim}
int subtractPol(biaPol *ptPol1, biaPol *ptPol2, biaPol *ptRes);
\end{verbatim}
%
where:
%
\begin{description}
\item[*ptPol1] pointer to a \texttt{biaPol} struct with the first polynomial.
\item[*ptPol2] pointer to a \texttt{biaPol} struct with the second polynomial.
\item[*ptRes] pointer to a \texttt{biaPol} struct where the multiplication operation will be stored.
\end{description}
%
The following codes are returned:
%
\begin{center}
\begin{tabular}{|l|l|}
\hline
\textbf{BIA\_MEM\_ALLOC} & Memory allocation error \\
\hline
\textbf{BIA\_TRUE} & Success \\
\hline
\end{tabular}
\end{center}
%
\note{\texttt{ptRes} will be released and memory allocation will be carried out to store the derivative.}
%
\warning{\texttt{ptRes} member \texttt{dblCoefs} has to be initialized to a \texttt{NULL} pointer to avoid a \textbf{Segment Fault} error if it was not previously initialized.}








%
% COMPLEJO.H
%

\include{complejo}

%
% EDO.H
%

\include{edo}

%
% ENTEROS.H
%

\include{enteros}

%
% MATRIZ.H
%

\include{matriz}

%
% POLINOMIOS.H
%

\include{polinomios}

%
% PRIMOS.H
%

\include{primos}

%
% RAIZFUNC.H
%

\include{raizfunc}

%
% RNGKUTTA.H
%

%
% RNGKUTTA.H
%

\chapter{Runge-Kutta methods (rngkutta.h)}

\section{Introduction}

\textbf{Runge-Kutta} are a family of implicit and explicit iterative methods used to approximate solutions of ordinary differential equations or \textbf{ODE}.\\

Butcher matricial notation is used in this implementation.\\

\section{Data structures}

\subsection{\texttt{biaButcherArray} data structure}

This structure is used to store the Butcher matricial notation.\\

Data structure is defined in figure \ref{fig:biaButcherArray} where:
%
\begin{description}
%
\item[intStages] method stages.
%
\item[*dblC] $c_i$ coefficients stored in an array with size \texttt{intStages}.
%
\item[*dblB] $b_i$ coefficients stored in an array with size \texttt{intStages}.
%
\item[**dblMatrix] matrix to store $a_{i,j}$ method's coeficients.  
%
\end{description}

\begin{figure}[!h]
\begin{verbatim}
typedef struct {
  double  *dblC,
          *dblB,
          **dblMatrix;

  int     intStages;
} biaButcherArray;
\end{verbatim}
\caption{biaButcherArray data structure.} \label{fig:biaButcherArray}
\end{figure}
%
\FloatBarrier

\subsection{\texttt{DataRK} data structure}

This structure is used to store all the data needed to apply a Runge-Kutta method.\\

Data structure is defined in figure \ref{fig:biaDataRK} where:
%
\begin{description}
%
\item[intNumApprox] number of approximations to be done (size of the array \texttt{dblPoints}).
%
\item[intImplicit] when the Runge-Kutta method is implicit or not. The following constants are defined in the header file:
%
  \begin{center}
  \begin{tabular}{|c|c|}
    \hline
    \textbf{Name} & \textbf{Value} \\
    \hline
    \textbf{BIA\_IMPLICIT\_RK\_TRUE} & $0$ \\
    \hline
    \textbf{BIA\_IMPLICIT\_RK\_FALSE} & $1$ \\
    \hline
  \end{tabular}
  \end{center}  
%
\item[*dblPoints]
%
\item[dblStepSize]
%
\end{description}

\begin{figure}[!h]
\begin{verbatim}
typedef struct {
  int intNumApprox,
      intImplicit;

  double  *dblPoints,
          dblStepSize,
          dblFirst,
          dblLast;

  biaButcherArray strCoefs;
} biaDataRK;
\end{verbatim}
\caption{biaDataRK data structure.} \label{fig:biaDataRK}
\end{figure}
%
\FloatBarrier

\section{Explicit Runge-Kutta methods}

Let's assume that the initial value problem (I.V.P.) or Cauchy problem we want to solve is:
%
\begin{eqnarray*}
  y'(x) & = & f(x, y(x)) \\
  y(x_0) & = & y_0
\end{eqnarray*}
%
The family of explicit \textbf{Runge-Kutta} methods is given by:
%
\begin{displaymath}
  y_{n+1} = y_n  + h \cdot \sum_{i=1}^s b_i \cdot k_i
\end{displaymath}
%
where:
%
\begin{eqnarray*}
  k_i & = & f(x_n + (h\cdot c_i), y_n + h\cdot \left(\sum_{j=1}^{i-1} a_{i,j} \cdot k_j \right) )\\
  c_i & = & \sum_{j=1}^{i-1} a_{i,j} \qquad \textrm{where} \qquad i \in \{2,\dots,s\}
\end{eqnarray*}

\subsection{\texttt{ExplicitRungeKutta} function}

\subsection{\texttt{RungeKuttaClasico} function}

Funci\'on que inicializa los coeficientes para el m\'etodo \emph{Runge-Kutta 
Cl\'asico}, el cual es un m\'etodo de $4$ etapas y orden $4$.\newline

La notaci\'on matricial del m\'etodo es la siguiente:

\begin{center}
$
\begin{array}{c|cccc}
0 & 0 \\
\frac{1}{2} & \frac{1}{2} & 0 \\
\frac{1}{2} & 0 & \frac{1}{2} & 0 \\
1 & 0 & 0 & 1 & 0 \\
\hline
 & \frac{1}{6} & \frac{1}{3} & \frac{1}{3} & \frac{1}{6} \\
\end{array}
$
\end{center}

El prototipo de esta funci\'on es el siguiente:

\begin{center}
\emph{int \textbf{RungeKuttaClasico}(DatosRK *ptstrDatos)}
\end{center}

\begin{description}
\item[ptstrDatos] puntero a una variable de \emph{estructura} del tipo
\emph{DatosRK}.
\end{description}

La funci\'on devuelve los siguientes c\'odigos:

\begin{center}
\begin{tabular}{|l|l|}
\hline
\textbf{ERR\_AMEM} & Hubo un error en la asignaci\'on de memoria. \\
\hline
\textbf{TRUE} & Se inicializaron con \'exito los coeficientes. \\
\hline
\end{tabular}
\end{center}

Por ejemplo:

\begin{center}
\emph{intResultado = \textbf{RungeKuttaClasico}(\&varstrDatRK);}
\end{center}

Inicializar\'{\i}a los coeficientes del m\'etodo en la variable 
\emph{varstrDatRK}, en \emph{intResultado} el valor \textbf{TRUE} si se pudieron
inicializar los coeficientes y en caso contrario \textbf{ERR\_AMEM}.

\subsection{MetodoHeun}
Funci\'on que inicializa los coeficientes para el m\'etodo de \emph{Heun}, el
cual es un m\'etodo \emph{Runge-Kutta} de $3$ etapas y orden $3$.\newline

La notaci\'on matricial del m\'etodo es la siguiente:

\begin{center}
$
\begin{array}{c|ccc}
0 & 0 \\
\frac{1}{3} & \frac{1}{3} & 0 \\
\frac{2}{3} & 0 & \frac{2}{3} & 0 \\
\hline
 & \frac{1}{4} & 0 & \frac{3}{4}
\end{array}
$
\end{center}

El prototipo de esta funci\'on es el siguiente:

\begin{center}
\emph{int \textbf{MetodoHeun}(DatosRK *ptstrDatos)}
\end{center}

\begin{description}
\item[ptstrDatos] puntero a una variable de \emph{estructura} del tipo
\emph{DatosRK}.
\end{description}

La funci\'on devuelve los siguientes c\'odigos:

\begin{center}
\begin{tabular}{|l|l|}
\hline
\textbf{ERR\_AMEM} & Hubo un error en la asignaci\'on de memoria. \\
\hline
\textbf{TRUE} & Se inicializaron con \'exito los coeficientes. \\
\hline
\end{tabular}
\end{center}

Por ejemplo:

\begin{center}
\emph{intResultados = \textbf{MetodoHeun}(\&varstrDatRK);}
\end{center}

Inicializar\'{\i}a los coeficientes del m\'etodo en la variable
\emph{varstrDatRK}, en \emph{intResultado} el valor \textbf{TRUE} si se pudieron
inicializar los coeficientes y en caso contrario \textbf{ERR\_AMEM}.

\subsection{MetodoKutta}

Funci\'on que inicializa los coeficientes para el m\'etodo de \emph{Kutta}, el
cual es un m\'etodo \emph{Runge-Kutta} de $3$ etapas y orden $3$.\newline

La notaci\'on matricial del m\'etodo es la siguiente:

\begin{center}
$
\begin{array}{c|ccc}
0 & 0 \\
\frac{1}{2} & \frac{1}{2} & 0 \\
1 & -1 & 2 & 0 \\
\hline
 & \frac{1}{6} & \frac{2}{3} & \frac{1}{6}
\end{array}
$
\end{center}

El prototipo de esta funci\'on es el siguiente:

\begin{center}
\emph{int \textbf{MetodoKutta}(DatosRK *ptstrDatos)}
\end{center}

\begin{description}
\item[ptstrDatos] puntero a una variable de \emph{estructura} del tipo
\emph{DatosRK}.
\end{description}

La funci\'on devuelve los siguientes c\'odigos:

\begin{center}
\begin{tabular}{|l|l|}
\hline
\textbf{ERR\_AMEM} & Hubo un error en la asignaci\'on de memoria. \\
\hline
\textbf{TRUE} & Se inicializaron con \'exito los coeficientes. \\
\hline
\end{tabular}
\end{center}

Por ejemplo:

\begin{center}
\emph{intResultado = \textbf{MetodoKutta}(\&varstrDatRK);}
\end{center}

Inicializar\'{\i}a los coeficientes del m\'etodo en la variable
\emph{varstrDatRK}, en \emph{intResultado} el valor \textbf{TRUE} si se pudieron
inicializar los coeficientes y en caso contrario \textbf{ERR\_AMEM}.

\subsection{\texttt{EulerModificado} function}

Funci\'on que inicializa los coeficientes para el m\'etodo de \emph{Euler 
modificado}, el cual es un m\'etodo \emph{Runge-Kutta} de $2$ etapas y 
orden $2$.\newline

La notaci\'on matricial del m\'etodo es la siguiente:

\begin{center}
$
\begin{array}{c|cc}
0 & 0 \\
\frac{1}{2} & \frac{1}{2} & 0 \\
\hline
 & 0 & 1
\end{array}
$
\end{center}

El prototipo de esta funci\'on es el siguiente:

\begin{center}
\emph{int \textbf{EulerModificado}(DatosRK *ptstrDatos)}
\end{center}

\begin{description}
\item[ptstrDatos] puntero a una variable de \emph{estructura} del tipo
\emph{DatosRK}.
\end{description}

La funci\'on devuelve los siguientes c\'odigos:

\begin{center}
\begin{tabular}{|l|l|}
\hline
\textbf{ERR\_AMEM} & Hubo un error en la asignaci\'on de memoria. \\
\hline
\textbf{TRUE} & Se inicializaron con \'exito los coeficientes. \\
\hline
\end{tabular}
\end{center}

Por ejemplo:

\begin{center}
\emph{intResultado = \textbf{EulerModificado}(\&varstrDatRK);}
\end{center}


Inicializar\'{\i}a los coeficientes del m\'etodo en la variable
\emph{varstrDatRK}, en \emph{intResultado} el valor \textbf{TRUE} si se pudieron
inicializar los coeficientes y en caso contrario \textbf{ERR\_AMEM}.

\subsection{\texttt{EulerMejorado} function}

Funci\'on que inicializa los coeficientes para el m\'etodo de \emph{Euler mejorado},
el cual es un m\'etodo \emph{Runge-Kutta} de $2$ etapas y orden $2$.\newline

La notaci\'on matricial del m\'etodo es la siguiente:

\begin{center}
$
\begin{array}{c|cc}
0 & 0 \\
1 & 1 & 0 \\
\hline
 & \frac{1}{2} & \frac{1}{2}
\end{array}
$
\end{center}

El prototipo de esta funci\'on es el siguiente:

\begin{center}
\emph{int \textbf{EulerMejorado}(DatosRK *ptstrDatos)}
\end{center}

\begin{description}
\item[ptstrDatos] puntero a una variable de \emph{estructura} del tipo
\emph{DatosRK}.
\end{description}

La funci\'on devuelve los siguientes c\'odigos:

\begin{center}
\begin{tabular}{|l|l|}
\hline
\textbf{ERR\_AMEM} & Hubo un error en la asignaci\'on de memoria. \\
\hline
\textbf{TRUE} & Se inicializaron con \'exito los coeficientes. \\
\hline
\end{tabular}
\end{center}

Por ejemplo:

\begin{center}
\emph{intResultado = \textbf{EulerMejorado}(\&varstrDatRK);}
\end{center}


Inicializar\'{\i}a los coeficientes del m\'etodo en la variable
\emph{varstrDatRK}, en \emph{intResultado} el valor \textbf{TRUE} si se pudieron
inicializar los coeficientes y en caso contrario \textbf{ERR\_AMEM}.





Todas estas funciones suponen que la variable de \emph{estructura}, del tipo
\emph{DatosRK}\footnote{Apartado (\ref{sec:datosRK}) en la p\'agina 
\pageref{sec:datosRK}}, no tienen dimensionados los punteros en ella 
contenidos, raz\'on por la cual ser\'a necesario liberar la memoria asignada
a estos antes de pasarle como parametro una variable de este tipo a una de
las siguientes funciones(siempre y cuando se hayan dimensionado dichos
punteros).\newline

Hay que destacar que \textbf{NO} se inicializan todos los miembros de esta
estructura, s\'olo aquellos miembros que contienen los coeficientes del 
m\'etodo.\newline

Los siguientes miembros \textbf{NO} se inicializan:
%
\begin{description}
\item[intNumAprox]
\item[dblPuntos]
\item[dblPaso]
\item[dblInicio]
\item[dblFinal]
\end{description}

Estos miembros son independientes del m\'etodo, dependen del problema que
se quiera resolver y tendr\'an que ser inicializados por el usuario.


%
% struct.h
%

%
% struct.h
%

\chapter{struct.h}

\section{Introducci\'on}

En este fichero de cabezera se encuentran las estructuras de datos que utiliza
\BI para la resoluci\'on de problemas.\newline

\par Como ya se dijo anteriormente todas las estructuras ser\'an declaradas
mediante la palabra reservada \emph{typedef}.

\newpage

\section{Estructuras de datos para E.D.O's} \label{sec:datosEDO}

Estas estructuras de datos se utilizan para la resoluci\'on num\'erica de
E.D.O's\footnote{Ecuaciones Diferenciales Ordinarias.}.

\subsection{ButcherArray}

Esta estructura se utiliza para almacenar la \emph{notaci\'on matricial}
de los m\'etodos \emph{Runge-Kutta}\footnote{Ver ap\'endice sobre 
Runge-Kutta en la p\'agina \pageref{sec:Runge}.}.\newline

La declaraci\'on de esta estructura es la siguiente:

\begin{verbatim}
typedef struct
        {
        double  *dblC,
                *dblB,
                **dblMatriz;

        int intEtapas;
        } ButcherArray;
\end{verbatim}

El significado de cada uno de los miembros de esta estructura es el
siguiente:

\begin{description}
\item[dblC] es un vector de dimensi\'on \emph{strVariable.intEtapas} el cual 
contiene los
elementos $c_i$ del \emph{Runge-Kutta}, donde $0 \leq i < 
\emph{strVariable.intEtapas}$.
\item[dblB] es un vector de dimensi\'on \emph{strVariable.intEtapas} el cual 
contiene los
elementos $b_i$ del \emph{Runge-Kutta}, donde $0 \leq i < 
strVariable.intEtapas$.
\item[dblMatriz] es una matriz la cual contiene la matriz de coeficientes del
\emph{Runge-Kutta}.
\item[intEtapas] es el n\'umero de etapas que tiene el m\'etodo.
\end{description}

Para declarar una variable de este tipo:
\begin{center}
\emph{\textbf{ButcherArray} strVariable;}
\end{center}

\emph{strVariable} ser\'{\i}a una variable del tipo 
\emph{\textbf{ButcherArray}}.

\newpage

\subsection{DatosRK} \label{sec:datosRK}

Esta estructura se utiliza para almacenar todos los datos necesarios
en la ejecuci\'on del \emph{Runge-Kutta}, desde su notaci\'on matricial
hasta su inicializaci\'on, pasando por el paso utilizado.\newline

La declaraci\'on de esta \emph{estructura} es la siguiente:

\begin{verbatim}
typedef struct
        {	
        int     intNumAprox,
                intImplicito;
	        
        double  *dblPuntos,
                dblPaso,
                dblInicio,
                dblFinal;

        ButcherArray strCoefi;
        } DatosRK;
\end{verbatim}

El significado de cada uno de los miembros de esta estructura es el
siguiente:

\begin{description}
\item[intNumAprox] n\'umero de aproximaciones que se realizar\'an con el
m\'etodo, es decir contendr\'a la dimensi\'on de \emph{strVariable.dblPuntos}.
\item[intImplicito] contendr\'a el valor \textbf{TRUE} si se trata de un 
m\'etodo \emph{impl\'{\i}cito}, en caso contrario su valor ser\'a distinto de
\textbf{TRUE}.
\item[dblPuntos] vector de dimensi\'on \emph{strVariable.intNumAprox} el 
cual contendr\'a, en \emph{strVariable.dblPuntos[i],} 
las aproximaciones en los diferentes $x_i$, donde:

\begin{description}
\item[]$x_i = (\emph{strVariable.dblInicio}) + i * (\emph{strVariable.dblPaso})$
\item[]$0 \leq i < \emph{strVariable.intNumAprox}$
\end{description}

\item[dblPaso] tama\~no del paso que utilizara el m\'etodo.\newpage
\item[dblInicio] primer punto, en el que es conocido el valor de la funci\'on,
en el cual nos apoyamos para calcular las dem\'as aproximaciones.
\item[dblFinal] es el \'ultimo punto en el que calcularemos una aproximaci\'on
de la ecuaci\'on diferencial.
\item[strCoefi] es una variable del tipo \emph{\textbf{ButcherArray}}, la cual
contendr\'a la notaci\'on matricial del m\'etodo.
\end{description}

Para declarar una variable de este tipo:

\begin{center}
\emph{\textbf{DatosRK} strVariable;}
\end{center}

\emph{strVariable} ser\'{\i}a una variable del tipo \emph{\textbf{DatosRK}}.

\newpage

\section{Polynomial data structures}

\subsection{Polynomials}

\begin{equation}
p(x) = a_0 + a_1 \cdot x + \cdots + a_n \cdot x^n = \sum_{i=0}^n a_i \cdot x^i
\end{equation}

\subsection{biaPol data structure} \label{sec:biaPol}

\textbf{biaPol} data structure is defined in figure \ref{fig:biaPol} where:

\begin{description}
\item[intDegree] polynomial degree.
\item[intRealRoots] number of real roots (if any).
\item[intCompRoots] number of complex roots (if any).
\item[*dblCoef] pointer to store polynomial coeficients.
\end{description}

\begin{figure}[!h]
\begin{verbatim}
typedef struct {
  int  intDegree    = 0,
       intRealRoots = 0,
       intCompRoots = 0;

  double  *dblCoefs;
  } biaPol;
\end{verbatim}
\caption{Polynomial data structure.} \label{fig:biaPol}
\end{figure}

\subsection{How to use it}


\section{Estructuras de datos para la aproximaci\'on de funciones}

\subsection{DatosAprxFunc} \label{sec:DatosAprxFunc}

La declaraci\'on de esta estructura es la siguiente:

\begin{verbatim}
typedef struct

        {
        int     intNMI;

        double  dblx0,
                dblSolucion,
                dblTol,
                dblError; 
        } DatosAprxFunc;
\end{verbatim}

El significado de cada uno de los miembros de esta estructura es el siguiente:

\begin{description}
\item[intNMI] n\'umero m\'aximo de iteraciones.
\item[dblx0] aproximaci\'on inicial.
\item[dblSolucion] aproximaci\'on final de la raiz.
\item[dblTol] tolerancia con la que se va a aproximar la raiz.
\item[dblError] error cometido al aproximar la raiz, valor absoluto de la 
distancia entre las dos \'ultimas aproximaciones.
\end{description}

Para declarar una variable de este tipo:

\begin{center}
\emph{\textbf{DatosAprxFunc} strVariable;}
\end{center}

\emph{strVariable} ser\'{\i}a una variable del tipo \textbf{DatosAprxFunc}.

\section{Estructuras de datos para matrices}

Estas estructuras de datos se utilizan para el uso de matrices.

\subsection{Matriz} \label{sec:strMatriz}

Esta estructura se utiliza para almacenar matrices.\newline

La declaraci\'on de esta estructura es la siguiente:

\begin{verbatim}
typedef struct

        {
        int     intFilas,
                intColumnas;

        double  **dblCoefi;
        } Matriz;
\end{verbatim}

El significado de cada uno de los miembros de esta estructura es el siguiente:

\begin{description}
\item[intFilas] n\'umero de filas de la matriz.
\item[intColumnas] n\'umero de columnas de la matriz.
\item[dblCoefi] matriz que contiene los coeficientes de la matriz que estamos
representando mediante esta estructura de datos.
\end{description}

Para declarar una variable de este tipo:

\begin{center}
\emph{\textbf{Matriz} strVariable;}
\end{center}

\emph{strVariable} ser\'{\i}a una variable del tipo \textbf{Matriz}.

\section{Estructuras de datos para n\'umeros complejos} \label{sec:complejos}

Estas estructuras de datos se utilizan para el uso de n\'umeros complejos.

\subsection{Complejo}
Esta estructura se utiliza para almacenar las coordenadas de un n\'umero
complejo, en coordenadas cartesianas.\newline

La declaraci\'on de esta \emph{estructura} es la siguiente:

\begin{verbatim}
typedef struct

        {
        double  dblReal,
                dblImag;
        } Complejo;
\end{verbatim}

El significado de cada uno de los miembros de esta estructura es el siguiente:

\begin{description}
\item[dblReal] parte real del n\'umero complejo.
\item[dblImag] parte imaginaria del n\'umero complejo.
\end{description}

Para declarar una variable de este tipo:

\begin{center}
\emph{\textbf{Complejo} strVariable;}
\end{center}

\emph{strVariable} ser\'{\i}a una variable del tipo \textbf{Complejo}.

\newpage

\subsection{Polar}
Esta estructura se utiliza para almacenar un n\'umero complejo en 
coordenadas polares.\newline

La declaraci\'on de esta \emph{estructura} es la siguiente:

\begin{verbatim}
typedef struct

        {
        double  dblMod,
                dblArg;	
        } Polar;
\end{verbatim}

El significado de cada uno de los miembros de esta estructura es el siguiente:

\begin{description}
\item[dblMod] m\'odulo del n\'umero complejo.
\item[dblArg] argumento del n\'umero complejo.
\end{description}

Para declarar una variable de este tipo:

\begin{center}
\emph{\textbf{Polar} strVariable;}
\end{center}

\emph{strVariable} ser\'{\i}a una variable del tipo \textbf{Polar}.



%
% TIPOMATRIZ.H
%

\include{tipomatriz}

%%%%%%%%%%%%%
% Apendices %
%%%%%%%%%%%%%

\appendix

%
% Apendice sobre metodos Runge - Kutta 
%

%
% Apendice sobre metodos Runge - Kutta 
%

\chapter{Runge-Kutta methods} \label{sec:Runge}

This appendix is intended to help to know how Runge-Kutta methods are implemented and used in this library.

\section{What is a Runge-Kutta method?}

\textbf{Runge-Kutta} methods are a family of numerical methods to approach solutions of ordinary differential equations (O.D.E). These methods are iterative methods used to solve ``\emph{initial problem value}'' (\textbf{I.P.V}) or ``\emph{Cauchy problem}''.\\

These methods are only-one-step methods with a fixed size for the method step\footnote{It is also possible to implement methods with a variable step known as \emph{embedding}.}.\\

\subsection{What is a I.V.P.?}

An \emph{I.V.P.} is:

\begin{equation} \label{eq:IVP}
\left\{ \begin{array}{l}
y' = f(x, y(x))\\
y(x_0) = y_0\\
\end{array} \right.
\end{equation}
%
So $y'$ is a function depending on the variable $x$, and the function $y(x)$. $y(x)$ is the solution of the equation \ref{eq:IVP} and the point $(x_0,y_0)$ belongs to the curve $y(x)$.\\

Solving the \emph{I.V.P.} \ref{eq:IVP} is finding a function $y(x)$ such as the equation \ref{eq:IVP} is met.\\

An example of a \emph{I.V.P.}:
%
\begin{equation} \label{eq:IVPej}
\left\{ \begin{array}{l}
y' = \frac{x * y(x) - y(x)^2}{x^2} \\
y(1) = 2 \\
\end{array} \right.
\end{equation}
%
The solution of the \ref{eq:IVPej} will be:
%
\begin{equation}
y(x) = \frac{x}{\frac{1}{2}+\ln x}
\end{equation}

\section{Runge-Kutta's method notation}

$y(x_i)$ will be the exact value of the function $y(x)$ evaluated in $x_i$.\\ \\
$y_i$ will be the approximation of the function $y(x)$ in the point $x_i$.\\ \\
$h$ is the step used by the method in each iteration.

\subsection{General formulation}

A $s$-stages \textbf{Runge-Kutta}'s method formulation is:
%
\begin{equation}
y_{n+1} = y_{n} + h \cdot \sum_{i=0}^{s-1} b_i \cdot k_i
\end{equation}
%
where:
%
\begin{equation}
k_i = f(x_n + c_i \cdot h, y_n + h \cdot \sum_{j=0}^{s-1} a_{i,j} \cdot k_j)
\end{equation}
%
satisfying:
%
\begin{equation}
\sum_{j=0}^{s-1} a_{i,j} = c_i
\end{equation}

\subsection{Matricial notation (Butcher's)}

Matricial notation is used to represent method's coeficients using a matrix.\\

For a $s$-stages \textbf{Runge-Kutta} method the matricial notation will be:
%
\begin{center}
\begin{displaymath}
\begin{array}{c|ccc}
c_0 & a_{0,0} & \cdots \cdots & a_{0,s-1} \\
\vdots & \vdots & & \vdots \\
\vdots & \vdots & & \vdots \\
c_{s-1} & a_{s-1,0} & \cdots \cdots & a_{s-1,s-1} \\
\hline
 & b_0 & \cdots \cdots & b_{s-1} \\
\end{array}
\end{displaymath}
\end{center}

\note{In section \ref{sec:biaButcherArray} is shown a data structure used to store the Butcher array.}

\section{Runge-Kutta types}

There are several types of \textbf{Runge-Kutta} methods.

\subsection{Implicit Runge-Kutta}

A \textbf{Runge-Kutta} method is said to be implicit when the $a_{i,j} \neq 0$ for some $j > i$.\\

The $2$-stages Gauss method is an implicit \textbf{Runge-Kutta} method of $2$-stages:
%
\begin{center}
\begin{displaymath}
\begin{array}{c|cc}
\frac{3-\sqrt 3}{6} & \frac{1}{4} & \frac{3-2*\sqrt 3}{12} \\
\frac{3+\sqrt 3}{6} & \frac{3+2*\sqrt 3}{12} &\frac{1}{4} \\
\hline
 & \frac{1}{2} & \frac{1}{2}
\end{array}
\end{displaymath}
\end{center}

\subsection{Semi-implicit Runge-Kutta}

A \textbf{Runge-Kutta} method is said to be semi-implicit when the $a_{i,j} = 0$ when $j > i$.\\

A $2$-stages semi-implicit \textbf{Runge-Kutta} method:
%
\begin{center}
\begin{displaymath}
\begin{array}{c|cc}
\frac{3+\sqrt 3}{6} & \frac{3+\sqrt 3}{6} & 0 \\
\frac{3-\sqrt 3}{6} & \frac{-\sqrt 3}{3} & \frac {3+\sqrt 3}{6} \\
\hline
 & \frac{1}{2} & \frac{1}{2}
\end{array}
\end{displaymath}
\end{center}

\subsection{Explicit Runge-Kutta}

A \textbf{Runge-Kutta} method is said to be explicit when the $a_{i,j} = 0$ when $j \geq i$.\\

A $4$-stages explicit \textbf{Runge-Kutta} method also known as ``\textbf{classic Runge-Kutta}'':
%
\begin{center}
\begin{displaymath}
\begin{array}{c|cccc}
0 & 0 \\
\frac{1}{2} & \frac{1}{2} & 0 \\
\frac{1}{2} & 0 & \frac{1}{2} & 0 \\
1 & 0 & 0 & 1 & 0 \\
\hline
 & \frac{1}{6} & \frac{1}{3} & \frac{1}{3} & \frac{1}{6}
\end{array}
\end{displaymath}
\end{center}


\end{document}
