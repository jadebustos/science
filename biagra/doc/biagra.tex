\documentclass[a4paper,12pt,twoside,openright]{report}

%\addtolength{\voffset}{-4cm}
\usepackage[english]{babel}
\usepackage[colorlinks = true,
            linkcolor = blue,
            urlcolor  = blue,
            citecolor = blue,
            anchorcolor = blue]{hyperref}
\usepackage{amssymb}
\usepackage{draftwatermark}
\usepackage{placeins}
\usepackage{balloons}
\usepackage{fancyhdr}

%\pagestyle{headings}

\fancyhf{}
\pagestyle{fancy}
\fancyhead [L] {Scientific programming library} %Encabezado
\fancyhead [R]{Version \ver}
\fancyfoot[R]{\thepage} %Pie de página
\fancyfoot[L]{\BI}


\SetWatermarkText{Draft}
\SetWatermarkScale{5}
\SetWatermarkColor{blue}

\author{Jos\'e Angel de Bustos P\'erez}

\newcommand{\BI}{\emph{B.I.A.G.R.A} }
\newcommand{\ver}{1.0}
\newcommand{\me}
	{\emph{Jos\'e Angel de Bustos P\'erez}}

\frenchspacing

\hyphenation{si-guien-te apro-xi-ma-cio-nes FORTRAN double intGrado 
dblDerivada dblResultado intNMI intOrden dblMatriz Newton intResultado
cabezera coor-de-na-das de-pen-dien-do di-men-sio-na-do re-pre-sen-tan-do par-ti-cu-lar a-ppro-xi-ma-tion me-mo-ry po-ly-no-mials e-ve-ry}

\begin{document}

\thispagestyle{empty}

\textbf{
\begin{center}
\Huge{B.I.A.G.R.A.} \\[.75cm]
%\end{center}
\LARGE BI\Large bliotec\LARGE A \Large de pro\LARGE GR\Large amaci\'on 
cient\'{\i}fic\LARGE A\\[5cm]
\end{center}
}

\large 
\begin{flushright}
\me \\
$<$jadebustos@gmail.com$>$\\ \ \\ 
Version \ver, \today .\\
\textbf{\LaTeXe}
\end{flushright}

\normalsize

\tableofcontents

\part{Introduction}

%
% INTRODUCTION
%

%
% INTRODUCCION
%

\chapter{Introducci\'on}
En este primer cap\'{\i}tulo vamos a dar una introducci\'on sobre:

\begin{itemize}
\item Predicci\'on de ``\emph{Series Temporales}''.
\item ``\emph{Redes Neuronales}''.
\end{itemize}

%
% SERIES TEMPORALES
%

%
% SERIES TEMPORALES
%

\section{Series Temporales}

Las ``\emph{Series Temporales}'' tambi\'en reciben el nombre de ``\emph{Series
Cronol\'ogicas}'' o ``\emph{Series Hist\'oricas}''. Se basan en el estudio del
comportamiento de una determinada variable, observaciones, con el fin de poder
predecir el comportamiento de dicha variable a lo largo del tiempo.

\subsection{?`Para que sirven?}

Intentaremos crear un modelo matem\'atico para predecir la evoluci\'on de una
variable a lo largo del tiempo. Esta variable puede ser de cualquier tipo:

\begin{itemize}
\item \textbf{Econ\'omica}: \emph{\'{\i}ndice de precios al consumo},
\emph{demanda de electricidad}, \emph{producto nacional bruto}, \emph{ventas
de un determinado tipo}, \ldots
\item \textbf{F\'{\i}sica}: \emph{velocidad del viento}, \emph{temperatura en
un proceso}, \emph{concentraci\'on en la atm\'osfera de un contaminante}, \ldots
\item \textbf{Social}: \emph{n\'umero de nacimientos}, \emph{n\'umero de
defunciones}, \emph{votos a un partido pol\'{\i}tico}, \ldots 
\end{itemize}

\newpage

\subsection{?`Que se necesita?}

Necesitaremos disponer de datos sobre la evoluci\'on de la variable que nos 
interesa a intervalos regulares de tiempo (horas, d\'{\i}as, meses, a\~nos,
\ldots). Utilizando estos datos intentaremos ``adivinar'' el comportamiento de
dicha variable en el futuro, suponiendo que el comportamiento de la variable es
uniforme a lo largo del tiempo, es decir que su comportamiento en el
futuro\footnote{O al menos en un futuro no muy lejano.} es similar al 
comportamiento que tuvo dicha variable en el pasado. Es decir que su
comportamiento no experimenta cambios bruscos entre dos espacios de tiempo
proximos.\\

Este tipo de analisis recibe el nombre de \emph{an\'alisis univariante}, el cual
es util para predicciones a corto plazo. Si necesitaramos hacer predicciones a
medio o largo plazo necesitariamos tener en cuenta una serie de variables
relacionadas con la variable que queremos estudiar. Este tipo de an\'alisis
recibe el nombre de \emph{an\'alisis multivariante}.
 
\subsection{Modelo matem\'atico}

El modelo matem\'atico para una \emph{serie temporal} es el concepto de
``\emph{proceso estoc\'astico}''. Supondremos que el valor observado de la serie
en un instante $t$ es una extracci\'on al azar de una variable aleatoria
definida en dicho instante $t$.\\

Una serie de $n$ datos ser\'a una muestra de un vector de $n$ variables
aleatorias ordenadas en el tiempo $(z_1,\ldots,z_n)$. Se denomina
\emph{proceso estoc\'astico} al conjunto de estas variables $\{z_t\}_{t=1}^n$,
y la serie observada se considera una realizaci\'on o trayectoria del proceso.\\

Resumiendo, una ``\emph{serie temporal}'' es una sucesi\'on de valores
observados de una variable en diferentes instantes de tiempo, en la que las
observaciones aparecen ordenadas cronol\'ogicamente. Los instantes de tiempo en
los que se toman las mediciones de la variable suelen ser intervalos regulares
de tiempo.\\

Hay ``\emph{series temporales}'' que tambi\'en reciben el nombre de
``\emph{series aleatorias}'' ya que son pr\'actiamente impredecibles, como por
ejemplo pueden ser:
\begin{itemize}
\item Los n\'umeros premiados en la Loter\'{\i}a Nacional.
\item El n\'umero resultante del lanzamiento de un dado equilibrado.
\end{itemize}

Sin embargo tambi\'en existen ``\emph{series temporales}'' cuyo comportamiento
es tan regular que se pueden hacer predicciones muy precisas:
\begin{itemize}
\item La posici\'on de los astros.
\item El horario de las mareas.
\end{itemize}

En un lugar intermedio, entre estos dos tipos de ``\emph{series temporales}'',
tenemos otro tipo de series que en su comportamiento tienen dos componentes,
una componente ``\emph{regular}'' y otra ``\emph{irregular}''. Un ejemplo de 
estas series pueden ser las ``\emph{series temporales econ\'omicas}''%
\footnote{Si no todas, al menos la mayor\'{\i}a.}.

\subsection{Procesos estoc\'asticos}

Un ``\emph{proceso estoc\'astico}'', $\{X_t\}$, es una colecci\'on de variables
aleatorias, $X_t$, ordenadas seg\'un un par\'ametro discreto, $t$, que para 
nosotros ser\'a el tiempo.\\

Como hemos observado antes los modelos estoc\'asticos de series temporales
conciben una serie temporal dada como una colecci\'on de observaciones
muestrales, cada una de ellas correspondiente a una variable del proceso.

\subsection{Definiciones sobre procesos estoc\'asticos}

\begin{definicion}[Funci\'on de medias]\label{def:funciondemedias}
Llamaremos ``funci\'on de medias'' del proceso a una funci\'on, dependiente del
tiempo, que proporciona las medias de las distribuciones marginales $z_t$ para
cada instante $t$:
\begin{displaymath}
\mu_t = E[z_t]
\end{displaymath}
\end{definicion}
Si todas la variables tienen la misma media, entonces la ``\emph{funci\'on de
medias}'' es constante y diremos que el proceso es \emph{estable} en la media.

\begin{definicion}[Funci\'on de varianzas]\label{def:funciondevarianzas}
Llamaremos ``funci\'on de varianzas'' del proceso a una funci\'on, dependiente
del tiempo, que proporciona las varianzas en cada instante $t$:
\begin{displaymath}
\sigma^2_t = Var\ (z_t)
\end{displaymath}
\end{definicion}
Si esta funci\'on es constante en el tiempo diremos que el proceso es
\emph{estable} en la varianza.

\begin{definicion}[Funci\'on de autocovarianzas]\label{def:fdeautocovarianzas}
Llamaremos ``funci\'on de autocorrelaci\'on'' del proceso a la funci\'on que
describe las covarianzas en dos instantes cualesquiera:
\begin{displaymath}
Cov\ (t,t+j)=Cov\ (z_t,z_{t+j}) = E\ [(z_t-\mu_t)(z_{t+j}-\mu_{t+j})]
\end{displaymath}
\end{definicion}

\begin{definicion}[Funci\'on de autocorrelaci\'on]\label{def:fdeautocorrelacion}
Llamaremos ``funci\'on de autocorrelaci\'on'' del proceso a la estandarizaci\'on
de la funci\'on de covarianzas:
\begin{displaymath}
\varrho_{(t,t+j)}=\frac{Cov\ (t,t+j)}{\sigma_t,\sigma_{t+j}}
\end{displaymath}
\end{definicion}
%
Estas dos \'ultimas funciones dependen de dos par\'ametros, $t$ y $j$, donde:
\begin{itemize}
\item $t$ es el instante inicial.
\item $j$ es el intervalo entre observaciones.
\end{itemize}
%
En muchos fen\'omenos din\'amicos se observa una condici\'on de estabilidad en
la cual la dependencia entre dos observaciones s\'olo depende del intervalo
entre ellas y no del origen considerado, es decir:
\begin{displaymath}
Cov\ (t_1,t_{1+k})=Cov\ (t_2,t_{2+k}) = \gamma_k\qquad \forall k
\end{displaymath}
es decir la relaci\'on entre $z_t$ y $z_{t+j}$ es siempre igual a la relaci\'on
entre $z_t$ y $z_{t-j}$.

\subsection{Procesos estacionarios}

Diremos que un proceso estoc\'astico (serie temporal) es estacionario en
``sentido d\'ebil'' si existen y son estables la media, la varianza y las
covarianzas, es decir, si para todo $t$:
\begin{enumerate}
\item $\mu_t$ = $\mu$ = cte
\item $\sigma^2_t$ = $\sigma^2$ = cte
\item $Cov\ (t,t+k)=Cov\ (t,t-k)=\gamma_k$ $\forall k$
\end{enumerate}
%
Teniendo en cuenta que $\gamma_0=\sigma^2$ la ``funci\'on de autocorrelaci\'on''
para un proceso estacionario es:
\begin{displaymath}
\varrho_{(t,t+k)} = \varrho_k = \frac{\gamma_k}{\gamma_0}
\end{displaymath}
adem\'as se verifica que $\varrho_{-k}=\varrho_{k}$.\\

Llamaremos ``\emph{funci\'on de autocorrelaci\'on simple}''(fas) o
``\emph{correlograma}'' a la representaci\'on de los coeficientes de
autocorrelaci\'on en funci\'on del retardo.\\

Si la dependencia entre observaciones tiende a cero al aumentar el retardo
diremos que el proceso es ``\emph{erg\'odico}''. En lo sucesivo supondremos
que los procesos estacionarios siempre son ``\emph{erg\'odicos}''.

\subsection{Procesos estacionales}

Que un proceso sea estacional implica que la serie temporal que lo representa
sigue una pauta regular de comportamiento peri\'odico.\\

Si la serie es mensual, por ejemplo, diremos que existir\'a estacionalidad si
los eneros tienden a ser similares en distintos a\~nos, y lo mismo con el resto
de meses. La estacionalidad de una serie la hace no estacionaria, ya que el
valor medio $\mu$ variar\'a de unos meses a otros.\\

\subsection{Procesos de ruido blanco}

Un proceso estacionario muy importante es el definido por:
\begin{enumerate}
\item $E\ [z_t] = 0$
\item $Var\ (z_t) = \sigma^2$
\item $Cov\ (z_t,z_{t-k})=0$ $\forall \ k$
\end{enumerate}
este proceso recibe el nombre de ``\emph{proceso de ruido}''. Si todas las
variables de un proceso de este tipo tienen una distribuci\'on normal entonces
diremos que es un proceso de ``\emph{ruido blanco}''.

\newpage


%
% REDES NEURONALES
%

%
% REDES NEURONALES 
%

\section{Redes Neuronales}

\subsection{?`Que son?}

Las ``\emph{Redes Neuronales}'' surgieron como consecuencia del intento de
modelizar el funcionamiento del cerebro humano, y en particular su capacidad de
aprendizaje, para poder dotar de ``inteligencia'' a los ordenadores.\\

Estas redes son modelos simplificados de las redes de neuronas que forman el
cerebro y tambi\'en reciben el nombre de ``\textbf{RNA}''\footnote{Redes
Neuronales Artificiales.} y al igual que el modelo biol\'ogico intentan 
``\emph{aprender}'' de los datos de los que le son suministrados.

\subsection{?`Para que sirven?}

Las \textbf{RNA} se utilizan para resolver aquellos problemas en los que su
complejidad hace casi imposible su resoluci\'on mediante las t\'ecnicas
habituales de programaci\'on, ya que el algoritmo utilizado para su
resoluci\'on es altamente complejo y dificilmente programable mediante los
sistemas de computaci\'on secuenciales.\\

Un ejemplo de estos problemas son los ``\emph{Sistemas de Reconocimiento Optico
de Caracteres}''\footnote{O.C.R.}. Estos sistemas tratan, basicamente, de que
un ordenador reconozca las letras y n\'umeros del abecedario a partir de una
imagen digital(reconocimiento de patrones).

\subsection{Historia}

Las primeras investigaciones en este campo empezaron en los a\~nos $40$ con los
trabajos de \emph{McCulloh} y \emph{Pitts}. Posteriormente \emph{Donald O. Hebb}
estableci\'o una conexi\'on entre la psicolog\'{\i}a y  f\'{\i}siologia en la
organizaci\'on del comportamiento, proponiendo un m\'etodo de aprendizaje que se
sigue aplicando en muchas redes neuronales actualmente como una de las reglas
m\'as potentes.
%
\newpage
%
Posteriormente \emph{Bernard Widrow} y \emph{Marcian Hoff} desarrollaron un
algoritmo de red adaptativa basado en un modelo simple de neurona al que
llamaron \textbf{Adaline} y \emph{Frank Rosenblantt} desarrollo el
\textbf{Perceptron} que era capaz de clasificar patrones contaminados con
ruido a la entrada\footnote{La informaci\'on original se deteriora antes de ser
procesada.} y fue inicialmente aplicado a ``O.C.R.''.\\

En $1.969$ hay un declive en la investigaci\'on sobre \textbf{RNA} a partir de
la publicaci\'on de un trabajo de \emph{Minsky} y \emph{Papert} que demostraba
que el \textbf{Perceptron} no era capaz de resolver problemas no lineales
simples, como la funci\'on l\'ogica \emph{XOR}\footnote{Una red de perceptrones
s\'{\i} era capaz de resolver el problema \emph{XOR} pero no existian
algoritmos que pudieran entrenar todos los casos.}.

En los a\~nos $80$ el campo de la computaci\'on neuronal resurge como
consecuencia de los trabajos de:

\begin{itemize}
\item \textbf{Kohonen}, que aport\'o trabajos sobre memorias adaptativas y
aprendizaje competitivo, quedando como recompensa la llamada \emph{red de
Kohonen}, en honor al investigador.
\item \textbf{Grossberg}, que trabaj\'o sobre dise\~no y construcci\'on de
modelos neuronales, introduciendo la funci\'on de transferencia de tipo
\emph{sigmoide} con la que era posible establecer valores de activaci\'on
reales y acotados. A partir de estos postulados se desarroll\'o su
\emph{Teor\'{\i}a de la Resonancia Adaptativa (ART)}.
\item \textbf{Hopfield}, que desarroll\'o un modelo que consist\'{\i}a en un
sistema de elementos totalmente interconectados y que buscaba la situaci\'on de
m\'{\i}nima energ\'{\i}a.
\item \textbf{Rumelhart}, que desarroll\'o el algoritmo de aprendizaje
``\emph{Backpropagation}'' o propagaci\'on del error hacia atr\'as.
\end{itemize}
%
\newpage
%
\subsection{Caracter\'{\i}sticas de las RNA}

Las \textbf{RNA} presentan una serie de caracter\'{\i}sticas que son:

\begin{description}
\item[Paralelismo:] debido a su dise\~no son entidades altamente paralelas, ya
que cada elemento de la red puede funcionar independientemente del resto (dentro
de una misma capa).
\item[Adaptabilidad:] las \textbf{RNA} son capaces de autoorganizarse y
aprender. Las redes no son programadas como los ordenadores tradicionales, sino
que son entrenadas a trav\'es de la repetida presentaci\'on de ejemplos.
\item[Aprendizaje adaptativo:] consigue el conocimiento en base a una etapa
caracter\'{\i}stica que es el aprendizaje, que se basa en el entrenamiento.
\item[Autoorganizaci\'on:] En base a ese entrenamiento la red va a crearse su
propia representaci\'on de todo el dominio de la informaci\'on que se le puede
dar. Es decir, aunque no se la entrene para todas las posibles entradas que le
pueden llegar es capaz de responder a todas ellas de una manera coherente.
\item[Reconocimiento de patrones:] las \textbf{RNA} tienen la ``habilidad'' de
clasificar patrones. Dado un patr\'on de entrada encuentra el patr\'on de
salida asociado.
\item[Reconstrucci\'on de patrones:] la red toma un patr\'on de entrada
incompleto\footnote{Del que se ha perdido informaci\'on.} y es capaz de
insertar la informaci\'on perdida en el patr\'on recuperando el mejor patr\'on
asociado con la entrada.
\item[Tolerancia a fallos:] la destrucci\'on o eliminaci\'on de elementos de
proceso de una red no causar\'a que la red falle e forma global. Dado que la
informaci\'on en una red es distribuida, peque\~nas porciones de informaci\'on
pueden perderse sin afectar seriamente al funcionamiento de la red.
\end{description}

\begin{ejemplo}[RNA aplicada al O.C.R.] \label{ej:OCRSimple}\ \\
Podemos emplear una \textbf{RNA} para el reconocimiento de
patrones. Supongamos que utilizamos una matriz rectangular para la
representaci\'on de las diferentes letras del abecedario. La matriz tendr\'a
dimensi\'on $7\times 5$.
%
\newpage
%
\begin{figure}[!ht]
\begin{displaymath}
\begin{tabular}{ccccc}
0&1&1&1&0\\
1&0&0&0&1\\
1&0&0&0&1\\
1&1&1&1&1\\
1&0&0&0&1\\
1&0&0&0&1\\
1&0&0&0&1
\end{tabular}
\end{displaymath}
\caption{Ideal de la letra A} \label{fig:IdealA}
\end{figure}
\begin{figure}[!ht]
\begin{displaymath}
\begin{tabular}{ccccc}
1&1&1&1&0\\
1&0&0&0&1\\
1&0&0&0&1\\
1&1&1&1&1\\
1&0&0&0&1\\
1&0&0&0&1\\
1&0&0&0&1
\end{tabular}
\end{displaymath}
\caption{Letra A borrosa} \label{fig:ABorrosa}
\end{figure}

Podemos entrenar una \textbf{RNA} para que reconozca la entrada mostrada en la
figura $\ref{fig:IdealA}$ como la letra \emph{A}. Pero por alg\'un motivo puede
ocurrir que se suministre la figura $\ref{fig:ABorrosa}$ como letra \emph{A},
puede ocurrir un error en la recepci\'on de los datos, o simplemente una
persona diferente a nosotros puede tener ``otro ideal de letra
\emph{A}''\footnote{Cada persona tiene un tipo de escritura propio.}. Mediante
el entrenamiento podemos lograr que una \textbf{RNA} reconozca el ``patr\'on''
de la letra \emph{A} en la entrada mostrada en la figura $\ref{fig:ABorrosa}$ y
lo asocie a la letra \emph{A}.
\end{ejemplo}

\subsection{Elementos de una RNA}

Una \textbf{RNA} consta de una serie de elementos interconectados entre s\'{\i}
a los que denominaremos ``EP'', los cuales son elementos simples de procesado,
neuronas. Las conexiones entre ``EP's'' reciben el nombre de
``\emph{sinapsis}'', las cuales ir\'an ponderadas con unos determinados valores
llamados ``\emph{pesos sin\'apticos}'' que var\'{\i}aran a medida que la red
aprende, por lo cual necesitaremos de un algoritmo\footnote{Algoritmo de
aprendizaje de la red.} que permita saber c\'omo y
cu\'anto deben variar estos.
\subsection{Elementos de proceso o EP's}

Los ``EP's'' son los elementos individuales de una \textbf{RNA} y los podemos
dividir en tres tipos:

\begin{itemize}
\item \textbf{Entrada:} reciben datos del ``mundo exterior'' para procesarlos.
\item \textbf{Salida:} emiten los datos procesados.
\item \textbf{Ocultos:} procesan los datos recibidos del ``mundo exterior'' o
de otros ``EP's'' ocultos y emiten la informaci\'on generada a otros ``EP's''
ocultos o a los de salida.
\end{itemize}

Todo ``EP'' tiene, como m\'{\i}nimo, una entrada y una salida la cual puede ser
aplicada a las entradas de otros ``EP's''(incluido el mismo EP).\\

Como hemos dicho antes todos los ``EP's'' que forman una \textbf{RNA} est\'an
interconectados entre s\'{\i}. Esta interconexi\'on permite transmitir los
datos de salida de un ``EP'', en forma de un n\'umero real, a todos los ``EP's''
con los que se encuentre conectado.

\begin{figure}[!ht]
\setlength{\unitlength}{3947sp}%
%
\begingroup\makeatletter\ifx\SetFigFont\undefined%
\gdef\SetFigFont#1#2#3#4#5{%
  \reset@font\fontsize{#1}{#2pt}%
  \fontfamily{#3}\fontseries{#4}\fontshape{#5}%
  \selectfont}%
\fi\endgroup%
\begin{picture}(6624,3928)(589,-3854)
\thinlines
\put(1576,-1261){\circle{670}}
\put(1576,-2161){\circle{670}}
\put(1576,-3061){\circle{670}}
\put(3976,-811){\circle{670}}
\put(3976,-1711){\circle{670}}
\put(3976,-2611){\circle{670}}
\put(3976,-3511){\circle{670}}
\put(6376,-1711){\circle{670}}
\put(6376,-2611){\circle{670}}
\put(1951,-1261){\vector( 4, 1){1641.177}}
\put(1951,-1261){\vector( 4,-1){1658.823}}
\put(1951,-1261){\vector( 4,-3){1704}}
\put(3601,-961){\makebox(1.6667,11.6667){\SetFigFont{5}{6}{\rmdefault}{\mddefault}{\updefault}.}}
\put(1951,-2161){\vector( 4, 1){1641.177}}
\put(1951,-2161){\vector( 4,-1){1658.823}}
\put(1951,-2161){\vector( 4, 3){1632}}
\put(1951,-3136){\vector( 3, 1){1642.500}}
\put(1951,-3136){\vector( 4, 3){1668}}
\put(1951,-3136){\vector( 3, 4){1602}}
\put(1951,-1261){\vector( 3,-4){1629}}
\put(1951,-2161){\vector( 4,-3){1656}}
\put(1951,-3136){\vector( 4,-1){1658.823}}
\put(4351,-811){\vector( 2,-1){1650}}
\put(4351,-811){\vector( 1,-1){1687.500}}
\put(4351,-1786){\vector( 1, 0){1575}}
\put(4351,-1786){\makebox(1.6667,11.6667){\SetFigFont{5}{6}{\rmdefault}{\mddefault}{\updefault}.}}
\put(4351,-1786){\vector( 2,-1){1590}}
\put(4351,-2611){\vector( 2, 1){1590}}
\put(4351,-2611){\vector( 1, 0){1575}}
\put(4351,-3511){\vector( 1, 1){1575}}
\put(4351,-3511){\vector( 2, 1){1560}}
\put(601,-1261){\vector( 1, 0){600}}
\put(601,-2161){\vector( 1, 0){600}}
\put(601,-3061){\vector( 1, 0){600}}
\put(6751,-1711){\vector( 1, 0){450}}
\put(6751,-2611){\vector( 1, 0){450}}
\put(1501,-1336){\makebox(0,0)[lb]{\smash{\SetFigFont{12}{14.4}{\rmdefault}{\mddefault}{\updefault}$1$}}}
\put(1501,-2236){\makebox(0,0)[lb]{\smash{\SetFigFont{12}{14.4}{\rmdefault}{\mddefault}{\updefault}$2$}}}
\put(1501,-3136){\makebox(0,0)[lb]{\smash{\SetFigFont{12}{14.4}{\rmdefault}{\mddefault}{\updefault}$3$}}}
\put(3901,-886){\makebox(0,0)[lb]{\smash{\SetFigFont{12}{14.4}{\rmdefault}{\mddefault}{\updefault}$4$}}}
\put(3901,-1786){\makebox(0,0)[lb]{\smash{\SetFigFont{12}{14.4}{\rmdefault}{\mddefault}{\updefault}$5$}}}
\put(3901,-2686){\makebox(0,0)[lb]{\smash{\SetFigFont{12}{14.4}{\rmdefault}{\mddefault}{\updefault}$6$}}}
\put(3901,-3586){\makebox(0,0)[lb]{\smash{\SetFigFont{12}{14.4}{\rmdefault}{\mddefault}{\updefault}$7$}}}
\put(6301,-1786){\makebox(0,0)[lb]{\smash{\SetFigFont{12}{14.4}{\rmdefault}{\mddefault}{\updefault}$8$}}}
\put(6301,-2686){\makebox(0,0)[lb]{\smash{\SetFigFont{12}{14.4}{\rmdefault}{\mddefault}{\updefault}$9$}}}
\put(901,-61){\makebox(0,0)[lb]{\smash{\SetFigFont{12}{14.4}{\rmdefault}{\mddefault}{\updefault}EP's de Entrada}}}
\put(3301,-61){\makebox(0,0)[lb]{\smash{\SetFigFont{12}{14.4}{\rmdefault}{\mddefault}{\updefault}EP's Ocultos}}}
\put(5851,-61){\makebox(0,0)[lb]{\smash{\SetFigFont{12}{14.4}{\rmdefault}{\mddefault}{\updefault}EP's de Salida}}}
\end{picture}

\caption{Ejemplo de RNA} \label{fig:RNA}
\end{figure}

\begin{ejemplo}[RNA aplicada al O.C.R.]\ \\
Siguiendo con el ejemplo $\ref{ej:OCRSimple}$, en la p\'agina
$\pageref{ej:OCRSimple}$, como cada letra esta representada por una matriz de
dimensi\'on $7\times 5$ entonces tendremos $7\times 5 = 35$ ``EP's'' de entrada
y tendremos tantos ``EP's'' como letras queramos reconocer con nuestra
\textbf{RNA}, por ejemplo $26$ para reconocer el siguiente conjunto de
caracteres:
\begin{center}
A B C D E F G H I J K L M N \~N O P Q R S T U V X Y Z
\end{center}
Como cada caracter est\'a representado por una matriz de $7\times 5$, si
ponemos todas las filas de la matriz seguidas tendremos $35$ digitos, cada uno
asociado con un ``EP'' de entrada. As\'{\i} mismo como el conjunto de
caracteres a reconocer es de $26$ tendremos $26$ ``EP's'' de salida, cada
``EP'' estar\'a asociado a un caracter del conjunto.
\end{ejemplo}

En algunas \textbf{RNA} aparece un ``EP'' especial, denominado \emph{bias} por
los inform\'aticos, el cual no es m\'as que un ``EP'' con salida constante que
est\'a conectado a todas la neuronas de una capa. En lo sucesivo supondremos que
dicho ``EP'' no existe o que tiene salida nula.
%
\newpage
%
\subsection{Partes de un EP}

Como hemos dicho antes una \textbf{RNA} esta formada por un conjunto de
``EP's'' interconectados entre s\'{\i}. Al estar, todos ellos, interconectados
entre s\'{\i} tiene que haber una forma en la que la informaci\'on procesada
por un ``EP'' se transmita a los dem\'as para seguir siendo procesada. De esto
se encargan las diferentes partes de un ``EP''.\\ \\
En un ``EP'' podemos encontrar:

\begin{itemize}
\item \textbf{Pesos Sin\'apticos}, valores asociados a la conexi\'on de un
``EP'' con otros ``EP's''. Estos valores se van modificando a medida que la
\textbf{RNA} va aprendiendo, mediante entrenamiento, para que de esta forma la
\textbf{RNA} se comporte adecuadamente.
\item \textbf{Entrada de Red o Entrada Total}, es una suma ponderada de los
valores de entrada al ``EP'', procedentes de los ``EP's'' conectados con \'el,
y de los pesos sin\'apticos asociados con dicho ``EP''.
\item \textbf{Funci\'on de Activaci\'on}, esta funci\'on nos va a indicar si el
``PE'' est\'a activo o no, y depende de la \emph{entrada de red} y/o \emph{valor
de la funci\'on de activaci\'on en el instante anterior}.
\item \textbf{Funci\'on de Salida}, en esta funci\'on se procesa el valor
obtenido en la \emph{funci\'on de activaci\'on} y el valor obtenido es el valor
de salida de la neurona, es decir, la informaci\'on que suministra a otras
neuronas. Normalmente se utiliza como \emph{funci\'on de salida} la funci\'on
identidad, con lo cual el valor devuelto por la \emph{funci\'on de activaci\'on}
es la informaci\'on mandada a otras neuronas. En lo sucesivo supondremos que
esta funci\'on es la funci\'on identidad, salvo que expresamente se diga lo
contrario.
\end{itemize}
Un ejemplo de esto aplicado a la capa oculta de la \textbf{RNA} que aparece en
la figura $\ref{fig:RNA}$ de la p\'agina $\pageref{fig:RNA}$:

\subsubsection{Pesos sin\'apticos para la RNA de la figura $\ref{fig:RNA}$}
\begin{center}
\begin{tabular}{|c|c|c|}
\hline
EP&EP's que suministran informaci\'on al EP &Pesos sin\'apticos\\
\hline
$4$&$1$, $2$ y $3$& $w_{1,4}$, $w_{2,4}$ y $w_{3,4}$\\
\hline
$5$&$1$, $2$ y $3$& $w_{1,5}$, $w_{2,5}$ y $w_{3,5}$\\
\hline
$6$&$1$, $2$ y $3$& $w_{1,6}$, $w_{2,6}$ y $w_{3,6}$\\
\hline
$7$&$1$, $2$ y $3$& $w_{1,7}$, $w_{2,7}$ y $w_{3,7}$\\
\hline
\end{tabular}
\end{center}
Los pesos asociados a cada ``EP'' se suelen determinar experimentalmente
mediante el ``entrenamiento'' de la red. El ``entrenamiento'' se suele realizar
introduciendo en la \textbf{RNA} las posibles entradas que se pueden presentar 
en la pr\'actica, o al menos un subconjunto significativo de las mismas, y
comparar la salida obtenida con la salida deseada y modificar los pesos mediante
alg\'un algoritmo de tal forma que el error cometido por la \textbf{RNA} sea
nulo o practicamente nulo.

\subsubsection{Entrada de red para la \textbf{RNA} de la figura $\ref{fig:RNA}$}

La \emph{entrada de red}, en el instante $t$, de los diferentes ``EP's'' es:

\begin{displaymath}
s_j(t)=x_1(t)\cdot w_{1,j}+x_2(t)\cdot w_{2,j}+x_3(t)\cdot w_{3,j} =\sum_{i=1}^3
x_i(t)\cdot w_{i,j}
\end{displaymath}
para $j=4,5,6,7$. Donde $x_i(t)$, $i=1,2,3$, es la entrada desde el $i$-esimo
``EP'', o lo que es lo mismo la salida del $i$-esimo ``EP''.

\subsubsection{Funci\'on de activaci\'on para la \textbf{RNA} de la figura
$\ref{fig:RNA}$}

La \emph{funci\'on de activaci\'on} no tiene que ser la misma para todos los
``EP's'' de la \textbf{RNA} y depender\'a del tipo de problema que estemos
resolviendo. Denotemos como $F_j$ a la funci\'on de activaci\'on para el
$j$-esimo ``EP''. Normalmente la activaci\'on, $A_j$, para el $j$-esimo ``EP''
en el momento depende de la \emph{entrada de red} en el momento $t$, es decir:

\begin{displaymath}
A_j(t+1) = F_j(s_j(t))
\end{displaymath}
Tambi\'en se puede dar el caso que la activaci\'on en el momento $t+1$ dependa
de la activaci\'on en el momento anterior $t$:

\begin{displaymath}
A_j(t+1)=F_j(A_j(t),s_j(t))
\end{displaymath}
A las \emph{funciones de activaci\'on} tambi\'en se las conoce con el nombre de 
\emph{funciones de transferencia}.\\

La \emph{funci\'on log\'{\i}stica} es comunmente utilizada como \emph{funci\'on de activaci\'on}.
\begin{displaymath}
F_j(t)=\frac{1}{1+e^{-t}}
\end{displaymath}
Esta funci\'on toma todos los valores comprendidos entre $0$ y $1$.\\

En algunos casos se tiene definido un valor ``\emph{umbral}'', $\theta$, y
dependiendo de si la ``entrada de red'' es mayor o no que este valor el ``EP''
correspondiente se activar\'a o no. En caso de activarse mandar\'{\i}a la 
se\~nal $1$ y en caso contrario la se\~nal $0$(permanecer\'{\i}a inactiva).

\subsection{Notaci\'on para redes neuronales}

En lo sucesivo utilizaremos la siguiente notaci\'on para una red neuronal:
\begin{itemize}
\item $n_s$ n\'umero de neuronas en la capa $s$ de la red neuronal.
\item $f(x)$ funci\'on de activaci\'on para la red neuronal.
\item $w_{j,i}[s]$ peso sin\'aptico entre la neurona $i$-\'esima de la capa
$(s-1)$ con la neurona $j$-\'esima en la capa $s$.
\item $x_j[s](p)$ salida de la $j$-\'esima neurona de la capa $s$ y patr\'on
$p$.
\begin{displaymath}
x_j[s](p) = f(I_j[s](p))
\end{displaymath}
\item $I_j[s](p)$ entrada total de red de la neurona $j$-\'esima en la capa $s$
y patr\'on $p$.
\begin{displaymath}
I_j[s](p) = \sum_{i=1}^{n_{(s-1)}} w_{j,i}[s]\cdot x_i[s-1](p)
\end{displaymath}
\end{itemize}

\subsection{Redes Feedforward}

En este tipo de redes las salidas de los ``EP's'' de una capa unicamente son
entrada de los ``EP's'' de la capa siguiente. Normalmente este tipo de redes
son las m\'as r\'apidas de todas y representan sistemas lineales. \\

Podemos ver un ejemplo de una red de este tipo en la figura $\ref{fig:RNA}$ 
situada en la p\'agina $\pageref{fig:RNA}$.

\subsection{Redes Feedback} 

En este tipo de redes las salidas de los ``EP's'' de una capa pueden ser la
entrada de los ``EP's'' de las capas anteriores. Este tipo de redes representa
sistemas no lineales.

\subsection{Redes Feedlateral}

En este tipo de redes las salidas de los ``EP's'' de una capa pueden ser la 
entrada de los ``EP's'' situados en la misma capa. Este tipo de redes representa
sistemas no lineales.

\subsection{Redes Recurrentes}

En este tipo de redes pueden existir lazos cerrados, es decir, un ``EP'' le
manda informaci\'on a otro ``EP'' y este \'ultimo despu\'es de procesar la
informaci\'on le manda informaci\'on al primer ``EP''.

\subsection{El aprendizaje en las RNA}

Basicamente el aprendizaje en las Redes Neuronales consiste en encontrar los
``pesos sin\'apticos'' adecuados para que la red pueda realizar de una forma
eficiente su trabajo.

\subsubsection{Aprendizaje por correcci\'on de error}

Consiste en tomar un conjunto de pares de datos, entradas con la correspondiente
salida en la red, y minimizar el error cometido por la red variando los ``pesos
sin\'apticos''. El conjunto de pares de datos recibe el nombre de
``\emph{conjunto de entrenamiento}''.

\subsubsection{Aprendizaje por refuerzo}

No se dispone de un conjunto completo del comportamiento deseado de la red, ya
que no se conoce la salida deseada exacta para cada entrada, sino que se conoce
el comportamiento de una manera global para diferentes entradas. La relaci\'on
entrada-salida se realiza a trav\'es de un proceso de \'exito o fracaso,
produciendo este una se\~nal de refuerzo que mide el buen funcionamiento del
sistema. La funci\'on del supervisor es m\'as la de un cr\'{\i}tico que la de
un maestro.

\subsubsection{Aprendizaje estoc\'astico}

Consiste, b\'asicamente, en realizar cambios aleatorios de los valores de los 
pesos y evaluar su efecto a partir del objetivo deseado.


%
% EL ALGORITMO DE BACKPROPAGATION
%

%
% EL ALGORITMO BACKPROPAGATION
%

\subsection{El algoritmo ``Backpropagation''}

Supongamos que tenemos una \textbf{RNA} con $s$ capas. El algoritmo de
``backpropagation'' consiste en ajustar los pesos sin\'apticos de la red
mediante la comparaci\'on de resultados obtenidos mediante un conjunto de
patrones con los resultados correctos que se deber\'{\i}an obtener con esos
mismos patrones.\\

El nombre de ``backpropagation''\footnote{Propagaci\'on hacia atras.} es debido
a que primero se ajustan los pesos de la capa de salida y luego se van ajustando
los pesos de las diferentes capas hacia atras.\\

Con este algoritmo se entrena a la \textbf{RNA} mediante un conjunto de
patrones, el cual se conoce como patrones de entrenamiento y lo denotaremos como
$Entr$.\\

Supongamos que el conjunto de patrones de entrenamiento para nuestra red es:
\begin{displaymath}
Entr = \{\ p_1,\dots,p_n\ \}
\end{displaymath}
Definamos los siguientes valores:
\begin{itemize}
\item $y_{p,k}$ salida correcta para el patr\'on de entrenamiento $p$ en la
neurona $k$-\'esima de la capa de salida.
\item $o_{p,k}$ salida real producida por el patr\'on de entrenamiento $p$ en
la neurona $k$-\'esima de la capa de salida.
\end{itemize}
Para calcular el error cometido en el procesamiento de cada patr\'on
utilizaremos el error cuadr\'atico medio. Luego el error cometido al procesar
el patr\'on $p$ ser\'a:
\begin{equation}\label{eq:ECM}
E_p = \frac{1}{2} \sum_{i=1}^{n_s} \delta_{p,k}^{^2} = \frac{1}{2}
\sum_{i=1}^{n_s} (y_{p,k}-o_{p,k})^2
\end{equation}
%
\newpage
%
Seg\'un las definiciones la entrada total de red para la neurona $k$-\'esima en
la capa $s$ ser\'a:
\begin{equation}\label{eq:ENTRADARED}
I_{k}[s](p) = \sum_{i=1}^{n_{s-1}} w_{k,i}[s]\cdot x_i[s-1](p)
\end{equation}
donde $x_i[s-1]$ es la salida de la $i$-\'esima neurona de la capa $s-1$. Luego
tendremos que:
\begin{equation}\label{eq:SALIDACORRECTA}
y_{p,k} = f(I_k[s](p))
\end{equation}
donde $f(x)$ es la funci\'on de activaci\'on de la red.\\

Para minimizar el error que comete la red tendremos que ajustar los pesos
sin\'apticos $w_{i,j}$, y esto lo haremos mediante el
``m\'etodo del gradiente''\footnote{No haremos referencia a las capas para no
complicar m\'as la notaci\'on}:
\begin{equation}\label{eq:GRADIENTE}
w_{j+1,k}-w_{j,k}=\Delta_p w_{j,k}=- \gamma \cdot
\frac{\partial E_p}{\partial w_{j,k}}
\end{equation}
donde $\gamma$ es el ``factor de aprendizaje'' de la red, el cual es positivo y
normalmente menor que uno.\\ \\
%
Utilizando la regla de la cadena tendremos:
\begin{equation}\label{eq:REGLACADENA}
\frac{\partial E_p}{\partial w_{j,k}} = \frac{\partial E_p}{\partial I_k(p)}
\cdot \frac{\partial I_k(p)}{\partial w_{j,k}}
\end{equation}
Definiendo:
\begin{equation}\label{eq:DEFINICION}
\delta_{p,k} = -\frac{\partial E_p}{\partial I_k(p)}
\end{equation}
y derivando $(\ref{eq:ENTRADARED})$ respecto de $w_{j,k}$ tendremos:
\begin{equation}\label{eq:DERIVANDO}
\frac{\partial I_k(p)}{\partial w_{j,k}} = x_j(p)
\end{equation}
Sustituyendo $(\ref{eq:DEFINICION})$ y $(\ref{eq:DERIVANDO})$ en
$(\ref{eq:REGLACADENA})$ entonces $(\ref{eq:GRADIENTE})$ nos queda:
\begin{equation}\label{eq:GRADIENTEFINAL}
\Delta_p w_{j,k} = \gamma \cdot \delta_{p,k}\cdot x_j(p)
\end{equation}
La formula $(\ref{eq:GRADIENTEFINAL})$ nos dar\'a el incremento necesario entre
dos pesos para el patr\'on $p$.\\

El calculo de los $\delta_{p,k}$ se puede hacer de forma recursiva desde la capa
de salida hac\'{\i}a las capas ocultas:
\begin{equation}\label{eq:DERFUNC}
\delta_{p,k}=-\frac{\partial E_p}{\partial I_k(p)}=
-\frac{\partial E_p}{\partial x_k(p)}\cdot
\frac{\partial x_k(p)}{\partial I_k(p)}
\end{equation}
En la ecuaci\'on $(\ref{eq:DERFUNC})$ el segundo factor es la derivada de la
funci\'on de activaci\'on $f(x)$:
\begin{displaymath}
x_k[s](p) = f(I_k[s](p))
\end{displaymath}
Para calcular el primer miembro de $(\ref{eq:DERFUNC})$ consideraremos los
siguientes casos:
\begin{itemize}
\item La neurona $k$-\'esima est\'a en la capa de salida.
\item La neurona $k$-\'esima est\'a en una capa oculta.
\end{itemize}

\subsubsection{En la capa de salida}

En este caso tendremos que:
\begin{displaymath}
\frac{\partial E_p}{\partial x_i(p)} = -(y_{p,i}-o_{p,i})
\end{displaymath}
ya que como estamos en la capa de salida tendremos que:
\begin{displaymath}
x_i(p)=x_i[s](p)=o_{p,i}
\end{displaymath}
Resultando que para las neuronas de la capa de salida tendremos de la ecuaci\'on
$(\ref{eq:DERFUNC})$:
\begin{equation}\label{eq:DeltaSalida}
\delta_{p,k} = (y_{p,i}-o_{p,i})\cdot f'(I_k[s](p))
\end{equation}
%
\newpage
%
\subsubsection{En las capas ocultas}
En la capa $z$-\'esima para el patr\'on de entrenamiento $p$ utilizando las
definiciones y la regla de la cadena tendremos:
\begin{eqnarray*}
\frac{\partial E_p}{\partial x_h[z](p)}&=& \sum_{i=1}^{n_z}
\frac{\partial E_p}{\partial I_i[z](p)}\cdot
\frac{\partial I_i[z](p)}{\partial x_h[z](p)} =\\
&=& \sum_{i=1} ^{n_z} \frac{\partial E_p}{\partial I_i[z](p)}\cdot
\frac{\partial }{\partial x_h[z](p)}\Big( \sum_{i=1}^{n_{z-1}}
w_{j,i}[z]\cdot x_i[z-1](p) \Big) =\\
&=& \sum_{i=1}^{n_z} \frac{\partial E_p}{\partial I_i[z](p)}\cdot w_{h,i}[z] =
-\sum_{i=1}^{n_z} \delta_{p,i}\cdot w_{h,i}[z]
\end{eqnarray*}
en el \'ultimo paso hemos utilizado la definici\'on dada en
$(\ref{eq:DEFINICION})$.\\ \\
%
Resultando que para las neuronas de las capas ocultas tendremos de la ecuaci\'on
$(\ref{eq:DERFUNC})$:	
\begin{equation}\label{eq:DeltaOcultas}
\delta_{p,k} = \Big( \sum_{i=1}^{n_z} \delta_{p,i}\cdot w_{k,i}[z] \Big) \cdot
f'(I_k[z](p))
\end{equation}

\subsubsection{Actualizaci\'on de los pesos}

Una vez que se conoce la expresi\'on de los $\delta_{p,k}$, utilizando las
ecuaciones $(\ref{eq:GRADIENTE})$, $(\ref{eq:REGLACADENA})$,
$(\ref{eq:DEFINICION})$ y $(\ref{eq:DERIVANDO})$ o lo que es lo mismo la
ecuaci\'on $(\ref{eq:GRADIENTEFINAL})$ obtenemos la siguiente formula para la
acutalizaci\'on de los pesos de la capa $z$-\'esima:
\begin{displaymath}
w_{j+1,k}=w_{j,k}+\gamma \cdot \delta_{p,k}\cdot x_j[z](p)
\end{displaymath}
donde $\delta_{p,k}$ viene determinada por $(\ref{eq:DeltaSalida})$ o
$(\ref{eq:DeltaOcultas})$, dependiendo si la capa en la que estamos 
actualizando los pesos es la capa de salida o es una capa oculta.
%
\newpage
%
\subsubsection{Factor de aprendizaje}

El valor $\gamma$ recibe el nombre de ``factor de aprendizaje'' y ello es 
debido a que seg\'un el valor que tenga el aprendizaje de la red ser\'a m\'as o
menos r\'apido.\\

Este factor es siempre positivo y normalmente es un n\'umero peque\~no, entre
$0.05$ y $0.25$, para asegurar que la red converge a una soluci\'on.\\

Un valor peque\~no de $\gamma$ significa que la red tendr\'a que hacer un gran
n\'umero de iteraciones. A veces es posible incrementar el valor de $\gamma$ a
medida que progresa el aprendizaje. Aumentando $\gamma$ a medida que disminuye
el error de la red suele acelerarse la convergencia incrementando el tama\~no
del paso conforme el error alcanza un valor m\'{\i}nimo, pero la red puede
rebotar, alej\'andose demasiado del valor m\'{\i}nimo verdadero si $\gamma$
llegara a ser demasiado grande.

\subsubsection{Regla Delta generalizada}

El algoritmo que hemos utilizado para ajustar los pesos sin\'apticos de la
\textbf{RNA}, utilizando ``backpropagation'', recibe el nombre de
\emph{Regla Delta generalizada}.



%
% COMO UTILIZAR BIAGRA
%

\include{como}

\part{B.I.A.G.R.A Data structures and constants}

%
% const.h
%

%
% const.h
%

\chapter{B.I.A.G.R.A constants (const.h)}

\section{Introduction}

\BI \ includes its own constants to be used if needed.\\

These constants are defined in \texttt{const.h}.

\section{Mathematical constants} \label{sec:mathematicalConsts}

Table \ref{tab:mathematicalConsts} shows the \BI's mathematical constanst.

\begin{table}[!h]
  \begin{center}
  \begin{tabular}{|c|c|c|}
    \hline
    \textbf{Constant} & \textbf{Name} & \textbf{Value} \\
    \hline
    $\texttt{e}$ & \texttt{BIA\_E} & $2.71828182845904523536029$ \\
    \hline
    $\pi$ & \texttt{BIA\_PI} & 3.14159265358979323846264 \\
    \hline
  \end{tabular}
  \end{center}
\caption{\BI\ mathematical constants.} \label{tab:mathematicalConsts}
\end{table}

\FloatBarrier

\section{Logical constants}

The following logical constants are defined:

\begin{description}
\item[BIA\_FALSE] when a condition is not met.
\item[BIA\_TRUE] when a condiction is met.
\end{description}

\section{Error constants}

The following error constants are defined:

\begin{description}
\item[BIA\_ZERO\_DIV] division by zero.
\item[BIA\_MEM\_ALLOC] error in memory allocation.
\end{description}


%
% dataaprox.h
%

\include{dataaprox}

\part{Memory allocation}

%
% resmem.h
%

%
% resmem.h
%

\chapter{Memory allocation (resmem.h)}

\section{Introduction}

\BI\ includes its own memory allocation functions which are defined in \texttt{resmem.h} file.

\section{Vector's memory allocation}

Some functions are provided to handle memory allocations for vectors.

\subsection{\texttt{dblPtMemAllocVec} function} \label{sec:dblPtMemAllocVec}

This functions allocates memory for a vector of doubles.\\

The definition of this function:
%
\begin{verbatim}
double *dblPtMemAllocVec(int intElements);  
\end{verbatim}
%
This function has only one argument, \texttt{intElements}, which is the dimension of the vector and a \texttt{double} pointer is returned.

\section{Matrix memory allocation}

Some functions are provided to handle memory allocations for vectors.

\subsection{\texttt{dblPtMemAllocMat} function} \label{sec:dblPtMemAllocMat}

This function allocates memory for a matrix of doubles.\\

The definition of this function:
%
\begin{verbatim}
double **dblPtMemAllocMat(int intRows, int intCols);
\end{verbatim}
%
where:
%
\begin{description}
\item[intRows] number of rows.
\item[intCols] number of columns.
\end{description}

\subsection{\texttt{dblPtMemAllocUpperTrMat} function} \label{sec:dblPtMemAllocUpperTrMat}

This function allocates memory for a upper triangular square matrix.\\

The definition of this function:
%
\begin{verbatim}
double **dblPtMemAllocUpperTrMat(int intOrder);
\end{verbatim}

This function has only one argument, \texttt{intOrder}, which is the order of the matrix and a \texttt{double} pointer to pointer is returned.\\

In a upper triangular square matrix all elements below the diagonal are zero: 
%
\begin{displaymath}
\left( \begin{array}{ccccc}
  a_{0,0} & a_{0,1} & a_{0,2} & a_{0,3} & a_{0,4} \\
  0      & a_{1,1} & a_{1,2} & a_{1,3} & a_{1,4} \\ 
  0      & 0      & a_{2,2} & a_{2,3} & a_{2,4} \\
  0      & 0      & 0      & a_{3,3} & a_{3,4} \\
  0      & 0      & 0      & 0      & a_{4,4} \\
\end{array} \right)
\end{displaymath}
%
For \texttt{intOrder = 5}:
%
\begin{verbatim}
myMatrix = dbpPtMemAllocUpperTrMat(5);  
\end{verbatim}
%
and:
\begin{center}
  \begin{tabular}{|c|c|c|c|}
    \hline
    \textbf{Pointer} & \textbf{\# elements} & \textbf{First element} & \textbf{Last element}\\
    \hline
    \texttt{myMatrix[0]} & 5 & 0 & 4\\
    \hline
    \texttt{myMatrix[1]} & 4 & 0 & 3\\
    \hline
    \texttt{myMatrix[2]} & 3 & 0 & 2\\
    \hline
    \texttt{myMatrix[3]} & 2 & 0 & 1\\
    \hline
    \texttt{myMatrix[4]} & 1 & 0 & 0\\
    \hline
  \end{tabular}
\end{center}
%
so:
%
\begin{displaymath}
  myMatrix[i][j] = *(*(myMatrix + i) + j) = \left\{ \begin{array}{ll}
    a_{i,j+i} & \forall \ i \le j \\
     & \\
    0 & \forall \ i > j
    \end{array} \right.    
\end{displaymath}

\subsection{\texttt{dblPtMemAllocLowerTrMat} function} \label{sec:dblPtMemAllocLowerTrMat}

This function allowcates memory for a lower triangular square matrix.\\

The definition of this function:
%
\begin{verbatim}
double **dblPtMemAllocLowerTrMat(int intOrder);
\end{verbatim}

This function has only one argument, \texttt{intOrder}, which is the order of the matrix and a \texttt{double} pointer to pointer is returned.\\

In a lower triangular square matrix all elements above the diagonal are zero: 
%
\begin{displaymath}
\left( \begin{array}{ccccc}
  a_{0,0} & 0      & 0      & 0      & 0 \\
  a_{1,0} & a_{1,1} & 0      & 0      & 0 \\ 
  a_{2,0} & a_{2,1} & a_{2,2} & 0      & 0 \\
  a_{3,0} & a_{3,1} & a_{3,2} & a_{3,3} & 0 \\
  a_{4,0} & a_{4,1} & a_{4,2} & a_{4,3} & a_{4,4} \\
\end{array} \right)
\end{displaymath}
%
For \texttt{intOrder = 5}:
%
\begin{verbatim}
myMatrix = dbpPtMemAllocLowerTrMat(5);  
\end{verbatim}
%
and:
\begin{center}
  \begin{tabular}{|c|c|c|c|}
    \hline
    \textbf{Pointer} & \textbf{\# elements} & \textbf{First element} & \textbf{Last element}\\
    \hline
    \texttt{myMatrix[0]} & 1 & 0 & 0\\
    \hline
    \texttt{myMatrix[1]} & 2 & 0 & 1\\
    \hline
    \texttt{myMatrix[2]} & 3 & 0 & 2\\
    \hline
    \texttt{myMatrix[3]} & 4 & 0 & 3\\
    \hline
    \texttt{myMatrix[4]} & 5 & 0 & 4\\
    \hline
  \end{tabular}
\end{center}
%
so:
%
\begin{displaymath}
  myMatrix[i][j] = *(*(myMatrix + i) + j) = \left\{ \begin{array}{ll}
    a_{i,j} & \forall \ i \le j \\
     & \\
    0 & \forall \ i < j
    \end{array} \right.    
\end{displaymath}

\section{Freeing memory} \label{sec:freeingMemory}

\BI\ includes its own functions to free memory.

\subsection{\texttt{freeMemDblMat} function} \label{sec:freeMemDblMat}




%
% MATHEMATICAL FUNCTIONS
%

\part{Mathematical functions}

%
% random.h
%

%
% random.h
%

\chapter{Pseudo random numbers (random.h)} \label{ch:random}

\section{Introduction}

\BI \ includes its own functions to pseudo random number generation and they are defined in \texttt{random.h} file.\\

\warning{This functions have not been tested to produce unpredictable sequences, so be careful when use them.}

\section{Pseudo random integer numbers}

\subsection{\texttt{intRandom} function} \label{sec:intRandom}

This function generates random integers.\\

The definition of this function:
%
\begin{verbatim}
int intRandom(int limit);  
\end{verbatim}
%
The pseudo random integer is placed in the interval $(-limit,limit)$.\\ \\
%
\note{Before using this function \texttt{srand} must be used to initialize \texttt{rand}. You can use \texttt{srand((unsigned)time(NULL))}.}
%
\ \\
%
The pseudo random number is generated with the following formula:
%
\begin{displaymath}
\left[ \frac{limit \cdot rand()}{RAND\_MAX + 1} \right] \in (-limit,limit)
\end{displaymath}
%
Then randomly is choosed if the number is positive or negative using the above formula with $limit=2$ and then taking modulus $2$. If modulus is $1$ then the number will be a negative one.

\subsection{\texttt{uintRandom} function} \label{sec:uintRandom}

This function generates random integers.\\

The definition of this function:
%
\begin{verbatim}
int uintRandom(int limit);  
\end{verbatim}
%
The pseudo random integer is placed in the interval $[0,limit)$.\\ \\
%
\note{Before using this function \texttt{srand} must be used to initialize \texttt{rand}. You can use \texttt{srand((unsigned)time(NULL))}.}
%
\ \\
%
The pseudo random number is generated with the following formula:
%
\begin{displaymath}
\left[ \frac{limit \cdot rand()}{RAND\_MAX + 1} \right] \in [0,limit)
\end{displaymath}
%

\section{Pseudo random floating point numbers}

\subsection{\texttt{dblRandom} function} \label{sec:dblRandom}

This function generates random floating point numbers.\\

The definition of this function:
%
\begin{verbatim}
int dblRandom(int limit);  
\end{verbatim}
%
The pseudo random floating point number is placed in the interval $(-limit,limit)$.\\ \\
%
\note{Before using this function \texttt{srand} must be used to initialize \texttt{rand}. You can use \texttt{srand((unsigned)time(NULL))}.}
%
\ \\
%
The pseudo random number is generated with the following formula:
%
\begin{displaymath}
\frac{limit \cdot rand()}{RAND\_MAX + 1} \in (-limit,limit)
\end{displaymath}
%
Then randomly is choosed if the number is positive or negative using the above formula with $limit=2$ and then taking modulus $2$. If modulus is $1$ then the number will be a negative one.

\subsection{\texttt{udblRandom} function} \label{sec:udblRandom}

This function generates random floating point numbers.\\

The definition of this function:
%
\begin{verbatim}
int udblRandom(int limit);  
\end{verbatim}
%
The pseudo random floating point number is placed in the interval $[0,limit)$.\\ \\
%
\note{Before using this function \texttt{srand} must be used to initialize \texttt{rand}. You can use \texttt{srand((unsigned)time(NULL))}.}
%
\ \\
%
The pseudo random number is generated with the following formula:
%
\begin{displaymath}
\frac{limit \cdot rand()}{RAND\_MAX + 1} \in [0,limit)
\end{displaymath}
%


%
% complex.h
%

%
% complex.h
%

\chapter{Complex numbers (complejo.h)}

\section{Introduction}

Functions to manage complex numbers are defined in \texttt{complex.h} file.

\section{Data structures}

Some data structures are defined in \BI \ to manage complex numbers.

\subsection{\texttt{biaComplex} data structure} \label{sec:biaComplex}

This data structure is used to handle polinomials $p(x) \in \mathbb{R}[x]$. \textbf{biaComplex} data structure is defined in figure \ref{fig:biaRealPol} where:

\begin{description}
\item[intDegree] polynomial degree.
\item[intRealRoots] number of real roots (if any).
\item[intCompRoots] number of complex roots (if any).
\item[*dblCoef] pointer to store polynomial coeficients.
\end{description}

\begin{figure}[!h]
\begin{verbatim}
typedef struct {
  double dblReal,
         dblImag;
  } biaComplex;
\end{verbatim}
\caption{biaComplex data structure.} \label{fig:biaComplex}
\end{figure}

\subsection{\texttt{biaPolar} data structure} \label{sec:biaPolar}

This data structure is used to store data for root approximation. Data structure is defined in figure \ref{fig:biaRealRoot} where:

\begin{description}
\item[intNMI] maximum number of iterations to get the root with a maximum error of \emph{dblTol}.
\item[intIte] iterations used to get the root.
\item[dblx0] initial approximation to get the root.
\item[dblRoot] root approximation.
\item[dblTol] maximum tolerance when calculating the root.
\item[dblError] error in root approximation. Difference between the las two root approximations.
\end{description}

\begin{figure}[!h]
\begin{verbatim}
typedef struct {
  double dblMod,
         dblArg;
  } biaPolar;
\end{verbatim}
\caption{biaPolar data structure.} \label{fig:biaPolar}
\end{figure}

\FloatBarrier

\section{Arithmetical operations using complex numbers}

\subsection{\texttt{addComplex} function}

This function adds two complex numbers.\\

The definition of this function:
%
\begin{verbatim}
void addComplex(biaComplex *ptCmplx1, biaComplex *ptCmplx2, 
                biaComplex *ptRes);
\end{verbatim}
%
where:
%
\begin{description}
\item[*ptCmplx1] first complex number to be added.
\item[*ptCmplx2] second complex number to be added.
\item[*ptRes] result of the operation.
\end{description}

\subsection{\texttt{subtractComplex} function}

This function subtracts two complex numbers.\\

The definition of this function:
%
\begin{verbatim}
void subtractComplex(biaComplex *ptCmplx1, biaComplex *ptCmplx2, 
                     biaComplex *ptRes);  
\end{verbatim}
%
where:
%
\begin{description}
\item[*ptCmplx1] complex number.
\item[*ptCmplx2] complex number to be subtracted to the above.
\item[*ptRes] result of the operation.
\end{description}

\subsection{\texttt{multiplyComplex} function}

This function multiplies two complex numbers.\\

The definition of this function:
%
\begin{verbatim}
void multiplyComplex(biaComplex *ptCmplx1, biaComplex *ptCmplx2, 
                     biaComplex *ptRes);  
\end{verbatim}
%
where:
%
\begin{description}
\item[ptCmplx1] first complex number to be multiplied. 
\item[ptCmplx2] second complex number to be multiplied.
\item[ptRes] result of the operation.
\end{description}

\subsection{\texttt{divideComplex} function}

This function divides one complex number by other:
%
\begin{displaymath}
\frac{\mathrm{a} + \mathrm{b} \cdot i}{\mathrm{c} + \mathrm{d} \cdot i} = (\mathrm{a} + \mathrm{b} \cdot i) \cdot (\mathrm{c} + \mathrm{d} \cdot i)^{-1}  
\end{displaymath}

The definition of this function:
%
\begin{verbatim}
int divideComplex(biaComplex *ptCmplx1, biaComplex *ptCmplx2, 
                  biaComplex *ptRes);
\end{verbatim}
%
where:
%
\begin{description}
\item[*ptCmplx1] complex number.
\item[*ptCmplx2] complex number used as divisor. 
\item[*ptRes] result of the operation.
\end{description}
%
The following codes are returned:
%
\begin{center}
\begin{tabular}{|l|l|}
\hline
\textbf{BIA\_ZERO\_DIV} & Division by zero \\
\hline
\textbf{BIA\_TRUE} & Success \\
\hline
\end{tabular}
\end{center} 

\subsection{\texttt{invSumComplex} function}

This function gets the additive inverse of a complex number:
%
\begin{displaymath}
\forall \ \ z_1 \in \mathbb{C} \quad \exists \ \ z_2 \in \mathbb{C} \ \ | \ \ z_1 + z_2 = 0
\end{displaymath}

The definition of this function:
%
\begin{verbatim}
void invSumComplex(biaComplex *ptCmplx, biaComplex *ptRes);  
\end{verbatim}
%
where:
%
\begin{description}
\item[*ptCmplx] complex number to get its additive inverse.
\item[*ptRes] where the additive inverse will be stored.
\end{description}

\subsection{\texttt{invMulComplex} function}

This function gets the multiplicative inverse of a complex number:
%
\begin{displaymath}
\forall \ \ z_1 \in \mathbb{C} - \{0\} = \mathbb{C}^{*} \quad \exists \ \ z_2 \in \mathbb{C} \ \ | \ \ z_1 \cdot z_2 = 1
\end{displaymath}

The definition of this function:
%
\begin{verbatim}
int invMulComplex(biaComplex *ptCmplx, biaComplex *ptRes) ;  
\end{verbatim}
%
where:
%
\begin{description}
\item[*ptCmplx] complex number to get its multiplicative inverse.
\item[*ptRes] where the additive multiplicative will be stored.
\end{description}

The following codes are returned:

\begin{center}
\begin{tabular}{|l|l|}
\hline
\textbf{BIA\_ZERO\_DIV} & Division by zero \\
\hline
\textbf{BIA\_TRUE} & Success \\
\hline
\end{tabular}
\end{center}

\section{Complex number operations}

\subsection{\texttt{dblComplexModulus} function}

This function gets the modulus of a complex number.\\

The definition of this function:
%
\begin{verbatim}
double dblComplexModule(biaComplex *ptCmplx);  
\end{verbatim}
%
where:
%
\begin{description}
\item[*ptCmplx] complex number to get its modulus.
\end{description}
%
This function returns the complex number modulus.

\subsection{\texttt{dblComplexArg} function}

This function gets the argument of a complex number.\\

The definition of this function:
%
\begin{verbatim}
double dblComplexArg(biaComplex *ptCmplx);  
\end{verbatim}
%
where:
%
\begin{description}
\item[*ptCmplx] complex number to get its argument.
\end{description}
%
This function returns the complex number argument (radians).

\subsection{\texttt{conjugateComplex} function}

This function gets the conjugate complex of a complex number:
%
\begin{displaymath}
z = \textrm{a} + \textrm{b} \cdot i \ \ \in \mathbb{C} \Rightarrow \overline{z} = \textrm{a} - \textrm{b} \cdot i \ \ \in \mathbb{C}
\end{displaymath}
%
The definition of this function:
%
\begin{verbatim}
void conjugateComplex(biaComplex *ptCmplx, biaComplex *ptRes);  
\end{verbatim}
%
where:
%
\begin{description}
\item[*ptCmplx] complex number to get its conjugate.
\item[*ptRes] complex conjugate.
\end{description}

\subsection{\texttt{complex2Polar} function}

This function gets the polar coordinates of a complex number.\\

The definition of this function:
%
\begin{verbatim}
void complex2Polar(biaComplex *ptCmplx, biaPolar *ptRes);  
\end{verbatim}
%
where:
%
\begin{description}
\item[*ptCmplx] complex number to calculate polar coordinates.
\item[*ptRes] polar coordinates.
\end{description}

\subsection{\texttt{polar2Complex} function}

This function gets the cartesian coordinates of a polar coordinates for a complex number.\\

The definition of this function:
%
\begin{verbatim}
void polar2Complex(biaPolar *ptPolar, biaComplex *ptRes);  
\end{verbatim}
%
where:
%
\begin{description}
\item[*ptPolar] polar coordinates.
\item[*ptRes] complex number in cartesian coordinates.
\end{description}

\note{Argument is supposed to be in radians.}


%
% integers.h
%

%
% integers.h
%

\chapter{Integer numbers (integers.h)} \label{ch:integers}

\section{Introduction}

\BI \ includes functions about integer numbers in \texttt{integers.h} file.\\

\section{Sum integers}

\subsection{\textbf{uintSumFirstNIntegers} function} \label{sec:uintSumFirstNIntegers}

This function gets the sum of the first $n$ integers.\\

The definition of this function:
%
\begin{verbatim}
unsigned uintSumFirstNIntegers(int n);
\end{verbatim}
%
If the sum is bigger than an unsigned int $0$ is returned.


%
% poliynomial.h
%

%
% polynomial.h
%

\chapter{Polynomial (polynomials.h)} \label{sec:polynomial}

\section{Introduction}

Functions to manage polynomials are defined in \texttt{polynomial.h} file.\\

Polynomials are stored using a \BI\ data structure named \texttt{biaRealPol}\footnote{Section \ref{sec:biaRealPol} in page \pageref{sec:biaRealPol}.}.

\section{Polynomial evaluation}

\subsection{\texttt{dblEvaluatePol} function}

This function evaluate a polynomial in a given point and returns it.\\

The definition of this function:
%
\begin{verbatim}
double dblEvaluatePol(biaRealPol *ptPol, double dblX);  
\end{verbatim}
%
where:
\begin{description} 
\item[*ptPol] pointer to a \texttt{biaRealPol} struct.
\item[dblX] point in which the polynomial has to be evaluated.
\end{description}
%
The \texttt{biaRealPol} has to be initialized with the polynomial degree and polynomial coeficients.

\note{See section \ref{sec:biaRealPol} to understand how \texttt{biaRealPol} is used.}

\section{Polynomial derivatives}

\subsection{\texttt{derivativePol} function}

This function gets the $n$-th derivative of a polynomial.\\

The definition of this function:
%
\begin{verbatim}
int derivativePol(biaPol *ptPol, biaPol *ptDer, int intN);
\end{verbatim}
%
where:
\begin{description} 
\item[*ptPol] pointer to a \texttt{biaRealPol} struct with the polynomial to get its derivative is stored.
\item[*ptDer] pointer to a \texttt{biaRealPol} struct where the derivative will be stored.
\item[intN] order of the derivative to get.
\end{description}
%
The following codes are returned:
%
\begin{center}
\begin{tabular}{|l|l|}
\hline
\textbf{BIA\_MEM\_ALLOC} & Memory allocation error \\
\hline
\textbf{BIA\_TRUE} & Success \\
\hline
\end{tabular}
\end{center}
%
\note{\texttt{ptDer} will be released and memory allocation will be carried out to store the derivative.}
%
\warning{\texttt{ptDer} member \texttt{dblCoefs} has to be initialized to a \texttt{NULL} pointer to avoid a \textbf{Segment Fault} error if it was not previously initialized.}

\section{Arithmetical operations using polynomials}

\subsection{\texttt{addPol} function}

This function adds two polynomials.\\

The definition of this function:
%
\begin{verbatim}
int addPol(biaPol *ptPol1, biaPol *ptPol2, biaPol *ptRes);  
\end{verbatim}
%
where:
%
\begin{description}
\item[*ptPol1] pointer to a \texttt{biaPol} struct with the first polynomial to be added.
\item[*ptPol2] pointer to a \texttt{biaPol} struct with the second polynomial to be added.
\item[*ptRes] pointer to a \texttt{biaPol} struct where the add operation will be stored.
\end{description}
%
The following codes are returned:
%
\begin{center}
\begin{tabular}{|l|l|}
\hline
\textbf{BIA\_MEM\_ALLOC} & Memory allocation error \\
\hline
\textbf{BIA\_TRUE} & Success \\
\hline
\end{tabular}
\end{center}
%
\note{\texttt{ptRes} will be released and memory allocation will be carried out to store the derivative.}
%
\warning{\texttt{ptRes} member \texttt{dblCoefs} has to be initialized to a \texttt{NULL} pointer to avoid a \textbf{Segment Fault} error if it was not previously initialized.}

\subsection{\texttt{subtractPol} function}

This function subtracts two polynomials.\\

The definition of this function:
%
\begin{verbatim}
int subtractPol(biaPol *ptPol1, biaPol *ptPol2, biaPol *ptRes);
\end{verbatim}
%
where:
%
\begin{description}
\item[*ptPol1] pointer to a \texttt{biaPol} struct with the first polynomial.
\item[*ptPol2] pointer to a \texttt{biaPol} struct with the polynomial to be subtracted from the above.
\item[*ptRes] pointer to a \texttt{biaPol} struct where the subtract operation will be stored.
\end{description}
%
The following codes are returned:
%
\begin{center}
\begin{tabular}{|l|l|}
\hline
\textbf{BIA\_MEM\_ALLOC} & Memory allocation error \\
\hline
\textbf{BIA\_TRUE} & Success \\
\hline
\end{tabular}
\end{center}
%
\note{\texttt{ptRes} will be released and memory allocation will be carried out to store the derivative.}
%
\warning{\texttt{ptRes} member \texttt{dblCoefs} has to be initialized to a \texttt{NULL} pointer to avoid a \textbf{Segment Fault} error if it was not previously initialized.}

\subsection{\texttt{multiplyPol} function}

This functions multiplies two polynomials.\\

The definition of this function:
%
\begin{verbatim}
int subtractPol(biaPol *ptPol1, biaPol *ptPol2, biaPol *ptRes);
\end{verbatim}
%
where:
%
\begin{description}
\item[*ptPol1] pointer to a \texttt{biaPol} struct with the first polynomial.
\item[*ptPol2] pointer to a \texttt{biaPol} struct with the second polynomial.
\item[*ptRes] pointer to a \texttt{biaPol} struct where the multiplication operation will be stored.
\end{description}
%
The following codes are returned:
%
\begin{center}
\begin{tabular}{|l|l|}
\hline
\textbf{BIA\_MEM\_ALLOC} & Memory allocation error \\
\hline
\textbf{BIA\_TRUE} & Success \\
\hline
\end{tabular}
\end{center}
%
\note{\texttt{ptRes} will be released and memory allocation will be carried out to store the derivative.}
%
\warning{\texttt{ptRes} member \texttt{dblCoefs} has to be initialized to a \texttt{NULL} pointer to avoid a \textbf{Segment Fault} error if it was not previously initialized.}

\section{Polynomial roots approximations}

\subsection{\texttt{newtonPol} function}

This function approaches a polynomial root using the \textbf{Newton} method.\\

The definition of this function:
%
\begin{verbatim}
int newtonPol(biaPol *ptPol, biaRealRoot *ptRoot);  
\end{verbatim}
%
The following codes are returned:
%
\begin{center}
\begin{tabular}{|l|l|}
\hline
\textbf{BIA\_MEM\_ALLOC} & Memory allocation error \\
\hline
\textbf{BIA\_ZERO\_DIV} & Division by zero \\
\hline
\textbf{BIA\_TRUE} & Success \\
\hline
\textbf{BIA\_FALSE} & Root approximation could not be calculated \\
                    & satisfying the requirements (\texttt{intMNI} and \texttt{dblTol}). \\
\hline
\end{tabular}
\end{center}

\texttt{biaRealRoot *ptRoot} has to be initialized:
%
\begin{description}
\item[intMNI] maximun number of iterations.
\item[dblx0] initial approximation.
\item[dblTol] tolerance to approximate the root.
\end{description}
%
\note{When two consecutive approximations are close enough, \texttt{dblTol}, last approximation will be considered as good and will be stored in \texttt{*biaRealRoot *ptRoot} in \texttt{dblRoot}.}
%
\ \\ \\ \\
%
\note{In \texttt{intIte} will be stored the number of iterations used to get the root and in \texttt{dblError} will be stored the error between the two last approximations.}







%
% matrix.h
%

%
% matrix.h
%

\chapter{Matrix (matrix.h)}

\section{Introduction}

Functions to manage matrices are defined in \texttt{matrix.h} file.\\

\section{Data structures}

Some data structures are defined in \BI \ to manage matrices.

\subsection{\texttt{biaMatrix} data structure} \label{sec:biaMatrix}

This data structure is used to store a matrix. \textbf{biaMatrix} data structure is defined in figure \ref{fig:biaMatrix} where:

\begin{description}
\item[intRows] number of rows.
\item[intCols] number of columns.
\item[**dblCoefs] pointer to store matrix coeficients.
\end{description}

\begin{figure}[!h]
\begin{verbatim}
typedef struct {
  int intRows,
      intCols;

  double **dblCoefs;
  } biaMatrix;   
\end{verbatim}
\caption{biaMatrix data structure.} \label{fig:biaMatrix}
\end{figure}

\FloatBarrier

\section{Matrix creation}

\BI \ includes functions to create some kind of matrices.

\subsection{\texttt{identityMatrix} function}

This function stores the identity matrix with order taken from intRows member of \texttt{ptMatrix}:
%
\begin{displaymath}
  \left( \begin{array}{ccccc}
    1 & 0 & 0 & 0 & 0 \\
    0 & 1 & \ddots & 0 & 0 \\
    0 & \ddots & \ddots & \ddots & 0 \\
    0 & 0 & \ddots & 1 & 0 \\
    0 & 0 & 0 & 0 & 1 \\
  \end{array} \right)
\end{displaymath}
%
The definition of this function:
%
\begin{verbatim}
void identityMatrix(biaMatrix *ptMatrix);  
\end{verbatim}
%
where:
%
\begin{description}
\item[*ptMatrix] matrix that has to be created before using this function. Memory allocation for \texttt{dblCoefs} must be done before using this function.
\end{description}
%
\ \\
%
\note{\texttt{intRows} is used to get the matrix order.}

\subsection{\texttt{scalingMatrix} function}

This function stores the scaling matrix with factor $\lambda$ and order taken from intRows member of \texttt{ptMatrix}:
%
\begin{displaymath}
  \left( \begin{array}{ccccc}
    \lambda & 0 & 0 & 0 & 0 \\
    0 & \lambda & \ddots & 0 & 0 \\
    0 & \ddots & \ddots & \ddots & 0 \\
    0 & 0 & \ddots & \lambda & 0 \\
    0 & 0 & 0 & 0 & \lambda \\
  \end{array} \right)
\end{displaymath}
%
The definition of this function:
%
\begin{verbatim}
void scalingMatrix(biaMatrix *ptMatrix, double dblFactor);  
\end{verbatim}
%
where:
%
\begin{description}
\item[*ptMatrix] matrix that has to be created before using this function. Memory allocation for \texttt{dblCoefs} must be done before using this function.
\end{description}
%
\ \\
%
\note{\texttt{intRows} is used to get the matrix order.}

\subsection{\texttt{nullMatrix} function}

This function stores the null matrix with order taken from intRows member of \texttt{ptMatrix}:
%
\begin{displaymath}
  \left( \begin{array}{ccccc}
    0 & 0 & 0 & 0 & 0 \\
    0 & 0 & \ddots & 0 & 0 \\
    0 & \ddots & \ddots & \ddots & 0 \\
    0 & 0 & \ddots & 0 & 0 \\
    0 & 0 & 0 & 0 & 0 \\
  \end{array} \right)
\end{displaymath}
%
The definition of this function:
%
\begin{verbatim}
void nullMatrix(biaMatrix *ptMatrix);  
\end{verbatim}
%
where:
%
\begin{description}
\item[*ptMatrix] matrix that has to be created before using this function. Memory allocation for \texttt{dblCoefs} must be done before using this function.
\end{description}
%
\ \\
%
\note{\texttt{intRows} and \texttt{intCols} is used to get the matrix order.}

\section{Matrix operations}

\subsection{\texttt{transposeMatrix} function}

This function stores the transpose matrix of a given matrix.\\
%
The definition of this function:
%
\begin{verbatim}
void transposeMatrix(biaMatrix *ptMatrix, biaMatrix *ptRes);  
\end{verbatim}
%
where:
%
\begin{description}
\item[*ptMatrix] matrix to get its transpose matrix.
\item[*ptRes] matrix to store the transpose matrix. Memory has to be preallocated before using this function.
\end{description}
%
\ \\
%
\note{\texttt{intRows} and \texttt{intCols} is used to get the matrix order.}

\section{Matrix checks}

\subsection{\texttt{isIdentityMatrix} function}

This function checks if a matrix is the identity matrix.\\
%
The definition of this function:
%
\begin{verbatim}
int isIdentityMatrix(biaMatrix *ptMatrix);
\end{verbatim}
%
where:
%
\begin{description}
\item[*ptMatrix] matrix to check.
\end{description}
%

\subsection{\texttt{isNullMatrix} function}

This function checks if a matrix is a null matrix.\\
%
The definition of this function:
%
\begin{verbatim}
int isNullMatrix(biaMatrix *ptMatrix, double dblTol);
\end{verbatim}
%
where:
%
\begin{description}
\item[*ptMatrix] matrix to check.
\item[dblTol] if a matrix element is minor than this value it is assumed it is a null element.
\end{description}
%

\subsection{\texttt{isSymmetricMatrix} function}

This function checks if a matrix is a symmetric matrix.\\
%
The definition of this function:
%
\begin{verbatim}
int isSymmetricMatrix(biaMatrix *ptMatrix);
\end{verbatim}
%
where:
%
\begin{description}
\item[*ptMatrix] matrix to check.
\end{description}
%


%
% roots.h
%

%
% roots.h
%

\chapter{Roots approximation (roots.h)} \label{sec:roots}

\section{Introduction}

Functions to compute function's roots approximation are defined in \texttt{roots.h} file.

\section{Data structures}

Some data structures are defined in \BI to manage roots.

\subsection{\texttt{biaRealRoot} data structure} \label{sec:biaRealRoot}

This data structure is used to store data for root approximation.\\

Data structure is defined in figure \ref{fig:biaRealRoot} where:
%
\begin{description}
\item[intNMI] maximum number of iterations to get the root with a maximum error of \emph{dblTol}.
\item[intIte] iterations used to get the root.
\item[dblx0] initial approximation to get the root.
\item[dblRoot] root approximation.
\item[dblTol] maximum tolerance when calculating the root.
\item[dblError] error in root approximation. Difference between the las two root approximations.
\end{description}
%
\begin{figure}[!h]
\begin{verbatim}
typedef struct {
  int intNMI,
      intIte;

  double dblx0,
         dblRoot,
         dblTol,
         dblError;
  } biaRealRoot;
\end{verbatim}
\caption{biaRealRoot data structure.} \label{fig:biaRealRoot}
\end{figure}

\FloatBarrier

\section{Function roots approximation}

\subsection{\texttt{newtonMethod} function}

This function approaches a function's root using the Newton method.\\ \\
%
The definition of this function:
%
\begin{verbatim}
int newtonMethod(biaRealRoot *ptRoot, 
       int (*func)(double dblx0, double *ptRes),
       int (*der)(double dblx0, double *ptRes));  
\end{verbatim}




%
% RNGKUTTA.H
%

%
% RNGKUTTA.H
%

\chapter{Runge-Kutta methods (rngkutta.h)}

\section{Introduction}

\textbf{Runge-Kutta} are a family of implicit and explicit iterative methods used to approximate solutions of ordinary differential equations or \textbf{ODE}.\\

Butcher matricial notation is used in this implementation.\\

Let's assume that the initial value problem or Cauchy problem we want to solve is:
%
\begin{eqnarray*}
  y'(x) & = & f(x, y(x)) \\
  y(x_0) & = & y_0
\end{eqnarray*}
%
\begin{displaymath}
  y_{n+1} = y_n  + h \cdot \sum_{i=1}^s b_i \cdot k_i
\end{displaymath}
%
Where:
%
\begin{eqnarray*}
  k_i & = & f(x_n + (h\cdot c_i), y_n + h\cdot \left(\sum_{j=1}^{i-1} a_{ij} \cdot k_j \right) )\\
  c_i & = & \sum_{j=1}^s a_{ij}
\end{eqnarray*}

Todas estas funciones suponen que la variable de \emph{estructura}, del tipo
\emph{DatosRK}\footnote{Apartado (\ref{sec:datosRK}) en la p\'agina 
\pageref{sec:datosRK}}, no tienen dimensionados los punteros en ella 
contenidos, raz\'on por la cual ser\'a necesario liberar la memoria asignada
a estos antes de pasarle como parametro una variable de este tipo a una de
las siguientes funciones(siempre y cuando se hayan dimensionado dichos
punteros).\newline

Hay que destacar que \textbf{NO} se inicializan todos los miembros de esta
estructura, s\'olo aquellos miembros que contienen los coeficientes del 
m\'etodo.\newline

Los siguientes miembros \textbf{NO} se inicializan:
%
\begin{description}
\item[intNumAprox]
\item[dblPuntos]
\item[dblPaso]
\item[dblInicio]
\item[dblFinal]
\end{description}

Estos miembros son independientes del m\'etodo, dependen del problema que
se quiera resolver y tendr\'an que ser inicializados por el usuario.

\section{M\'etodos expl\'{\i}citos}

\subsection{RungeKuttaClasico}
Funci\'on que inicializa los coeficientes para el m\'etodo \emph{Runge-Kutta 
Cl\'asico}, el cual es un m\'etodo de $4$ etapas y orden $4$.\newline

La notaci\'on matricial del m\'etodo es la siguiente:

\begin{center}
$
\begin{array}{c|cccc}
0 & 0 \\
\frac{1}{2} & \frac{1}{2} & 0 \\
\frac{1}{2} & 0 & \frac{1}{2} & 0 \\
1 & 0 & 0 & 1 & 0 \\
\hline
 & \frac{1}{6} & \frac{1}{3} & \frac{1}{3} & \frac{1}{6} \\
\end{array}
$
\end{center}

El prototipo de esta funci\'on es el siguiente:

\begin{center}
\emph{int \textbf{RungeKuttaClasico}(DatosRK *ptstrDatos)}
\end{center}

\begin{description}
\item[ptstrDatos] puntero a una variable de \emph{estructura} del tipo
\emph{DatosRK}.
\end{description}

La funci\'on devuelve los siguientes c\'odigos:

\begin{center}
\begin{tabular}{|l|l|}
\hline
\textbf{ERR\_AMEM} & Hubo un error en la asignaci\'on de memoria. \\
\hline
\textbf{TRUE} & Se inicializaron con \'exito los coeficientes. \\
\hline
\end{tabular}
\end{center}

Por ejemplo:

\begin{center}
\emph{intResultado = \textbf{RungeKuttaClasico}(\&varstrDatRK);}
\end{center}

Inicializar\'{\i}a los coeficientes del m\'etodo en la variable 
\emph{varstrDatRK}, en \emph{intResultado} el valor \textbf{TRUE} si se pudieron
inicializar los coeficientes y en caso contrario \textbf{ERR\_AMEM}.

\newpage

\subsection{MetodoHeun}
Funci\'on que inicializa los coeficientes para el m\'etodo de \emph{Heun}, el
cual es un m\'etodo \emph{Runge-Kutta} de $3$ etapas y orden $3$.\newline

La notaci\'on matricial del m\'etodo es la siguiente:

\begin{center}
$
\begin{array}{c|ccc}
0 & 0 \\
\frac{1}{3} & \frac{1}{3} & 0 \\
\frac{2}{3} & 0 & \frac{2}{3} & 0 \\
\hline
 & \frac{1}{4} & 0 & \frac{3}{4}
\end{array}
$
\end{center}

El prototipo de esta funci\'on es el siguiente:

\begin{center}
\emph{int \textbf{MetodoHeun}(DatosRK *ptstrDatos)}
\end{center}

\begin{description}
\item[ptstrDatos] puntero a una variable de \emph{estructura} del tipo
\emph{DatosRK}.
\end{description}

La funci\'on devuelve los siguientes c\'odigos:

\begin{center}
\begin{tabular}{|l|l|}
\hline
\textbf{ERR\_AMEM} & Hubo un error en la asignaci\'on de memoria. \\
\hline
\textbf{TRUE} & Se inicializaron con \'exito los coeficientes. \\
\hline
\end{tabular}
\end{center}

Por ejemplo:

\begin{center}
\emph{intResultados = \textbf{MetodoHeun}(\&varstrDatRK);}
\end{center}

Inicializar\'{\i}a los coeficientes del m\'etodo en la variable
\emph{varstrDatRK}, en \emph{intResultado} el valor \textbf{TRUE} si se pudieron
inicializar los coeficientes y en caso contrario \textbf{ERR\_AMEM}.

\newpage

\subsection{MetodoKutta}
Funci\'on que inicializa los coeficientes para el m\'etodo de \emph{Kutta}, el
cual es un m\'etodo \emph{Runge-Kutta} de $3$ etapas y orden $3$.\newline

La notaci\'on matricial del m\'etodo es la siguiente:

\begin{center}
$
\begin{array}{c|ccc}
0 & 0 \\
\frac{1}{2} & \frac{1}{2} & 0 \\
1 & -1 & 2 & 0 \\
\hline
 & \frac{1}{6} & \frac{2}{3} & \frac{1}{6}
\end{array}
$
\end{center}

El prototipo de esta funci\'on es el siguiente:

\begin{center}
\emph{int \textbf{MetodoKutta}(DatosRK *ptstrDatos)}
\end{center}

\begin{description}
\item[ptstrDatos] puntero a una variable de \emph{estructura} del tipo
\emph{DatosRK}.
\end{description}

La funci\'on devuelve los siguientes c\'odigos:

\begin{center}
\begin{tabular}{|l|l|}
\hline
\textbf{ERR\_AMEM} & Hubo un error en la asignaci\'on de memoria. \\
\hline
\textbf{TRUE} & Se inicializaron con \'exito los coeficientes. \\
\hline
\end{tabular}
\end{center}

Por ejemplo:

\begin{center}
\emph{intResultado = \textbf{MetodoKutta}(\&varstrDatRK);}
\end{center}

Inicializar\'{\i}a los coeficientes del m\'etodo en la variable
\emph{varstrDatRK}, en \emph{intResultado} el valor \textbf{TRUE} si se pudieron
inicializar los coeficientes y en caso contrario \textbf{ERR\_AMEM}.

\newpage

\subsection{EulerModificado}

Funci\'on que inicializa los coeficientes para el m\'etodo de \emph{Euler 
modificado}, el cual es un m\'etodo \emph{Runge-Kutta} de $2$ etapas y 
orden $2$.\newline

La notaci\'on matricial del m\'etodo es la siguiente:

\begin{center}
$
\begin{array}{c|cc}
0 & 0 \\
\frac{1}{2} & \frac{1}{2} & 0 \\
\hline
 & 0 & 1
\end{array}
$
\end{center}

El prototipo de esta funci\'on es el siguiente:

\begin{center}
\emph{int \textbf{EulerModificado}(DatosRK *ptstrDatos)}
\end{center}

\begin{description}
\item[ptstrDatos] puntero a una variable de \emph{estructura} del tipo
\emph{DatosRK}.
\end{description}

La funci\'on devuelve los siguientes c\'odigos:

\begin{center}
\begin{tabular}{|l|l|}
\hline
\textbf{ERR\_AMEM} & Hubo un error en la asignaci\'on de memoria. \\
\hline
\textbf{TRUE} & Se inicializaron con \'exito los coeficientes. \\
\hline
\end{tabular}
\end{center}

Por ejemplo:

\begin{center}
\emph{intResultado = \textbf{EulerModificado}(\&varstrDatRK);}
\end{center}


Inicializar\'{\i}a los coeficientes del m\'etodo en la variable
\emph{varstrDatRK}, en \emph{intResultado} el valor \textbf{TRUE} si se pudieron
inicializar los coeficientes y en caso contrario \textbf{ERR\_AMEM}.

\newpage

\subsection{EulerMejorado}
Funci\'on que inicializa los coeficientes para el m\'etodo de \emph{Euler mejorado},
el cual es un m\'etodo \emph{Runge-Kutta} de $2$ etapas y orden $2$.\newline

La notaci\'on matricial del m\'etodo es la siguiente:

\begin{center}
$
\begin{array}{c|cc}
0 & 0 \\
1 & 1 & 0 \\
\hline
 & \frac{1}{2} & \frac{1}{2}
\end{array}
$
\end{center}

El prototipo de esta funci\'on es el siguiente:

\begin{center}
\emph{int \textbf{EulerMejorado}(DatosRK *ptstrDatos)}
\end{center}

\begin{description}
\item[ptstrDatos] puntero a una variable de \emph{estructura} del tipo
\emph{DatosRK}.
\end{description}

La funci\'on devuelve los siguientes c\'odigos:

\begin{center}
\begin{tabular}{|l|l|}
\hline
\textbf{ERR\_AMEM} & Hubo un error en la asignaci\'on de memoria. \\
\hline
\textbf{TRUE} & Se inicializaron con \'exito los coeficientes. \\
\hline
\end{tabular}
\end{center}

Por ejemplo:

\begin{center}
\emph{intResultado = \textbf{EulerMejorado}(\&varstrDatRK);}
\end{center}


Inicializar\'{\i}a los coeficientes del m\'etodo en la variable
\emph{varstrDatRK}, en \emph{intResultado} el valor \textbf{TRUE} si se pudieron
inicializar los coeficientes y en caso contrario \textbf{ERR\_AMEM}.

\section{M\'etodos impl\'{\i}citos}

\section{M\'etodos semiimpl\'{\i}citos}



%%%%%%%%%%%%%
% Apendices %
%%%%%%%%%%%%%

\appendix

%
% Apendice sobre metodos Runge - Kutta 
%

%
% Apendice sobre metodos Runge - Kutta 
%

\chapter{Runge-Kutta methods} \label{sec:Runge}

This appendix is intended to help to know how Runge-Kutta methods are implemented and used in this library.

\section{What is a Runge-Kutta method?}

\textbf{Runge-Kutta} methods are a family of numerical methods to approach solutions of ordinary differential equations (O.D.E). These methods are iterative methods used to solve ``\emph{initial problem value}'' (\textbf{I.P.V}) or ``\emph{Cauchy problem}''.\\

These methods are only-one-step methods with a fixed size for the method step\footnote{It is also possible to implement methods with a variable step known as \emph{embedding}.}.\\

\subsection{What is a I.V.P.?}

An \emph{I.V.P.} is:

\begin{equation} \label{eq:IVP}
\left\{ \begin{array}{l}
y' = f(x, y(x))\\
y(x_0) = y_0\\
\end{array} \right.
\end{equation}
%
So $y'$ is a function depending on the variable $x$, and the function $y(x)$. $y(x)$ is the solution of the equation \ref{eq:IVP} and the point $(x_0,y_0)$ belongs to the curve $y(x)$.\\

Solving the \emph{I.V.P.} \ref{eq:IVP} is finding a function $y(x)$ such as the equation \ref{eq:IVP} is met.\\

An example of a \emph{I.V.P.}:
%
\begin{equation} \label{eq:IVPej}
\left\{ \begin{array}{l}
y' = \frac{x * y(x) - y(x)^2}{x^2} \\
y(1) = 2 \\
\end{array} \right.
\end{equation}
%
The solution of the \ref{eq:IVPej} will be:
%
\begin{equation}
y(x) = \frac{x}{\frac{1}{2}+\ln x}
\end{equation}

\section{Runge-Kutta's method notation}

$y(x_i)$ will be the exact value of the function $y(x)$ evaluated in $x_i$.\\ \\
$y_i$ will be the approximation of the function $y(x)$ in the point $x_i$.\\ \\
$h$ is the step used by the method in each iteration.

\subsection{General formulation}

A $s$-stages \textbf{Runge-Kutta}'s method formulation is:
%
\begin{equation}
y_{n+1} = y_{n} + h \cdot \sum_{i=0}^{s-1} b_i \cdot k_i
\end{equation}
%
where:
%
\begin{equation}
k_i = f(x_n + c_i \cdot h, y_n + h \cdot \sum_{j=0}^{s-1} a_{i,j} \cdot k_j)
\end{equation}
%
satisfying:
%
\begin{equation}
\sum_{j=0}^{s-1} a_{i,j} = c_i
\end{equation}

\subsection{Matricial notation (Butcher's)}

Matricial notation is used to represent method's coeficients using a matrix.\\

For a $s$-stages \textbf{Runge-Kutta} method the matricial notation will be:
%
\begin{center}
\begin{displaymath}
\begin{array}{c|ccc}
c_0 & a_{0,0} & \cdots \cdots & a_{0,s-1} \\
\vdots & \vdots & & \vdots \\
\vdots & \vdots & & \vdots \\
c_{s-1} & a_{s-1,0} & \cdots \cdots & a_{s-1,s-1} \\
\hline
 & b_0 & \cdots \cdots & b_{s-1} \\
\end{array}
\end{displaymath}
\end{center}

\note{In section \ref{sec:biaButcherArray} is shown a data structure used to store the Butcher array.}

\section{Runge-Kutta types}

There are several types of \textbf{Runge-Kutta} methods.

\subsection{Implicit Runge-Kutta}

A \textbf{Runge-Kutta} method is said to be implicit when the $a_{i,j} \neq 0$ for some $j > i$.\\

The $2$-stages Gauss method is an implicit \textbf{Runge-Kutta} method of $2$-stages:
%
\begin{center}
\begin{displaymath}
\begin{array}{c|cc}
\frac{3-\sqrt 3}{6} & \frac{1}{4} & \frac{3-2*\sqrt 3}{12} \\
\frac{3+\sqrt 3}{6} & \frac{3+2*\sqrt 3}{12} &\frac{1}{4} \\
\hline
 & \frac{1}{2} & \frac{1}{2}
\end{array}
\end{displaymath}
\end{center}

\subsection{Semi-implicit Runge-Kutta}

A \textbf{Runge-Kutta} method is said to be semi-implicit when the $a_{i,j} = 0$ when $j > i$.\\

A $2$-stages semi-implicit \textbf{Runge-Kutta} method:
%
\begin{center}
\begin{displaymath}
\begin{array}{c|cc}
\frac{3+\sqrt 3}{6} & \frac{3+\sqrt 3}{6} & 0 \\
\frac{3-\sqrt 3}{6} & \frac{-\sqrt 3}{3} & \frac {3+\sqrt 3}{6} \\
\hline
 & \frac{1}{2} & \frac{1}{2}
\end{array}
\end{displaymath}
\end{center}

\subsection{Explicit Runge-Kutta}

A \textbf{Runge-Kutta} method is said to be explicit when the $a_{i,j} = 0$ when $j \geq i$.\\

A $4$-stages explicit \textbf{Runge-Kutta} method also known as ``\textbf{classic Runge-Kutta}'':
%
\begin{center}
\begin{displaymath}
\begin{array}{c|cccc}
0 & 0 \\
\frac{1}{2} & \frac{1}{2} & 0 \\
\frac{1}{2} & 0 & \frac{1}{2} & 0 \\
1 & 0 & 0 & 1 & 0 \\
\hline
 & \frac{1}{6} & \frac{1}{3} & \frac{1}{3} & \frac{1}{6}
\end{array}
\end{displaymath}
\end{center}


\end{document}
