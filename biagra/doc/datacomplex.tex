%
% datacomplex.h
%

\chapter{Complex number data structures (datacomplex.h)}

\section{Introduction}

Data structures for complex numbers are defined in \texttt{datacomplex.h} file.

\section{\textbf{biaComplex} data structure} \label{sec:biaComplex}

This data structure is used to handle polinomials $p(x) \in \mathbb{R}[x]$. \textbf{biaComplex} data structure is defined in figure \ref{fig:biaRealPol} where:

\begin{description}
\item[intDegree] polynomial degree.
\item[intRealRoots] number of real roots (if any).
\item[intCompRoots] number of complex roots (if any).
\item[*dblCoef] pointer to store polynomial coeficients.
\end{description}

\begin{figure}[!h]
\begin{verbatim}
typedef struct {
  double dblReal,
         dblImag;
  } biaComplex;
\end{verbatim}
\caption{biaComplex data structure.} \label{fig:biaComplex}
\end{figure}

\section{\textbf{biaPolar} data structure} \label{sec:biaPolar}

This data structure is used to store data for root approximation. Data structure is defined in figure \ref{fig:biaRealRoot} where:

\begin{description}
\item[intNMI] maximum number of iterations to get the root with a maximum error of \emph{dblTol}.
\item[intIte] iterations used to get the root.
\item[dblx0] initial approximation to get the root.
\item[dblRoot] root approximation.
\item[dblTol] maximum tolerance when calculating the root.
\item[dblError] error in root approximation. Difference between the las two root approximations.
\end{description}

\begin{figure}[!h]
\begin{verbatim}
typedef struct {
  double dblMod,
         dblArg;
  } biaPolar;
\end{verbatim}
\caption{biaPolar data structure.} \label{fig:biaPolar}
\end{figure}
