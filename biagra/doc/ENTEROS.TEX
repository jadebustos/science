%
% ENTEROS.H
%

\chapter{enteros.h}

\section{Introducci\'on}

En este fichero de cabezera estan los prototipos de funciones para el 
tratamiento de n\'umeros enteros.

\section{Suma de n\'umeros enteros}

\subsection{SumaNumerosN}

Funci\'on que suma los $n$ primeros n\'umeros naturales.\newline

El prototipo de esta funci\'on es el siguiente:

\begin{center}
\emph{\textbf{unsigned} SumaNumerosN(int intNumero)};
\end{center}

Esta funci\'on suma todos los n\'umeros enteros entre el $1$ y \emph{intNumero},
ambos inclusive, y la devuelve como un entero ``sin signo''. Si la suma excede
el rango entonces devolver\'a un cero.\newpage

Por ejemplo:

\begin{center}
\emph{unsgResultado = \textbf{SumaNumerosN}(intNumero)};
\end{center}

Almacenar\'{\i}a en \emph{ungsResultado} la siguiente suma:

\begin{displaymath}
\sum_{i=1}^{intNumero} i
\end{displaymath}

siempre y cuando no exceda el rango de \emph{unsigned}\footnote{El m\'aximo
valor almacenable en una variable de este tipo viene dada por la constante
simb\'olica \textbf{UINT\_MAX} que esta definida en el fichero de cabezera
\textbf{limits,h}.}, en caso contrario almacenar\'a el valor $0$.\newline

Observaciones:

\begin{itemize}
\item La funci\'on considera \emph{intNumero} como un n\'umero positivo.
\end{itemize}
